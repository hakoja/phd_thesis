\chapter{Introduction}\label{sec:intro}


\begingroup
\hypersetup{linkcolor=black}
\minitoc
\endgroup


\vspace{1ex}

Designing secure cryptographic protocols is difficult.
Over the years a large number of security protocols have been proposed that later turned out to be flawed. 
This is mostly due to the inherent complexity of the protocols themselves,
but it can also be partly ascribed to the  paradigm in which they were traditionally designed.
Typically,
a protocol designer would start out by proposing some concrete protocol construction $P$.
Next,
the protocol would get analyzed,
often revealing some flaw.
The designer would then revise the original design of $P$ to (hopefully) include a fix for the discovered flaw.
The whole cycle would then repeat itself,
with a new round of analysis discovering new flaws,
yielding more fixes,
and so on.    


Over time,
a body of prudent practices emerged~\cite{AbadiN:1996:prudent_engineering},
identifying common pitfalls when designing cryptographic protocols. 
However,
these practices represented no more than useful heuristics and guidelines,
rather than necessary and sufficient criteria for creating secure protocols.
Within the academic cryptography community this realization led to an interest in finding more rigorous and formal approaches towards assessing the security of a protocol.  

Traditionally,
two distinct approaches have been taken in order to formally model cryptographic protocols.
The first,
and the one we will be following in this thesis,
is the \emph{computational} approach.
As its name suggests,
it has its roots in computational complexity theory and views cryptographic operations as algorithms working on bitstrings.
Adversaries are modeled as probabilistic Turing machines and security is expressed in terms of the probability and computational complexity of carrying out a successful attack.
We will have more to say about the computational model below,
as well as in \cref{sec:definitions}.


The second approach is the \emph{symbolic} approach,
also called \emph{formal methods}.
It has its roots in logic and formal language theory and views cryptographic operations as functions on formal symbolic expressions.
A symbolic security model consists of a set of axioms and inference rules that can be applied to the symbolic expressions.
For example,
a formal expression of the form $\symbolicencrypt{M}_K$ could represent the encryption of a message $M$ under some key $K$.
Note,
however,
that both $M$ and $K$ are also formal expressions,
and thus carry no inherent meaning.
An inference rule could say that,
given $\symbolicencrypt{M}_K$ and $K$,
one can conclude $M$.
That is,
the inference rule allows you to decrypt $M$ from  $\symbolicencrypt{M}_K$  given $K$. 
On the other hand,
without $K$ it is impossible to deduce $M$.
In particular,
since only operations derivable from the inference rules are possible,
cryptographic primitives in the symbolic model are \emph{perfect}.
Security in the symbolic model is expressed as saying that one cannot reach a certain configuration by applying the inference rules,
starting from the given axioms.
Unlike in the computational approach,
there is no probabilistic reasoning in the symbolic world.

A major benefit of the symbolic model is that it readily allows for machine-checkable proofs,
or even automatic derivation of proofs.
Many tools exist for this purpose,
including ProVerif~\cite{Blanchet:2016:ProVerif}, Scyther~\cite{Cremers:2008:Scyther}, and Tamarin~\cite{MeierSCB:2013:Tamarin}.
On the other hand,
a common criticism of the symbolic approach is that its assumption of perfect black-box primitives is unrealistic.
A~protocol proven secure in the symbolic model may nevertheless have an attack in the computational model.
%This makes it difficult to assess what a proof in the symbolic model has to say about a real-world protocol
%(although a similar thing could criticism can also be  about proofs in the computational model). 
Still,
there have been attempts to bridge the gap between the computational model and the symbolic model,
beginning with~\cite{AbadiR:2000:two_views}.

Finally,
there have also been much recent development in tools that can automatically verify proofs in the computational model,
such as CryptoVerif~\cite{Blanchet:2008:CryptoVerif},
EasyCrypt~\cite{BartheDGKSS:2013:EasyCrypt},
and miTLS~\cite{SP:BFKPS13}.
%However,
%these tools are generally less automatic than those working in the symbolic model.
Although this thesis will be based on the computational model,
it will,
however,
not be making use of any of these tools.
Instead,
it will follow the more traditional style of ``pen-and-paper'' proofs.
Moreover,
since our security models will be in the computational setting,
we will not be saying more about the symbolic model in this thesis.
As a result,
most of our literature references will be to results in the computational model.
At the same time,
we acknowledge that there is a vast body of cryptographic research that consequently will not be covered here.




\section[Computational modeling of cryptographic protocols]{Computational modeling of cryptographic protocols}

The idea of formalizing cryptography within a computational complexity theoretical setting was introduced by Goldwasser and Micali in 1984~\cite{GoldwasserM:1984:probabilistic_encryption}.
Central to their work was the formal \emph{definition} of what it means for a cryptographic scheme to be \emph{secure}.
Specifically,
they focused on the goal of public-key encryption and formalized the now fundamental definition of \emph{semantic security}.
To go along with their new definition,
they also created a concrete scheme which they could now \emph{prove} satisfied the definition of semantic security by giving a reduction to a number-theoretic assumption.
Soon after,
many other common cryptographic primitives,
like digital signatures,
symmetric encryption,
pseudorandom functions,
and
message-authentication schemes
were formalized (and proven secure) in a similar manner.

However,
it would go almost 10 years from Goldwasser and Micali's initial paper until the first formal model for cryptographic protocols was presented by Bellare and Rogaway~\cite{C:BelRog93} in 1993.
On the other hand,
their model became highly influential for the formal research on protocols,
in particular \emph{key exchange} protocols,
and it is still the basis for many of the models used today. 

%
%Below we present a high-level overview of their model---simply called the 
%BR-model---as an illustration of how KE protocols can be formally modeled.
%Technically speaking,
%we do not actually present the BR-model as originally described by Bellare and Rogaway~\cite{Bellare:1994:BR93},
%but instead we use the term more broadly to encompass a number of technically different models that are nevertheless similar in spirit.
%






\paragraph{The BR-model.}
The starting point of the BR-model is a set of principals that want to communicate over an insecure network.
Every pair of principals shares a common long-term key,
and their goal is to negotiate a temporary \emph{session key} which they will use to secure their further communication.
In the formal model the details of the communication network is mostly abstracted away,
leaving only the principals themselves and a specification of how they behave on receiving input from the network.
How the messages are delivered to each principal is left to the adversary's discretion,
i.e., in the BR-model the adversary \emph{is} the network.
In particular,
while the adversary can chose to forward messages as intended by the protocol,
it also has full freedom to arbitrarily change, delay, reorder, reroute or  drop messages as it sees fit.
It is important that we allow the attacker this kind of flexibility since we want our protocols to be secure from \emph{any} choice of adversarial strategy.
That is,
in general it is impossible to enumerate every possible way that a protocol might get attacked,
so the only thing we can reasonably make assumptions about is the attacker's \emph{computational powers}.

Depending on the type of protocol,
its security goals may vary.
Classically,
the goals considered by Bellare and Rogaway~\cite{C:BelRog93},
were those of \emph{authenticated key exchange} and \emph{entity authentication}.
The first property focuses on the security of the established session keys themselves.
The formal definition of this borrows from the idea of semantic security for public-key encryption schemes,
and demands that an adversary should learn nothing about the distributed keys.
The second property focuses on the authenticity of the protocol conversation,
meaning that two protocol participants can be assured that they have in fact been speaking to each other at the end of the protocol run.
There are also protocol goals beyond those of authenticated key exchange and entity authentication,
for example goals focusing on the secure \emph{usage} of the distributed session keys.
This will all be covered in  detail in \cref{sec:definitions}.



\paragraph{Simulation-based vs. game-based security.}
Within the computational setting,
there are two main approaches to defining the security of a protocol. 
One is the \emph{simulation-based} approach
and the other is the \emph{game-based} approach.
In the simulation-based approach,
security is defined by considering two ``worlds'': 
an \emph{ideal} world where the protocol is replaced with some \emph{idealized functionality}  that is secure by design;
and a \emph{real} world where the actual protocol is being used.
Security is expressed by saying that for any attacker $\A$ against the protocol in the real world,
there should exist a corresponding \emph{simulator} $\A[S]$ in the ideal world,
such that the transcript that $\A$ generates through its interactions with the real protocol,
is \emph{computationally indistinguishable} from the transcript that $\A[S]$ generates through its interaction with the ideal functionality.
Since the ideal functionality is secure by design,
the existence of $\A[S]$ means that $\A$'s ability to break the real protocol must be limited.

A number of simulation-based models have been developed in order to analyze protocols.
Examples include the model of Shoup~\cite{EPRINT:Shoup99b},
the UC model of Canetti~\cite{FOCS:Canetti01},
the IITM model of Küsters and Tuengerthal~\cite{EPRINT:KueTue13},
and the GNUC model of Hofheinz and Shoup~\cite{JC:HofSho15}.
Of these,
the latter three are so-called \emph{universal composability} models,
where the emphasis is on very general composition results that allow secure sub-protocols to be arbitrarily composed in order to form larger and still secure protocols.
Due to their generality,
universal composability models tend to be quite complex.  

The alternative to simulation-based models is \emph{game-based} models.
Here,
security properties are formulated directly as \emph{winning conditions} in a formal experiment,
called a \emph{game},
played between an honest entity $\A[C]$ called the \emph{challenger},
and an adversary $\A$.
A protocol is said to be \emph{secure} with respect to the property modeled by the game,
if no computationally efficient adversary can manage to win in the game except with a small probability.
What ``efficient'' and ``small'' means in this setting can be formalized in different ways;
again we refer to~\cref{sec:definitions} for further details.

The original BR-model~\cite{C:BelRog93} was in the game-based setting,
and naturally so were also the large number of extensions and follow-up works that built on it,
for example~\cite{STOC:BelRog95,Blake-WilsonM:1997:BR93_asymmetric,EC:BelPoiRog00,EC:CanKra01,PROVSEC:LaMLauMit07,C:JKSS12}.
In this thesis we are going to take the game-based approach to security.



\section{Content and contribution of thesis}


This thesis provides a formal security analysis of the Extensible Authentication Protocol (EAP) \cite{IETF:RFC3748:EAP} and the IEEE~802.11~\cite{IEEE:2012:802.11} protocol in a computational game-based setting.
Compared to the Transport Layer Security (TLS)~\cite{IETF:RFC5246:TLS} protocol,
which has been subject to a large amount of formal analysis,
both EAP and IEEE~802.11 have received considerably less scrutiny. 
That is not to say that EAP and IEEE~802.11 are little used;
quite the contrary.
For instance,
according to the Wireless Geographic Logging Engine (WiGLE)\footnote{\url{https://wigle.net/}.
} project,
there are more than 350 million Wi-Fi networks available worldwide today---Wi-Fi being the name more commonly associated with IEEE~802.11.
Similarly,
%EAP is also used by a large number of organizations and enterprises.
the \emph{eduroam}\footnote{\url{https://www.eduroam.org/}} network,
which is a roaming service provided to students and employees of educational institutions around the world,
alone accounted for more than 3 billion user authentications in 2016\footnote{\url{https://www.eduroam.org/2017/03/07/2016-a-record-breaking-year-for-eduroam/}
}---all of which use EAP.
The importance of these protocols should thus be clear from the sheer scale of their deployment.
  
The main contribution of this thesis is a formal analysis of the EAP and IEEE~802.11 protocols in a computational setting based on the BR-model.
Our analysis will cover these protocols both separately and when combined 
(since EAP and IEEE~802.11 are often used together). 
\cref{sec:descriptions:EAP_&_802.11} will describe EAP and IEEE~802.11 in detail,
but here we nevertheless give a very brief description of these protocols so as to illustrate the main results of the thesis.
Hopefully,
Wi-Fi,
and thus IEEE~802.11,
should be well-known to everyone:
a wireless client and an access point use a shared secret,
typically a password,
to protect the wireless link between them. 
This involves an initial key exchange phase,
where the client and access point derive a cryptographic key from the common secret,
and a channel encryption phase,
where the application data is being sent.
At the same time,
IEEE~802.11 can also be used in situations where the client and access point do not share a common secret beforehand.
This is exactly the setting of the eduroam network mentioned above.
Here,
they will first use a trusted third-party to help them establish a common secret.
The protocol used to facilitate this is EAP.

EAP specifies a way for two parties to establish a common secret through the help of a trusted third-party.
However,
rather than viewing EAP as a single protocol,
it can be better thought of as a protocol \emph{framework} used to compose other concrete protocols.
For the EAP framework to be secure the concrete protocols need to be safely instantiated,
but EAP itself does not specify them.
IEEE~802.11 is commonly used as one of the concrete sub-protocols in the EAP framework,
but it does not have to be; 
EAP is mostly protocol agnostic.



Given these high-level descriptions of EAP and IEEE~802.11,
our results can be summarized as follows.
Below we refer to security notions such as authenticated key exchange and secure channel protocols only informally.
Their formal definitions will be made precise in \cref{sec:definitions}. 


\begin{itemize}

	\item 
	Our first result is a game-based security analysis of the general EAP framework.
	This involves two generic composition theorems that abstract away the concrete protocols used within EAP.
	Instead,
	the theorems establish sufficient criteria on the protocol building blocks in order for the EAP framework to be instantiated securely.
	The overall security goal of EAP that we aim for is that of a three-party authenticated key exchange.
	Having proven the soundness of the general EAP framework,
	it remains to find suitable concrete protocols that satisfy the security criteria laid down by the composition results. 
	
	\item 
	One such concrete protocol is EAP-TLS~\cite{IETF:RFC5216:EAP-TLS},
	which within the EAP framework is used between the  client and the trusted third-party.
	We prove that EAP-TLS is a secure two-party authenticated key exchange,
	which is sufficient in order to be used in our  compositions results. 
	However,
	this result also has independent interest outside of the EAP framework,
	because of the way it is established.
	Essentially,
	we give a generic transformation that shows how secure channel protocols can be turned into secure key exchange protocols by exporting additional session keys from their handshake protocols.
	This has applications to the practice of exporting extra keys from the TLS handshake,
	since TLS is a secure channel protocol,
	but \emph{not} a secure key exchange protocol
	(we return to this point in \cref{sec:definitions}).
	
	
	
	\item Finally,
	we analyze the IEEE~802.11 protocol.
	Again, this analysis has independent interest outside of the EAP framework,
	since IEEE~802.11 is often used without EAP.
	Recall from our brief description above that IEEE~802.11 proper consists of a key exchange phase followed by a channel encryption phase.
	We prove that the former constitutes a secure two-party authenticated key exchange protocol,
	and that the latter satisfies the notion of a secure stateful authenticated encryption scheme.  
	Although these results are of independent interest,
	they also combine with our EAP composition theorems to culminate in our biggest main result:
	namely the security of EAP and IEEE~802.11 used together.
	
\end{itemize}

The results outlined above roughly correspond to the contents of \cref{sec:generic_composition_results},
\cref{sec:EAP-TLS-security},
and \cref{sec:802.11},
respectively.





\paragraph{Modularity.}
A common theme among all the results established in this thesis is an emphasis on reusing  existing security results as far as possible.
%This is especially apparent in our composition results for the EAP framework,
%as well as the generic transformation we use to prove EAP-TLS.
For example,
the TLS protocol is an important component in both EAP and EAP-TLS,
but we do not want to redo any analysis of TLS for the purposes of establishing our results.
Instead,
we want to be able to leverage the large amount of already existing analysis of TLS in a black-box manner.
This requires generic and modular results,
but it also requires security models that are comparable.
This is one of the reasons why we have chosen to use a game-based formulation of security over a simulation-based formulation.
Many of the existing results on the real-world protocols we care about,
such as TLS, IPsec, and SSH, 
are for the most part proven in a game-based setting.
%Admittedly,
%this is somewhat ironic given the compositional nature of our results,
%since this is exactly what universal composability models were created for.
%


\subsection{Publications}

The material in this thesis is primarily based on the following two papers:
\begin{itemize}[leftmargin=*,labelindent=2.5em]
	\item[\cite{PKC:BrzJac17}] \fullcite{PKC:BrzJac17}. 
	
	\item[\cite{EC:BrzJacSte16}] \fullcite{EC:BrzJacSte16}. 
\end{itemize}

Specifically,
the material found in \cref{sec:generic_composition_results} and \cref{sec:802.11} of this thesis is taken from~\cite{PKC:BrzJac17},
while the material in \cref{sec:EAP-TLS-security} comes from~\cite{EC:BrzJacSte16}.
However,
the content as it appears in this thesis has undergone major revisions compared to the original publications.
Moreover,
this thesis also introduces some new material not found in either of the published papers.
In particular,
\cref{sec:802.11:CCMP} and \cref{sec:802.11:multi-cipher} present some additional results and discussions on IEEE~802.11.




\subsection{Outline of thesis}

In \cref{sec:descriptions:EAP_&_802.11} we give detailed protocol descriptions of EAP and IEEE~802.11.
In \cref{sec:definitions} we provide the formal framework that will be used to analyze EAP, EAP-TLS and IEEE~802.11 in the later chapters.
Since our analyses cover a wide range of different protocols,
a great number of definitions and notions are needed.
We have tried to discuss and justify all of our definitional choices to the greatest extent possible.


In \cref{sec:generic_composition_results} we conduct our first security analysis,
beginning with the general EAP framework.
The main results of the chapter are two modular and generic protocol composition theorems.
Then,
in \cref{sec:EAP-TLS-security},
we analyze one specific component in the EAP framework,
namely the EAP-TLS protocol. 
However,
although the starting point is the concrete EAP-TLS protocol,
the main result of the chapter is again a generic result with applications beyond the immediate scope of EAP-TLS.

In \cref{sec:802.11} we analyze the IEEE~802.11 protocol.
The main technical result is an analysis of the IEEE~802.11 key exchange protocol when considered as a standalone protocol---as it is typically used in home networks.
However,
the result additionally combines with the composition theorems of \cref{sec:generic_composition_results} to yield a result for IEEE~802.11 combined with EAP. 
Furthermore,
\cref{sec:802.11} also presents some new material on IEEE~802.11 which have not appeared elsewhere,
including and analysis of the IEEE~802.11 data encryption algorithm called CCMP,
as well as a discussion of the multi-ciphersuite and negotiation security of IEEE~802.11.


Finally,
in \cref{sec:conclusions} we conclude the thesis by putting our work in a larger context and by discussing some of its limitations.
We also point to some possible directions for future work.
 

\paragraph{Note.}
We use the symbol ``$\blacktriangle$'' to denote the end of a remark or example,
and use the symbol ``$\blacksquare$'' to denote the end of a proof.