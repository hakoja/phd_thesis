\subsection{Full proof of Theorem~\ref{thm:protocol_5:3P-AKE}}

\begin{proof}[of Theorem~\ref{thm:protocol_5:3P-AKE}]
First we need to define a matching function for protocol $\protocol_5$.
To this end, we construct $f_5$ from the assumed partner functions $f_3$ and $f_4$,
just like how $f_3$ was constructed from $f_1$ and $f_2$ in the previous composition theorem (Theorem~\ref{thm:protocol_3:3P-AKE-}).
\setcounter{gamehop}{0}
\item
\paragraph{Game~\game:}
This is the real 3P-AKE security game, hence
\begin{equation*}
	\adv_{\protocol_5,\A,f_5}^{\mathsf{G_\game}}(\secparam) = \adv_{\protocol_5{},\A,f_5}^{\operatorname{\mathsf{3P-AKE}}}(\secparam) \ .
\end{equation*}

\newgame
\paragraph{Game~\game:}
In this game, the challenger aborts if a session accepts maliciously in protocol $\Pi_5$.
By the Difference Lemma~\cite{Shoup:2004:sequence_of_games}, it follows that
\begin{equation*}
	\adv_{\protocol_5,\A,f_5}^{\mathsf{G_{\prevgame{}}}}(\secparam) 
	\leq \adv_{\protocol_5,\A,f_5}^{\mathsf{G_\game}}(\secparam) 
	+ \adv_{{\Pi_5},\A,{f_5}}^{\operatorname{\mathsf{3P-AKE-EA}}}(\secparam) ,
\end{equation*}
and $\adv_{{\Pi_5},\A,{f_5}}^{\operatorname{\mathsf{3P-AKE-EA}}}(\secparam)$ will be bounded in Lemma~\ref{thm:protocol_5:3P-AKE-EA}.

\newgame
\paragraph{Game~\game:}
In this game the challenger guesses two sessions, i.e, two pairs of indices $(i,s) \gets \mathcal{P} \times \mathbb{N}$ and $(j,t) \gets (\mathcal{P} \times \mathbb{N})$. 
If it picks the same pair twice, this session is not expected to get a partner, we call such a session ``single''. The game challenger aborts, if the following bad event occurs:
\begin{quote}
(i) The  challenger guessed that the session was a single session, but it gets a partner in $\protocol_3$, or (ii) one of the two guessed session becomes partnered in $\protocol_3$ with another session, or (iii) adversary $\A$ makes a $\Q{Corrupt}$ query that would make any of the guessed sessions unfresh in $\protocol_3$, or (iv) adversary issues a $\Q{Test}$ query to a session that is neither $(i,s)$ nor $(j,t)$, or (v) adversary terminates and the two guessed sessions are not partnered in $\protocol_3$.\footnote{Assuming again that the adversary always makes a $\Q{Test}$ query before terminating.}
\end{quote}

\noindent
Note that item (iv) is different from the similar bad event in the previous composition theorem.
This time, we do not make a distinction between $(i,s)$ being tested and $(j,t)$ being tested. 
This will be used in the next game-hop.
\begin{lemma}
$
	\adv_{\protocol_5,\A,f_5}^{\mathsf{G_{\prevgame{}}}}(\secparam) 
	\leq n^2 \cdot \adv_{\protocol_5,\A,f_5}^{\mathsf{G_\game}}(\secparam),
$
\end{lemma}

\begin{proof}
Note that, if the challenger guesses the test session (and its partner) correctly, 
then the bad event does not occur, and the challenger guesses the right (pair of) session(s) with probability at least $1 / n^2$. 
If the bad event does not occur, then $\A$'s success probability is the same in $G_{\prevgame}$ and $G_{\game}$, and the lemma follows.
\qed
\end{proof}
In the remaining games,
let $\oracle^* = \oracle[i][s]$ denote the guessed test-session and $\oracle' = \oracle[j][t]/\bot$ its partner.
%Define $\oracle[T][u]$ to be the \emph{co-partner} of $\oracle^* = \oracle[i][s]$
%if $(P_T, \ttp) \in \oracle^*.\peer$,
%and either $\oracle^*.\role = \initiator \wedge f_{1,T_1}(\oracle^*) = \oracle[T] $
%or $\oracle^*.\role = \responder \wedge f_{2,T_2}(\oracle^*) = \oracle[T]$. 
%That is, $\oracle[T][u]$ is the trusted third-party session in the protocol run involving~$\oracle^*$ (and $\oracle'$). 

\newgame
\paragraph{Game~\game:}
When one of the two guessed sessions $\oracle^*$ or $\oracle'$ accepts in $\protocol_3$, 
replace the ``intermediate'' key $\KDkey$
%output by protocol $\protocol_3$
with a random key $\KDkeyrand$. 
When the other session accepts, use the same key.
%\item
%\vspace{-\baselineskip}

\begin{lemma}\label{lemma:3P-AKE:3P-KD-swap-random}
$
	\adv_{\protocol_5,\A,f_5}^{\mathsf{G_{\prevgame{}}}}(\secparam) 
	\leq \adv_{\protocol_5,\A,f_5}^{\mathsf{G_\game}}(\secparam) 
	+ \adv_{\protocol_3, \A[B]_1,f_3}^{\operatorname{\mathsf{3P-AKE}}^w}(\secparam)  .
$
\end{lemma}


\begin{proof}
From $\A$ we build an adversary $\A[B]_1$ against the AKE$^w$-security of the 3P-AKE protocol $\protocol_3$ as follows.
$\A[B]_1$ begins by choosing a random bit $b$ and picking twice random indices $(i,s)$ and $(j,t)$ to determine
session $\oracle^*$ and $\oracle'$, respectively. It starts running $\A$.
For all of $\A$'s $\Q{Send}$ queries pertaining to the 3P-AKE protocol $\protocol_3$,
adversary $\A[B]_1$ answers them by making the corresponding queries to its own 3P-AKE experiment. 
When the first of $\oracle^*$ or $\oracle'$ accepts, $\A[B]_1$ makes a $\Q{Test}$ query to obtain its session key in sub-protocol $\protocol_3$
(which is either its actual key or a random key).
When the second session out of $\oracle$ or $\oracle'$ accepts in $\protocol_3$, 
then $\A[B]_1$ uses the same key that was answered by the $\Q{Test}$ query.
Whenever any other session accepts in sub-protocol $\protocol_3$, then
$\A[B]_1$ makes a $\Q{Reveal}$ query to obtain the session key it derived.
In all cases, the so-obtained key becomes the ``intermediate'' session key $\KDkey$ for use as the long-term key in the emulated sub-protocol $\protocol_4$. All of $\A$'s $\Q{Send}$ queries pertaining to the 2P-AKE protocol $\protocol_4$ can now be answered as normal using the derived ``intermediate'' keys $\KDkey$. 
$\A[B]_1$ can answer $\A$'s $\Q{Test}$ queries using the bit $b$ it selected and the session keys it derived in sub-protocol $\protocol_4$.
Throughout the game, $\A[B]_1$ checks whether the next query would cause a malicious accept or make the bad event happen. If so, it aborts and outputs a random bit. Note again that the malicious accept and the bad event are checkable based on public information.
Let $b'$ be the output of $\A$.
If $b' = b$,
then $\A[B]_1$ outputs $1$.\todo{HJ: should it be 0 or 1 here?}
Otherwise, $\A[B]_1$ outputs $0$.

\medskip
We need to argue that if $\A[B]_1$ makes a $\Q{Corrupt}$ query that does not harm the freshness of $\Pi_5$, then it also does not harm freshness in $\Pi_3$. If $\A[B]_1$ makes a $\Q{Corrupt}$ query to the identity of $\pi^*$ or its intended partner, without harming the freshness of $\pi^*$, then this means that $\pi^*$ has already accepted by definition of freshness. Therefore, as malicious accept has
not happened, $\pi^*$ has a partner session, which is $\pi'$, and by definition of $f_5$, $\pi'$ is also the partner of $\pi^*$ according
to $f_3$ and $f_4$ and $\pi'$ has accepted in sub-protocol $\protocol_3$. Hence, also the \AKEm model allows a $\Q{Corrupt}$ query to the identity of $\pi^*$ or its intended partner without making $\pi^*$ unfresh.

Therefore, when the 3P-AKE game returns a real key as an answer to the $\Q{Test}$ query, then $\A[B]_1$ simulates Game~\prevgame{} perfectly,
%, because the bad event can be efficiently checked based on public information on $\protocol_3$ and hence, in this case, 
i.e., $\A[B]_1$ outputs $1$ with exactly the same probability that $\A$ wins in Game~\prevgame{}. Likewise, if the 3P-AKE game returns random key as an answer to the $\Q{Test}$ query, then $\A[B]_1$ perfectly simulates Game~\game{} and hence, it outputs $1$ with exactly the same probability that $\A$ wins in Game~\game{}. Thus, the advantage of $\A[B]_1$ in breaking $\protocol_3$ is equal to the difference that $\A$ wins in Game~\prevgame{} and that $\A$ wins in Game~\game{}.
\qed
\end{proof}
\item
Finally, we show that any attack of $\A$ in Game $G_\game$ can be transformed into an attack on $\protocol_4$.
\begin{lemma}\label{lemma:3P-AKE:2P-AKE-swap-random}
$
	\adv_{\protocol_5\A,f_5}^{\mathsf{G_{\game{}}}}(\secparam) 
	\leq  \adv_{\protocol_4,\A[B]_2,f_4}^{\operatorname{\mathsf{2P-AKE-static}}}(\secparam)  .
$
\end{lemma}

The proof of this lemma uses $\A$ to build an adversary $\A[B]_2$ against the security of the 2P-AKE-static protocol $\protocol_4$. 
The proof is straightforward and uses the same simulation techniques as the previous proofs,
hence we omit it. 
Putting together all the above lemmas yields a proof of Theorem~\ref{thm:protocol_5:3P-AKE}.
\qed
\end{proof} %% End of 2nd composition theorem