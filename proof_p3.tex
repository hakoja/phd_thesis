\subsection{Full proof of Theorem~\ref{thm:protocol_3:3P-AKE-}}\label{sec:full_proof:pi3}




\begin{proof}
We start by defining the partnering function $f_3$ based on the partnering functions for $\protocol_1$ and $\protocol_2$.
\item
\paragraph{Defining the partnering function for $\protocol_3$.}
Intuitively,
we define the partner function $f_3$ for protocol $\protocol_3$ by ``composing'' the two partner functions $f_1$ and $f_2$ assumed to exist for sub-protocols $\protocol_1$ and $\protocol_2$.
For example,
if $\oracle$ is an initiator session,
then $f_{3}(\oracle) = \oracle''$ iff there exists a trusted third-party session $\oracle'$,
such that $f_1(\oracle) = \oracle'$ and  $f_2(\oracle') = \oracle''$.
That is, $\oracle$ and $\oracle''$ are $f_3$-partners,
if there exists a trusted third-party session $\oracle'$ that acts as the ``link'' between them in the two sub-protocols $\protocol_1$ and $\protocol_2$.

To make the above intuition precise, 
one needs to be able to extract from the 3P-AKE$^w$ transcript two transcripts $T_1$ and $T_2$,
so that running $f_1$ and $f_2$ on them is well-defined.
The way to do this is specified in Table~\ref{tab:f3_parsing}.
Essentially,
the parsing rules ensure that to each initiator session in $\protocol_3$,
there exists a corresponding initiator session in $\protocol_1$.
To each responder session in $\protocol_3$ there exists a responder session in~$\protocol_2$.
And,
finally,
to each trusted-third party session in $\protocol_3$, there exist \emph{two} sessions:
one \emph{responder} session in $\protocol_1$;
and one \emph{initiator} session in $\protocol_2$.
However, the latter is only created ``on-demand'' after the trusted third-party session accepts in sub-protocol $\protocol_1$.
Based on $T_1$ and $T_2$ created by the parsing rules of Table~\ref{tab:f3_parsing},
we can formally define $f_3$ as follows.

When $\oracle$ is a session at party $P_A$ (initiator),
having intended  peers $P_B$ (responder) and $P_T$ (trusted third-party),
then:
\begin{itemize}
	\item $f_{3,T_3}(\oracle) = \oracle'$ if,
	\begin{enumerate}
		\item $f_{1,T_1}(\oracle) = \oracle''$ and $f_{2,T_2}(\oracle'') = \oracle'$\footnote{Here 
		we are abusing notation and writing $\oracle$, $\oracle'$, $\oracle''$ irregardless of whether they are sessions defined on transcripts $T_1$, $T_2$ or $T_3$.},
		\item $\oracle'.\peer = \lbrace P_A, P_T \rbrace$,
		\item $\oracle''.\peer = \lbrace P_A, P_B \rbrace$,
	\end{enumerate}

	\item $f_{3,T_3}(\oracle) =  \bot$, otherwise
	
\end{itemize}
\medskip

When $\oracle$ is a session at party $P_B$,
having intended peers $P_A$ and $P_T$,
then $f_3$ is defined similarly by ``reversing'' the order of $f_1$ and~$f_2$:
\begin{itemize}
	\item $f_{3,T_3}(\oracle) = \oracle'$ if,
	\begin{enumerate}
		\item $f_{2,T_2}(\oracle) = \oracle''$ and $f_{1,T_1}(\oracle'') = \oracle'$;
		\item $\oracle'.\peer = \lbrace P_B, P_T \rbrace$,
		\item $\oracle''.\peer = \lbrace P_A, P_B \rbrace$,
	\end{enumerate}

	\item $f_{3,T_3}(\oracle) =  \bot$, otherwise
\end{itemize}





 
\newcommand{\NULL}{\multicolumn{1}{|c|}{$-$}}

\begin{sidewaystable}
	\scriptsize
	\centering
	\caption{Parsing rules for extracting transcripts $T_1$ and $T_2$ from $T_3$.}\label{tab:f3_parsing}
	\todo[inline]{TODO: add more description on how to read this table.}
%	\makebox[\textwidth][c]{
	\begin{tabular}{ | l | l | l |}
		\hline 
		
		\multicolumn{1}{|c|}{$T_3$} & \multicolumn{1}{|c|}{$T_1$} &  \multicolumn{1}{|c|}{$T_2$}\\
		
		\hline 
	
		$(\Q{NewSession}(P_i, \initiator, \lbrace (P_j, \responder), (P_k, \ttp)\rbrace),\oracle[i][s], m)$  & $(\Q{NewSession}(P_i, \initiator, \lbrace (P_k, \responder)\rbrace),\oracle[i][s], m)$& \NULL \\
		
		$(\Q{NewSession}(P_j, \responder, \lbrace (P_i, \initiator), (P_k, \ttp)\rbrace),\oracle[j][t], \bot)$  &\NULL &  $(\Q{NewSession}(P_j, \responder, \lbrace (P_k, \initiator) \rbrace),\oracle[j][t], \bot )$ \\
		
		$(\Q{NewSession}(P_k, \ttp, \lbrace (P_i, \initiator), (P_j, \responder)\rbrace),\oracle[k][u], \bot)$  & $(\Q{NewSession}(P_k, \responder, \lbrace (P_i, \initiator) \rbrace),\oracle[k][u], \bot )$ & \NULL \\
		
%		\hline
		
		&& \\
		
%		\hline
		
		$(\Q{Send}(\oracle[i][s],m), m^*, (\running,\running,\running))$ &$(\Q{Send}(\oracle[i][s],m), m^*, (\running))$& \NULL  \\
		
		$(\Q{Send}(\oracle[i][s],m), m^*, (\accepted,\accepted,\accepted))$ &$(\Q{Send}(\oracle[i][s],m), m^*, (\accepted))$& \NULL \\
		
		$(\Q{Send}(\oracle[i][s],m), \bot, (\rejected,\rejected,\rejected))$ &$(\Q{Send}(\oracle[i][s],m), \bot, (\rejected))$& \NULL \\
		
%		\hline
		
		&&\\
		
%		\hline
		
		$(\Q{Send}(\oracle[j][t],m), m^*, (\accepted,\running,\running))$ & \NULL &$(\Q{Send}(\oracle[j][t],m), m^*, (\running))$  \\
		
		$(\Q{Send}(\oracle[j][t],m), m^*, (\accepted,\accepted,\running))$ & \NULL & $(\Q{Send}(\oracle[j][t],m), m^*, (\accepted))$ \\
		
		
		$(\Q{Send}(\oracle[j][t],m), \bot, (\accepted,\rejected,\rejected))$ & \NULL & $(\Q{Send}(\oracle[j][t],m), \bot, (\rejected))$ \\
		
		$(\Q{Send}(\oracle[j][t],\Ckey), \bot, (\accepted,\accepted,\accepted))$ &  \NULL & \NULL  \\
		
		$(\Q{Send}(\oracle[j][t],\Ckey'), \bot, (\accepted,\accepted,\rejected))$ & \NULL & \NULL  \\
		
%		\hline
		
		&& \\
		
%		\hline
		
		$(\Q{Send}(\oracle[k][u],m), m^*, (\running,\running,\running))$ &$(\Q{Send}(\oracle[k][u],m), m^*, (\running))$& \NULL  \\
		
		$(\Q{Send}(\oracle[k][u],m), \bot, (\rejected,\rejected,\rejected))$ & $(\Q{Send}(\oracle[k][u],m), \bot, (\rejected))$ & \NULL  \\
		
		$(\Q{Send}(\oracle[k][u],m), m^*, (\accepted,\running,\running))$\footnote{Only if $\oracle[k][u].\vv{\runstate} = (\running,\running,\running)$ for all prior $\Q{Send}$ queries to $\oracle[k][u]$.} &$(\Q{Send}(\oracle[k][u],m), \bot, (\accepted))$& $(\Q{NewSession}(P_k, \initiator, \lbrace (P_j, \responder) \rbrace),\oracle[k][u], m^* )$ \\
		
		$(\Q{Send}(\oracle[k][u],m), m^*, (\accepted,\running,\running))$ & \NULL & $(\Q{Send}(\oracle[k][u], m), m^*, (\running))$\\
		
		$(\Q{Send}(\oracle[k][u],m), \Ckey, (\accepted,\accepted,\accepted))$ & \NULL &$(\Q{Send}(\oracle[k][u], m), \bot, (\accepted))$\\
		

		
		$(\Q{Send}(\oracle[k][u],m), \bot, (\accepted,\rejected,\rejected))$ & \NULL & $(\Q{Send}(\oracle[k][u],m), \bot, (\rejected))$  \\
		
		\hline
	
	\end{tabular}
%	} % makebox
\end{sidewaystable} 



		


\paragraph{Soundness.}
\todo[inline]{TODO}


\paragraph{AKE$^w$-security.}
\setcounter{gamehop}{0}

%\item
\paragraph{Game~\game:}
This is the real 3P-AKE security game, hence
\begin{equation*}
	\adv_{\protocol_3,\A,f_3}^{\mathsf{G_\game}}(\secparam) = \adv_{\protocol_3,\A,f_3}^{\operatorname{\mathsf{3P-AKE}}^w}(\secparam) \ .
\end{equation*}

\newgame
\paragraph{Game~\game:}\label{proof:3P-KD:game_hop:EA_in_ACCE}
This game proceeds as the previous one, 
but aborts if a session accepts maliciously in sub-protocol $\protocol_2$.
Technically,
this means to check whether the session has an $f_2$-partner based on transcript $T_2$.

%\vspace{-\baselineskip}                        
%\item
%\vspace{-\baselineskip}
\begin{lemma}\label{lemma:ACCEX-EA}
$
	\adv_{\protocol_3,\A,f_3}^{\mathsf{G_{\prevgame{}}}}(\secparam) 
	\leq \adv_{\protocol_3,\A,f_3}^{\mathsf{G_\game}}(\secparam) 
	+ \adv_{\protocol_2,\A[B]_1, f_2}^{\operatorname{\mathsf{ACCEX-EA}}}(\secparam)  .
$
\end{lemma}

\begin{proof}

Reduction $\A[B]_1$ begins by creating all the long-term keys for sub-protocol $\protocol_1$ and selecting a random bit $b$.
Essentially,
$\A[B]_1$ will simulate the $\protocol_1$ part of $\protocol_3$ itself,
while forwarding all messages pertaining\footnote{$\A[B]_1$ uses the parsing rules of Table~\ref{tab:f3_parsing} to
know whether a message pertains to $\protocol_1$ or~$\protocol_2$.
} 
to $\protocol_2$ to its 2P-ACCEX challenger.
While conceptually simple,
we provide the full details of the simulation made by $\A[B]_1$ below.
In the remaining proofs we will only refer to how a reduction differs with respect to the simulation provided by $\A[B]_1$.


\small
$\A[B]_1$ forwards all the public-keys to $\A$ and answers its queries as follows.
\begin{itemize}
	\item $\Q{NewSession}(P_i, \role_a, \peer = \lbrace (P_j, \role_b), (P_k, \role_c) \rbrace)$\todo{HJ: fix this!}: 
	$\A[B]_1$ instantiates a session $\oracle$	with role $\role_a$ and peers $\peer$.	
	If $\role_a = \responder$ and,
	say $\role_c = \ttp$,
	then $\A[B]_1$ also makes a $\Q{NewSession}(P_i,\responder, (P_k, \initiator))$ query to its ACCEX challenger.
	We write $\oracle*$ for the corresponding proxy session created in $\A[B]_1$'s ACCEX experiment.
 



	\item $\Q{Corrupt}(P_i)$: 
	$\A[B]_1$ returns the private key it created for $P_j$.
	
	\item $\Q{Corrupt}(P_i, P_j)$: 	
 	$\A[B]_1$ forwards the $\Q{Corrupt}$ query to its ACCEX challenger to obtain the long-term symmetric key shared between $P_i$ and $P_j$.
		 	
	
	\item $\Q{Send}(\oracle, m)$: 
	If 	$m$ pertains to a message of sub-protocol $\protocol_1$,
	then $\A[B]_1$ answers it using the long-term keys it created itself. 
	
	\smallskip
	Moreover,
	if $\oracle$ is a trusted-party session and it accepted in $\protocol_1$ on receiving $m$,
	then $\A[B]_1$ also makes a $\Q{NewSession}(\oracle.\pid,\initiator, (P_j, \responder))$ query to its ACCEX challenger.
	Again, we write $\oracle*$ for the corresponding proxy session created in $\A[B]_1$'s ACCEX experiment.  
	

	\smallskip
	If $m$ is a handshake message of $\protocol_2$,
	then $\A[B]_1$ forwards the $\Q{Send}$ query to the proxy session $\oracle*$ in its own ACCEX security experiment.
	Moreover,
	if $\oracle$ is a trusted third-party session and $m$ caused the proxy session $\overline{\oracle}$ to accept in $\A[B]_1$'s ACCEX experiment,
	then $\A[B]_1$ creates $\oracle$'s $\Ckey$ message by making the query $\Q{Encrypt}(\oracle*, \ell, \key_{AB}, \key_{AB}, H)$.

	\smallskip
	Finally,
	if $m$ is a $\Ckey$ message and $\oracle$ is responder session,
	then $\A[B]_1$ makes a $\Q{Decrypt}(\oracle*, m, H)$ query and sets $\oracle$'s session key to be the output.


	\item $\Q{Reveal}(\oracle)$:
	If $\oracle.\runstate_F \neq \accepted$,
	return $\bot$.
	Otherwise, return $\oracle.\key$.

	\item $\Q{Test}(\oracle)$: 
	Return $\oracle.\key$ if $b=0$,
	or a random key if $b = 1$.
		
\end{itemize}
\normalsize

If $\A$ terminates,
then $\A[B]_1$ terminates too
(in this case no malicious accept has occurred).
To analyze $\A[B]_1$'s winning probability,
we only have to observe that $\A[B]_1$ provides a perfect simulation of $\protocol_3$ for $\A$.
This means that if a malicious accept occurs in sub-protocol $\protocol_2$,
then a malicious accept also occurs in $\A[B]_1$ ACCEX experiment.
\qed
\end{proof}

\newgame
\paragraph{Game~\game:}\label{proof:3P-KD:game_hop:int_in_ACCE}
This game proceeds as the previous one, 
but it aborts if a $\fresh[AKE]^w$ responder session accepts on receiving a $\Ckey$ message that was not legitimately produced by its partner in $\protocol_2$.
\begin{lemma}\label{lemma:3P-AKE-:game:ACCEX-int}
$
	\adv_{\protocol_3,\A,f_3}^{\mathsf{G_{\prevgame{}}}}(\secparam) 
	\leq \adv_{\protocol_3,\A,f_3}^{\mathsf{G_\game}}(\secparam) 
	+ \adv_{\protocol_2,\A[B]_2,f_2}^{\operatorname{\mathsf{ACCEX-int}}}(\secparam)  .
$
\end{lemma}

\begin{proof}
$\A[B]_2$ works exactly like algorithm $\A[B]_1$ in the previous proof,
but it also simulates the abort on malicious
accept. 
This simulation is possible because the partnering function $f_2$ is based on the \emph{public} transcript $T_2$.
It only remains to argue that the new abort event of Game~\game{} implies a forgery in $\A[B]_2$'s ACCEX experiment.
This amounts to showing that if a session is $\fresh[AKE]^w$ in $\protocol_3$, 
then the corresponding session in $\protocol_2$ is $\fresh[ACCEX]$.
But this is true because the $\fresh[AKE]^w$ predicate is more restrictive than the $\fresh[ACCEX]$ predicate.
\qed
\end{proof}

\newgame
\paragraph{Game~\game:}
In this game the challenger guesses two sessions, i.e, two pairs of indices $(i,s) \gets \mathcal{P} \times \mathbb{N}$ and $(j,t) \gets (\mathcal{P} \times \mathbb{N})$. 
By convention,
if it picks the same pair twice, then this session is not expected  to get a partner.
We call such a session ``single''. 
The challenger aborts if one of the following bad event occurs:
%Unless $\oracle[i][s]$ is selected as the test-session by $\A$,
%and $\oracle[j][t]/\bot$ is its partner,
%the challenger aborts the game. That is, the challenger aborts as soon as the following bad event occurs:
\begin{quote}
(i) The guessed session was a single session and gets a partner in $\protocol_3$, or (ii) one of the two guessed session becomes partnered in $\protocol_3$ with another session, or (iii) adversary $\A$ makes a $\Q{corrupt}$ query that would make any of the guessed sessions unfresh in $\protocol_3$, or (iv) the adversary issues a $\Q{Test}$ query to a session that is not $(i,s)$, or (v) adversary terminates and the two guessed sessions are not partnered in $\protocol_3$.\footnote{We assume w.l.o.g. that the adversary always makes a $\Q{Test}$ query before terminating.}
\end{quote}

\begin{lemma}
\begin{equation}
	\adv_{\protocol_3,\A,f_3}^{\mathsf{G_{\prevgame{}}}}(\secparam) 
	\leq n^2 \cdot \adv_{\protocol_3,\A,f_3}^{\mathsf{G_\game}}(\secparam),
\end{equation}
where $n = \numparties\numsessions$ is an upper bound on the total number of sessions.
\end{lemma}

\begin{proof}
If the bad event does not occur, then $\A$'s success probability is the same in $G_{\prevgame}$ and $G_{\game}$, and the challenger guesses the right (pair of) session(s) at least with probability $\frac{1}{n^2}$.
\qed
\end{proof}

In the remaining games,
let $\oracle^* = \oracle[i][s]$ denote the guessed test-session and $\oracle' = \oracle[j][t]/\bot$ its partner.
Define $\oracle[T][u]$ to be the \emph{co-partner} of $\oracle^* = \oracle[i][s]$
if $(P_T, \ttp) \in \oracle^*.\peer$,
and either $\oracle^*.\role = \initiator \wedge f_{1,T_1}(\oracle^*) = \oracle[T] $
or $\oracle^*.\role = \responder \wedge f_{2,T_2}(\oracle^*) = \oracle[T]$. 
That is, $\oracle[T][u]$ is the trusted third-party session in the protocol run involving~$\oracle^*$ (and $\oracle'$). 


\newgame
\paragraph{Game~\game:}\label{proof:3P-KD:game_hop:C_key_encrypt_0}
In this game, when creating the $\Ckey$ message for the co-partner of $\oracle^*$,
the challenger encrypts the ``dummy'' string $0^\secparam$
instead of the actual session key $\key_{AB}$.
However,
if the ``dummy'' $\Ckey$ message is eventually delivered 
(which by Game~\ref{proof:3P-KD:game_hop:int_in_ACCE} must be to the correct recipient),
it is nevertheless ``decrypted'' to the actual session key $\key_{AB}$.



\begin{lemma}\label{lemma:3P-KD:abort_on_forgery}
$
	\adv_{\protocol_3,\A,f_3}^{\mathsf{G_{\prevgame{}}}}(\secparam) 
	\leq \adv_{\protocol_3,\A,f_3}^{\mathsf{G_\game}}(\secparam) 
	+ \adv_{\protocol_2, \A[B]_3,f_2}^{\operatorname{\mathsf{ACCEX-priv}}}(\secparam)  .
$
\end{lemma}

We omit the proof as it is similar to Lemma~\ref{lemma:ACCEX-EA}, 
and only briefly discuss the technically interesting considerations. 
The reduction can implement all abort conditions introduced so far based on
public information and the queries available to it in the ACCEX experiment. 
In particular $\A[B]_3$ creates the $\Ckey$ message for the test-session by querying the pair $(\key_{AB}, 0^\secparam)$ to the encryption oracle for the trusted third-party session that is relevant to the test-session. 
Finally, as in Lemma~\ref{lemma:3P-AKE-:game:ACCEX-int}, the freshness condition in 
\AKEm{} implies freshness in ACCEX.

\iffalse
\begin{proof}
If $\A$ distinguishes between Game~\prevgame{} and Game~\game{},
then we can construct an adversary $\A[B]_3$ against the ACCEX security of sub-protocol $\protocol_2$ as follows.
Adversary $\A[B]_3$ begins by choosing a random bit $b$ and guessing the test-session $\oracle^*$ and its partner $\oracle'$.
It then creates all the long-term keys for sub-protocol $\protocol_1$.
To simulate protocol $\protocol_3$ for $\A$,
$\A[B]_3$ will forward all messages that pertains to sub-protocol $\protocol_2$ to its own ACCEX challenger,
while all messages that pertains to sub-protocol $\protocol_1$,
$\A[B]_3$ can answer itself using the long-term keys it created. 
Technically, $\A[B]_3$ will implement the same parsing rules as the partner function $f_3$
(described in Table~\ref{tab:f3_parsing});
maintaining two transcripts $T_1$ and $T_2$ which corresponds to sub-protocols $\protocol_1$ and $\protocol_2$,
respectively. 
In particular, 
to each session having role $\responder$ in protocol $\protocol_3$,
$\A[B]_3$ will create a corresponding ``proxy session'' in its own ACCEX experiment with role $\responder$,
while to each session having role $\ttp$,
it will create a proxy session with role $\initiator$.

After forwarding the public keys of sub-protocol $\protocol_1$ to $\A$,
algorithm $\A[B]_3$ begins to run $\A$,
answering its queries as described in the parsing table. Moreover, $\A[B]_3$ aborts whenever
the bad event introduced in $\game_3$ occurs. Note that these conditions are
publicly checkeable so that $\A[B]_3$ can indeed simulate them. Morever, $\A[B]_3$ 
also aborts if a session accepts maliciously in sub-protocol $\protocol_2$
(in accordance with Game~\ref{proof:3P-KD:game_hop:EA_in_ACCE}) or if
$\A$ delivers an improper $\Ckey$ message 
(according to Game~\ref{proof:3P-KD:game_hop:int_in_ACCE}).
%or $\A$ makes a query that breaks the $\fresh[\AKEm]$ predicate for $\oracle^*$.  
\begin{itemize}
	\item $\Q{NewSession}(P_i, \role_a, \peer = \lbrace (P_j, \role_b), (P_k, \role_c) \rbrace)$: 
	$\A[B]_3$ instantiates a session $\oracle$	with role $\role_a$ and peers $\peer$.	
	If $\role_a = \responder$ and,
	say $\role_c = \ttp$,
	then $\A[B]_3$ also makes a $\Q{NewSession}(P_i,\responder, (P_k, \initiator))$ query to its ACCEX challenger.
	We write $\oracle*$ for the corresponding proxy session created in $\A[B]_3$'s ACCEX experiment.
 



	\item $\Q{Corrupt}(P_i)$: 
	$\A[B]_3$ returns the private key it created for $P_j$.
	
	\item $\Q{Corrupt}(P_i, P_j)$: 	
 	$\A[B]_3$ forwards the $\Q{Corrupt}$ query to its ACCEX challenger to obtain the long-term symmetric key shared between $P_i$ and $P_j$.
		 	
	
	\item $\Q{Send}(\oracle, m)$: 
	If 	$m$ pertains to a message of sub-protocol $\protocol_1$,
	then $\A[B]_3$ answers it using the long-term keys it created itself. 
	Moreover,
	if $\oracle.\role = \ttp$;
	$\oracle.\vv{\runstate} = \lbrace (P_i, \initiator), (P_j, \responder) \rbrace$;
	and $\oracle$ accepts in sub-protocol $\protocol_1$ on receiving $m$;
	then $\A[B]_3$ also makes a $\Q{NewSession}(\oracle.\pid,\initiator, (P_j, \responder))$ query to its ACCEX challenger,
	returning the responding initial message back to $\A$. 
	Again, we write $\oracle*$ for the corresponding proxy session created in $\A[B]_3$'s ACCEX experiment.  
	

	
	If $m$ is a handshake message of sub-protocol $\protocol_2$,
	then $\A[B]_3$ can answer it by asking a corresponding $\Q{Send}$ query to the proxy session $\oracle*$ in its own ACCEX security experiment.

	Additionally,
	if $\oracle[][].\rho = \ttp$,
	and $m$ caused the proxy session $\overline{\oracle}$ to accept in the ACCEX experiment,
	then $\A[B]_3$ creates $\oracle$'s $\Ckey$ as follows:
	\begin{itemize}
		\item if $\oracle$ is the co-partner of $\oracle^*$,
		then $\A[B]_3$ queries $\Q{Encrypt}(\oracle*, \ell, \key_{AB}, 0^\secparam, H)$
		to its ACCEX challenger,
		where $\key_{AB}$ is the actual session key,
		
		\item otherwise, $\A[B]_3$ queries $\Q{Encrypt}(\oracle*, \ell, \key_{AB}, \key_{AB}, H)$. 
		
	\end{itemize}
	
%	Otherwise, $m$ is rejected.
	
	Finally, if $\oracle.\vv{\runstate} = (\accepted, \accepted, \running)$ and $m = \Ckey$,
	then set $\oracle.\vv{\runstate} = (\accepted, \accepted, \accepted)$ and set $\oracle.\key = \key_{AB}$.

	\item $\Q{Reveal}(\oracle)$:
	If $\oracle.\runstate_F \neq \accepted$,
	return $\bot$.
	Otherwise, return $\oracle.\key$.

	\item $\Q{Test}(\oracle^*)$: 
	Return $\oracle^*.\key$ if $b=0$,
	or a random key if $b = 1$.
		
\end{itemize}
Let $b'$ be the output of $\A$ in Game~\game{}.
Note that if $\oracle{}^*.\role = \initiator$,
then it is possible that $\A$ outputs $b'$ without triggering the creation of a ``dummy'' $\Ckey$ message
(e.g. if $\A$ simply does not run sub-protocol $\protocol_2$).
However,
in this case Game~\prevgame{} and Game~\game{} are identical,
so $\A$'s advantages in both games are identical, too.
Now if a $\Ckey$ message is indeed created by $\oracle^*$'s co-partner $\oracle[T][s]$,
let $\oracle*[T][s]$ be the corresponding proxy session in $\A[B]_3$'s ACCEX experiment.
If $b' = b$, 
then $\A[B]_3$ outputs $(\oracle*[T][s], 1)$ to its ACCEX experiment,
and $(\oracle*[T][s], 0)$ otherwise.

Since $\oracle^*$ is $\fresh[\AKEm]$,
then $\oracle*[T][s]$ is $\fresh[ACCE]$ since the requirements of freshness are stronger in the former than in the latter.
Moreover,
when $\oracle*[T][s].b = 1$, 
then $\A[B]_3$ provides a perfect simulation of Game~\game{},
while if $\oracle[T][s].b = 0$,
then $\A[B]_3$ provides a perfect simulation of Game~$\prevgame$,
hence
\begin{align}
	| \adv_{\protocol_3}^{\mathsf{G_{\game{}}}}(\A)
		&- \adv_{\protocol_3}^{\mathsf{G_{\prevgame}}}(\A)  | 
	 \leq | \Pr[G_{\game}^{\A} \Rightarrow 1 \mid \Ckey \text{ created}] \\
		&\qquad\qquad\qquad\qquad - \Pr[G_{\prevgame}^{\A} \Rightarrow 1 \mid \Ckey \text{ created}] | \\ 
%	 &= \left| \Pr[b' = b \mid \oracle[T][s].b = 1] - \Pr[b' = b \mid \oracle[T][s].b = 0] \right| \\
	 &= \left| \Pr[\A[B]_3 \Rightarrow (\oracle[T][s],1) \mid \oracle[T][s].b = 1] 
		 - \Pr[\A[B]_3 \Rightarrow (\oracle[T][s],1) \mid \oracle[T][s].b = 0] \right| \\
	 &= \adv_{\protocol_2}^{\operatorname{\mathsf{ACCEX-priv}}}(\A[B]_3) \ .
\end{align}\qed
\end{proof} 
\fi

\newgame
\paragraph{Game~\game:}
This game proceeds as the previous one,
except that when answering the $\Q{Test}$ query in the case $b = 0$,
the challenger responds with $\key_{AB} \gets \protocol_3.\mathsf{PRF}(\widetilde{\key_{AT}},P_A, P_B)$,
where $\widetilde{\key_{AT}} \gets \bits^\secparam$ is an independently and uniformly distributed string.
%and $\lbrace P_A, P_B \rbrace = \lbrace \oracle^*.\pid, \oracle'.\pid \rbrace$.
\begin{lemma}\label{lemma:3P-KD:2P-AKE-swap-random}
$
	\adv_{\protocol_3,\A,f_3}^{\mathsf{G_{\prevgame{}}}}(\secparam) 
	\leq \adv_{\protocol_3,\A,f_3}^{\mathsf{G_\game}}(\secparam) 
	+ \adv_{\protocol_1,\A[B]_4, f_1}^{\operatorname{\mathsf{2P-AKE}}}(\secparam) \ .
$
\end{lemma}

We omit the proof as it is an analogous reduction as in Lemma~\ref{lemma:ACCEX-EA}, except that this time the reduction is to sub-protocol $\protocol_1$ instead of $\protocol_2$.
We briefly discuss the technically interesting considerations for the reduction. Namely, the \AKEm freshness condition for $\protocol_3$ puts more restrictions on the adversary than the freshness condition of $\protocol_3$, thus, the reduction can forward all queries. Moreover, the reduction can implement all the abort conditions from previous games via public simulation.

\iffalse
\begin{proof}\todo{HJ: can this be shortened?}
If $\A$ distinguishes between Game~\prevgame{} and Game~\game{},
then we can construct an adversary $\A[B]_4$ against the 2P-AKE security of sub-protocol $\protocol_1$ analogously to the way we constructed adversary $\A[B]_3$ against the 2P-ACCE security of sub-protocol $\protocol_2$ in the previous game-hop. 
%
Adversary $\A[B]_4$ begins by choosing a random bit $b$ and guessing the test-session $\oracle^*$ and its partner $\oracle'$.
It creates all long-term keys for sub-protocol $\protocol_2$ and  receives all the public keys of sub-protocol $\protocol_1$ from its 2P-AKE challenger. 
To simulate protocol $\protocol_3$ for $\A$,
$\A[B]_4$ will forward all messages that pertain to sub-protocol $\protocol_1$ to its own 2P-AKE challenger,
while all messages that pertain to sub-protocol $\A[B]_4$ can answer itself using the pre-shared keys it created. 
Technically, $\A[B]_4$ will implement the same parsing rules as the partner function $f_3$
(described in Table~\ref{tab:f3_parsing}),
maintaining two running transcripts $T_1$ and $T_2$ corresponding to sub-protocols $\protocol_1$ and $\protocol_2$,
respectively. 
In particular, 
to each session having role $\initiator$ in protocol $\protocol_3$,
$\A[B]_4$ will create a corresponding ``proxy session'' in its own 2P-AKE experiment with role $\initiator$;
while to each session having role $\ttp$,
it will create a proxy session with role $\responder$. After forwarding all the public keys to $\A$, $\A[B]_4$ answers its queries as described in the table.
Again, adversary $\A[B]_4$ implements all the abort conditions that were introduced in previous games as well.

\begin{itemize}
	\item $\Q{NewSession}(P_i, \role_a, \peer = \lbrace (P_j, \role_b), (P_k, \role_c) \rbrace)$: 
	$\A[B]_4$ instantiates a session $\oracle$	with role $\role_a$ and peers $\peer$.
	
	Additionally, 
	if $\role_a = \initiator$ and,
	say $\role_c = \ttp$,
	then $\A[B]_4$ also makes a $\Q{NewSession}(P_i,\initiator, (P_k, \responder))$ query to its 2P-AKE challenger.
	Similarly,
	if $\role_a = \ttp$ and $\role_c = \initiator$,
	then $\A[B]_4$ makes the query $\Q{NewSession}(P_i,\responder, (P_k, \initiator))$.
	We write $\oracle*$ for the corresponding proxy session created in $\A[B]_4$'s 2P-AKE experiment.
	
	\item $\Q{Send}(\oracle, m)$: 
	If $m$ pertains to a message of sub-protocol  $\protocol_1$,
	then $\A[B]_4$ answers it by making the same $\Q{Send}$ query to the corresponding proxy  session $\oracle*$ in its own 2P-AKE experiment.

	If $m$ pertains to a message of sub-protocol $\protocol_2$,
	then $\A[B]_4$ answers it itself by using the long-term keys it created. 
	In particular,
	if $\oracle$ is the co-partner of $\oracle$ (see the definition in Game~\ref{proof:3P-KD:game_hop:int_in_ACCE}),
	then its $\Ckey$ message is created by encrypting the string $0^\secparam$.

	
	\item $\Q{Corrupt}(P_i)$: 
	$\A[B]_4$ makes a corresponding $\Q{Corrupt}$ query to its 2P-AKE-RoR challenger to obtain $P_i$'s long-term private key which it returns to $\A$.
	
	\item $\Q{Corrupt}(P_i, P_j)$: 
	$\A[B]_4$ returns the long-term symmetric key $\Key_{ij}$ it created for $P_i$ and $P_j$ in sub-protocol $\protocol_2$.
		 
	\item $\Q{Test}(\oracle^*)$:
	If $b = 0$,
	then $\A[B]_4$ queries $\Q{Test}(\oracle*{}^*)$ to the proxy session in its own 2P-AKE experiment to get a challenge key $\key_{AT}^*$,
	and then it returns $\key_{AB} \gets \protocol_3.\mathsf{PRF}(\key_{AT}^*, P_A, P_B)$ back to $\A$.
	If $b = 1$,
	then $\A[B]_4$ returns a random key. 
%	If $\oracle$ has accepted,
%	then $\A[B]_4$ returns $\oracle.\key$ if $b=0$,
%	or a random key if $b = 1$
%	(returning the same random key if a $\Q{Test}$-query is later made to $\oracle$'s partner).\todo{FIX: deal with corruptions and so on.}
	
	
\end{itemize}
Let $b'$ be the output of $\A$.
If $b' = b$,
then $\A[B]_4$ outputs $1$ to its 2P-AKE security experiment.
Otherwise,
it outputs $0$.
When the $\Q{Test}$ query in $\A[B]_4$'s 2P-AKE security experiment returns a real key,
then $\A[B]_4$ perfectly simulates Game~\prevgame{},
and when the $\Q{Test}$ query returns a random key,
then $\A[B]_4$ perfectly simulates Game~\game{}.
%Thus, $\A[B]_4$ outputs 1 when the $\Q{Test}$ queries in its own 2P-AKE-RoR security experiment returns real keys 
%is exactly the probability that $\A$ correctly guesses the hidden bit $b$ in Game~\prevgame{}.
%Similarly,
%the probability that $\A[B]_4$ outputs 1 when the $\Q{Test}$ queries returns random keys
%is exactly the probability that $\A$ guesses the hidden bit $b$ in Game~\game{}.
\qed
\end{proof}
\fi

\newgame
\paragraph{Game~\game:}\label{proof:3P-KD:game_hop:PRF->RF}
This game proceeds as the previous one,
except that for the $\protocol_1$ session key that belongs to the 
guessed  test session $\pi^*$, for both partners in $\protocol_1$, the
key $\key_{AB}$ is derived using a random function $\$(\cdot, \cdot)$
rather than $\mathsf{PRF}(\key_{AT}, \cdot, \cdot)$.
\begin{lemma}
$
	\adv_{\protocol_3,\A,f_3}^{\mathsf{G_{\prevgame{}}}}(\secparam) 
	\leq \adv_{\protocol_3,\A,f_3}^{\mathsf{G_\game}}(\secparam) 
	+  \adv_{\protocol_3.\mathsf{PRF}}^{\operatorname{\mathsf{PRF}}}(\A[D])  .
$
\end{lemma}

\begin{proof}
Algorithm $\A[D]$ has access to an oracle $\mathcal{O}$,
which either implements the function $\protocol_3.\mathsf{PRF}(\widetilde{\key}, \cdot, \cdot)$,
for an independently and uniformly distributed key $\widetilde{\key}$;
or a random function $\$(\cdot, \cdot)$.
$\A[D]$ begins by choosing a random bit $b$ and creating all the long-term keys for sub-protocols $\protocol_1$ and $\protocol_2$.
It also guesses a test-session $\oracle^*$ and its partner $\oracle'$.
Next,
it runs $\A$ and simulates all the session oracles according to Game~\prevgame{} by using the keys it created. 
However,
when $b=0$,
then $\A[D]$ answers the $\Q{Test}$ query by returning $\mathcal{O}(P_A,P_B)$,
where $\lbrace P_A, P_B \rbrace = \lbrace \oracle^*.\pid, \oracle'.\pid \rbrace$.
Let $b'$ be the output of $\A$.
If $b' = b$,
then $\A[D]$ outputs $1$.
Otherwise,
it outputs~$0$.

When $\A[D]$'s oracle $\mathcal{O}$ implements $\protocol_3.\mathsf{PRF}$,
then $\A[D]$ perfectly simulates game $\mathsf{G}_{\game}$, whereas if $\mathcal{O}$ 
implements a random function $\$(\cdot,\cdot)$, then $\A[D]$ perfectly simulates game $\mathsf{G}_{\prevgame}$
concluding the proof of the lemma.
%Hence, the probability that $\A[D]$ outputs $1$ in this case,
%is exactly the probability that $\A$ guesses the hidden bit in game $\mathsf{G}_{\prevgame.j}$.
%Similarly, 
%when the PRF oracle implements $\protocol_3.\mathsf{PRF}$,
%then $\A[D]$ perfectly simulates game $\mathsf{G}_{\prevgame.{j-1}}$,
%and so the probability that $\A[D]$ outputs 1 is exactly the probability that $\A$ guesses the hidden bit in game $\mathsf{G}_{\prevgame.j-1}$.
%Thus:
%\begin{align}
%	| \adv_{\protocol_3}^{\mathsf{G_{\game{}}}}(\A)  - \adv_{\protocol_3,\A,f_3}^{\mathsf{G_{\prevgame{}}}}(\secparam)  | 
%	 &= \left| \Pr[b' = b \mid \A[D]^{\$(\cdot)}] - \Pr[b' = b \mid \A[D]^{\mathsf{PRF}(\cdot)}] \right| \\
%	 &= \left| \Pr[ \A[D]^{\$(\cdot)} \Rightarrow 1] - \Pr[\A[D]^{\mathsf{PRF}(\cdot)}
%	  \Rightarrow 1] \right| \\
%	 &= \adv_{\protocol_3.\mathsf{PRF}}^{\operatorname{\mathsf{PRF}}}(\A[D]) \ .
%\end{align}
\qed
\end{proof}


\paragraph{Concluding the proof of Theorem~\ref{thm:protocol_3:3P-AKE-}.}
We argue that $\adv_{\protocol_3,\A,f_3}^{\mathsf{G_\game}}(\secparam)=0$, which is where, crucially, we will rely on the channel binding.
By the change in Game~\ref{proof:3P-KD:game_hop:C_key_encrypt_0}, 
no information on the session key of the test-session is leaked through the $\Ckey$ message,
and by the change in Game~\ref{proof:3P-KD:game_hop:PRF->RF},
the $\Q{Test}$ query is always answered by a random function, regardless of the secret bit.
The same key $k_{AT}$ is used at most at $\pi^*$ and its $f_1$-partner session. If the key
derivation function used the same identities in both sessions, then no information about the derived key
is leaked due to the change in Game~\ref{proof:3P-KD:game_hop:PRF->RF}. If it used incorrect
identities, then the two derived keys are independent from each other. Hence, $\adv_{\protocol_3,\A,f_3}^{\mathsf{G_\game}}(\secparam)=0$,
as the adversary receives a uniformly random key, regardless of the secret bit $b$.
Combined with the previous lemmas, this concludes the proof of Theorem~\ref{thm:protocol_3:3P-AKE-}.
\qed
\end{proof}