\section{Application to EAP}\label{sec:EAP}

Our two composition theorems (\cref{thm:protocol_3:3P-AKE-} and \cref{thm:protocol_5:3P-AKE})
apply to the basic and full variants of EAP respectively,
as defined in \cref{sec:generic_composition_results:modeling_EAP}.
Specifically,
in \cref{thm:protocol_3:3P-AKE-} we identify sub-protocol $\protocol_1$ with the EAP method run between the client and the server,
and sub-protocol $\protocol_2$ with the key-transport protocol between the server and the authenticator. 
By a suitable instantiation of these building blocks,
and assuming that the EAP method provides channel binding,
we immediately get from \cref{thm:protocol_3:3P-AKE-} that basic EAP ($\protocol_3$) is a secure 3P-AKE protocol in the weak forward secrecy model 3P-\akewfstext.
This result can then be combined with \cref{thm:protocol_5:3P-AKE},
where we identify sub-protocol $\protocol_3$ with basic EAP and sub-protocol $\protocol_4$ with the subsequent Security Association Protocol between the client and the authenticator.
It follows immediately that full EAP ($\protocol_5$) is a secure 3P-AKE protocol in the full forward secrecy model \akefstext.

In \cref{sec:EAP-TLS-security} we will show that the EAP method EAP-TLS~\cite{IETF:RFC5216:EAP-TLS} satisfies the requirements on sub-protocol $\protocol_1$ in \cref{thm:protocol_3:3P-AKE-}.
Likewise,
in \cref{sec:802.11} we show that IEEE~802.11~\cite{IEEE:2012:802.11} satisfies the requirements on sub-protocol~$\protocol_4$ in \cref{thm:protocol_5:3P-AKE}.
Thus,
it remains to demonstrate that the key-transport protocol between the server and the authenticator satisfies the requirements on sub-protocol $\protocol_2$ in \cref{thm:protocol_3:3P-AKE-}.

As we argued in \cref{sec:generic_composition_results:modeling_EAP:AAA},
the security properties provided by the MPPE encryption method employed by RADIUS are largely unknown.
It is therefore difficult to assess whether RADIUS alone can safely instantiate our first composition theorem.
On the other hand,
RADIUS is commonly run on top of a secure channel protocol like TLS or IPsec.
%This is also the case for RADIUS' successor protocol Diameter~\cite{IETF:RFC6733:DIAMETER},
%which is another popular key-transport protocol.
In this case the security reduces to that of the underlying secure channel protocol.
Both TLS and IPsec are well-studied,
and have received large amounts of formal analysis.
In particular,
a number of works have shown TLS to be a secure ACCE protocol
\cite{C:JKSS12,C:KraPatWee13,EPRINT:KohSchSch13,BrzuskaFSWW:2012:less_is_more,PKC:LSYKS14},
so in \cref{thm:protocol_3:3P-AKE-} we can for instance set sub-protocol $\protocol_2$ to be RADIUS-over-TLS~\cite{IETF:RFC6614:RADIUS_over_TLS}.


 



\subsection{EAP without channel binding}\label{sec:EAP:without_channel_binding}

\cref{thm:protocol_3:3P-AKE-} requires that the EAP method provides channel binding.
Without it,
EAP  becomes vulnerable to the UKS attack described in \cref{sec:generic_composition_results:modeling_EAP:EAP_method}.
Unfortunately,
many concrete EAP methods do not provide channel binding.
Because the communication between the client and the server is normally routed through the authenticator,
a malicious authenticator can trivially modify the information presented to the two sides. 
As a consequence,
without channel binding it suffices to compromise a \emph{single} authenticator in order to compromise an entire network. 
Since authenticators are typically low-protected devices such as wireless access points,
this represents  a substantial attack vector on larger enterprise networks.


Interestingly,
a very similar situation can be found in the  UMTS and LTE mobile networks.
UMTS and LTE employ a key exchange protocol called AKA which is structured almost identically to the EAP protocol:
a mobile client that wants to connect to a base station first has to authenticate to its home operator.
The home operator then transmits a list of so-called \emph{authentication vectors}
(which in particular includes a session key) to the base station in much the same way the server forwards the session key to the authenticator in EAP.  
Moreover,
similar to many EAP methods,
the AKA protocol lacks channel binding for its authentication vectors. 
In their analysis of the AKA protocol,
Alt et al.~\cite[§5]{ACNS:AFMOR16} noted this lack of channel binding,
and suggested a fix which is almost identical to the key-derivation approach analyzed in this \lcnamecref{sec:generic_composition_results}. 


\subsection{Channel binding scope}\label{sec:EAP:channel-binding_scope}
\cref{thm:protocol_3:3P-AKE-} assumes that the channel binding includes the identity of the client and the authenticator in order to bind them cryptographically to the session key.  
However,
this fine-grained scope of the channel binding might not be relevant in all situations.
For example,
in a WLAN supported by many access points,
the client does not actually care about \emph{which} specific access point it connects to,
as long as it connects to a legitimate access point of that WLAN.
Thus, in this case the granularity of the channel binding does not have to be at the individual access point level but rather at the WLAN level,
defined by all the access points broadcasting the same network identifier (SSID).  
Of course,
by doing so the protection  provided by the channel binding will be weaker. 
In particular,
when channel binding occurs at the individual level,
then the corruption of a single access point will not influence clients connecting to access points having a \emph{different} identity.
On the other hand, 
when channel binding occurs at the network level,
then a single corrupted access point will affect \emph{all} connections within that WLAN.
In this case,
channel binding only protects connections occurring in networks having a different SSID.


More generally,
the information included in the channel binding defines the scope of the protection it provides,
and can include more than just identities.
For instance,
physical media types,
data rates, cost-information,
channel frequencies,
can all be used as input to the channel binding
(see \cite{ClancyH:2009:making_the_case_EAP_cb} for a discussion of these possibilities). 
The specifications for channel binding within EAP~\cite{IETF:draft:EAP-channel-binding,IETF:RFC6677:EAP-channel-binding}
leave open exactly what type of information should go into the binding,
because the amount of information that will be available to both the client and the server may vary.





