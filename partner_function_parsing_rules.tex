\chapter{Transcript parsing rules}\label{sec:appendix:full_proofs}
Let $T_3$ be a transcript produced by running experiment $\Exp_{\protocol_3,\queryset}(\A)$,
where  $\protocol_3$ is the protocol described in \cref{sec:generic_composition_results:first_composition_result}.
Table~\ref{tab:f3_parsing} defines how to extract from $T_3$
two other transcripts,
$T_1$ and $T_2$,
corresponding to runs of $\Exp_{\protocol_1,\queryset}(\A')$, and $\Exp_{\protocol_2,\queryset}(\A'')$,
respectively.
Essentially, $T_1$ and $T_2$ are extracted from $T_3$ as follows.
\begin{itemize}
	\item For each initiator session on~$T_3$,
	create a corresponding initiator session on~$T_1$.
	
	\item For each responder session on $T_3$,
	create a corresponding responder session on~$T_2$.

	\item For each server session on $T_3$, 
	create \emph{two} sessions:
	one \emph{responder} session on $T_1$;
	and one \emph{initiator} session on  $T_2$.
	However, the latter is only created if the server session reached the accept state in sub-protocol $\protocol_1$ on $T_3$.
	
	
	\item For each $\Send$ message on $T_3$ directed to an initiator session,
	copy the $\Send$ message to the corresponding session on~$T_1$.
		
	\item For each $\Send$ message on $T_3$ directed to a responder session,
	copy the $\Send$ message to the corresponding session on~$T_2$.
	
	\item For each $\Send$ message on $T_3$ directed to a server session which has \emph{not} accepted in sub-protocol $\protocol_1$,
	copy the $\Send$ message to the corresponding session on~$T_1$.
%	If the $\Send$ message on $T_3$ resulted in the server session accepting in sub-protocol $\protocol_1$,
%	create 

	
	\item For each $\Send$ message on $T_3$ directed to a server session which \emph{has} accepted in sub-protocol $\protocol_1$,
	copy the $\Send$ message to the corresponding session on~$T_2$.
	
\end{itemize}


\newcommand{\NULL}{\multicolumn{1}{c}{$-$}}
\colorlet{EmptyRowColor}{lightgray!25}
\colorlet{ProtocolOneColor}{white}
\colorlet{ProtocolTwoColor}{white}


\renewcommand{\arraystretch}{1.2}





%\begin{adjustwidth}{-4cm}{-8cm}

\newgeometry{left=1.8cm,right=1.2cm,top=4cm,bottom=4.3cm}

\begin{landscape}

\thispagestyle{empty}
%\centering

%\vspace*{10em}

\begin{table}

	\captionsetup{width=1.1\textwidth}

	\centering
	\scriptsize
	\caption{Parsing rules for extracting transcripts $T_1$ and $T_2$ from a transcript $T_3$
	generated by an execution of experiment $\Exp_{\protocol_3,\queryset}(\A)$,
	where $\protocol_3$ is the protocol defined in \cref{sec:generic_composition_results:first_composition_result}.
	The table assumes that $A \in \Inits$, $B \in \Responders$ and $S \in \Servers$ in protocol $\protocol_3$.
	Parsing is done as follows.
	For each entry in $T_3$, look up the row in the column marked ``$T_3$'' that matches this query.
	From this row, use the corresponding queries in the columns marked ``$T_1$'' and ``$T_2$''
	to create the entries on $T_1$ and $T_2$ respectively
	(``$-$'' means that no query should be created).}\label{tab:f3_parsing}
	
	
\begin{threeparttable}
	
	\begin{tabular}{  l  l  l }
		\toprule
		
		\multicolumn{1}{c}{\large$T_3$} & \multicolumn{1}{c}{\large$T_1$} &  \multicolumn{1}{c}{\large$T_2$}\\
		
		\midrule
		
		\rowcolor{ProtocolOneColor}
		$(\Q{NewSession}(A,B, S ),\oracle[A][i], m)$  & $(\Q{NewSession}(A,  S ),\oracle[A][i], m)$& \NULL \\
		
		\rowcolor{ProtocolTwoColor}
		$(\Q{NewSession}(B, A, S ),\oracle[B][j], \bot)$  &\NULL &  $(\Q{NewSession}(B,  S ),\oracle[B][j], \bot )$ \\
		
		\rowcolor{ProtocolOneColor}
		$(\Q{NewSession}(S,A, B),\oracle[S][k], \bot)$  & $(\Q{NewSession}(S,  A  ),\oracle[S][k], \bot )$ & \NULL \\
		
%		\hline
		
		\rowcolor{EmptyRowColor}
		
		&& \\
		
%		\hline
		\rowcolor{ProtocolOneColor}
		$(\Q{Send}(\oracle[A][i],m), m^*, (\running,\running,\running))$ &$(\Q{Send}(\oracle[A][i],m), m^*, (\running))$& \NULL  \\
		\rowcolor{ProtocolOneColor}
		$(\Q{Send}(\oracle[A][i],m), m^*, (\accepted,\accepted,\accepted))$ &$(\Q{Send}(\oracle[A][i],m), m^*, (\accepted))$& \NULL \\
		\rowcolor{ProtocolOneColor}
		$(\Q{Send}(\oracle[A][i],m), \bot, (\rejected,\rejected,\rejected))$ &$(\Q{Send}(\oracle[A][i],m), \bot, (\rejected))$& \NULL \\
		
%		\hline
		
		\rowcolor{EmptyRowColor}
		&&\\
		
%		\hline
		\rowcolor{ProtocolTwoColor}
		$(\Q{Send}(\oracle[B][j],m), m^*, (\accepted,\running,\running))$ & \NULL &$(\Q{Send}(\oracle[B][j],m), m^*, (\running))$  \\
		\rowcolor{ProtocolTwoColor}
		$(\Q{Send}(\oracle[B][j],m), m^*, (\accepted,\accepted,\running))$ & \NULL & $(\Q{Send}(\oracle[B][j],m), m^*, (\accepted))$ \\
		\rowcolor{ProtocolTwoColor}		
		$(\Q{Send}(\oracle[B][j],m), \bot, (\accepted,\rejected,\rejected))$ & \NULL & $(\Q{Send}(\oracle[B][j],m), \bot, (\rejected))$ \\
		\rowcolor{ProtocolTwoColor}
		$(\Q{Send}(\oracle[B][j],\Ckey), \bot, (\accepted,\accepted,\accepted))$ &  \NULL & \NULL  \\
		\rowcolor{ProtocolTwoColor}
		$(\Q{Send}(\oracle[B][j],\Ckey'), \bot, (\accepted,\accepted,\rejected))$ & \NULL & \NULL  \\
		
%		\hline
		
		\rowcolor{EmptyRowColor}
		&& \\
		
%		\hline
		
		\rowcolor{ProtocolOneColor}
		$(\Q{Send}(\oracle[S][k],m), m^*, (\running,\running,\running))$ &$(\Q{Send}(\oracle[S][k],m), m^*, (\running))$& \NULL  \\
		\rowcolor{ProtocolOneColor}
		$(\Q{Send}(\oracle[S][k],m), \bot, (\rejected,\rejected,\rejected))$ & $(\Q{Send}(\oracle[S][k],m), \bot, (\rejected))$ & \NULL  \\
		\rowcolor{ProtocolOneColor}
		$(\Q{Send}(\oracle[S][k],m), m^*, (\accepted,\running,\running))^\dagger$ &$(\Q{Send}(\oracle[S][k],m), \bot, (\accepted))$& $(\Q{NewSession}(S, B ),\oracle[S][k], m^* )$ \\
		\rowcolor{ProtocolTwoColor}
		$(\Q{Send}(\oracle[S][k],m), m^*, (\accepted,\running,\running))$ & \NULL & $(\Q{Send}(\oracle[S][k], m), m^*, (\running))$\\
		\rowcolor{ProtocolTwoColor}
		$(\Q{Send}(\oracle[S][k],m), \Ckey, (\accepted,\accepted,\accepted))$ & \NULL &$(\Q{Send}(\oracle[S][k], m), \bot, (\accepted))$\\
		\rowcolor{ProtocolTwoColor}		
		$(\Q{Send}(\oracle[S][k],m), \bot, (\accepted,\rejected,\rejected))$ & \NULL & $(\Q{Send}(\oracle[S][k],m), \bot, (\rejected))$  \\
		
		\bottomrule
	
	\end{tabular}
	
		\begin{tablenotes}
     \footnotesize
     \smallskip
     \item $^\dagger$This rule only applies if $\oracle[S][k].\vv{\runstate} = (\running,\running,\running)$ for all prior $\Q{Send}$ queries to $\oracle[S][k]$,
     i.e., if receiving message $m$ caused session $\oracle[S][k]$ to accept in sub-protocol $\protocol_1$.
   \end{tablenotes}

\end{threeparttable}


%\end{adjustbox}

\end{table}
%\vfill
%}

\end{landscape}

%\end{adjustwidth}
\restoregeometry



