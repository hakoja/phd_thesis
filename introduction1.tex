\section{Introduction}\label{sec:introduction}
The Extensible Authentication Protocol (EAP),
specified in RFC~3748~\cite{IETF:RFC3748:EAP},
is a widely used authentication framework for network access control. 
It is particularly common in wireless networks,
being used by protocols like IEEE~802.11 (Wi-Fi),
IEEE~802.16 (WiMAX) and various 3G/4G mobile networks. 
The typical use case of EAP is in settings where a \emph{client} seeks to gain access to a network controlled by an \emph{authenticator},
but where the client and authenticator do not share any common credentials.
EAP allows the client and authenticator to authenticate each other based on a mutually trusted \emph{server}.
Technically,
EAP is not a specific authentication mechanism on its own,
rather it specifies a generic three-party framework for composing other concrete authentication protocols.
This provides applications of EAP the freedom to choose whatever concrete instantiation is suitable for their own specific setting. 
The success of this approach is apparent by the huge and diverse set of real-life deployments using the EAP framework.

Surprisingly then,
given its prevalence and importance,
there has been no formal reduction-based provable security analysis of EAP.
One reason for this might be due to the general nature of EAP itself.
As mentioned above,
EAP is not a single protocol on its own,
but relies on other sub-protocols to instantiate it.
As such,
many things in the EAP specification are left unspecified or considered out of scope.
On the other hand,
in order to conduct a formal security analysis of EAP,
these details matter and require a careful treatment.
More generally, the need to make assumptions on protocols outside of the EAP standard makes it harder to analyze
as described by Hoeper and Chen~\cite{HoeperC:2007:EAP_claims_fail}.

Another reason for the lack of reductionist-based security results on EAP might be due to the fact that it is a three-party protocol.
As pointed out by Schwenk in his recent work on Kerberos~\cite{EPRINT:Schwenk16},
apart from a few papers like \cite{STOC:BelRog95,PKC:AbdFouPoi05,EPRINT:NCPW14,ACNS:AFMOR16,EPRINT:Schwenk16} relatively little work has been done on three-party protocols\footnote{Considered distinct from \emph{group-key exchange} protocols.} in the computational setting compared to the huge literature on two-party protocols.
%Three-party protocols can be more difficult to model formally.


In this paper we aim to remedy this state-of-affairs by providing a formal reductionist analysis of EAP in the computational setting.
Our result is modular and reflects the compositional nature of EAP.
Building upon this result we extend our analysis to cover a very common application of the EAP framework:
network authentication and access control in enterprise and university networks.
In particular,
we focus on wireless networks based on the IEEE~802.11 standard~\cite{IEEE:2012:802.11} when combined with EAP for centralized authentication.
This setting is often referred to as WPA2-Enterprise. 
Current results on IEEE~802.11 have so far only focused on the much simpler WPA2-PSK setting where the client and access point (authenticator)
already share a pre-established long-term key.
WPA2-PSK is typically used in wireless home-networks and small offices where sharing a single long-term key among many users is feasible,
while WPA2-Enterprise is used in larger organizations and businesses where central authentication is necessary. 
Based on our result on EAP we can now provide a reduction-based analysis of WPA2-Enterprise as well. 
%Below we will further expand upon our results,
%but first we provide a brief description of the EAP architecture and how IEEE~802.11 relates to it. 

\begin{figure}[t]
	\centerline{
		\includestandalone[width=0.85\textwidth,mode=build]{tikz_EAP}
%		\includegraphics[width=0.8\textwidth]{tikz_EAP}
	}
	\caption{The three-party EAP architecture.
	Concrete example protocols shown in parenthesis.}
	\label{fig:EAP-architecture}
\end{figure}

\paragraph{Review of EAP and IEEE~802.11.}
The general EAP architecture is shown in Fig.~\ref{fig:EAP-architecture}.
The exchange begins when a client wants to connect to some access-controlled service protected by an authenticator.
For most practical purposes the service is simply getting network access (e.g., to the Internet),
and the authenticator is a wireless access point or link-layer switch.
The client and authenticator do not share any common secrets a-priori
but instead share credentials with a mutually trusted server. 
The purpose of EAP is to allow the client to authenticate itself to the server using whatever authentication protocol they like,
for instance TLS or IPsec,
and then having the server ``vouch'' for the client to the authenticator.
In order to do this in a generic and uniform manner across different authentication protocols,
EAP  defines a frame format as well as a set of generic  messages known as Request/Response messages.
The Request and Response messages are used to encapsulate the concrete authentication protocol being used between the client and the server.
This frees the authenticator---which 
often operates in so-called \emph{pass-through} mode between the client and the server,
meaning that all their messages pass through it---from
having to support all the concrete authentication mechanisms itself.
Instead,
the authenticator only needs to inspect the generic EAP messages.
This is valuable in settings where it can be difficult to update the authenticator(s).



The combination of a concrete authentication protocol,
like TLS,
together with the EAP encapsulation is called an \emph{EAP method}.
Numerous EAP methods have been defined,
with EAP-TLS~\cite{IETF:2008:RFC5216-EAP-TLS} being one of the most widely supported.
In EAP-TLS the client and server mutually authenticate each other based on certificates.
Besides authentication, the EAP method usually also supports the derivation of a shared key between the client and the server.
In this paper we will assume that all EAP methods derive keys. 
The server will forward this key to the authenticator over some separate channel,
where the choice of channel protocol is orthogonal to the choice of EAP method used between the client and the server.
While the EAP standard does not specify the protocol to use between the server and the authenticator,
the de-facto standard in practice is RADIUS~\cite{IETF:RFC:2865-RADIUS}.\footnote{Within 
the EAP standard lingo,
the protocol run between the server and authenticator is generally referred to as an \emph{Authentication, Authorization and Accounting (AAA)} protocol.
Besides RADIUS is Diameter~\cite{IETF:RFC:6733:DIAMETER} another common AAA protocol. 
}


Once the key is transported from the server to the authenticator the EAP exchange is technically complete.   
Still,
at this point the client does not actually have any guarantee that the authenticator is in possession of the same key as itself,
since the communication between the server and the authenticator happens over a completely separate channel.
Thus,
in practice the client and the authenticator now typically run some link-layer specific protocol in order to prove mutual possession of the key distributed by EAP. 
Additionally,
this also serves to implicitly authenticate the client and the authenticator to each other,
since in order to get the same key they must have been able to authenticate themselves to the mutually trusted server.

Again,
the subsequent link-layer protocol run between the client and the authenticator is outside the scope of EAP
and could in principal be any one of a number of different protocols.
In this paper we are going to focus particularly on the a very common setting of wireless LANs provided by the IEEE~802.11 protocol\cite{IEEE:2012:802.11}.
In this case the ``key-confirmation'' protocol run between the client and the authenticator (i.e., access point) 
is known to as the ``4-Way-Handshake'' (4WHS).
The 4WHS is both an authentication protocol as well as a key-exchange protocol,
meaning that the client and the access point also derive fresh session keys.
This session key is used to protect the bulk data transfer between the client and the access point on the WLAN.
Although technically incorrect,
the security part of the IEEE~802.11 wireless standard is also commonly referred to by the name ``WPA2''.
When EAP and IEEE~802.11 are combined,
then the entire exchange is referred to as ``IEEE~802.11 with upper-layer authentication'' or WPA2-Enterprise.






\paragraph{On the difficulty of modeling EAP.}
In this paper we analyze the security of EAP
both when considered on its own as well as when combined with IEEE~802.11 in a formal reduction-based setting.
We do this in a modular way:
first considering the security properties provided by EAP and IEEE~802.11 in isolation,
then using a composition theorem to link them together.
However, 
since EAP inherently depends on other protocols,
assessing the exact security guarantees it provides is in one sense harder than for ``standalone'' protocols like TLS, IKE and SSH. 
While the EAP specification defines the security requirements of each EAP method (\cite[§7]{IETF:RFC3748:EAP}),
this only covers the communication between the client and the server.
%We claim that it is more accurate to think of EAP as a three-party protocol.
It leaves unspecified how, for example, the derived key should be transferred from the server to the authenticator.
Hence, 
solely using the security claims from RFC~3748 is not sufficient to decide the security of EAP considered as a three-party protocol.
In fact,
it is impossible to talk about ``the'' EAP and its security without making further assumptions on the various protocols that make up EAP.
Consequently,
in order to be able to analyze EAP,
we will have to make some assumptions on these protocols.
 

Firstly,
in this paper we are going to assume that the communication between the authenticator and the server takes place over a secure channel.
Specifically, we model the link as a two-party authenticated channel establishment protocol (2P-ACCE) based on symmetric long-term pre-shared keys\footnote{There is nothing fundamental about our assumption on symmetric PSKs here.
The choice is made simply because the trust-relationship between the server and authenticator is commonly based on symmetric PSKs in practice.
Our results work just as well for certificate-based authentication.} (see Section~\ref{sec:ACCE} for a formal definition).
Since most key-transport protocols used between the server and the authenticator support to be run inside a secure channel
(see e.g. RADIUS-over-TLS~\cite{IETF:RFC:6614:RADIUS_over_TLS} and Diameter~\cite{IETF:RFC:6733:DIAMETER}),
this assumption seems reasonable.

%Another\todo{HJ: One PKC reviewer said that this paragraph was badly written. Do you understand it? Should we mention this at all? Is it possible to skip this entire paragraph? CB: I tried to address is by inverting the order of text.}{} problem arises from an issue related to channel binding (explained after this paragraph). Namely, channel binding binds a key to a pair of identities, and thus, the name spaces of these identities are important. As common in security analysis, in this paper, we make use of a single unique identity per party.
%%
%% at the scope of client-server and server-authenticator,
%%and that the identities presented at the client-authenticator link are sufficient for the server to identify the authenticator.
%%This is related to the issue of channel binding (explained next),
%%and will be further discussed in Section~\ref{sec:EAP}.
%%
%%
%Note however, that in practice, parties might use \emph{different} identities in different subprotocols and thus, identities might need to be mapped. These different identities arise from the different types of addressing schemes that different types of media links have.
%%Another issue is the fact that the communication between the parties in EAP usually takes place on different types of media links, having different types of addressing schemes.
%%This means that there might not be a single uniform and consistent identity space shared by all the protocol participants.
%For example,
%the identities visible at the client-server link might be based on username and certificates,
%while the only identities available at the client-authenticator link might be link-layer addresses.
%As pointed out by Hoeper and Chen~\cite{HoeperC:2007:EAP_claims_fail},
%this makes formal analysis more difficult.
 
Second,
since the authenticator often works in pass-through mode,
a well-known issue with the EAP architecture is the so-called ``lying authenticator problem''.
Namely,
a malicious authenticator may present false or inconsistent identity information to the client and the server. 
Unless the EAP method provides a feature known as \emph{channel binding}~\cite{IETF:RFC6677:EAP-channel-binding},
there is no way for the client and server to verify that they are in fact talking to the same authenticator
(see \cite[§3]{IETF:RFC6677:EAP-channel-binding} for examples of attacks that this may enable).
Hence,
in this paper we are generally going to assume that EAP provides channel binding.
However,
we will also briefly explore the (in)security of EAP without channel binding in Section~\ref{sec:EAP:without_channel_binding}. 
While there are a couple of suggested ways to achieve channel binding in EAP (see~\cite[§4.1]{IETF:RFC6677:EAP-channel-binding}),
here we are only going to focus on the cryptographically simplest one,
described in RFC draft \sloppy{\url{draft-ohba-eap-channel-binding-02}}~\cite{IETF:draft:EAP-channel-binding}. 
In this approach, 
the client and authenticator identities are being input to the key-derivation step of EAP,
cryptographically binding the session key to the right pair of identities (see Section~\ref{sec:EAP:channel-binding_scope} for details). 




\paragraph{Our contributions.}
The main contributions of this paper are the following. 
\begin{itemize}
	\item We provide the first reduction-based security result for EAP assuming it employs channel binding. 
	
	\item We show how the security guarantees of EAP can be upgraded by adding an additional key-confirmation step (modeled as a 2P-AKE). 
	This corresponds to the common scenario where EAP is first used to bootstrap the establishment of a common key among the client and the authenticator,
	then some link-layer specific protocol is run between the client and the authenticator in order to prove mutual possession of that key
	(in addition to establishing session keys for the lower-layer link).   
	
	\item Our technical means for obtaining the above results are two modular composition theorems which may be of separate interest.
	Namely,
	the two theorems consider a fairly generic way of constructing a 3P-AKE protocol,
	using generic 2P-AKEs and secure channels as building blocks.
	For instance,
	both Kerberos and the AKA protocol used within the UMTS and LTE mobile networks,
	fit the description of our 3P-AKE construction.
	In particular,
	for the latter protocol,
	our theorems might enable a more general and modular analysis than the one recently provided by Alt et al.~\cite{ACNS:AFMOR16}.
	
		  
	
	\item As a stepping stone towards our final result,
	we provide a reduction-based security result for the IEEE~802.11 4-Way-Handshake protocol in the pre-shared key setting without the use of EAP
	(i.e., WPA2-PSK).
	This corresponds to the  setting typically found in home WLANs.
	To the best of our knowledge,
	we are the first do to such an analysis using standard game-based definitions of AKE. 
	Previous analyses of WPA2-PSK have either been based on formal methods~\cite{CCS:HSDDM05} or on universal composition frameworks~\cite{CCS:KusTue11,RSA:KusTue11}.
	
	\item Finally,
	the results above combine to provide the first reduction-based security result for EAP combined with IEEE~802.11 (WPA2-Enterprise).
	This corresponds to the setting usually found in enterprise and university WLANs.
	For instance,
	the \emph{eduroam} network\footnote{\url{https://www.eduroam.org}},
	which is used to provide wireless roaming services to university and research institutions,
	uses EAP and IEEE~802.11.
	

\end{itemize}

The structure of our paper is as follows.
In Section~\ref{sec:definitions} we provide our formal security definitions,
including 2P/3P-AKE, ACCE (secure channels) and explicit entity authentication.
In Section~\ref{sec:generic_composition_results} we prove our two composition results for two generic protocol constructions.
In Section~\ref{sec:EAP} we show how the security of standalone EAP follows directly from  our first composition result by making appropriate assumptions on the concrete protocols used to instantiate the EAP framework.
Finally, in Section~\ref{sec:802.11},
we prove the security of the IEEE~802.11 4WHS protocol.
Combined with our result on EAP in Section~\ref{sec:EAP} and our second composition result,
this immediately yields a result for IEEE~802.11 combined with EAP.




\paragraph{Technical overview of our results.}
The main technical contributions of this paper are two fairly generic composition theorems which correspond to the ``cryptographic core'' of EAP with or without a subsequent key-confirmation step, respectively. 
To obtain these theorems we first have to provide an appropriate security model.
Our starting point is the original 3P-AKE model of Bellare and Rogaway~\cite{STOC:BelRog95}.
However,
due the different security guarantees provided by standalone EAP,
EAP combined with IEEE~802.11 and standalone IEEE~802.11,
we in fact define \emph{three} different models of varying strength.
The main difference between these models lays in the level of adaptivity afforded to the adversary in terms of long-term key leakage,
capturing full, weak and no forward secrecy,
respectively. 
The distinction between full and weak forward secrecy follows the definition given in the eCK model\footnote{We 
do not consider ephemeral key-leakage in this paper however.}~\cite{EPRINT:LaMLauMit06}.
 

Briefly,
the only difference between the strongest security model (full forward secrecy) and the intermediate one
(weak forward secrecy) depends on what happens if the test-session does not have a partner.
When this happens in the strongest model the adversary is still allowed to learn the long-term keys of the parties involved,
provided this happens after the test-session accepted.
On the other hand,
in the intermediate model,
if the test-session does not have a partner then the adversary is forbidden from learning these long-term keys.
Finally,
if the test-session \emph{does} have a partner,
then there is no difference between the two models:
the adversary is allowed to learn any long-term key at any time.
The formal definitions of these models are provided in Section~\ref{sec:defs:AKE}.
	
Preempting our own results a bit,
we show that standalone EAP can achieve weak forward secrecy,
while IEEE 802.11 \emph{without} upper-layer authentication achieves no forward secrecy at all
(this is natural since the 4WHS relies exclusively on symmetric primitives).
However,
when EAP and IEEE~802.11 are \emph{combined},
the security is upgraded to achieve full forward secrecy in our strongest corruption model.
	

Intuitively,
the reason why standalone EAP does not achieve full forward secrecy is because it does not provide \emph{key confirmation}.
Namely,
after completing the EAP method with the server,
the client has no guarantee that the key-transport protocol between the server and the authenticator actually took place.
Specifically,
the following attack illustrates why EAP does not provide full forward secrecy. 
Suppose that \emph{after} the client accepted,
but \emph{before} the key-transport protocol between the server and authenticator starts running,
an adversary learns the long-term PSK of the server and the authenticator.
Now the adversary simply impersonates the authenticator towards the server and have it send over the session key it previously established with the client. 
According to the full forward secrecy model this attack would be valid since the exposure of the PSK happened after the client accepted.
On the other hand,
in the weak forward secrecy model the attack is not considered valid because client session doesn't have a partner,
hence the PSK cannot be exposed.

Essentially,
the purpose of the link-layer protocol is to provide key-confirmation to the standalone EAP protocol,
which ensures that the client will always have a partner before it accepts.
This is similar to how the security of the two-flow variant of HMQV can be upgraded from only providing weak forward secrecy to providing full forward secrecy simply by adding a third flow to it (see~\cite[§3]{C:Krawczyk05}).
While the property of key confirmation was recently formalized in~\cite{SP:FGSW16},
in this paper we model the key-confirmation step by assuming that the link-layer protocol provides \emph{explicit} entity authentication 
(formally defined in Section~\ref{sec:enitity_authentication}).
 


Besides the introduction of the three different corruption models,
we only provide a few other changes to the original 3P-AKE model of Bellare and Rogaway~\cite{STOC:BelRog95}. 
For example, 
we support both asymmetric and symmetric long-term keys,
and dispense with the explicit $\Q{SendS}$ query to the server 
(now modeled simply as a regular $\Q{Send}$ query). 

One thing we \emph{do} keep from~\cite{STOC:BelRog95} however,
is the concept of \emph{partner functions}.
Interestingly,
the use of partner functions has seen rather limited adoption when compared to partnering based on matching conversations~\cite{C:BelRog93} or session identifiers (SIDs)~\cite{EC:BelPoiRog00}.
However,
when modeling EAP,
we are in the peculiar situation that the parties that we need to partner (the client and the authenticator) do not have any messages in common! 
Naturally,
this makes partnering based on matching conversations more difficult,
but it also severely limits our choice of SIDs: 
we are essentially forced to pick their session keys as the SID.
While using the session key as the SID is reasonable in many settings (cf.~\cite{ASIACCS:KobShiStr09}),
it does not necessarily guarantee \emph{public} partnering (see \cite{CCS:BFWW11}).
%\footnote{Note that even when matching based on the session key, one could, potentially, provide some matching function that tells based on public information which sessions have the same key --- however, if such a public partnering function needs to exist for composition, then why not use it from the start?}
This is important for modular composition proofs like our own.
While partnering functions have been criticized for being non-intuitive and hard to work with 
(even by Rogaway himself~\cite[§6]{Rogaway:2004:role_of_definitions}),
they generalize more naturally to the three-party setting than SIDs.
Essentially,
this is because partner functions can take global transcript information into consideration rather than only the local views of the two partners.
%In a companion manuscript~\cite{BrzuskaCJKW:2017:partner_mechanisms} we explore partner functions in more detail,
%showing their soundness as a partnering tool for analyzing key exchange protocols.\todo{HJ: This reference will piss of the reader since the paper is not available to him. Can we do this better? Can we just drop the reference? CB: We could reference ePrint without number. This way, when someone reads the proceedings version in 8 months, they can have a look at the ePrint version of our paper.}{}
%

