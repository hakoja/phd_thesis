\chapter{Security of IEEE 802.11}\label{sec:802.11}

\begingroup
\hypersetup{linkcolor=black}
\minitoc
\endgroup

In \cref{sec:EAP-TLS-security} we proved that EAP-TLS is a secure 2P-AKE protocol.
By the first composition theorem this means that basic EAP using EAP-TLS as its EAP method is a secure 3P-AKE protocol with weak forward secrecy.
To complete the picture on full EAP having full forward,
we need to establish that there is a link-layer protocol which satisfies the requirements of the second composition theorem.
In this \lcnamecref{sec:802.11} we do exactly that for the IEEE~802.11 protocol.
IEEE~802.11 is also of independent interest outside of its use in EAP,
since it is the most widely used standard for creating wireless LANs.
%Millions of home networks are today secured by the IEEE~802.11 protocol.




\section{Summary of the IEEE~802.11 protocol}

%Recall from \cref{sec:descriptions:EAP_&_802.11:802:11} that IEEE~802.11 supports three modes of operation: 
%infrastructure mode, ad-hoc mode, and mesh network mode. 
%In ad-hoc mode and mesh-networking mode there is no central infrastructure and the wireless clients talk directly to each other.
%On the other hand,
%in infrastructure mode the clients only communicate through an access point which also provides connectivity to a larger WAN. 
%Infrastructure mode is by far the most common mode of IEEE~802.11,
%and is the only one that will be covered in this \lcnamecref{sec:802.11}. 

IEEE~802.11  is a link-layer protocol,
aiming to secure the wireless link between a client and an access point. 
As explained in \cref{sec:description:EAP_&_802.11:802.11:security_protocol},
IEEE 802.11 consists of two main security protocols for this purpose:
the 4-Way-Handshake (4WHS) protocol used to authenticate and establish a session key between the client and access point;
and CCMP used to secure the actual application data.


The 4WHS is based on a symmetric \emph{pairwise master key (PMK)} shared between the client and the access point.
The PMK can either be a pre-shared key (PSK)
or distributed through some other means,
for instance using EAP.
The first alternative is  most typically found in home networks where a static PMK is manually configured at the access point and at every connecting device.\footnote{The PMK is usually not configured directly,
but instead derived from a password using the Password Based Key Derivation Function 2 (PBKDF2)~\cite{IETF:RFC8018:PBKDF2}. 
We ignore this detail here.
}
This variant is also commonly referred to as WPA2-PSK.
The second alternative,
often referred to as WPA2-Enterprise,
is normally used in large organization like universities and big companies where there are many users and access points.
In this setting it is infeasible for every user and access point to share the same PMK.
Instead,
a central authentication server is used to manage authentication as well as distributing new  PMKs for every established session.
The protocol used to access the authentication server is normally EAP. 
  
In \cref{sec:802.11:4WHS} we will analyze the PSK variant of the 4WHS protocol,
and in \cref{sec:802.11:upper-layer_authentication} we describe how this result can be combined with the composition theorems of \cref{sec:generic_composition_results}
to also get a result for the enterprise variant of IEEE~802.11.
In \cref{sec:802.11:CCMP} we analyze the CCMP algorithm.
Finally,
in \cref{sec:802.11:multi-cipher} we informally discuss how our results on the 4WHS protocol can be extended to also cover multi-ciphersuite and negotiation security.


\subsection{Related work on IEEE~802.11}
As explained in \cref{sec:802.11:security_history},
IEEE~802.11 has been subject to a large amount of cryptanalysis,
especially against WEP and TKIP.
Here we only discuss related work as it pertains to the formal analysis of IEEE~802.11.
In the symbolic setting, 
He et al.~\cite{CCS:HSDDM05} have conducted a formal analysis of the 4WHS protocol using their Protocol Compositional Logic.
In the computational setting,
Küsters and Tuengerthal~\cite{RSA:KusTue11, CCS:KusTue11} have analyzed both the 4WHS protocol and CCMP in their universal composability framework called IITM.
In the game-based setting the only work we are aware of that attempts to analyze the 4WHS protocol is~\cite{ZhangMM:2005:802.11_claimed}.
However,
this work is quite rudimentary;
security definitions and theorems are only outlined and it provides no proofs nor proof sketches.
To the best of our knowledge,
there is no existing analysis of CCMP in the game-based setting.







\section{Analyzing the 4-Way Handshake}\label{sec:802.11:4WHS}



\subsection{Formal modeling}

The 4WHS protocol was described in detail in \cref{sec:description:EAP_&_802.11:802.11:security_protocol},
and our formal modeling of it is shown in \cref{fig:4WHS}.
The 4WHS depends on a pseudorandom function $\mathsf{PRF}$ and a MAC scheme $\mac = (\mactag, \macvrfy)$.
%see \cref{sec:other_definitions} for their formal definitions.
We use the notation $[x]_k \defeq x \concat \sigma$ to denote a message $x$ together with its MAC tag $\sigma \gets \mactag(k,x)$.
%We now explain this model more carefully.


An IEEE~802.11 network is identified by its SSID.
In the PSK setting each SSID is associated with a single 256 bit pairwise master key ($\PMK$).
However,
the same SSID can be broadcasted by multiple different access points.
This could happen either by chance if independent networks unknowingly configuring the same SSID,
or deliberately if multiple access points are combined to form an extended service set (ESS) in order increase the coverage of the network.
In the former case,
the PMK will (usually) be different,
while in the latter case the same PMK will be shared by all the access points of the ESS.
Technically speaking,
if two independent networks configure the same SSID \emph{and} PMK,
then they are in fact part of the same ESS.


An access point can also broadcast multiple SSIDs at the same time,
and hence belong to more than one ESS
(using different PMKs).
%In order to indicate which PMK is being used by the ESSs it serves,
%an access point will include a PMK \emph{identifier} in its first handshake message. 
%However,
For simplicity we are going to assume that every ESS has a unique SSID.
%and thus every client will know which PMK to use when connecting to a specific network.  
In the PSK setting all clients connecting to the same ESS will share the same~PMK.
 

We are mostly going to ignore the details of the IEEE~802.11 frame format used in the real 4WHS protocol.
For our purposes it is sufficient to model the four handshake messages as consisting of a nonce plus some value $p_i = i \concat x$ which uniquely determines each message $m_i$.
If a received message does not match the expected format it is silently discarded.

For $p_1$ in particular we moreover assume that $x$ is a constant, 
which means that $p_1$ itself is a constant.
Thus, 
although the first handshake message lacks integrity protection,
an attacker can in effect only modify the nonce value because a client will always check that it matches the excepted format of ``$\nonce_{AP} \concat 1 \concat x$''.
Of course,
a real IEEE~802.11 frame consists of many bit fields,
but for message $m_1$ they all have pre-determined values except for the nonce field.
So modeling $p_1$ as a constant faithfully represents the real IEEE~802.11 frame.
 
For the other three handshake messages there \emph{are} variable bit fields that an attacker could potentially influence.
But since these messages are protected by a MAC,
the adversary will be unable to modify them (as we will show).
%Whether we model $p_2$, $p_3$ and $p_4$ as constants or as variable, but distinct, values makes no difference for our analysis. 



\begin{figure}

	\centering

	\includestandalone[width=0.95\textwidth,mode=build,center]{tikz_4WHS}

	\caption{Our formal model of the IEEE~802.11 4-Way Handshake protocol. 
	The client $C$ and access point $AP$ share a symmetric key $\PMK$.
%	and $\mac = (\mathsf{KG}, \mactag, \macvrfy)$ is MAC scheme.
%	Not shown: each side verifies that incoming messages have the right format.}
	}
	\label{fig:4WHS}
\end{figure}


Recall from \cref{sec:description:EAP_&_802.11:802.11:security_protocol} that prior to the 4WHS there is a negotiation phase where the client and access point agree upon the ciphersuite to use. 
This includes the choice of $\mathsf{PRF}$ and $\mac$.
In this \lcnamecref{sec:802.11:4WHS} we assume that there is a single fixed ciphersuite being used.
The topic of multi-ciphersuite and negotiation security will be treated in \cref{sec:802.11:multi-cipher}. 

Identities in the 4WHS protocol are based on the parties' 48 bit link-layer addresses.
The functions  $\min$ and $\max$ compute the minimum and maximum of two link-layer addresses when treated as 48 bit unsigned integers.
In the following,
let
\begin{equation}
	\minmax(X,Y) = \min \lbrace X, Y \rbrace || \max \lbrace X, Y \rbrace.
\end{equation}

The 4WHS protocol proceeds as follows.



\begin{enumerate}
	\item  The exchange begins with the access point $AP$ sending the message $m_1 = \nonce_{AP} \concat p_1$ to the client $C$,
	where $\nonce_{AP}$ is a 256 bit nonce and $p_1$ is a constant.
%	\footnote{We
%	are ignoring the exact encoding of the IEEE~802.11 packets here.
%	For our purposes it sufficient to model them as consisting of a nonce plus some other unspecified data.
%	}  
	
	\item On receiving $m_1 = \nonce_{AP} \concat p_1$, 
	the client $C$ generates its own 256 bit nonce $\nonce_C$
	and derives a  \emph{pairwise transient key (PTK)} as
	\begin{equation}\label{eq:802.11:4WHS:derive_PTK}
		\PTK \defeq \key_\mu \concat \key_\epsilon \concat \key_\alpha \gets \mathsf{PRF}(\PMK, P \concat \nonce) ,
	\end{equation}
	where 
	$P \gets \minmax(C, AP)$
	and
	$\nonce \gets  \minmax(\nonce_{C}, \nonce_{AP})$. 

	The sub-key $\key_\alpha$ will be the session key eventually output by the client in the 4WHS.
	The sub-key $\key_\mu$ is the MAC key used to protect the handshake messages.
	The sub-key $\key_\epsilon$	is an encryption key used to protect a group key $\GTK$ transmitted from $AP$ to $C$.
	Since we do not model any group aspect of IEEE~802.11 in this thesis,
	we ignore $\key_\epsilon$ and set it to be the empty string $\emptystring$.
	
	
	After deriving $\PTK$,
	$C$ creates and sends the next protocol message $m_2 = [\nonce_C \concat p_2]_{\key_\mu}$.
	%Additionally,
	%it set its ``SID'' to be $P \concat \nonce$.
	
	\item On receiving $m_2 = [\nonce_C \concat p_2]_{\key_\mu}$,
	the access point $AP$ derives the pairwise transient key $\PTK = \key_\mu \concat \key_\alpha \gets \mathsf{PRF}(\PMK, P \concat \nonce)$ according to \cref{eq:802.11:4WHS:derive_PTK}.
	With the sub-key $\key_\mu$ it verifies the MAC tag on $m_2$.
	
	If the verification succeeds,
	then $AP$ stores $\PTK \gets \key_{\mu} \concat \key_\alpha$ as its $\PTK$ 
	%then $AP$ sets its ``SID'' to $P \concat \nonce$,
	and sends out the third protocol message $m_3 = [\nonce_{AP} \concat p_3]_{\key_\mu}$.
	Additionally,
	$AP$,
	or rather the corresponding session at $AP$,
	sets the accept state to $\runstate = \accepted$
	(since the 4WHS does not consist of any sub-protocols, we simplify the accept vector $\vrunstate$ to a single value $\runstate = \runstate_F$).
	
	 

	If the verification fails,
	then $AP$ silently discards $m_2$,
	as well as the derived PTK,  
	and continues running.
	
	\item 
	On receiving $m_3 = [\nonce_{AP} \concat p_3]_{\key_{\mu}}$, 
	the client $C$  checks that $\nonce_{AP}$ is the same as the nonce it received in message~$m_1$ 
	(denoted ``$m_1.\nonce_{AP}$'' in \cref{fig:4WHS})
	and verifies that the MAC tag on message $m_3$ is valid.

	If either of these checks fail,
	then $C$ silently discards $m_3$ and continues running.

	Otherwise,
	$C$ sends out the final handshake message $m_4 = [p_4]_{\key_\mu}$.
	Additionally,
	it sets its own acceptance state to $\runstate = \accepted$. 
	

	
	\item On receiving $m_4$, the access point $AP$ verifies the MAC with the key $\key_\mu$.

	If the verification succeeds,
	then the 4WHS is over and $AP$ is ready to receive encrypted messages under the key $\key_\alpha$.

	If the verification fails,
	then $AP$ silently discards the message and continues running.

\end{enumerate}





\begin{remark}
The error handling semantics of the 4WHS is different from protocols like TLS and SSH.
Specifically,
rather than rejecting immediately on receiving a bad message,
a session will instead silently discard it.
Combined with the fact that the key used to verify the handshake messages ($\key_{\mu}$)
is derived from the handshake messages themselves,
modeling the error handling semantics of the 4WHS protocol will make our analysis a little more complicated
(specifically the proof of explicit entity authentication in \cref{sec:802.11:4WHS:EA}).
\end{remark}

\begin{remark}
While the IEEE 802.11 standard claims that message $m_4$ serves no cryptographic purpose \cite[§11.6.6.8.4]{IEEE:2012:802.11},
it is in fact crucial for downgrade protection as shown by Vanhoef et al.~\cite{ASIACCS:VanSchPie17}.
Nevertheless,
within the context of this \lcnamecref{sec:802.11:4WHS},
where we assume that there is a \emph{single} ciphersuite being used,
message $m_4$ \emph{is} indeed superfluous.
\end{remark}







\subsection{\texorpdfstring{\akenfstext }{AKEnfs} security}\label{sec:802.11:4WHS:AKE}
We begin by proving that the 4WHS constitutes a secure 2P-AKE protocol in the \akenfstext  model,
when all PMKs are pre-shared keys. 
Remember that we have assumed that each SSID belongs to a unique ESS,
which is potentially served by multiple access points all sharing the same PMK. 
All clients connecting to this ESS will also use the same PMK.
Since a client might share multiple PMKs with the same access point if the latter serves multiple ESSs,
we slightly change the syntax of the $\Corrupt$ query  to instead take an SSID as input,
identifying the PMK of a specific ESS.
%Following that,
%we show that it also achieves explicit entity authentication.
Since the access point has the initiator role and the client has the responder role in the 4WHS,
we write $\Parties_{AP} = \Inits$ and $\Parties_{C} = \Responders$.



\begin{theorem}\label{thm:4WHS:2P-AKE}
For any adversary $\A$ in security experiment \akenfstext against the 4WHS protocol as described above,
we can create a partner function $f$ and an algorithm $\A[D]$, such that
\begin{equation}
	\adv_{\mathsf{4WHS},f}^{\operatorname{\mathsf{2P-\akenfs}}}(\A)  
	\leq 2 \cdot \numssid  \cdot \adv_{\mathsf{PRF}}^{\prf}(\A[D]) 
	+\frac{(\numparties  \numsessions)^2}{2^{\noncelen+1}} ,
\end{equation}
where $\numssid$ is the number of unique SSIDs, 
%$\numparties$ is the number of parties, 
$\numsessions$ is the number of sessions that $\A$ creates at each party,
$\noncelen$ is the length of the nonces,
and $\numparties = |\mathcal{P}_C | + |\mathcal{P}_{AP}|$.	
\end{theorem}


By our assumption above,
$\numssid$ corresponds to the number of ESSs and thus also gives an upper bound on the number of PMKs in the system.
Moreover,
by assuming that no access point belongs to more than $c$ different ESSs,
then $c \cdot | \Parties_{AP} |$ is an upper bound on $\numssid$.



\begin{proof}$\ $



\paragraph{Defining the partner function \texorpdfstring{$f$}{f}.}
For the analysis of the 4HWS it would be natural to use session identifiers as the partnering mechanism.
Namely,
the session identifier of a session $\oracle$ would be the string $\sid = P \concat \nonce$ that $\oracle$ input to its $\mathsf{PRF}$ in order to create the session key (see \cref{eq:802.11:4WHS:derive_PTK}).
%\footnote{For an access point, the SID would only be set if the verification of the received $m_2$ message succeeded.
%}
However, because our security model is phrased in terms of partner functions, 
we instead synthetically encode the session identifier as a partner function by saying that $\oracle$'s partner is the \emph{first} session---different from $\oracle$---that sets the same session identifier as $\oracle$. 
Taking the first such session is important in order to make the partner function well-defined.

In more detail,
suppose  $\oracle[C][i]$ is a client session and $\oracle[AP][i]$ is an access point session.
For the purposes of this description,
let us associate an extra variable $\sid$ to each session.
Session $\oracle[C][i]$ sets its value of $\sid$ to be the string that it input to the $\mathsf{PRF}$
after having received the first handshake message.
Session $\oracle[AP][i]$ also sets its value of $\sid$ to be the input to the $\mathsf{PRF}$,
but it only sets its value \emph{after} it has verified the MAC of the second handshake message.
Partner function $f$ can now be defined as follows.

\begin{itemize}
\item \emph{\underline{Definition of $f$}}:  $\oracle[C][i]$ and $\oracle[AP][j]$ are partners if and only if

	\begin{enumerate}
		\item $\oracle[C][i].\sid = \oracle[A][j].\sid$, and
	
		\item $\oracle[C][i]$ and $\oracle[A][i]$ where the \emph{first} sessions at $C$ and $AP$,
		respectively, for which Item 1 holds.
	\end{enumerate}
\end{itemize}

Note that since the party identities of the session's intended peers are included in the $\sid$ string,
we do not need to include agreement on peers as an explicit requirement.
The soundness of $f$ is immediate from the value of $\sid$ and $\mathsf{PRF}$ being a deterministic function.
In fact,
$f$ has perfect soundness and is also a local partner function
(\cref{def:local_partnering}).
This will be important when we look at IEEE~802.11 combined with upper-layer authentication in \cref{sec:802.11:upper-layer_authentication}. 
%\todo{HJ: true? CB: I don't think so. We have to use sid soundness as an ingredient here. CB: I was wrong. Yes, it is true.}


\setcounter{gamehop}{0}
\paragraph{Game~\game:}
This is the real 2P-\akenfstext security game, hence
\begin{equation*}
	\adv_{\mathsf{4WHS},f}^{\mathsf{G_\game}}(\A) = \adv_{\mathsf{4WHS},f}^{\operatorname{\mathsf{2P-\akenfs}}}(\A) .
\end{equation*}


\newgame
\paragraph{Game~\game:}\label{proof:4WHS:game:nonce_collisions}
This game proceeds as the previous one,
but aborts if not all the nonces in the game are distinct, hence
\begin{equation}
	\adv_{\mathsf{4WHS},f}^{\mathsf{G_{\prevgame{}}}}(\A) 
	\leq \adv_{\mathsf{4WHS},f}^{\mathsf{G_\game}}(\A) 
	+ \frac{(\numparties \numsessions)^2}{2^{\noncelen+1}}  .
\end{equation} \todo{This does not take into account that AP creates many nonces (one for each received $m_2$ attempt).}

\newgame
\paragraph{Game~\game:}\label{game_hop:4WHS:AKE:selective_sec}
This game implements a selective AKE security game where at the beginning of the game the adversary has to commit to the ESS which the test-session will be connected to.
Specifically,
at the beginning of the game the adversary has to output an SSID and the game aborts if the test-session was not connected to the ESS having this SSID.
%Game~\game{} proceeds as in Game~\prevgame{} except that it aborts if the test-session did not belong to the .
\begin{claim}\label{lemma:4WHS:AKE:selective_security}
\begin{equation}
	\adv_{\mathsf{4WHS},f}^{\mathsf{G_{\prevgame{}}}}(\A) 
		\leq  \numssid \cdot \adv_{\mathsf{4WHS},f}^{\mathsf{G_{\game{}}}}(\A') .
\end{equation}
\end{claim}


\begin{proof}
From an adversary $\A$ that wins against the adaptive game in Game~\prevgame{},
we create an adversary $\A'$ that wins against the selective game in Game~\game{}. 
$\A'$ randomly selects an SSID
(and thus an ESS)
and outputs this as its choice to the selective security game it is playing.
$\A'$ then runs $\A$ and answers all of its queries by forwarding them to its own selective security game. 
If the test-session selected by $\A$ does not belong to the SSID network guessed by $\A'$,
then $\A'$ stops its simulation an outputs a random bit.
Else,
it outputs the same bit as $\A$.
Algorithm $\A'$ perfectly simulates Game~\prevgame{} for $\A$,
and since its guess is correct with probability at least $1 / \numssid$ the \lcnamecref{lemma:4WHS:AKE:selective_security} follows.
\end{proof}

In the remainder of the proof,
let $\ssid^*$ denote the SSID that the adversary commits to in \cref{game_hop:4WHS:AKE:selective_sec},
and let $\PMK^*$ denote the corresponding PMK used in the ESS identified by $\ssid^*$. 
Note that by the requirements of the $\fresh_{\akenfs}$ predicate  
(\cref{fig:freshness:AKE}),
$\PMK^*$ cannot be exposed if the test-session is to be fresh. 
In particular,
this means that the adversary cannot make a $\Corrupt(\ssid^*)$ query.

\newgame
\paragraph{Game~\game:}\label{game_hop:4WHS:AKE:PRF-to-random}
In this game the challenger replaces the pseudorandom function $\mathsf{PRF}$
with a random function $\$(\cdot)$ in all evaluations involving  $\PMK^*$.
That is, calls of the form $\mathsf{PRF}(\PMK^*,\cdot)$ are instead answered by~$\$(\cdot)$.

\begin{claim}\label{lemma:4WHS:AKE:PRF-to-random}
\begin{equation}
	\adv_{\mathsf{4WHS},f}^{\mathsf{G_{\prevgame{}}}}(\A) 
	\leq  \adv_{\mathsf{4WHS},f}^{\mathsf{G_{\game{}}}}(\A)
	+ 2 \cdot \adv_{\mathsf{PRF}}^{\mathsf{prf}}(\A[D]) .
\end{equation}
\end{claim}

\begin{proof}
If it is possible to distinguish between Game~\prevgame{} and Game~\game{},
then we can create a distinguisher algorithm $\A[D]$ against the PRF security of the function $\mathsf{PRF}$.
Algorithm $\A[D]$ has access to an oracle $\mathcal{O}$,
which either implements the function $\mathsf{PRF}(\widetilde{\PMK}, \cdot)$ for some independent and uniformly distributed key $\widetilde{\PMK}$,
or it implements a truly random function $\$(\cdot)$.
Algorithm $\A[D]$ begins by choosing a random bit $\bsim$ and creating all the PMKs for all ESSs different from the one identified by $\ssid^*$.
It then runs $\A$,
answering its queries as follows.

For all of $\A$'s queries that do not involve the computations of $\PMK^*$,
$\A[D]$ answers itself using the $\PMK$s it created.
On the other hand,
for queries that would normally involve computations of $\PMK^*$,
algorithm $\A[D]$ instead uses its oracle $\mathcal{O}$ to do these computations.
Finally,
when $\A$ stops with some output~$b'$,
then $\A[D]$ stops and outputs~$0$ to its PRF security game if $b' = \bsim$,
and outputs $1$ otherwise.

When $\mathcal{O} = \mathsf{PRF}(\widetilde{\PMK}, \cdot)$
then $\A[D]$ perfectly simulates Game~\prevgame{} since all the $\PMK$s are chosen independently and uniformly at random;
while when $\mathcal{O} = \$(\cdot)$, then $\A[D]$ perfectly simulates Game~\game{}.
The \lcnamecref{lemma:4WHS:AKE:PRF-to-random} follows.
%\qed
\end{proof}

\paragraph{Concluding the proof of Theorem~\ref{thm:4WHS:2P-AKE}.}
Suppose the test-session in Game~\game{} accepted with the $\sid$ variable set to $P \concat \nonce$.
Since all nonces in the game are unique by Game~\ref{proof:4WHS:game:nonce_collisions},
the only sessions that evaluated the pseudorandom function on $\sid$ was the test-session and possibly its partner.
However, by Game~\ref{game_hop:4WHS:AKE:PRF-to-random} the PRF is now a truly random function,
% which
%provided the test-session remains fresh,
%is unavailable to the adversary provided the test-session is to remain \akenfstext fresh.
so the $\PTK$ derived by the test-session (and possibly its partner) is a truly random string $\widetilde{\PTK} = \widetilde{\key_\mu} \concat \widetilde{\key_\alpha} \gets \bits^{2\keylen}$.
Moreover,
$\widetilde{\key_\alpha}$ is independent of all other values.
Thus,
$\adv_{\mathsf{4WHS},f}^{\mathsf{G_{\game{}}}}(\A) = 0$,
and \cref{thm:4WHS:2P-AKE} follows.
%\qed
\end{proof} %% End Theorem on 4WHS









\subsection{Explicit entity authentication}\label{sec:802.11:4WHS:EA}
We now prove that the 4WHS protocol additionally provides explicit entity authentication.
The proof of this fact follows the same outline as for the key-indistinguishability part of the proof,
using essentially the same game hops.
However,
instead of bounding the key-indistinguishability advantage of the adversary in the final game,
we instead bound the probability that a session will accept maliciously.
Intuitively,
we can translate this event into a forgery for the MAC algorithm $\mac$ since the adversary will either have to forge an $m_2$ message to the access point or an $m_3$ message to the client in order for a malicious accept to happen.


Alas,
the proof is complicated by the aforementioned error handling semantics of the 4WHS protocol.
That is,
when a session receives a bad message it silently discards it and continues running the protocol instead of immediately rejecting.
This means that the adversary can make many attempts at getting an access point to accept an $m_2$ message or a client to accept an $m_3$ message.
The first case is especially subtle to deal with since the access point will derive a new PTK for each received $m_2$ message.
%We do this by reducing to the SUF-CMA security of the MAC~$\mac$ 
%(see \cref{sec:other_definitions} for a formal definition of the SUF-CMA notion).
To better align our reduction to the possibility of an adversary making many attempts at the $m_2$ and $m_3$ messages,
we reduce to a variant of SUF-CMA security that allows multiple verification attempts;
see \cref{sec:other_definitions:MAC} for the formal definition.

While single-verification and multi-verification are not equivalent in the traditional UF-CMA setting,
they \emph{are} equivalent in the stronger SUF-CMA setting;
see~\cite{EPRINT:BelGolMit04}.
Moreover,
for message authentication \emph{codes}---as opposed to message authentication schemes in general---UF-CMA 
security implies SUF-CMA security.
Since the IEEE 802.11 4WHS protocol only uses MACs and not general message authentication schemes,
the multi-verification SUF-CMA assumption is justified provided the MAC scheme is UF-CMA secure.
%More specifically, 
%the MACs standardized for IEEE 802.11 are HMAC~\cite{IETF:RFC2104:HMAC} and CMAC~\cite{FIPS:SP-800-38B:CMAC}.\todo{HJ: add citation to their proofs of security.}
The security of the HMAC~\cite{IETF:RFC2104:HMAC} algorithm and the CMAC~\cite{FIPS:SP-800-38B:CMAC} 
algorithm used by the IEEE~802.11 standard is well-studied;
see \cite{C:GazPieRyb14,JC:Bellare15,FSE:IwaKur03,INDOCRYPT:IwaKur03}.




%\small
\begin{theorem}\label{thm:4WHS:EA}
For any adversary $\A$ in security experiment \akenfstext-EA against the 4WHS,
we can create algorithms $\A[D]$ and $\A[F]$, such that
%\small
\begin{align}
	\begin{split}
	\adv_{\mathsf{4WHS},f}^{\operatorname{\mathsf{2P-\akenfs-\EA}}}(\A)  
		&\leq 2 \cdot \numssid \cdot  \adv_{\mathsf{PRF}}^{\prf}(\A[D]) 
			+ \frac{(\numparties  \numsessions)^2}{2^{\noncelen+1}} \\
		&\qquad + 2 \numsessionsess \cdot  (q + 1) \cdot \numssid \cdot  \adv^{\sufcma}_{\mac}(\A[F]) ,
	\end{split}
\end{align}
\normalsize
%where $\noncelen$ is the length of the nonces,
%%$\numparties$ is the number of parties, 
%$\numsessions$ is the number of sessions at each party,
%and $\numparties = |\mathcal{P}_C | + |\mathcal{P}_{AP}|$.	
where $f$, $\numssid$, $\numsessions$, $\numparties$ and $\noncelen$ are the same as in \cref{thm:4WHS:2P-AKE},
and where  $\numsessionsess$ is the maximum number of sessions $\A$ creates in each ESS,
and $q$ is the maximum number of $m_2$ messages that $\A$ sends to an access point session.
\end{theorem}
\normalsize

\begin{proof}
The initial part of the proof proceeds through  three game hops that are completely analogous to the first three game hops of the proof of \cref{thm:4WHS:2P-AKE}. 


\setcounter{gamehop}{0}
\paragraph{Game~\game:}
This is the real explicit entity authentication security game,
hence
\begin{equation*}
	\adv_{\mathsf{4WHS},f}^{\mathsf{G_\game}}(\A) = \adv_{\mathsf{4WHS},f}^{\operatorname{\mathsf{2P-\akenfs-\EA}}}(\A) .
\end{equation*}

\newgame
\paragraph{Game~\game:}\label{game_hop:4WHS:EA:nonce_collisions}
This game proceeds as the previous one,
but aborts if not all the nonces in the game are distinct, hence
\begin{equation}
	\adv_{\mathsf{4WHS},f}^{\mathsf{G_{\prevgame{}}}}(\A) 
	\leq \adv_{\mathsf{4WHS},f}^{\mathsf{G_\game}}(\A) 
	+ \frac{(\numparties \numsessions)^2}{2^{\noncelen+1}} .
\end{equation} 


\newgame
\paragraph{Game~\game:}\label{game_hop:4WHS:EA:game:guess_SSID}
This game implements a selective security game where at the beginning of the game the adversary has to commit to the ESS which the first session that accepts maliciously connects to.
Just like for Game~\ref{game_hop:4WHS:AKE:selective_sec} of \cref{thm:4WHS:2P-AKE} (\cref{lemma:4WHS:AKE:selective_security}),
we have
\begin{equation}
	\adv_{\mathsf{4WHS},f}^{\mathsf{G_{\prevgame{}}}}(\A) 
		\leq  \numssid \cdot \adv_{\mathsf{4WHS},f}^{\mathsf{G_{\game{}}}}(\A')  .
\end{equation}


In the remainder of the proof,
let $\ssid^*$ denote the SSID that the adversary commits to in \cref{game_hop:4WHS:AKE:selective_sec},
let $\ess^*$ denote the ESS identified by $\ssid^*$,
and let $\PMK^*$ denote the corresponding PMK used in $\ess^*$.


\newgame
\paragraph{Game~\game:}\label{game_hop:4WHS:EA:PRF-to-random}
In this game the challenger replaces the pseudorandom function $\mathsf{PRF}$
with a random function $\$(\cdot)$ in all evaluations involving $\PMK^*$.
That is, calls of the form $\mathsf{PRF}(\PMK^*,\cdot)$ are instead answered by~$\$(\cdot)$.
By the same arguments as for Game~\ref{game_hop:4WHS:AKE:PRF-to-random} of \cref{thm:4WHS:2P-AKE} (\cref{lemma:4WHS:AKE:PRF-to-random}),
we have
\begin{equation}
	\adv_{\mathsf{4WHS},f}^{\mathsf{G_{\prevgame{}}}}(\A) 
		\leq  \adv_{\mathsf{4WHS},f}^{\mathsf{G_{\game{}}}}(\A) 
			+ 2\cdot \adv_{\mathsf{PRF}}^{\prf}(\A[D]) .
\end{equation}


In the next game the adversary additionally has to commit to the session (in $\ess^*$) that will accept maliciously first.


\newgame
\paragraph{Game~\game:}\label{game_hop:4WHS:EA:game:guess_session}
This game implements a selective security game where at the beginning of the game the adversary has to commit to the session in $\ess^*$ which accepts maliciously first.
With the same type of reduction as for \cref{game_hop:4WHS:EA:game:guess_SSID},
we have
\begin{equation}
	\adv_{\mathsf{4WHS},f}^{\mathsf{G_{\prevgame{}}}}(\A) 
		\leq  \numsessionsess \cdot \adv_{\mathsf{4WHS},f}^{\mathsf{G_{\game{}}}}(\A')  .
\end{equation}

%\todo{Discuss bound on $\numsessionsess$}

In the following let $\eatarget$ denote the session that $\A$ commits to in \cref{game_hop:4WHS:EA:game:guess_session}.
We conclude the proof of \cref{thm:4WHS:EA} by showing that if $\A$ gets $\eatarget$ to accept maliciously in Game~\game{},
then we can construct an algorithm $\A[F]$ that creates forgeries for the MAC algorithm $\mac$.

\begin{claim}\label{lemma:4WHS:EA:forgery}
\begin{equation}
	\adv_{\mathsf{4WHS},f}^{\operatorname{\mathsf{G_{\game}-\EA}}}(\A)
	\leq 2  (q + 1) \cdot  \adv^{\sufcma}_{\mac}(\A[F]) .
\end{equation}
\end{claim}

\begin{proof}

From an adversary $\A$ in \cref{game_hop:4WHS:EA:game:guess_session} we create an algorithm $\A[F]$ against the SUF-CMA security of MAC algorithm $\mac$.
The algorithm $\A[F]$ has access to two oracles 
$\mathcal{O}^{\mactag}(\cdot)$ and $\mathcal{O}^{\macvrfy}(\cdot,\cdot)$
which implements the functions $\mac.\mactag(\widetilde{\key},\cdot)$ and $\mac.\macvrfy(\widetilde{\key}, \cdot, \cdot)$ for some independent uniformly distributed key $\widetilde{\key}$.
The idea of $\A[F]$ is to embed the oracles $\mathcal{O}^{\mactag}$ and $\mathcal{O}^{\macvrfy}$ into computations that would normally involve using the $\PTK$ of the target session $\eatarget$.





Algorithm $\A[F]$ begins by drawing a random bit $b_{sim}$
and waits for $\A$ to commit to a pair $(\ssid^*, \eatarget)$ according to \cref{game_hop:4WHS:EA:game:guess_SSID} and \cref{game_hop:4WHS:EA:game:guess_session}.
If $\eatarget$ belongs to a client,
say $U^* = C$,
let $AP$ be its intended peer,
i.e., $\eatarget.\peers = \lbrace C, AP \rbrace$
(remember that even if an ESS potentially contains many clients and access points,
a connection will nonetheless be between two specific parties in ESS).
Conversely,
if $\eatarget$ belongs to an access point $AP$,
let $C$ be the client it wants to talk to.




For sessions not pertaining to network $\ess^*$,
i.e.,
those not using $\PMK^*$ as their long-term key,
$\A[F]$ simulates everything itself by creating their PMKs.
For sessions in $\ess^*$,
$\A[F]$ still mostly simulates everything itself,
but this time by implementing a random function $\$(\cdot)$ rather than the function $\mathsf{PRF}(\PMK^*, \cdot)$.
This can be done using lazy sampling~\cite[Section~4.3]{EPRINT:BelRog04}.
However,
for certain specific computations which we describe below,
$\A[F]$ will embed its MAC oracles $\mathcal{O}^{\mactag}$ and $\mathcal{O}^{\macvrfy}$. 
We split our proof into two cases,
depending on whether $\eatarget$ belongs to the client $C$ or the access point $AP$.






\subparagraph{Case \texorpdfstring{$U^* = C$}{U* = C}.}
Assume that $\eatarget$ belongs to the client $C$.
In this case $\A[F]$ uses its tagging oracle $\mathcal{O}^{\mactag}$ to create the $m_2$ message of $\eatarget$.
Similarly,
when $\eatarget$ receives an $m_3$ message,
then $\A[F]$ uses the oracle $\mathcal{O}^{\macvrfy}$ to verify it.
If the verification succeeds,
then we will argue below that $\A[F]$ has created a forgery in its SUF-CMA security game.
If the verification fails,
then $\A[F]$ discards the message and continues the simulation.

Besides this,
there is one additional place where $\A[B]$ embeds its $\mathcal{O}^{\macvrfy}$ oracle.
Namely,
suppose the nonce $\eatarget$ received in the first handshake message,
say $\nonce_{AP}$, was created by a session $\oracle[AP][j]$ at the access point $AP$.
If $\A$ forwards $\eatarget$'s $m_2$ message back to $\oracle[AP][j]$,
or at least the nonce $\nonce_{C}$ contained in it,
then $\eatarget$ and $\oracle[AP][j]$ will derive the same $\PTK$
since the they will input the same nonces $\nonce_{C}$, $\nonce_{AP}$ to $\$(\cdot)$.
Thus,
in order for the simulation to be consistent,
$\A[F]$ needs to embed the verification oracle $\mathcal{O}^{\macvrfy}$ into $\oracle[AP][j]$'s verification of this $m_2$ message.
If the verification fails,
then $\A[F]$ discards the message and continues the simulation.
Else,
$\A[F]$ aborts with a failure.
The procedure labeled $\simC$ in \cref{fig:proof:4WHS-EA:reduction_to_MAC} on \cpageref{fig:proof:4WHS-EA:reduction_to_MAC} makes this high-level description precise.

\begin{figure}

\small
%\algrenewcommand\alglinenumber[1]{\scriptsize #1:}
%\algrenewcommand\algorithmicindent{6pt}

\begin{adjustbox}{margin*=2ex,width=0.997\textwidth,frame,center}
	\begin{minipage}[t]{0.53\textwidth}
			$\underline{\simC(\eatarget)}$: 
			\begin{algorithmic}[1]
				
				\Statex
			
				\Statex \underline{Initialization}:
				\setcounter{ALG@line}{100}
				\Indent
					\State $\nonce_{C}^*, \nonce_{AP}^* \gets \bot$
					\State $\oracle[AP][*] \gets \bot$
	%				\State $\nonce_{C}^* \gets \bot$
				\EndIndent
				
				\Statex
				
				\Statex \underline{On receiving $m_1$}:\label{code:simC:line:m_1_start}
				\setcounter{ALG@line}{200} 
				\Indent
					\State parse $m_1$ as $(\nonce_{AP}, p_1)$
					\State $\nonce_{AP}^* \gets \nonce_{AP}$
					\If{session $\oracle[AP][j]$ created $\nonce_{AP}$}
						\State $\oracle[AP][*] \gets \oracle[AP][j]$
					\EndIf
					
					\State $\nonce_{C}^* \getsr \bits^\noncelen$
%					\State $\key_\mu \concat \key_\alpha \gets \$(AP, C, \nonce_{AP}^*, \nonce_{C}^*)$
%					\State $\eatarget.\key \gets \key_\alpha$
					\State $m_2 \gets \mathcal{O}^{\mactag}(\nonce_{C}^* \concat p_2)$
%					\State $\eatarget.\sid \gets \minmax(C, AP) \concat \minmax(\nonce_{C}^*, \nonce_{AP})$
					\State send $m_2$ \label{code:simC:line:m_1_end}

				\EndIndent
		
				\Statex
				
				\Statex \underline{On receiving an $m_3$ message}: \label{code:simC:line:m_3_start}
				\setcounter{ALG@line}{300} 
				\Indent
					\State parse $m_3$ as $\nonce_{AP} \concat p_3 \concat \tau$
					\If{$\nonce_{AP} \neq \nonce_{AP}^*$}
						\State discard $m_3$
						\State continue simulation
					\Else
						\State $d \gets \mathcal{O}^{\macvrfy}(\nonce_{AP} \concat p_3, \tau)$ \label{code:simC:line:verify_m_3}
						\If{$d = 1$}
							\State stop simulation \label{code:simC:line:verify_m_3:win}
						\Else
							\State discard $m_3$
							\State continue simulation \label{code:simC:line:m_3_end}
						\EndIf
					\EndIf
				\EndIndent
	
				\Statex
				
				\Statex \underline{On forwarding $\nonce_{C}^*$ to $\oracle[AP][*]$}:  \label{code:simC:line:forward_C_nonce_start}
				\setcounter{ALG@line}{400} 
				\Indent
					\State parse $m_2$ attempt as $\nonce_C^* \concat p_2 \concat \tau$
					\State $d \gets \mathcal{O}^{\macvrfy}(\nonce_{C}^* \concat p_2, \tau)$ \label{code:simC:line:nonce_forwarding:embed_MAC_oracle}
					\If{$d = 1$}
						\State abort with failure \label{code:simC:line:nonce_forwarding:stop_simulation}
					\Else
						\State discard $m_2$
						\State continue simulation \label{code:simC:line:forward_C_nonce_end}
					\EndIf
				\EndIndent	
	
			\end{algorithmic}
	\end{minipage}
	\begin{minipage}[t]{0.5\textwidth}
			$\underline{\simAP(\eatarget)}$: 
			\begin{algorithmic}[1]
				\Statex
			
				\Statex \underline{Initialization}: \label{code:simAP:line:init_start}
				\setcounter{ALG@line}{100}
				\Indent
					\State $\nonce_{AP}^* \gets \bot$
%					\vspace{2pt}
					\State $q^* \getsr [1, q]$ \label{code:simAP:line:init_end}
					\State $\mathsf{distinct} \gets 0$
					\State $\vv{\mathsf{N}} \gets \emptyset$
					\State $\mathsf{Fwd} \gets \emptyset$
	%				\State $\mathsf{uniq}_\nonce \gets 0$
				\EndIndent
					
				\Statex
				
				\Statex \underline{Creating $m_1$}: \label{code:simAP:line:m_1_start}
				\setcounter{ALG@line}{200}
				\Indent
					\State $\nonce_{AP}^* \getsr \bits^\noncelen$
					\State $m_1 \gets \nonce_{AP}^* \concat p_1$
					\State send $m_1$ \label{code:simAP:line:m_1_end}
				\EndIndent
				

				
				
				\Statex
				
				\Statex \underline{On receiving an $m_2$ message}: \label{code:simAP:line:m_2_start}
				\setcounter{ALG@line}{300}
				
				\Indent
					\State parse $m_2$ attempt as $\nonce_C \concat p_2 \concat \tau$
					
					\State
%					\State \textcolor{blue}{\emph{// when receiving the $q^*$-th distinct}}
%					\State \textcolor{blue}{\emph{// nonce, record it in variable $\nonce_{C}^*$}}
					\If{$\nonce_{C} \notin \vv{\mathsf{N}}$}
						\State $\mathsf{distinct} \gets \mathsf{distinct} + 1$
						\State $\vv{\mathsf{N}}[\mathsf{distinct}] \gets  \nonce_{C}$
%						\If{$| N_{\mathsf{distinct}} | = q^*$}
%							\State $\nonce^*_C \gets \nonce_{C}$
%							\If{$\nonce_{C}^* \in N_{\mathsf{fwd}}$}
%								\State stop simulation \label{code:simAP:line:abort_on_nonce_forward}
%							\EndIf
							
%						\EndIf
					\EndIf
					
					\State
					\State $d \gets 0$ \label{code:simAP:line:d_bit_init}
%					\State \codecomment{if $\nonce_{C}$ equals the $q^*$-th distinct nonce}
%					\State \codecomment{and $\nonce_{C}$ was not nonce-forwarded:}
%					\State \codecomment{embed the $\mathcal{O}^{\macvrfy}$ oracle}
										
%					\If{$(\nonce_{C} = \vv{\mathsf{N}}[q^*]) \land (\nonce_{C} \in \mathsf{Fwd})$}
%						\State abort with failure
					\If{$(\nonce_{C} = \vv{\mathsf{N}}[q^*]) \land (\nonce_{C} \notin \mathsf{Fwd})$} \label{code:simAP:line:embed_MAC_oracle:check}
						\State $d \gets \mathcal{O}^{\macvrfy}(\nonce_{C} \concat p_2, \tau)$ \label{code:simAP:line:embed_MAC_oracle}
					\Else
						\State $\key_{\mu} \concat \key_\alpha \gets \$(AP, C, \nonce_{AP}^*, \nonce_{C})$ \label{code:simAP:line:derive_PTK}
						\State $d \gets \mac.\macvrfy(\key_{\mu}, \nonce_{C} \concat p_2, \tau)$ \label{code:simAP:line:local_MAC_check}
					\EndIf
					
					\State
					\If{$d = 1$} \label{code:simAP:line:IFd=1}
						\State stop simulation \label{code:simAP:line:stop_simulation}
					\Else
						\State discard $m_2$
						\State continue simulation \label{code:simAP:line:m_2_end}
					\EndIf
					
				\EndIndent
				
				\Statex
				
				\Statex \underline{On forwarding $\nonce_{AP}^*$ to any $\oracle[C][j]$}: \label{code:simAP:line:forward_AP_nonce_start}
				\setcounter{ALG@line}{400} 
				\Indent
					\State create $\oracle[C][j]$'s response message 
					\State $m_2 \gets \nonce_C \concat p_2 \concat \tau$ as normal  
					\State $\mathsf{Fwd} \gets \mathsf{Fwd} \cup \lbrace \nonce_C \rbrace$ \label{code:simAP:line:forward_AP_nonce_end}
				\EndIndent	
		
	
			\end{algorithmic}
	\end{minipage}
	

\end{adjustbox}

\caption{Description of $\A[F]$'s simulation in the proof of \cref{lemma:4WHS:EA:forgery}.
The simulation depends on whether $U^* = C$ (shown in $\simC$) or $U^* = AP$ (shown in $\simAP$).
The random function that $\A[F]$ implements for key derivation between $AP$ and $C$ is denoted by~$\$(\cdot)$.
In both $\simC$ and $\simAP$ it is assumed that if the parsing of a received message fails,
then the message is silently dropped and the simulation continues;
for clarity this behavior is omitted from the code.}
\label{fig:proof:4WHS-EA:reduction_to_MAC}

\end{figure}



Algorithm $\A[F]$'s simulation of \cref{game_hop:4WHS:EA:game:guess_session} is perfect.
We argue that if $\eatarget$ accepted maliciously,
then $\A[F]$ must also have made a valid forgery in its SUF-CMA game.
By definition,
for $\eatarget$ to have accepted maliciously it must have received an $m_3$ message that verified correctly,
and no session at $AP$ can have set the same $\sid$ as $\eatarget$. 
Since $\A[F]$ uses its $\mathcal{O}^{\macvrfy}$ oracle to verify $m_3$ messages
(\cref{code:simC:line:verify_m_3} in the $\simC$ procedure),
and because $\A[F]$ never asks for a tag on an $m_3$ message to its $\mathcal{O}^{\mactag}$ oracle
(since $p_2 \neq p_3$);
this implies that whenever $\eatarget$ accepts maliciously,
then $\A[F]$ creates a valid forgery in its SUF-CMA game
(\cref{code:simC:line:verify_m_3:win} in $\simC$).

Note that $\A[F]$ aborts with failure at \cref{code:simC:line:nonce_forwarding:stop_simulation} in $\simC$ only if  $\A$ failed.
That is,
$\A[F]$ embeds its $\mathcal{O}^{\macvrfy}$ oracle also when verifying $m_2$ messages delivered to $\oracle[AP][*] = \oracle[AP][j]$ that contains $\eatarget$'s nonce $\nonce_{C}^*$ 
(\cref{code:simC:line:nonce_forwarding:embed_MAC_oracle} in the $\simC$ procedure).
However,
if such a verification were to succeed,
then $\oracle[AP][j]$ would accept and store the nonces $\nonce_{C}^*$, $\nonce_{AP}^*$ in $\oracle[AP][j].\sid$.
But this would yield the same $\sid$ as that of $\eatarget$.
So if $\eatarget$ were to accept on receiving an $m_3$ message,
$\eatarget$ and $\oracle[AP][j]$ would be partners,
contradicting the fact that $\eatarget$ was supposed to accept maliciously.









\subparagraph{Case \texorpdfstring{$U^* = AP$}{U* = AP}.}
Now suppose $\eatarget$ belongs to the access point $AP$.
For $\eatarget$ to accept maliciously,
it must receive some $m_2$ message that verifies correctly.
Again,
$\A[F]$ will embed its MAC oracles into some of $\eatarget$'s computations,
in particular the $\mathcal{O}^{\macvrfy}$ oracle.
However,
$\A[F]$ cannot verify \emph{every} $m_2$ message with $\mathcal{O}^{\macvrfy}$.
The problem is that $\A$ might forward $\eatarget$'s initial message,
containing the nonce $\nonce_{AP}$,
to \emph{multiple} sessions at $C$.
Since these sessions will all generate their own unique nonces $\nonce_{C}$,
they will also derive distinct\footnote{At 
least with high probability.
} 
$\PTK$s which they use to create their $m_2$ messages.
But the MAC oracles only represent a \emph{single} key $\key_{\alpha}$,
so it would not be correct to embed the $\mathcal{O}^{\macvrfy}$ oracle to verify \emph{all} these $m_2$ messages.
Moreover,
some $m_2$ messages may not even have been created by sessions at client $C$ at all because $\A$ could have forged the nonces itself or taken them from sessions at other clients.
So which $m_2$ messages should be verified with $\mathcal{O}^{\macvrfy}$?

Since each nonce $\nonce_{C}$ combined with $\eatarget$'s own nonce $\nonce_{AP}$ determines a single PTK,
$\A[F]$ must guess one nonce and use its $\mathcal{O}^{\macvrfy}$ oracle to verify all $m_2$ messages that contain this nonce. 
For all other $m_2$ messages,
$\A[F]$ will instead derive a $\PTK$ using $\$(\cdot)$, 
and use the MAC algorithm $\mac.\macvrfy$ to ``locally'' verify the message without calling $\mathcal{O}^{\macvrfy}$.
This strategy is described in the $\simAP$ procedure shown in \cref{fig:proof:4WHS-EA:reduction_to_MAC}.  

In $\simAP$, 
the value $q$ represents the maximum number of \emph{unique} nonces that $\A$ will ever send to an access point session.
It is upper bounded by the number of $\Send$ queries made by $\A$.
Algorithm
$\A[F]$ makes a guess $q^* \getsr [1, q]$ and hopes that $m_2$ messages that contain the $q^*$-th unique nonce will lead $\eatarget$ to accept maliciously. 
We emphasize that this does not mean that $\eatarget$ must necessarily accept maliciously after receiving  the $q^*$-th $m_2$ message in total
(since $\A$ could make repeated attempts with some of the earlier nonces first); 
nor does it mean that $\eatarget$ must necessarily accept maliciously after receiving an $m_2$ message containing the $q^*$-th unique nonce for the \emph{first} time
(since $\A$ can make many $m_2$ attempts with this particular nonce). 

The counter $\mathsf{distinct}$ is used to keep track of how many unique nonces $\eatarget$ have received so far.
The array $\vv{\mathsf{N}}$ stores all the distinct nonces. 
In particular,
the nonce in $\vv{\mathsf{N}}[q^*]$ is the one for which $\A[F]$ will embed the $\mathcal{O}^{\macvrfy}$  oracle.
Additionally,
$\A[F]$ maintains a list $\mathsf{Fwd}$ which is used to record situations where $\eatarget$ cannot accept malicously
(discussed below).
The variable $\nonce_{AP}^*$ stores the nonce created by $\eatarget$.

We first argue that $\A[F]$ perfectly simulates \cref{game_hop:4WHS:EA:game:guess_session}.
%We only need to focus on the verification of the $m_2$ messages.
Looking at the $\simAP$ procedure,
it is clear that we only need to focus on the verification of $m_2$ messages.
If $\eatarget$ receives an $m_2$ message which contains a nonce $\nonce_{C}$ which is different from the $q^*$-th nonce $\vv{\mathsf{N}}[q^*]$,
or if $\nonce_{C}$ has been forwarded from a session at $C$ which first received $\eatarget$'s nonce $\nonce_{AP}^*$
(meaning $\nonce_{C} \in  \mathsf{Fwd}$),
then $\A[F]$ derives the $\PTK$ itself and verifies with the MAC algorithm $\mac.\macvrfy$ 
(\cref{code:simAP:line:derive_PTK} and \cref{code:simAP:line:local_MAC_check} in $\simAP$).
This gives a correct simulation.


For the remaining $m_2$ messages,
$\nonce_{C}$ is equal to the $q^*$-th unique nonce and $\nonce_{C} \notin \mathsf{Fwd}$,
so $\A[F]$ embeds its $\mathcal{O}^{\macvrfy}$ oracle
(\cref{code:simAP:line:embed_MAC_oracle}).
The condition $\nonce_{C} \notin \mathsf{Fwd}$ implies that no session at $C$ have input both $\nonce_{AP}^*$ and $\nonce_{C}$ to $\$(\cdot)$.
This implies that the MAC keys used by the sessions at $C$ are independent from the MAC key
(if any) used to produce these specific $m_2$ messages.
Consequently,
using oracle $\mathcal{O}^{\macvrfy}$ to verify these $m_2$ messages lead to  answers that are identically distributed to those one would get if $\A[F]$ derived $\PTK$ from $\$(\cdot)$ itself and verified ``locally'' with $\mac.\macvrfy$.

It remains to analyze $\A[F]$'s probability of  making a valid forgery in its SUF-CMA game whenever $\eatarget$ accepts maliciously in \cref{game_hop:4WHS:EA:game:guess_session}.
For $\eatarget$ to have accepted maliciously it must have successfully verified an $m_2$ message,
so at some point we must have had $d = 1$ in $\simAP$
(\cref{code:simAP:line:IFd=1}).
Moreover,
by the same arguments as for the ``abort with failure'' condition in the $\simC$ procedure,
the nonce $\nonce_{C}$ which $\eatarget$ accepted on cannot have been produced by a session $\oracle[C][j]$ which received $\eatarget$'s nonce $\nonce_{AP}^*$.
In other words,
we must have $\nonce_{C} \notin \mathsf{Fwd}$.

Thus, 
if $\A[F]$'s guess of $q^*$ was correct,
$\A[F]$ will have used the oracle $\mathcal{O}^{\macvrfy}$ to verify the $m_2 = \nonce_{C} \concat p_2 \concat \tau$ message on which $\eatarget$ accepted maliciously.
Furthermore,
since $\A[F]$ never makes a query to its tagging oracle $\mathcal{O}^{\mactag}$,
if it happens that $\mathcal{O}^{\macvrfy}(\nonce_{C} \concat p_2, \tau) = 1$ then this must necessarily be a valid forgery in the SUF-CMA experiment.
On the other hand,
if $\A[F]$'s guess of $q^*$ was wrong,
then it will not have used $\mathcal{O}^{\macvrfy}$ to calculate $d$,
in which case it clearly does not win in its SUF-CMA game.

Since $\A[F]$'s simulation of \cref{game_hop:4WHS:EA:game:guess_session} is perfect,
and since $q^*$ was drawn independently of $\A$,
$\A[F]$'s guess was correct with probability $1 / q$.
Hence $\A[F]$ winning probability in its SUF-CMA experiment is at least $1 / q$ times  $\A$'s winning probability in Game~\game{}.
 


\subparagraph{Concluding the proof of \cref{lemma:4WHS:EA:forgery}.}

Up to a factor of $1 / q$,
we see that $\A[F]$ successfully forges whenever $\eatarget$ accepts maliciously in \cref{game_hop:4WHS:EA:game:guess_session} regardless of whether $U^* = C$ or $U^* = AP$.
This proves \cref{lemma:4WHS:EA:forgery}. 
\end{proof}


\paragraph{Concluding the proof of Theorem~\ref{thm:4WHS:EA}.}

Combining all the bounds from Game~0 to Game~\ref{game_hop:4WHS:EA:game:guess_session} with \cref{lemma:4WHS:EA:forgery} yields \cref{thm:4WHS:EA}.
\end{proof} %% End of 4WHS EA
 


%\begin{remark}
%It is interesting to note that the proof of \cref{lemma:4WHS:EA:forgery} crucially relies on the fact that parties only implement one role.
%In fact,
%if a party could play both the role of client and access point using the same $\PMK$,
%then the 4WHS would be vulnerable to a simple reflection attack;
%see \cite[Section~4.4]{NDSS:HeMit05}.
%\end{remark}

\subsection{Security of IEEE 802.11 with upper-layer authentication}\label{sec:802.11:upper-layer_authentication}

\cref{thm:4WHS:2P-AKE} and \cref{thm:4WHS:EA} apply to the WPA2-PSK variant of IEEE~802.11.
To also address the security of IEEE~802.11 in its WPA2-Enterprise variant,
we need to analyze the setting where the PMK is provided by some upper-layer authentication protocol.
Technically,
IEEE~802.11 can be combined with any type of upper-layer authentication protocol,
but in practice the \emph{de facto} standard is EAP.
Consequently,
our second composition result from \cref{sec:generic_composition_results} can immediately be applied to obtain a result on IEEE~802.11 in its  WPA2-Enterprise variant as well.

More precisely,
by setting  $\protocol_3 = \text{basic EAP}$ and $\protocol_4 = \operatorname{4WHS}$ in \cref{thm:protocol_5:3P-AKE}, 
we get that the combination $\protocol_5 = \text{EAP} + \text{4WHS}$ is a secure 3P-AKE protocol in our strongest security model \akefstext.
However,
technically speaking,
in order to apply \cref{thm:protocol_5:3P-AKE} we also need to show that the probability that two sessions end up with the same local transcript in the 4WHS protocol is small.
Fortunately,
this is trivial since each side in the 4WHS protocol creates a random 256 bit nonce.
In detail,
the function $\transcriptcollisionfunc$ required by \cref{thm:protocol_5:3P-AKE} is in the 4WHS bounded by the probability of a nonce collision among the sessions at a specific party,
namely
\begin{equation}
	\transcriptcollisionfunc  
		\leq \frac{| \Inits \cup \Responders | \cdot  \numsessions^2}{2^\noncelen} 
%		= \frac{| \Parties_{C} \cup \Parties_{AP} | \cdot  \numsessions^2}{2^\noncelen} 
		= \frac{\numparties \cdot  \numsessions^2}{2^{256}} .
\end{equation}
Note that the bound is proportional to $\numparties \cdot  \numsessions^2$ and not $(\numparties \cdot \numsessions)^2$,
since the collision needs to happen at a specific party.
 







\section{Analyzing CCMP}\label{sec:802.11:CCMP}

While this \lcnamecref{sec:802.11} is primarily about the AKE security of the 4WHS key exchange protocol,
for completeness we also include an analysis of the CCMP algorithm used to protect the IEEE~802.11 application data.
CCMP is a stateful authenticated encryption (stAE) scheme built out of the CCM mode of operation~\cite{IETF:RFC3610:CCM} using AES as its underlying block cipher.
Since CCMP is only defined within the context of IEEE~802.11,
we specialize all of our descriptions to this setting,
including that of CCM.





\subsection{Description of CCMP}



An IEEE~802.11 frame consists of a \emph{header} $A = A_1 \concat A_2 \concat \dotsb \concat A_r$ which will be integrity protected but not encrypted,
and a \emph{plaintext} message $P = P_1 \concat P_2 \concat \dotsb \concat P_s$ which will be both integrity protected and encrypted.
Each block of $A$ and $P$ is 128 bits,
except possibly for the final blocks $A_r$, $P_s$ which might be shorter.

\begin{figure}
	\centerline{
		\hspace{-0.7em}\includestandalone[width=0.8\textwidth,mode=build]{tikz_CCM}
	}
	
	\caption{The CCM mode of operation.}
	\label{fig:CCM}
\end{figure}



\paragraph{CCM.}
The CCM mode of operation is shown in \cref{fig:CCM} with one header block $A_1$ and two plaintext blocks $P_1 \concat P_2$.
CCM combines a CBC-MAC with CTR mode encryption in the style of MAC-then-encrypt,
and can be summarized as follows.
On input a key $\Key$,
a message $A \concat P$,
and a 104 bit nonce $N$;
CCM first derives the initial value $IV$ needed by CBC-MAC, and the initial counter value $\ctr$ needed by CTR mode,
from the nonce $N$ and two distinct 8~bit flags $\flags_1$ and $\flags_2$.
Then, CBC-MAC is computed over the whole message $A \concat P$ to produce a tag $T$.
Next, the plaintext message $P$ is encrypted using CTR mode to produce a ciphertext $C_P$.
Finally,
the tag $T$ is encrypted with a single counter block to produce a ciphertext $C_T$.
The combination $C \gets C_P \concat C_T$ is the output of CCM.
Decryption works in the obvious manner.


\begin{figure}
%\algrenewcommand\alglinenumber[1]{\scriptsize #1:}
%\algrenewcommand\algorithmicindent{6pt}
\begin{adjustbox}{margin*=2ex,width=0.9\textwidth,frame,center}

	\begin{minipage}[t]{0.48\textwidth}
	
		$\underline{\ccmp.\staeenc(\Key, P, A)}$: 
		\begin{algorithmic}[1]
			
			\State $\sent \gets \sent + 1$
			
			\State $\widehat{U} \gets \addrextract(A)$
			
			\State $N \gets \flags \concat \widehat{U} \concat \sent $
			
			\State 
			
			\State $C \gets \ccm.\staeenc(\Key, N, P, A)$
			
			\State 
			\State \Return $(\sent, C)$
	
		\end{algorithmic}
	\end{minipage}
	
	\begin{minipage}[t]{0.42\textwidth}		
		$\underline{\ccmp.\staedec(\Key, \sent \concat C, A)}$: 
		\begin{algorithmic}[1]
			
%			\State $(\sent, A, C) \gets F^{-1}(Z)$
			\State $\widehat{U} \gets \addrextract(A)$
			\State $N \gets \flags \concat \widehat{U} \concat \sent$
			
			\State
			\State $P \gets \ccm.\staedec(\Key, N, C, A)$ \label{alg:CCMP:line:CCM_decrypt}
			
			\State
			\If{$P = \bot$}
				\State \Return $\bot$
			\EndIf
			
			\State
			\If{$\sent \leq \received$} \label{alg:CCMP:line:check_replay}
				\State \Return $\bot$
			\EndIf
			
			\State
			\State $\received \gets \sent$
			
			\State 
			\State \Return $P$

		\end{algorithmic}

	\end{minipage}
\end{adjustbox}

\caption{The CCMP encryption and decryption procedures.}
\label{fig:CCMP}
\end{figure}


\paragraph{CCMP.}
The CCMP encrypt and decrypt procedures are shown in \cref{fig:CCMP}.
The two main responsibilities of CCMP are to create the nonce $N$ that will be used as input to CCM,
and to ensure replay protection for the IEEE~802.11 frames.
CCMP achieves both by maintaining a 48~bit counter $\sent$,
which is incremented for each sent IEEE~802.11 frame;
and a 48~bit counter $\received$,
which is (potentially) updated for each received IEEE~802.11 frame.\footnote{The
$\sent$ counter is called the \emph{packet number} in the IEEE~802.11 standard~\cite{IEEE:2012:802.11},
while the $\received$ counter is called the \emph{replay counter}.
}
The $\sent$ counter is initialized to 1 and the $\received$ counter is initialized to~0.




In order to encrypt an IEEE~802.11 frame consisting of a header $A$ and a~plaintext $P$,
CCMP first increments the $\sent$ counter and creates the 104 bit nonce $N$ as
\begin{equation}\label{eq:802.11:CCMP:nonce_creation}
	N \gets \flags \concat \widehat{U} \concat \sent ,
\end{equation}
where $\widehat{U}$ is the 48 bit link-layer address of the sender,
and $\flags$ is an 8 bit value encoding various IEEE~802.11 settings.
We treat it as a constant.
The link-layer address $\widehat{U}$ is always part of the header $A$,
so in \cref{fig:CCMP} we use a function $\addrextract$ to indicate the process of extracting $\widehat{U}$ from $A$.
Given the nonce $N$,
CCMP then encrypts the IEEE~802.11 frame consisting of header data $A$ and plaintext $P$ using the CCM mode of operation to produce the ciphertext $C$.
The output of CCMP is the concatenation $\sent \concat C$.

Remember that CCM will add additional elements to the nonce $N$ in order to create the CBC-MAC IV and the CTR mode initial counter as indicated in \cref{fig:CCM}.
Particularly,
the IV and initial counter for CCM when used in the context of CCMP are the following two 128 bit values.
\begin{align}
	IV &\gets \flags_1 \concat \flags \concat \widehat{U} \concat \sent \concat \mathsf{length}_{16}(A + P) \\
	\ctr &\gets \flags_2 \concat \flags \concat \widehat{U} \concat \sent \concat 0^{16}
\end{align}

Here, $\mathsf{length}_{16}(A + P)$ is the length of $A$ plus $P$ in bytes,
encoded as 16 bits.
Note that 16 bits is sufficient to accommodate the length of the maximum size IEEE~802.11 frame.





To decrypt an IEEE~802.11 frame $Z = (\sent \concat C, A)$,
CCMP first recreates the nonce $N$ from $A$ and $\sent$,
and then decrypts $C$ with CCM.
If the decryption fails,
then CCMP outputs~$\bot$.
Else,
it checks that the value $\sent$ contained in $Z$ is strictly greater than the internally maintained $\received$ counter.
If not,
then CCMP outputs $\bot$ again.
Otherwise,
it updates the value of $\received$ to match that of the received $\sent$,
and returns the plaintext $P$. 





\subsection{Analysis of CCMP}\label{sec:802.11:CCMP:analysis}





Jonsson~\cite{SAC:Jonsson02} has shown that the CCM mode of operation is a secure authenticated encryption scheme.
He proved that CCM satisfies the two separate security notions of indistinguishability under chosen-plaintext attacks (IND-CPA) and integrity of ciphertexts (INT-CTXT);
see~\cite{AC:BelNam00} for their formal definitions.
In \cref{sec:other_definitions:AE},
we show that this is equivalent to our all-in-one definition of AE security.
Thus,
in order to prove the stateful AE security of CCMP,
it is sufficient to reduce to the (stateless) AE security of CCM. 

However,
on closer inspection,
we cannot,
in fact,
prove that CCMP is a secure stAE scheme according to \cref{def:stae:security}. 
The reason is that the security experiment used to define stAE security in \cref{sec:other_definitions:stAE},
targets a different \emph{integrity semantic} than what is provided by CCMP.
Namely,
the security experiment in \cref{fig:stAE_security_experiment} is adapted from \cite{C:JKSS12},
which presented the definition in the context of analyzing TLS.
But the integrity guarantees provided by the TLS Record Layer protocol are stronger than those provided by CCMP.

Specifically,
in terms of integrity,
the TLS Record Layer is supposed to provide protection against: 
\begin{enumerate}
	\item forgeries,
	
	\item replays,
	
	\item reordering, and
	
	\item dropping of messages.
\end{enumerate} 

However,
CCMP does \emph{not} provide protection against messages being dropped.
To see this,
suppose a sender transmits as its first encrypted CCMP messages,
the frames $Z_1$, $Z_2$ and $Z_3$,
thus having corresponding packet numbers $2$, $3$ and $4$
(remember that the $\sent$ counter starts at $1$).
Now suppose an attacker drops messages $Z_1$ and $Z_2$,
but delivers $Z_3$ to the receiver.
Since $Z_3$ is a validly created CCMP frame,
the CCM decryption at \cref{alg:CCMP:line:CCM_decrypt} of the $\ccmp.\dec$ procedure  will succeed.
Moreover,
when the $\ccmp.\dec$ procedure continues to check whether $Z_3$ is a replay at \cref{alg:CCMP:line:check_replay},
then this will also succeed,
since the receiver's $\received$ counter is set to $0$ and so we have $\received < \sent$.
Thus,
$Z_3$ will be accepted by the receiver even though $Z_1$ and $Z_2$ were lost.


%Note that this is not an oversight by the IEEE~802.11 standard,
%but rather a wanted feature.
%IEEE~802.11 is a connectionless protocol which does not provide in-order guarantees of the received frames. 



By contrast,
the TLS Record Layer demands that the decryption process happens in exactly the same order as the ciphertexts were created by the encryption process.
Thus,
it does not accept any messages being dropped.
Other integrity semantics are also possible by considering different combinations of the four properties listed above.
Boyd et al.~\cite{RSA:BHMS16} have analyzed this question in depth.
 

To summarize,
CCMP cannot be proven secure according to the stAE security definition given in \cref{sec:other_definitions:stAE} because it implies a \emph{stronger} integrity semantics than what CCMP achieves.
Consequently,
we have to consider a \emph{weakening} of the stAE model that allows for messages to be dropped.
%In order to make the model \emph{weaker},
%we have to make the test that decides whether a message is considered out-of-sync in the $\staedec$ oracle in \cref{fig:stAE_security_experiment} \emph{stricter}.
Particularly,
at  \cref{alg:stAE:experiment:Dec:check_synch} of the $\staedec$ oracle in \cref{fig:stAE_security_experiment},
we change the condition  from
%\mbox{``\textbf{if} $(C,A) \neq S[\received]$\textbf{:} \dots''} to instead read:
\begin{quote}
	\hspace*{\algorithmicindent}\textbf{if} $(C,A) \neq S[\received]$\textbf{:}  \hspace*{1cm} \phantom{to} \hspace*{1cm}	\textbf{if} $(C,A) \notin S[\received \dotso \sent]$\textbf{:}
	
	\hspace*{\algorithmicindent}\hspace*{0.3cm} $\phase \gets \FALSE$ \hspace*{1cm} \raisebox{0.5\baselineskip}[0pt]{to} \hspace*{0.7cm}	
	\hspace*{\algorithmicindent} $\phase \gets \FALSE$
\end{quote}
where $S[i \dotso j] = \lbrace S[i], S[i+1], \dotsc, S[j] \rbrace$ if $i \leq j$,
and $\emptyset$ otherwise.
Notice that this widens the scope of which messages are considered in-sync,
from one message $S[\received]$,
to potentially a large range $S[\received \dotso \sent]$.
Hence, this effectively weakens the model since it restricts the adversary more. 


Let $\adv_{\protocol}^{\staesecd}(\A)$ denote the stAE advantage of an adversary $\A$ against some stAE scheme $\protocol$ within this weakened model.
We then have the following result.


\begin{theorem}
Let $\A$ be an adversary against the stAE security of CCMP,
then we can create an adversary $\A[B]$ against the AE security of CCM,
such that
\begin{equation}
	\adv_{\ccmp}^{\staesecd}(\A) \leq \adv_{\ccm}^{\aesec}(\A[B]) .
\end{equation}
\end{theorem}

\begin{proof}
From an adversary $\A$ that breaks the stAE security of CCMP where message drops are allowed,
we can construct an algorithm $\A[B]$ that breaks the CCM mode of operation.
Specifically,
algorithm $\A[B]$ creates and maintains the counters $\sent$ and $\received$ of the CCMP scheme,
as well as the variables $\sent$, $\received$, $S[\cdot]$, and $\phase$ of security game $\Exp_{\ccmp}^{\staesecd}(\A)$.
%It doesn't need to simulate the $\received$ counter of the CCMP scheme,
%nor the $\sent$ counter of experiment $\Exp_{\ccmp}^{\staesecd}(\A)$.
We write $\sent_\ccmp$, $\received_\ccmp$ for the counters that $\A[B]$ maintains for the CCMP scheme,
and $\sent_\Exp$, $\received_\Exp$ for the counters that $\A[B]$ maintains for the stAE experiment.


When $\A$ makes an encryption query $(M_0, M_1, A)$,
$\A[B]$ first increments $\sent_\ccmp$ and creates a nonce $N$ from $\sent_\ccmp$ according \cref{eq:802.11:CCMP:nonce_creation}.
It then queries the $\enc$ oracle in its own AE security game on $(N, M_0, M_1, A)$.
$\A[B]$ returns the resulting ciphertext $C$ together with $\sent_\ccmp$ to $\A$,
and updates the variables $\sent_\Exp$ and $S[\sent_\Exp]$ accordingly.

When $\A$ makes a decryption query $(\sent' \concat C, A)$,
$\A[B]$ first increments the value of the counter $\received_\Exp$ by 1 and then proceeds as follows.
\begin{itemize}
	\item If $(\sent' \concat C, A) \in S[\received_\Exp \dotso \sent']$,
	then $\A[B]$
%	updates the value of $\received_\ccmp$ to $\sent'$ and 
	returns $\bot$ to $\A$
	(since this query is in-sync).
	
%	\item Else, if $\sent' \leq \received_\ccmp$, then $\A[B]$ also returns $\bot$
%	(following the description of CCMP).
	
	\item Else,
	$\A[B]$ creates the nonce $N \gets \flags \concat \widehat{U} \concat \sent'$,
	and queries $(N, C, A)$ to the $\dec$ oracle in its own AE security game to produce a message $M$.
	If $M \neq \bot$ then $\A[B]$ stops its simulation and outputs $1$ to its AE security game.
	Otherwise,
	$\A[B]$ returns $\bot$ to $\A$. 
	
\end{itemize}

Finally,
when $\A$ stops with some output $b'$,
then $\A[B]$ stops and outputs the same bit to its AE security game.


Notice that when the secret bit in $\A[B]$'s own AE security game is $0$,
then $\A[B]$ perfectly simulates the ``left-world'' of experiment $\Exp_{\ccmp}^{\staesecd}(\A)$,
i.e., where the encryption query returns the encryption of $M_0$ and the decryption query always returns $\bot$.
This is because in this scenario $\A[B]$'s own AE decryption oracle always returns $\bot$,
and so $\A[B]$ answers all of $\A$'s decryption queries with $\bot$ too.
Moreover,
$\A[B]$'s simulation of $\A$'s encryption queries is perfect no matter what the value of the secret bit in its own AE security game is.
This is because $\A[B]$ properly creates the nonce $N$ from the counter $\sent_\ccmp$,
which it creates and maintains itself.


Thus,
it only remains to argue that $\A[B]$ perfectly simulates the decryption query of the ``right-world'' of experiment $\Exp_{\ccmp}^{\staesecd}(\A)$
when the secret bit in $\A[B]$'s own AE security game is $1$.
To this end,
it is sufficient to note that $\A[B]$ forwards the decryption query to its own AE security game exactly when the query is out-of-sync according to the requirements of game $\Exp_{\ccmp}^{\staesecd}(\A)$.
This is so because $\A[B]$ itself has created and maintains the counter $\received_\Exp$ in accordance with $\Exp_{\ccmp}^{\staesecd}(\A)$.
Moreover,
if $\A[B]$'s own decryption oracle returns something other than $\bot$,
which for instance could happen if $\A$ replays an old ciphertext but with a new counter $\sent'$ such that $\received_\Exp < \sent'$,
then $\A[B]$ stops and immediately wins in its AE security game.\footnote{Note
that aborting early is a tiny optimization which is not strictly necessary in order to bound $\A$'s winning probability in terms of $\A[B]$'s.
Without it,
$\A[B]$ would additionally have to maintain the $\received_\ccmp$ counter and properly check $\sent'$ against $\received_\ccmp$ before answering $\A$.
The benefit of the optimization is that it makes the description of $\A[B]$ simpler,
and can only strictly increase $\A[B]$'s winning chances. 
}

To summarize:
$\A[B]$ perfectly simulates the $\Exp_{\ccmp}^{\staesecd}(\A)$ game for $\A$,
and wins in its own AE security against CCM with at least the same probability as~$\A$.
\end{proof}

Since CCMP maintains a 48 bit counter $\sent$ in order to create the nonces for CCM,
it can technically be used to encrypt up to $2^{48}$ IEEE~802.11 frames.
The maximum allowable IEEE~802.11 frame size is around $2^{13} \text{ bytes} = 8$ kB,
so a total of $2^{48 + 13}$ bytes can be encrypted under the same key.
However,
the CCM security bound proven by Jonsson~\cite{SAC:Jonsson02} includes a ``birthday bound'' term of the form $c \cdot \ell^2 \cdot 2^{-\keylen_b}$,
where $c$ is a small constant,
$\ell$ is the number of invocations of the underlying block cipher,
and $\keylen_b$ is the block length.
In CCMP the block cipher is AES,
so $\keylen_b = 128$.
Moreover,
since CCM makes roughly 2 block cipher calls per input block,
this implies that an IEEE~802.11 frame of $B$ bytes will involve around  $2 \cdot (8B / 128) = B / 8$ block cipher calls.
If we set $c = 1$,
then in order for $\ell^2 \cdot 2^{-\keylen_b}$ to stay below $\epsilon$,
we need $(B / 8)^2 \cdot 2^{-128} \leq \epsilon$.
In other words,
no more than $2^{67} \cdot B^{-1} \cdot \epsilon^{1/2}$ frames of $B$ bytes can be encrypted under the same key.


For example,
if we set the frame size to be $B = 2^{10} \text{ bytes} = 1$~kB,
and we want $\epsilon < 2^{-60}$,
then Jonsson's bound only justifies up to $2^{27}$  IEEE~802.11 frames being encrypted with the same key,
or roughly $2^{37} \approx 137$~GB of data.
Alternatively,
if the requirement is reduced to $\epsilon < 2^{-50}$,
then we get the more tolerable bound of $\approx 4$~TB of data;
while if the requirement is increased to $\epsilon < 2^{-70}$,
we get virtually no guarantees ($\approx 4$~GB of data).




\section{Multi-ciphersuite and negotiation security of IEEE 802.11}\label{sec:802.11:multi-cipher}
\sectionmark{Multi-ciphersuite and negotiation security}

In \cref{sec:802.11:4WHS} we analyzed the 4WHS protocol under the assumption that there is only a single version of it.
However,
as explained in \cref{sec:description:EAP_&_802.11:802.11:security_protocol},
IEEE~802.11 actually supports several different \emph{ciphersuites},
leading to slightly different instances of the 4WHS protocol.
A ciphersuite in IEEE~802.11 essentially determines five things:
\begin{itemize}
	\item Whether to use upper-layer authentication or not.
	
	\item Which KDF to use inside the 4WHS.
	There are two possible options: a PRF based on HMAC-SHA1
	and a PRF based on HMAC-SHA256.
	
	\item Which MAC algorithm to use inside the 4WHS.
	There are three possible options: HMAC-MD5~\cite{IETF:RFC2104:HMAC} (deprecated),
	HMAC-SHA1-128 \cite{IETF:RFC2104:HMAC},
	and AES-CMAC-128~\cite{FIPS:SP-800-38B:CMAC}.
	
	\item Which encryption algorithm to use inside the 4WHS.
	There are two possible options: 
	RC4 (deprecated) and AES Key Wrap~\cite{IETF:RFC3394:AES-key-wrap}.
	
	\item Which encryption algorithm to use for the IEEE~802.11 application data.
	There are two possible options:
	TKIP (deprecated) and CCMP. 
	
\end{itemize}

Although in principle this gives a total number of $2^4 \cdot 3 = 48$ ciphersuites,
the actual number is smaller since some options need to be used together.
The ciphersuite to use is determined by a \emph{negotiation protocol} run prior to the 4WHS.
In particular,
recall from \cref{sec:description:EAP_&_802.11:802.11:security_protocol} that during the association phase of the IEEE~802.11 protocol, the access point will send the client a list of supported ciphersuites $\vcnf$.
From this list the client will chose a single preferred ciphersuite $\cnf_c$ based on some negotiation function $\operatorname{Nego}$.
Leaving out the irrelevant messages from \cref{fig:RSN_establishment},
this gives the simplified protocol shown in \cref{fig:4WHS_negotiation}. 
Compared to \cref{fig:4WHS},
we have made the values of the constants $p_1, \dotsc, p_4$ explicit in order to reflect the presence of ciphersuite-specific information in the 4WHS.  

\begin{figure}
	\centering
	\includestandalone[width=1\textwidth,mode=build]{tikz_4WHS_nego}
	
	
	\caption{Ciphersuite negotiation in the 4WHS protocol.}
	\label{fig:4WHS_negotiation}
\end{figure}

In particular,
in \cref{fig:4WHS_negotiation} the chosen ciphersuite $\cnf_c$ determines a specific KDF $\mathsf{PRF}_c$,
a MAC algorithm $\mac_c = (\mactag_c, \macvrfy_c)$,
a handshake encryption algorithm $\mathcal{E}_c = (\mathcal{E}_c, \mathcal{D}_c)$,
and an application data encryption algorithm $\stae_c = (\staeinit_c, \staeenc_c, \staedec_c)$ (not shown). 
Both the client and the access points repeat their respective messages from the negotiation protocol inside the 4WHS
(integrity protected by the MAC algorithm $\mac_c$).
If the received values $\cnf_c$ and $\vcnf$ are not identical to what the client and access point received during the negotiation protocol,
%respectively $\cnf_C$ and $\vcnf_{AP}$,
they abort the 4WHS.
Note that the access point encrypts its ciphersuite list $\vcnf$ with the encryption algorithm $\mathcal{E}_c$ in the third message of the 4WHS
(which also includes a group key~$\GTK$). 

Given that IEEE~802.11 supports multiple ciphersuites,
there are two related questions to ask.
The first is whether the 4WHS protocol is \emph{multi-ciphersuite secure},
meaning that it is still secure as an AKE protocol when multiple different ciphersuites can be run in parallel.
The second is whether the negotiation protocol in combination with the 4WHS achieves \emph{negotiation security},
meaning that the client and access point end up with the ciphersuite prescribed by their initial configurations.

In the sections below we examine both of these issues and indicate how one could prove the multi-ciphersuite and negotiation security of IEEE~802.11.
We stress that unlike the other sections of this \lcnamecref{sec:802.11},
we do not provide any formal theorem statements or proofs,
but only keep the discussion at an informal level.









\subsection{Multi-ciphersuite security}\label{sec:802.11:multi-cipher:multi-cipher}
If a multi-ciphersuite protocol uses different long-term keys for each ciphersuite,
then the multi-ciphersuite security of the protocol follows trivially from the security of the individual ciphersuites.
%This can be shown by a straightforward reduction that simulates all the other ciphersuites by creating all the long-term keys itself,
However,
if the same long-term key is used across different ciphersuites,
then this does not necessarily have to be the case.
Chatterjee et al.~\cite{INDOCRYPT:ChaMenUst09} give several examples of attacks that are possible when long-term keys are reused across different protocols.
Since the same PSK can be reused across different ciphersuites in IEEE~802.11,
we cannot automatically conclude that the 4WHS achieves multi-ciphersuite security.


\paragraph{Generic composition.}
Seemingly,
a possible approach to proving the multi\-cipher\-suite security of the 4WHS protocol would be to adapt the multi-cipher\-suite framework of Bergsma et al. (BDKSS)~\cite{CCS:BDKSS14}. 
In their framework a multi-cipher\-suite protocol $\protocolneg \concat \vprotocol$
is the composition of a negotiation protocol $\protocolneg$,
followed by a sub-protocol $\protocol_c \in \vprotocol$,
corresponding to specific ciphersuite $c$.
Importantly,
BDKSS do not demand that a different long-term key be used for each sub-protocol $\protocol_c$.
One of the central results of BDKSS is a generic composition theorem that tries to recover as much as possible of the intuition that the security of $\multiprotocol$ should follow from the security of the individual sub-protocols.

The main issue in reducing the security of $\multiprotocol$ to the security of a sub-protocol $\protocol_c$,
is that the reduction also needs to be able to simulate the \emph{other} sub-protocols $\protocol_d$ that use the same long-term key as $\protocol_c$.
BDKSS solve this problem by considering an extension to the single-ciphersuite security game where the adversary additionally gets access to an \emph{auxiliary oracle} that runs a certain operation $\Aux(sk, \cdot)$ based on a private key $sk$.\footnote{Technically,
BDKSS only considered multi-ciphersuite security for ACCE protocols,
but their results can easily be adapted to other protocol types as well.
}
For instance,
if $sk$ was the private key for some signature scheme,
then $\Aux(sk, \cdot)$ could be a signing operation under $sk$.  
The idea of introducing the auxiliary oracle is that it can make it possible to simulate the other ciphersuites which uses the same private key $sk$.
This is the central ingredient of the composition theorem of BDKSS:
if the auxiliary oracle allows simulation of other ciphersuites that use the same private key as $\protocol_c$,
then the multi-ciphersuite security of $\multiprotocol$ can be reduced to the single-ciphersuite security of $\protocol_c$,
in the extended auxiliary oracle model.

Still,
this instead shifts the problem to showing that $\protocol_c$ satisfies single-ciphersuite security in the possibly stronger auxiliary oracle model.
In particular,
if the operation $\Aux(sk, \cdot)$ is very liberal in terms of what function it computes with $sk$,
then it will be easy to simulate the other ciphersuites,
but at the cost of making it more difficult to show security of $\protocol_c$ in the (single-ciphersuite) auxiliary oracle model.
To counteract this,
BDKSS additionally introduced the notion of a \emph{constraint predicate} $\auxpred$.
If the adversary is to win in the extended single-ciphersuite security game,
then it cannot make an auxiliary oracle query that satisfies $\auxpred$.
The stricter $\auxpred$ is,
the easier it becomes to prove the security of $\protocol_c$ in the single-ciphersuite setting.
On the other hand,
a stricter $\auxpred$ could also make it harder to simulate the other ciphersuites,
which is needed by the composition theorem.

Let $\protocolneg$ be the negotiation protocol in \cref{fig:4WHS_negotiation} 
and let $\vprotocol$ be the collection of all the different 4WHS ciphersuite variants.
To prove the multi-ciphersuite security of $\multiprotocol$ using the composition theorem of BDKSS~\cite{CCS:BDKSS14},
we have to come up with a suitable auxiliary oracle and constraint predicate $\auxpred$.
Although BDKSS's original formulation was in the public-key setting,
we can adapt it to the PSK setting in a straightforward manner.
However,
what is not so straightforward is constructing the PSK operation $\Aux(\PMK, \cdot)$ and the corresponding constraint predicate $\auxpred$.
In order for an adversary against a ciphersuite $\protocol_c$ to simulate another ciphersuite $\protocol_d$ sharing the same $\PMK$,
it needs to be able to create the application keys $\key_\alpha$ of $\protocol_d$.
For simplicity,
suppose ciphersuites $c$ and $d$ share the same KDF $\mathsf{PRF}$.
If we let $\Aux(\PMK, x)$ return $\mathsf{PRF}(\PMK, x)$ then we can certainly simulate protocol $\protocol_d$.
The problem is that having access to this operation also makes it trivial to break protocol $\protocol_c$
in the single-ciphersuite setting.
Unfortunately,
there does not seem to be a way to constrain the inputs to $\Aux(\PMK, \cdot)$ using $\auxpred$
which simultaneously  protects $\protocol_c$ and at the same time enables simulation of sub-protocol $\protocol_d$.
This is because the chosen ciphersuite is not cryptographically bound to the derived keys through the KDF input
(in the sense of channel binding).
Hence there is no difference between how the key $\key_{\alpha}$ is derived in $\protocol_c$ and $\protocol_d$.
In the end,
we do not see how the composition theorem of BDKSS~\cite{CCS:BDKSS14} can be used to prove the multi-ciphersuite security of IEEE~802.11.

\paragraph{Agile security.}
An alternative to using the modular composition theorem of BDKSS~\cite{CCS:BDKSS14}
is to prove the multi-ciphersuite security of IEEE~802.11 directly.
Essentially,
a proof of multi-ciphersuite AKE security  or multi-ciphersuite explicit entity authentication would mostly mirror the corresponding single-ciphersuite proofs in \cref{sec:802.11:4WHS:AKE} and \cref{sec:802.11:4WHS:EA}, respectively.
However,
there is one major difference:
the multi-ciphersuite proofs would have to rely on a so-called cryptographic \emph{agility} assumption~\cite{EC:ABBC10,C:BFKPSZ14}.
Cryptographic agility refers to a setting where the same key is being used across multiple members of the same type of function,
e.g., a PRF or a MAC.

In IEEE~802.11, the same $\PMK$ is used in two different PRFs: 
one based on HMAC-SHA1,
and one based on HMAC-SHA256.
A key agility assumption would then say that each of these schemes is secure
(as a PRF),
even when the adversary has oracle access to the other scheme under the same key.
Additionally,
it is also possible that the same MAC key could be used across the different MAC algorithms supported by IEEE~802.11,
namely HMAC-MD5, HMAC-SHA1, and AES-CMAC.
This could happen if an attacker in the negotiation protocol (see~\cref{fig:4WHS_negotiation}) replaced the client's choice of ciphersuite $\cnf_c$ with another ciphersuite $\cnf_c'$ that selects a different MAC algorithm. 
Thus,
we would also need an agility assumption on the MAC security of the collective $\lbrace \operatorname{HMAC-MD5}, \operatorname{HMAC-SHA1}, \operatorname{AES-CMAC}  \rbrace$.


In more detail,
during a multi-ciphersuite security proof of the 4WHS protocol, the PRF agility assumption would be invoked in the game hop where we replace the KDF with a random function
(\cref{game_hop:4WHS:AKE:PRF-to-random} in both \cref{thm:4WHS:2P-AKE} and \cref{thm:4WHS:EA}),
and the MAC agility assumption would be invoked in the analysis of the final game in the proof of explicit entity authentication 
(\cref{lemma:4WHS:EA:forgery} in \cref{thm:4WHS:EA}).
Except for the added assumptions of agility security,
we expect the proofs to be straightforward extensions of those presented in \cref{sec:802.11:4WHS}. 


Incidentally,
the 4WHS protocol is quite similar to the PSK based variant of the IKEv1~\cite{IETF:RFC2409:IKEv1} protocol in ``aggressive'' mode,
which Bhargavan et al.~\cite{SP:BBFGKB16} have conducted a multi-ciphersuite security analysis of.



\subsection{Negotiation security}
Intuitively,
the goal of negotiation security is that no attacker should be able to get two parties to successfully agree upon a worse ciphersuite than the best ciphersuite they mutually support.
If an adversary succeeds in getting the parties to use a worse ciphersuite then what is prescribed by  their mutual configurations,
it is said to have performed a \emph{downgrade} attack.
As mentioned in \cref{sec:description:EAP_&_802.11:802.11:security_protocol},
if WEP is supported alongside Robust Security Network (RSN) ciphersuites,
then IEEE~802.11 cannot provide any protection against downgrade attacks.
This is because when WEP is used,
the 4WHS protocol is not run at all,
and hence there is no way for the client and access point to verify that a downgrade attack has occurred.
The IEEE~802.11 standard~\cite{IEEE:2012:802.11} requires that WEP and RSN should not be enabled together.
In the remainder we only discuss the negotiation security of IEEE~802.11 when using RSN ciphersuites exclusively.

Similar to the modular composition theorem of BDKSS~\cite{CCS:BDKSS14} for the multi-ciphersuite security of a protocol $\multiprotocol$,
Dowling and Stebila~\cite{ACISP:DowSte15} proposed a generic composition theorem that relates the negotiation security of $\multiprotocol$
to the authentication security of the individual sub-protocols $\protocol_c$.
However,
their theorem assumes that different sub-protocols use different long-term keys,
and so cannot be applied to IEEE~802.11.

In contrast,
Bhargavan et al.~\cite{SP:BBFGKB16} formulate negotiation security in the setting of key reuse across ciphersuites.
They also prove a generic theorem that allows the negotiation security of $\multiprotocol$ to be lifted from a simpler \emph{core negotiation protocol} $\protocolneg'$ extracted from $\protocolneg$ and $\vprotocol$.
Thus,
it is sufficient to focus on the negotiation security of protocol $\protocolneg'$.
%Somewhat similar to composition theorem of BDKSS~\cite{CCS:BDKSS14},
%the theorem of Bhargavan et al.~\cite{SP:BBFGKB16} also has a requirement of simulatability  of the full protocol $\multiprotocol$ given $\protocolneg'$,
%but we ignore this issue here.
Using their generic theorem,
Bhargavan et al.~\cite{SP:BBFGKB16} proved the negotiation security of the SSH~\cite{IETF:RFC4253:SSH_transport} protocol under agile assumptions on its cryptographic primitives.

Admittedly,
the value of applying the composition theorem of Bhargavan et al.~\cite{SP:BBFGKB16} to IEEE~802.11 is rather limited,
since the core negotiation protocol one can extract for IEEE~802.11 is almost the same as the whole protocol itself;
essentially corresponding to the protocol we have shown in \cref{fig:4WHS_negotiation}.
A proof of negotiation security for IEEE~802.11 would thus proceed in more or less the same way as the proofs of multi-ciphersuite security---which 
themselves are essentially the same as our proofs of single-ciphersuite AKE security (\cref{thm:4WHS:2P-AKE})
and explicit entity authentication (\cref{thm:4WHS:EA}).
But again,
it would require agile security assumptions on the KDFs and MACs.

 
As mentioned in \cref{sec:802.11:multi-cipher:multi-cipher},
Bhargavan et al.~\cite{SP:BBFGKB16} also analyzed the negotiation security of IKEv1-PSK in aggressive mode,
which is very similar to the 4WHS protocol.
However,
for simplicity they assumed that only a single KDF and a single MAC algorithm was being used in order to rely on more traditional non-agile security assumptions. 





