\chapter*{Abstract}

This thesis presents a computational reduction-based security analysis of the Extensible Authentication Protocol (EAP)
and the IEEE~802.11 protocol.
EAP is a widely used authentication framework while IEEE~802.11 is the most commonly used standard for creating wireless local area networks (WLANs),
better known as Wi-Fi.
The typical use case of EAP is to allow a client on a WLAN to connect to an access point through the use of mutually trusted server.
%Rather than being a single protocol on its own,
EAP is a general framework that specifies how different sub-protocols can be combined to securely achieve this goal.
IEEE~802.11 is usually one of the sub-protocols used within the EAP framework.


There are three main contributions of this thesis.
The first is a modular security analysis of the general EAP framework.
This includes two generic composition theorems that reflect the modular nature of EAP,
and which establish sufficient criteria on its sub-protocols in order for the whole framework to be instantiated securely. 
Having proven the soundness of the general EAP framework,
it remains to find suitable sub-protocols that satisfy the security criteria of the composition results. 

Our second main contribution is a security analysis of one such concrete sub-protocol,
namely the EAP-TLS protocol which is used to establish a shared key between the wireless client and the trusted server.
We prove that EAP-TLS is a secure two-party authenticated key exchange protocol
by presenting a generic compiler that transforms secure channel protocols into secure key exchange protocols.

Our third main contribution is a thorough security analysis of the IEEE 802.11 protocol.
We study both the handshake protocol as well as the encryption algorithm used to protect the application data.
On their own,
our results on IEEE~802.11 apply to the usage found in wireless home networks where a key is shared between the client and access point \emph{a priori},
e.g. using a password.
However,
by combining this with our composition theorems for the EAP framework,
we also obtain a result for IEEE~802.11 in its ``enterprise'' variant,
where the common key is instead established using a mutually trusted server.

