\section{RSN Establishment Procedures}
IEEE~802.11~\cite{IEEE:2012:802.11} is the most widely used standard for creating wireless local area networks (WLANs).
The subset of the standard that deals with security specifies a set of procedures called the \emph{Robust Security Network (RSN)}.
We will use the terms IEEE~802.11 and RSN interchangeably to denote the protocol procedures defined by the RSN specification.
%We will use RSN and IEEE~802.11 interchangeably to denote the  protocol that deals with se 
IEEE~802.11 supports three \emph{modes of operation}
depending on the network topology: \emph{infrastructure mode},
\emph{ad-hoc mode},
and \emph{mesh network mode}.
Infrastructure mode is the most common,
and involves one or more wireless clients connecting to an \emph{access point (AP)} which is connected by wire to a wide-area network (WAN), e.g. the Internet.
All communication in the WLAN goes through the AP. 
In ad-hoc mode and mesh-networking mode there is no central infrastructure.
Wireless clients can talk directly to each other and there is possibly no connection to a larger WAN.
While the RSN security specification covers all three modes,
we will primarily deal with infrastructure mode here.

RSN in infrastructure mode is a three-party protocol involving the wireless client,
the AP,
and an \emph{authentication server (AS)}.
The AP controls access to the network by conferring with the AS.
The AS holds the client's credentials and is responsible for validating and authorizing the user to the network.
This authentication framework is lifted from the IEEE 802.1X standard for port-based access control~\cite{IEEE:2010:802.1X}. 
Using the terminology of IEEE~802.1X,
the wireless client is usually called the \emph{Supplicant},
and the AP is called the \emph{Authenticator}.
The AS might be co-located on the same device as the AP,
or exist as a separate network device.
The complete RSN establishment procedure (in infrastructure mode) consists of six stages and is shown in Fig.~\ref{fig:RSN_establishment}. 

\begin{figure}
	\centerline{\includestandalone[scale=0.7,mode=build]{tikz_RSNA}}
	\caption{The RSN protocol in infrastructure mode. 
	Diagram adapted from~\cite{He:05:securityanalysis80211}.}
	\label{fig:RSN_establishment}
\end{figure}




\paragraph{Stage~1. Network and Security Capability Discovery.} 
In this stage the Supplicant discovers available networks and their security capabilities.
Particularly, the AP will advertise a list of supported cipher suites
%(denoted by ``$\mathrm{ID_A}$ RSN IE'' in message (1) and (3) in Fig.~\ref{fig:RSN_establishment}) 
in so-called Beacon frames transmitted at regular intervals (message (1) in Fig.~\ref{fig:RSN_establishment}),
and also in Probe Response frames sent as answers to Probe Request frames from the Supplicant (message~(3)). 

\paragraph{Stage~2. 802.11 Open System Authentication and Association.}
In this stage the Supplicant selects the AP to which it wants to connect.
This involves doing a procedure called IEEE 802.11 Open Authentication which is a null operation in terms of security.
It is included to maintain backward compatibility with previous IEEE 802.11 specifications.
After this, the Supplicant associates with the AP
(message (6) in Fig.~\ref{fig:RSN_establishment})
% which means that the AP creates a 
%mapping for this client to a specific 802.1X port (used in the next stage).
%In the association stage 
in which it selects the highest cipher suite it supports from the list previously advertised by the AP.
This includes the kind of authentication that will be performed in the next stage.

If the AP and Supplicant already shares a long-term key then they proceed directly to Stage~4 in Fig.~\ref{fig:RSN_establishment},
skipping Stage~3 altogether.
This is called pre-shared key mode, or RSN-PSK,
and is typically used in home-networks where the PSK is derived from a manually configured password.
On the other hand,
if the Supplicant and AP do \emph{not} share a long-term key,
then the AP needs to involve the AS in order to authenticate the Supplicant.
This means running a higher-level authentication protocol between the Supplicant and AS which is described next.

%Note that AP will not let any application data traffic through until it has authenticated the Supplicant. 


\paragraph{Stage~3. EAP/802.1X/RADIUS Authentication.}
%The 802.1X port associated to the Supplicant in Stage~2 is currently blocked until it has successfully completed an IEEE~802.1X authentication procedure.  
%Only 802.1X authentication frames carrying EAP messages, also known as EAP-over-LAN (EAPOL), are allowed to pass through the authenticator. 
%The 802.1X protocol itself does not enforce any particular authentication method,
%but it can be based either on pre-shared keys (RSN-PSK),
%or some upper-level mutual authentication protocol run between the Supplicant and AS.
%In both cases, 
%this stage ends when a secret key called the Pairwise Master Key (PMK) has been configured at both the client and the AP.
%Note, however, that the client and AP has \emph{not} authenticated each other at this point.
%They will use the PMK in the next stage (the \HS{}) to authenticate each other and derive session keys.
%
%When RSN-PSK is used to configure the PMK,
%it is either manually typed into the equipment or, more typically, derived from a password.
%In any case, when preshared keys are used, Stage~3 does not involve any of the transactions shown in Fig.~\ref{fig:RSN_establishment}, 
%but continues immediately to the next stage. 

When upper-level authentication is used then the AP mediates the packets of the higher-level authentication protocol between the Supplicant and the AS by encapsulating them in \emph{EAP (Extensible Authentication Protocol)}~\cite{IETF:RFC3748:EAP} frames 
(message (8)--(14) in Fig.~\ref{fig:RSN_establishment}).
The IEEE~802.11 standard does not specify which upper-level protocol to be used between the Supplicant and the AS,
but it is required that it provides mutual authentication.
Without this requirement, it would be vulnerable to a simple man-in-the-middle attack~\cite{Asokan:2003:MitM-tunneling}.
The most common choice of upper-level authentication is EAP-TLS~\cite{IETF:2008:RFC5216-EAP-TLS}. 
%which we will also assume is the choice for the rest of this paper.

The end result of a successful run of the upper-level authentication protocol between the Supplicant and the AS is a shared key called the \emph{Master Session Key (MSK)}.
The MSK will be delivered from the AS to the AP over a pre-established secure channel (message~(15)).
Again, the standard does not specify which protocol to use between the AP and the AS,
but typically the RADIUS protocol~\cite{IETF:RFC:2865-RADIUS} is used.
A portion of the MSK becomes the so-called \emph{Pairwise Master Key (PMK)},
which will be the basis for the authentication procedure between the Supplicant and the AP in the next stage 
(recall that the Supplicant and AP have \emph{not} authenticated each other at this stage, 
only the Supplicant and the AS have).


%Note that when no upper-level authentication is used (i.e. RSN-PSK),
%then $\PMK=\PSK$.




\paragraph{Stage~4. The 4-Way Handshake (4WHS).}
At the beginning of this stage both the Supplicant and the AP possess a common PMK,
either derived from an upper-level authentication protocol or being their pre-shared key (i.e. RSN-PSK).
In any case,
the PMK is used in the 4WHS to mutually authenticate the Supplicant and AP to each other and to derive a \emph{Pairwise Transient Key (PTK)}.
The $\PTK$ is computed as follows
\begin{align}\label{eq:ptk_derivation_function}
	\PTK := \key_\mu \concat \key_\varepsilon \concat \key_{\alpha} 
		= \mathsf{PRF}(\PMK, & \min \lbrace\ida, \ids\rbrace \conc \max \lbrace \ida, \ids \rbrace \conc \\
			& \min \lbrace \nonce[A], \nonce[S] \rbrace \conc \max \lbrace \nonce[A], \nonce[S] \rbrace) \ , 
\end{align}
where $\mathsf{PRF}$ is a pseudorandom function based on HMAC~\cite{IETF:1997:RFC2104-HMAC},
$A$ and $S$ are the physical addresses of the wireless network cards of the AP and Supplicant,
and $\nonce[A]$, $\nonce[S]$ are the nonces they sent out during the 4WHS
(message (16) and (17) in Fig.~\ref{fig:RSN_establishment}).
The PTK is parsed into three separate keys: 
$\key_\mu$ is used to key a Message Authentication Code (MAC) scheme used  within the \HS{} itself; 
$\key_\varepsilon$ is used to encrypt the group session key distributed in Stage~5;
and finally, 
$\key_\alpha$ is the session key used to protect the data traffic in Stage~6.

%A more detailed description of the \HS{} is shown in Fig.~\ref{fig:4WHS}. 

While the first message of the \HS{} (message (16)) is not cryptographically protected,
the other three messages (message (17)--(19)) are protected by the aforementioned MAC scheme. 
The notation $\overline{m} = [m]_{\key_\mu}$ is used to denote a message $m$ being authenticated (but not encrypted) by the MAC algorithm using key $\key_\mu$.
Similarly, $\lbrace m \rbrace_{\key_\varepsilon}$ denotes the encryption of $m$ under an encryption scheme $\mathcal{E}_\key$ using key $\key_\varepsilon$.

To protect against cipher suite degradation attacks,
the second and third message of the \HS{} also includes the cipher suites that were advertised earlier in the RSN establishment procedures
(message (1), (3) and (6) in Fig.~\ref{fig:RSN_establishment}).


Some additional points about the \HS{} are worth emphasis.
\begin{itemize}
	\item (No forward secrecy) The \HS{} does not provide forward secrecy.
	 Anyone who knows the PMK and captures $\ANonce{}$ and $\SNonce$ can compute the $\PTK$.
	 Additionally, in the case of RSN-PSK, 
	 the PMK is subject to a brute-force attack if the PSK is derived from a password of low entropy.
%	 Consequently, we have to assume that the PSK has been selected with sufficient entropy to prevent brute-force attacks on the $\PMK$.
	 
	\item (Replay protection mechanism) The \HS{} employs a somewhat unusual approach for protecting against replay attacks.
	Instead of explicitly acknowledging a nonce by repeating it in the response message, 
	the \HS{} instead mixes $\ANonce$ and $\SNonce$ into the derivation of the $\PTK$ and replay attacks are detected implicitly by MAC failures. 
	
	The \HS{} also includes an explicit counter value $\seq$, called the Key Replay Counter. 
	This counter serves no cryptographic purpose in the handshake, 
	but is simply used as an optimization technique to discard replayed messages. 

	\item (Accept stage) In Fig.~\ref{fig:RSN_establishment} it is shown that the AP reaches the \emph{accept state} after receiving the second message of the handshake, while the Supplicant accepts after receiving the third handshake message.
	These states do not correspond to any variables defined in the actual IEEE 802.11 specification.
	Rather, they represent the accept stages within our formal analysis of the \HS{} in Section~\ref{sec:4WHS-security}.
%	 and are used in the proof of partnering security. 
	
\end{itemize}
  


\paragraph{Stage~5. Group Key Handshake.}
This is an optional stage, providing a Group Temporal Key (GTK) to all clients associated to this AP. 
The GTK is used to protect broadcast messages sent from the AP to the group.
%For simplicity, the distribution of GTKs is ignored in this paper. 

\paragraph{Stage~6.}
At this stage both authenticator and Supplicant have authenticated each other and installed the session key $\key_\alpha$,
so secure data transfer can start to flow.
%Either TKIP or CCMP can be used, 
%but recent attacks on TKIP~\cite{Tews:2009:PAA, Vanhoef:2013:PVW,paterson:2014:plaintext_recovery_TKIP} has shown that this option does not implement a secure channel.
%On the other hand, CCMP is based on the CCM mode of operation, 
%which has been proved to be IND-CCA secure~\cite{Jonsson:2002:CCM-IND-CCA}.
%For the rest of this paper we will assume that CCMP is being used. 
RSN uses a stateful authenticated encryption (sAE) scheme called the \emph{Counter Mode-CBC MAC Protocol (CCMP)}. 
