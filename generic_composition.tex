\chapter{Security of EAP}\label{sec:generic_composition_results}


\begingroup
\hypersetup{linkcolor=black}
\minitoc
\endgroup

In this \lcnamecref{sec:generic_composition_results} we analyze the security of EAP using the formal models introduced in \cref{sec:definitions}.
However,
first we need to precisely define what we mean by EAP and what type of security properties we expect it to provide.
Recall from \cref{sec:EAP:description} that EAP is not a single protocol
but rather a protocol framework which inherently depends on other concrete protocols.
As summarized in \cref{fig:EAP-architecture},
EAP is essentially a composition of three separate protocols:
an EAP method between the client and the server,
a key-transport protocol between the server and the authenticator,
and a link-layer specific protocol between the client and the authenticator.
The EAP standard~\cite{IETF:RFC3748:EAP} is mostly agnostic as to which concrete protocol to actually use for each of these different parts.

\section{Modeling EAP}\label{sec:generic_composition_results:modeling_EAP}

With respect to security requirements,
the base EAP standard~\cite{IETF:RFC3748:EAP} only specifies what the security properties of the EAP method should be.
However,
a supplementary RFC~\cite{IETF:RFC5247:EAP-key-management} describes the security goals of EAP as a whole.  
Section 1.5 of this document spells out the main requirements:
\begin{quote}
%1.5. Security Goals

	The goal of the EAP conversation is to derive fresh session keys
   between the EAP [client] and authenticator that are known only to those
   parties, and for both the EAP [client] and authenticator to demonstrate
   that they are authorized to perform their roles either by each other
   or by a trusted third party (the backend authentication server).
   
%   Completion of an EAP method exchange (phase 1a) supporting key
%  derivation results in the derivation of EAP keying material (MSK,
%  EMSK, TEKs) known only to the EAP [client] (identified by the Peer-Id(s))
%  and EAP server (identified by the Server-Id(s)).
%  [\dots]
%  
%  Completion of the AAA exchange (phase 1b) results in the transport of
% keying material from the EAP server (identified by the Server-Id(s))
% to the EAP authenticator (identified by the NAS-Identifier) without
% disclosure to any other party.[\dots]

	[\dots]
   
   The backend authentication server is trusted to transport keying
  material only to the authenticator that was established with the
  [client], and it is trusted to transport that keying material to no other
  parties. [\dots] 
%  The backend authentication server is trusted to
%  refrain from deriving these same keys or acting as a
%  man-in-the-middle even though it has access to the keying material
%  that is needed to do so.

\end{quote}

In other words,
the goal of EAP is to be a secure 3P-AKE protocol.
We now explain how we are going to model each of the components that make up the EAP framework.
Since EAP is agnostic with respect to its components,
we want to reflect this in our modeling as well.
As far as possible,
we try not to make any assumptions on the internal structure of the sub-protocols that are used to instantiate the EAP framework.


\subsection{Client--server EAP method}\label{sec:generic_composition_results:modeling_EAP:EAP_method}

The modeling of the EAP method between the client and the server is fairly straightforward.
Since its goal is to distribute a shared key between the two parties,
it can be naturally modeled as a 2P-AKE protocol with mutual (implicit) authentication.
For simplicity,
we are going to assume that the EAP method is based on public keys for its long-term credentials.
This corresponds to EAP methods such as EAP-TLS~\cite{IETF:RFC5216:EAP-TLS} which we will study further in \cref{sec:EAP-TLS-security}.
Nevertheless,
there is nothing fundamental about this assumption.
Our results can easily be modified to also handle symmetric long-term keys,
or even a combination of the two.


\paragraph{Channel binding.}
There is a well-known issue with the EAP architecture called the ``lying authenticator problem''~\cite[Section~7.15]{IETF:RFC3748:EAP},
where a malicious authenticator may present false or inconsistent identity information to the different sides. 
Specifically,
during the EAP method the client needs to signal to the server which authenticator it is connecting to,
so that the server can know where it is supposed to transfer the shared key.
Unfortunately,
many EAP methods do not authenticate this information.
This can enable a rogue authenticator to impersonate another authenticator towards the client.

Concretely,
suppose client $A$ wants to connect to an authenticator $B$.
Assume that they are both associated with the mutually trusted server $S$.
Additionally,
suppose there is also a malicious authenticator $C$ associated with the same server $S$.
When client $A$ begins its EAP method exchange with server $S$,
it also communicates the identity ``$B$'' to $S$.
However,
since this information is not authenticated,
the authenticator $C$ can change it to say ``$C$''.
Consequently,
once the EAP method completes,
the server $S$ will believe that $A$ wanted to talk to $C$.
As a result,
$S$ transfer the key it established with $A$ to $C$ instead of~$B$.



Unless the EAP method authenticates the identify information there is no way for the client and server to verify that they are talking to the same authenticator.
More generally,
the process of ensuring that what the authenticator said to the client is consistent with what it said to the server is known in EAP as \emph{channel binding}.
There are two principal ways in which one can  achieve channel binding in EAP~\cite[Section~4.1]{IETF:RFC6677:EAP-channel-binding}.
The first is to have the EAP method authenticate the necessary information directly during the exchange
or in a separate integrity-protected channel after the shared key between the client and the server has been established. 
The second is to have the information that needs to be authenticated included into the derivation of the EAP session key.

There are advantages and disadvantages to both approaches.
For example,
the former allows for policy-based comparison of network properties where not all information necessarily have to match bit-for-bit on both ends,
while in the latter this does not work.
In contrast,
authenticating the information by explicitly transferring it in an integrity-protected channel might require larger changes to the existing EAP methods than just including it into the key derivation.

Since we find it to be the cryptographically cleanest,
we only consider channel binding based on key derivation in this thesis.
Consequently,
we are going to assume that there is a pseudorandom function $\mathsf{PRF}$ associated to each EAP method.
On the other hand,
we are not going to assume that the EAP method itself provides integrity protection of the identity information in any way.
In fact,
we are going to treat the communication of the authenticator's identity from the client to the server as being completely independent of the EAP method.
This has the benefit of making it possible to analyze the EAP methods purely in terms of their concrete underlying authentication protocol.




\subsection{Server--authenticator key transport protocol}\label{sec:generic_composition_results:modeling_EAP:AAA}

After the EAP session key has been established between the client and the server,
it needs to be securely transmitted to the authenticator.
Without a doubt the most popular protocol for this purpose is RADIUS~\cite{IETF:RFC2865:RADIUS}.
It is based on a long-term symmetric secret,
i.e., a PSK,
shared between the server and the authenticator.
In order to transfer the EAP session key from the RADIUS server to the authenticator,
the RADIUS protocol specifies a custom encryption scheme based on the Microsoft Point-to-Point Encryption (MPPE) algorithm~\cite{IETF:RFC2548:MS-MPPE-RADIUS-attributes}. 
Basically,
MPPE is a stream cipher based on the cipher feedback (CFB) mode of operation using the MD5 hash function as its internal pseudorandom function.

To encrypt an EAP session key $K = K_1 \concat K_2 \concat \dotsb \concat K_t$,
consisting of 128 bit blocks $K_i$ (with $K_t$ possibly zero-padded),
MPPE proceeds as follows:
\begin{align}\label{eq:EAP:MPPE}
	C_1 &\gets \operatorname{MD5}(S \concat R \concat A) \oplus K_1 , \\
	C_i &\gets \operatorname{MD5}(S \concat C_{i-1}) \oplus K_i .
\end{align} 
Here $S$ is the PSK shared between the server and the authenticator, 
$R$ is a 128 bit random nonce,
and $A$ is a 16 bit salt.
Peculiarly,
the nonce $R$ is \emph{not} chosen by the server itself.
Instead, it is generated by the authenticator and sent to the server in a previous RADIUS message.
The ciphertext $C = C_1 \concat C_2 \concat \dotsb \concat C_t$ is integrity protected using HMAC-MD5\footnote{The original RADIUS standard~\cite{IETF:RFC2865:RADIUS} does not specify HMAC-MD5,
but rather the MAC construction $\operatorname{MD5}(M \concat S)$. 
However, a later RFC~\cite{IETF:RFC3579:RADIUS+EAP} dealing specifically with the combination of EAP + RADIUS,
prescribes the use of HMAC-MD5.
} 
before being sent to the authenticator.
The HMAC tag is computed with the same secret $S$ that was used to encrypt~$K$. 


Although RADIUS is the most common server-to-authenticator protocol when using EAP,
we choose not to model it explicitly in this thesis.
There are a couple of reasons for this.
First,
while RADIUS is certainly the predominant choice of key transport protocol used together with EAP,
it is not the \emph{only} one.
In particular,
protocols like Diameter~\cite{IETF:RFC6733:DIAMETER} and Cisco's TACACS+\cite{IETF:DRAFT:TACACS+},
are also frequently used.
Thus,
in keeping with our  goal of capturing the generality of the EAP framework,
we want our modeling to cover these protocols as well.
Second,
the RADIUS encryption mechanism described above has received very little scrutiny. 
That is to say,
its CFB and HMAC building blocks have been heavily analyzed and are well understood
(see e.g.~\cite{EPRINT:Wooding08} and \cite{JC:Bellare15}),
but their specific usage within the RADIUS protocol is not.
In particular,
the non-standard way in which the random nonce $R$ is chosen,
as well as the reuse of the secret $S$ in both MPPE and HMAC,
are cause for concern.
Ultimately,
the security of RADIUS is largely unknown.

On the other hand,
RADIUS is often used on top of some other security protocol,
like IPsec and TLS
(see e.g. RADIUS-over-TLS~\cite{IETF:RFC6614:RADIUS_over_TLS}).
Thus,
it seems reasonable to model the key transport protocol between the server and the authenticator as a generic ACCE protocol based on  a symmetric PSK.
Again,
there is nothing fundamental about our choice of PSKs for long-term credentials,
and our model could very well have included public keys as well.
However,
since RADIUS is so often configured with PSKs,
it seems like a natural choice.
It also has the added benefit of making our security analyses cleaner,
since the long-term keys used by the EAP method and the key transport protocol are of distinct types
(recall that we have assumed that EAP methods use public keys for long-term credentials).  




\subsection{Client--authenticator protocol}\label{sec:generic_composition_results:modeling_EAP:link-layer}

Let us call the combination of an EAP method and the subsequent key transport protocol \emph{basic EAP}.
Normally,
basic EAP is followed by a link-layer specific protocol between the client and the authenticator,
called a \emph{Security Association Protocol} in \cite{IETF:RFC5247:EAP-key-management}.
Like the key transport protocol between the server and the authenticator,
the Security Association Protocol is technically out of scope of the base EAP standard~\cite{IETF:RFC3748:EAP}.
Nevertheless,
Section~3.1 of \cite{IETF:RFC5247:EAP-key-management} lists a number of recommended features that it ought to have.
Chief among these are:
mutual proof of possession of the EAP session key,
generation of link-layer specific encryption keys,
entity authentication,
and secure negotiation of protocol capabilities.
Save possibly for the last one,
these are all features we expect from a PSK-based 2P-AKE protocol providing explicit entity authentication.
%Thus, we will model the Security Association Protocol as a 2P-AKE protocol based on symmetric long-term keys.

Let us call the combination of basic EAP with a subsequent Security Association Protocol \emph{full EAP}.
Given that full EAP is usually what is used in practice,
our main aim in this \lcnamecref{sec:generic_composition_results} is to analyze this composition. 
However,
rather than analyzing the full EAP all at once,
we prefer a more modular approach.
First, we establish the security of basic EAP under the assumptions we made on the EAP method and the key transport protocol in \cref{sec:generic_composition_results:modeling_EAP:EAP_method} and \cref{sec:generic_composition_results:modeling_EAP:AAA}, respectively.
Then,
rather than viewing the full EAP as consisting of an EAP method,
a  key transport protocol, and a Security Association Protocol,
we instead think of it as consisting of a black-box 3P-AKE protocol
(i.e., basic EAP) combined with a PSK-based 2P-AKE protocol. 
Consequently,
the two main results in this \lcnamecref{sec:generic_composition_results} are two generic composition theorems which correspond to the ``cryptographic core'' of basic and full EAP, respectively. 
Our results are modular and capture the compositional nature of EAP.
\Cref{fig:generic_protocol} gives a roadmap for the two composition results.
 
\begin{figure}
	\centering
	\includestandalone[width=1\textwidth,mode=build]{tikz_3P-KD_construction}
	
	\caption{(Right) Construction of a 3P-\akewfstext secure protocol $\protocol_3$ from a 2P-\akefstext secure protocol $\protocol_1$, 
	an ACCE secure protocol $\protocol_2$, and a pseudorandom function $\mathsf{PRF}$.
	(Left) Construction of a 3P-\akefstext secure protocol $\protocol_5$ from a 3P-\akewfstext secure protocol $\protocol_3$ and a
	2P-\akenfstext secure protocol $\protocol_4$.
%	$B$ and $S$ share a common long-term symmetric key $\Key_{BT}$.
	}
	\label{fig:generic_protocol}
\end{figure}

Preempting our results a bit,
we show that basic EAP can achieve security in our weak forward secrecy model \akewfstext,
while full EAP can achieve security in our full forward secrecy model \akefstext.
Intuitively,
the reason why basic EAP does not achieve full forward secrecy is because it does not provide key-confirmation.
Namely,
after completing the EAP method with the server,
the client has no guarantee that the key-transport protocol between the server and the authenticator actually took place.
Specifically,
recall the following attack from \cref{sec:definitions:AKE}
which illustrates why basic EAP does not provide full forward secrecy. 

Suppose \emph{after} the client accepted,
but \emph{before} the key-transport protocol between the server and authenticator starts running,
an adversary learns the long-term PSK of the server and the authenticator.
The adversary can now impersonate the authenticator towards the server and have it send over the session key it previously established with the client. 
According to the full forward secrecy model \akefstext this attack is valid since the exposure of the PSK happened after the client accepted.
On the other hand,
in the weak forward secrecy model \akewfstext the attack is not valid,
because the client session does not have a partner and hence the PSK cannot be exposed.

To that end,
the purpose of the link-layer protocol can be seen as providing key-confirmation to the basic EAP protocol,
which ensures that the client will always have a partner before it accepts.
This is similar to how the security of the two-flow variant of HMQV can be upgraded from only providing weak forward secrecy to providing full forward secrecy by adding a third flow to it~\cite[§3]{C:Krawczyk05}.
Interestingly,
this also means that the forward secrecy of full EAP depends solely on the forward secrecy of the EAP method
and not at all on the forward secrecy of the Security Association Protocol.

Finally,
we point out that for technical reasons we cannot use the link-layer protocol in a completely black-box way,
but need to assume a little bit of structure on it.
Specifically,
we need to assume that the probability that two sessions at the same party end up with the same local transcript is statistically bounded.
Recall that a session's local transcript $\localtranscript$ consists of all the messages it has sent and received.
The reason for this assumption is that in the proof of our second composition result we need to rely on a local partner function,
which are only defined for unique (global) transcripts
(see \cref{def:local_partnering}).
%and local partner functions are only well-defined for (global) transcripts~$T$ where no local transcripts collide,
%i.e., so-called unique transcripts.

Note that this assumption is quite mild.
It is met by any protocol where each protocol participant contributes some randomness,
e.g., a nonce or Diffie-Hellman share.
In particular,
it is met by TLS, SSH, IKEv2 and the IEEE~802.11 4WHS protocol.

\subsection{Related work on EAP}\label{sec:generic_composition_results:modeling_EAP:related_work}
He et al.~\cite{CCS:HSDDM05} have conducted a formal analysis of the combination of EAP-TLS and IEEE~802.11 in a symbolic model called the Protocol Compositional Logic.
However,
they do not threat the full generality of the EAP framework since they assume that the server and authenticator belong to the same entity
(hence omitting the key transport stage).  
In the computational setting there are to the best of our knowledge no papers that have treated the general EAP framework. 
However,
from a proof-technical point of view,
our composition theorems are reminiscent of the composition result proven by Abdalla, Fouque, and Pointcheval~\cite{PKC:AbdFouPoi05}.
 

Hoeper and Chen~\cite{HoeperC:2007:EAP_claims_fail} have criticized the lack of a clear trust model and precise security definitions in the EAP standard~\cite{IETF:RFC3748:EAP},
pointing out that this makes it difficult to formally assess its security.
Regarding EAP methods specifically,
Clancy and Hoeper~\cite{ClancyH:2009:making_the_case_EAP_cb} have stressed the importance of channel binding,
demonstrating several real-world attacks that might be possible in its absence.
Somewhat related to channel binding is the concept of \emph{tunneled} EAP methods.
These are EAP methods that first establish a secure channel (or tunnel) between the client and the authenticator having only unilateral or even no authentication.
Then, a second authentication protocol is run inside the secure channel in order to provide mutual authentication.
Examples of tunneled EAP methods are EAP-TTLS~\cite{IETF:RFC5281:EAP-TTLS} and PEAP~\cite{IETF:draft:PEAPv2}
(see \cref{table:EAP_methods}).
A classic result by Asokan, Niemi and Nyberg~\cite{Asokan:2003:MitM-tunneling} shows that a man-in-the-middle attack is possible on tunneled authentication protocols unless the inner and outer layers are cryptographically bound together.
Hoeper and Chen~\cite{HoeperC:2010:inconvenient_tunneled} have demonstrated that several existing tunneled EAP methods fail to properly bind the layers together.
This has also been exploited in later attacks on EAP-TTLS and PEAP~\cite{SP:BDFPS14}.
However,
we stress that we are not covering tunneled EAP methods specifically in this thesis.

Finally,
Horst et al.~\cite{CANS:HGJS16} have cryptanalysed the Point-to-Point Tunneling (PPTP) protocol in combination with RADIUS.
PPTP is used to establish a virtual private network (VPN) between a client and a VPN endpoint with the help of a mutually trusted RADIUS server.
The VPN endpoint functions very much like the authenticator in the EAP framework,
and the RADIUS protocol is used to transfer a session key from the server to the VPN endpoint using the same MPPE encryption scheme described in \cref{sec:generic_composition_results:modeling_EAP:AAA}.
However,
a crucial difference between the usage of MPPE in PPTP  vs. EAP,
is that in the former,
the random nonce $R$  provided by the VPN endpoint to the RADIUS server and used as input to the MPPE algorithm
(see \cref{eq:EAP:MPPE}),
is \emph{not} protected by a MAC. 
This makes the attack of Horst et al.~\cite{CANS:HGJS16} possible against PPTP,
but not against EAP.

\section[First composition theorem]{First composition theorem: \\
{\large\texorpdfstring{2P-\akefstext + 2P-ACCE $\implies$ 3P-\akewfstext}{2P-AKEfs + 2P-ACCE + channel binding -> 3P-AKEwfs}}}\label{sec:generic_composition_results:first_composition_result}

In this \lcnamecref{sec:generic_composition_results:first_composition_result} we state and prove our first composition theorem.
This result corresponds to the cryptographic core of basic EAP.

\paragraph{Construction.}\label{sec:construction:PI_3}
From a 2P-AKE protocol $\protocol_1$
(based on public keys),
a 2P-ACCE protocol $\protocol_2$
(based on PSKs),
and a pseudorandom function $\mathsf{PRF}$,
we construct the 3P-AKE protocol $\protocol_3$ shown in \cref{fig:generic_protocol}.
Specifically,
protocol $\protocol_3$ works as follows.
First,
sub-protocol $\protocol_1$ is run between client $A$ and server $S$ to derive an intermediate key $\key_{AS}$. 
$A$ also communicates the identities ``$A$'' and ``$B$'' to $S$,
where $B$ is the identity of the authenticator that $A$ wants to talk to.
These identities are sent independently of sub-protocol~$\protocol_1$ and have no integrity protection. 


Note that $A$ knows the identities of both $S$ and $B$ at the beginning of the protocol
whereas $S$ learns about $B$ from the identities communicated by $A$.
Technically,
this means that a session at $A$ needs to be initialized with the identities of $S$ and $B$
(setting the $\peers$ variable accordingly),
while a session at $S$ will update its $\peers$ variable to include $B$ after receiving this identity from~$A$. 
%---meaning 
%that $\peers = \lbrace B, S \rbrace$ at the initialization of some session at $A$---

From the key $\key_{AS}$ derived in sub-protocol $\protocol_1$,
both $A$ and $S$ further derive the key $\key_{AB} \gets \mathsf{PRF}(\key_{AS},A,B)$.
The key $\key_{AB}$ will ultimately be the session key shared between $A$ and $B$ in protocol~$\protocol_3$.
In order for $S$ to transfer $\key_{AB}$ to $B$,
 they first establish a secure channel using sub-protocol $\protocol_2$.
Once established,
$S$ sends the session key $\key_{AB}$ together with the identity of $A$ over the channel to $B$.
For simplicity,
we assume that $\key_{AB}$ and ``$A$'' are transfered with a \emph{single} channel message,
which we call the \emph{$\Ckey$ message}.
Specifically,\footnote{To 
simplify our exposition we omit both the associated data and the encryption state when writing the inputs to the stAE scheme $\stae$ for the remainder of this \lcnamecref{sec:generic_composition_results}.  
}
\begin{equation}\label{eq:EAP:first_composition:def:C_key}
	\Ckey \gets \stae.\staeenc(ck, \key_{AB} \concat ``A") ,
\end{equation}
where $\stae$ is the stAE scheme associated with the ACCE protocol $\protocol_2$;
$ck$ being the channel key that $S$ and $B$ established in $\protocol_2$.
Unlike the identities ``$A$'', ``$B$'' sent over the $A$--$S$ link in \cref{fig:generic_protocol},
the identity ``$A$'' sent over the $S$--$B$ link enjoys privacy and integrity protection from the secure channel between $S$ and~$B$.

The initiator $A$ accepts in protocol $\protocol_3$ when it has derived $\key_{AB}$,
while the responder $B$ accepts once it has received---and properly decrypted---the $\Ckey$ message,
obtaining the session key $\key_{AB}$ as well as the identity ``$A$'' which it uses to update its $\peers$ variable.

\begin{remark}
Technically speaking,
it would be possible to include the identity ``$A$'' only as associated data when creating the $\Ckey$ message,
since it does not need privacy protection.
However,
when RADIUS is being run on top of a secure channel protocol,
like TLS or IPsec,
everything is transmitted inside the encrypted channel anyway,
so our approach in \cref{eq:EAP:first_composition:def:C_key} more closely matches real-world practice.
\end{remark}

\paragraph{Result.}
Our first composition result shows that protocol $\protocol_3$ is 3P-\akewfstext secure
if sub-protocol $\protocol_1$ is 2P-\akefstext secure,
sub-protocol $\protocol_2$ is 2P-ACCE secure,
and $\mathsf{PRF}$ is a pseudorandom function.
Note that protocol  $\protocol_3$ does not provide explicit entity authentication.
In fact,
no session at the initiator $A$ will have a partner at the time it accepts.
As a consequence,
protocol $\protocol_3$ cannot achieve security in the strongest \akefstext model due to the attack on basic EAP described in \cref{sec:generic_composition_results:modeling_EAP:link-layer}.

%
%Being modeled on the standalone EAP,
%protocol 
%
%\todo[inline]{HJ: FIX}
%We refer the reader to the last 2 pages of the introduction for a discussion of why 3P-\akewfstext is a the appropriate security notion for $\protocol_3$.




%\small
\begin{theorem}\label{thm:protocol_3:3P-AKE-}
Let $\protocol_3$ be the 3P-AKE protocol constructed from a 2P-AKE protocol $\protocol_1$, a ACCE protocol $\protocol_2$ and a pseudorandom function $\mathsf{PRF}$ as described in the construction above.
Let $f_1$ and $f_2$ be partner functions.
Then for any adversary $\A$ in security game \akewfstext against $\protocol_3$,
we can create a partner function~$f_3$
and algorithms $\A[B]_1$, $\A[B]_2$, $\A[B]_3$ and $\A[D]$,
such that
\begin{align}\label{eq:thm:P3}
	\begin{split}
	\adv_{\protocol_3,f_3}^{\operatorname{\mathsf{3P-}\akewfs}} (\A) 
	&\leq \adv_{\protocol_2,f_2}^{\operatorname{\mathsf{\acce-\EA}}}(\A[B]_1) 
		+ 2n^2 \cdot \adv_{\protocol_1,f_1}^{\operatorname{\mathsf{2P-}\akefs}}(\A[B]_2,) \\
	&\qquad + 2n^2 \cdot \adv_{\mathsf{PRF}}^{\prf}(\A[D]) 
		+ 4 n^2 \cdot \adv_{\protocol_2,f_2}^{\acce}(\A[B]_3) ,			
	\end{split}
\end{align} 
where $n = (\numsessions+1) \cdot \left| \mathcal{I} \cup \mathcal{R} \right|$,
and $\numsessions$  is the maximum number of sessions that $\A$ creates at each party.
%where $\numsessions$ is an upper bound on the number of sessions at each party.
%	$q_{ap}$ is the number of $\Ckey$ messages sent to the AP.
%	$|\mathcal{P}_S|$ is the number of Supplicant parties,
\end{theorem}

The proof of \cref{thm:protocol_3:3P-AKE-} roughly works as follows.
The 2P-\akefstext security of sub-protocol $\protocol_1$ allows us to swap out the intermediate key $\key_{AS}$ of the test-session with a random key.
The PRF-security of the key-derivation function $\mathsf{PRF}$ then allows us to replace the derived session key $\key_{AB}$ with random key.
Finally, the ACCE-security of sub-protocol $\protocol_2$ ensures that the adversary cannot modify any $\Ckey$ messages
nor can it learn anything about the session keys transferred inside them.
Thus,
at this point the adversary has zero advantage in winning in its 3P-AKE experiment.



\begin{proof}
We begin by defining the partner function $f_3$ using the given partner functions for sub-protocols $\protocol_1$ and $\protocol_2$.
Remember that throughout this thesis we always assume that all partners functions are symmetric and monotone.

\item
\paragraph{Defining the partner function for \texorpdfstring{$\protocol_3$}{Pi 3}.}
Intuitively,
$f_3$ is constructed by composing the two partner functions $f_1$ and $f_2$ as follows.
If $\oracle[A][i]$ is an initiator session and $\oracle[B][j]$ is a responder session,
then $\oracle[A][i]$ and  $\oracle[B][j]$ are partners in protocol $\protocol_3$ according to $f_3$ if and only if there exists a server session $\oracle[S][k]$,
such that $\oracle[A][i]$~and $\oracle[S][k]$ are partners in sub-protocol $\protocol_1$ according to $f_1$;
and  that $\oracle[S][k]$ and $\oracle[B][j]$ are partners in sub-protocol $\protocol_2$ according to $f_2$.
That is,
$\oracle[A][i]$ and $\oracle[B][j]$ are partners if there exists a server session $\oracle[S][k]$ that acts as the bridge between them in the two sub-protocols $\protocol_1$ and~$\protocol_2$.


To make this formally precise,
one needs to extract from the 3P-AKE transcript $T_3$ of experiment $\Exp_{\protocol_3,\queryset}(\A)$
two transcripts $T_1$ and $T_2$ that contain the queries pertaining to the two-party sub-protocols $\protocol_1$ and $\protocol_2$.
Then $f_3$ is defined on $T_3$ by running $f_1$ and $f_2$ on the corresponding transcripts $T_1$ and $T_2$.
Admittedly,
the details on how to do this are a bit tedious,
so they are relegated to \cref{sec:appendix:full_proofs}.
They can safely be skipped on first reading.


Suppose we have extracted transcripts $T_1$ and $T_2$ from $T_3$.
We say that $\oracle[A][i]$~and~$\oracle[S][k]$ are \emph{$f_1$-partners} if $f_{1,T_1}(\oracle[A][i]) = \oracle[S][k]$.
Since $f_1$ is symmetric this is equivalent to $f_{1,T_1}(\oracle[S][k]) = \oracle[A][i]$.
Similarly,
we say that $\oracle[S][k]$~and~$\oracle[B][j]$ are \emph{$f_2$-partners} if $f_{2,T_2}(\oracle[S][k]) = \oracle[B][j]$ 
(or equivalently $f_{2,T_2}(\oracle[B][j]) = \oracle[S][k]$).
Finally,
$\oracle[A][i]$~and~$\oracle[B][j]$ are \emph{$f_3$-partners},
or just \emph{partners},
if $f_{3,T_3}(\oracle[A][i]) = \oracle[B][j]$,
where $f_3$ is defined as follows.

\item
\begin{itemize}
\item 
$f_{3,T_3}(\oracle[A][i]) = \oracle[B][j]$ if and only if:

\begin{enumerate}
	\item $\oracle[A][i]$ and $\oracle[S][k]$ are $f_1$-partners,
	
	\item $\oracle[S][k]$ and $\oracle[B][j]$ are $f_2$-partners,
	
	\item $\oracle[A][i].\peers = \oracle[B][j].\peers = \oracle[S][k].\peers = \lbrace A, B, S \rbrace$. \label{def:proof:3P-AKE-:f_3:equal_peers}
	


\end{enumerate}
\end{itemize}

Note that \cref{def:proof:3P-AKE-:f_3:equal_peers}  implies that $\oracle[S][k]$ received the same identities that $\oracle[A][i]$ sent  over the $A$--$S$ link in \cref{fig:generic_protocol}. 
By its construction,
$f_3$ is monotone and symmetric provided $f_1$ and $f_2$ are.
The soundness of $f_3$ follows from the soundness of $f_1$ and $f_2$ and the ACCE security of sub-protocol $\protocol_2$
(particularly its channel integrity).
That is,
it can be shown that 
\begin{equation}\label{eq:proof:3P-AKE-:f_3:soundness}
	\adv_{\protocol_3,f_3}^{\sound}(\A) 
		\leq \adv_{\protocol_1,f_1}^{\sound}(\A') 
			+ \adv_{\protocol_2,f_2}^{\sound}(\A'') 
			+ \adv_{\protocol_2,f_2}^{\acce}(\A''') .
\end{equation}

However,
for simplicity we are going to assume that $f_1$ and $f_2$ have perfect soundness in this proof,
i.e., $\adv_{\protocol_1,f_1}^{\sound}(\A') = 0$ and $\adv_{\protocol_2,f_2}^{\sound}(\A'') = 0$.
Unfortunately,
this does not imply that $f_3$ has perfect soundness too,
since an adversary could potentially forge a $\Ckey$ message so that two partners end up with different keys.
Thus,
in order to enforce perfect soundness for $f_3$ as well,
we extend its definition with the following requirement:


\begin{enumerate}[4.]
	\item 	the $\Ckey$ message received by $\oracle[B][j]$ was identical to the one produced by~$\oracle[S][k]$.
\end{enumerate}
	

This removes the $\adv_{\protocol_2,f_2}^{\acce}(\A''')$ term in \cref{eq:proof:3P-AKE-:f_3:soundness} and gives $f_3$ perfect soundness.
In the remainder we can thus always assume that $f_3$-partners have the same session key.
		



\paragraph{\texorpdfstring{\akewfstext}{AKEwfs} security.}
Our proof for protocol $\protocol_3$,
as well as most of the other proofs in this thesis,
make use of a proof technique called \emph{game hopping} (\cite{EPRINT:Shoup04,EPRINT:BelRog04}).
In a game hopping proof,
one incrementally  introduce small changes to the original security game---each change being called a \emph{game hop}---which,
after a finite number of hops,
eventually leads to a situation where the adversary cannot win by definition,
or where it is easy to bound its advantage in terms of something else.
Each individual game hop is justified by showing that the change does not substantially affect the adversary's winning chances.
%The difference is often bounded by showing that if the adversary could discern the two games,
%then it could be transformed into solving some other problem.

The most fundamental result on game hoping proofs is the so-called Difference Lemma~\cite{EPRINT:Shoup04},
sometimes also called the Fundamental Lemma of Game-Playing~\cite{EPRINT:BelRog04}.
Basically,
the Difference Lemma states that if two games proceed identically unless the some event $F$ occurs,
then the difference between the adversary's advantage in the two games is bounded by $\Pr[F]$.
We will use this result many times in our thesis.



In the following,
when we say that a certain game \emph{aborts}, 
%due to some event occurring,
we mean that the challenger stops the execution of the experiment and outputs a random bit on $\A$'s behalf. 


\setcounter{gamehop}{0}

%\item
\paragraph{Game~\game:}
This is the real 3P-\akewfstext security game, hence
\begin{equation*}
	\adv_{\protocol_3,f_3}^{\mathsf{G_\game}}(\A) = \adv_{\protocol_3,f_3}^{\operatorname{\mathsf{3P-\akewfs}}}(\A) \ .
\end{equation*}

\newgame
\paragraph{Game~\game:}\label{game_hop:3P-KD:ACCE-EA}
This game proceeds as the previous one, 
but aborts if a fresh responder or server session \emph{accepts maliciously} in sub-protocol $\protocol_2$,
meaning that it accepted without a partner in $\protocol_2$ according to partner function $f_2$.

%\vspace{-\baselineskip}                        
%\item
%\vspace{-\baselineskip}
\begin{claim}\label{lemma:3P-KD:ACCE-EA}
\begin{equation}
	\adv_{\protocol_3,f_3}^{\mathsf{G_{\prevgame{}}}}(\A) 
	\leq \adv_{\protocol_3,f_3}^{\mathsf{G_\game}}(\A) 
	+ \adv_{\protocol_2,f_2}^{\operatorname{\mathsf{\acce-\EA}}}(\A[B]_1)  .
\end{equation}
\end{claim}

\begin{proof}
Let $E$ be the event that a server or responder session accepts maliciously in sub-protocol $\protocol_2$.
Game~\prevgame{} and Game~\game{} proceed identically as long as event $E$ does not occur,
so by the Difference Lemma we have 
\begin{equation}
	\adv_{\protocol_3,f_3}^{\mathsf{G_{\prevgame{}}}}(\A) 
	\leq \adv_{\protocol_3,f_3}^{\mathsf{G_\game}}(\A) 
		+ \Pr[E]  .
\end{equation}  

To bound $\Pr[E]$ we create an algorithm $\A[B]_1$ which breaks the explicit entity authentication of sub-protocol $\protocol_2$ whenever event $E$ occurs in Game~\prevgame.
Algorithm $\A[B]_1$ begins by creating all the long-term keys for sub-protocol $\protocol_1$,
selects a random bit $\bsim$,
and then runs $\A$.
Since $\A[B]_1$ has created all the long-term keys for sub-protocol $\protocol_1$,
it can derive all the session keys $\key_{AB}$ itself,
and simulate all of $\A$'s queries pertaining to sub-protocol $\protocol_1$.

When $\A$ makes a $\Send$ query that pertains to sub-protocol $\protocol_2$,
then $\A[B]_1$ answers it by making a corresponding $\Send$ query to its 2P-ACCE-EA security game.
When a server session $\oracle[S][k]$ accepts in the 2P-ACCE-EA security game,
then $\A[B]_1$ creates its $\Ckey$ message by making the query $\Encrypt(\oracle[S][k], \key_{AB}, \key_{AB})$\footnote{For
the remainder of this proof we omit the associated data input to the $\Encrypt$ query,
since the $\Ckey$ message does not depend on it.
},
where $\key_{AB}$ is session key $\A[B]_1$ derived for $\oracle[S][k]$ in sub-protocol $\protocol_1$.
When $\A$ issues a $\Test(\oracle[U][i])$ query,
then,
depending on bit $\bsim$, $\A[B]_1$ returns $\oracle[U][i]$'s  real session key or a random key.
Finally,
when $\A$ terminates,
then $\A[B]_1$ terminates too
(in this case event $E$ has not occurred).

To analyze $\A[B]_1$'s winning probability,
observe that $\A[B]_1$ provides a perfect simulation of protocol $\protocol_3$ for $\A$.
Moreover,
since the freshness predicate $\fresh_{\akewfs}$ is strictly more restrictive than predicate $\fresh_{\acce}$,
if event $E$ occurs in Game~\prevgame,
then a malicious accept also occurs in $\A[B]_1$'s 2P-ACCE-EA security game.
\end{proof}




\begin{remark}
The abort condition in Game~\game{} does not mean that every session in protocol \emph{$\protocol_3$} will have a partner (according to $f_3$).
In fact, 
no initiator session will have a partner at the time when it accepts, 
because at that point sub-protocol $\protocol_2$ has not even started yet.
%Rather, the abort condition only applies to the responder sessions.
\end{remark}






\newgame
\paragraph{Game~\game:}\label{game_hop:3P-KD:guess_test-session}
This game implements a \emph{selective} AKE security game~\cite[§3.3]{EPRINT:KraPatWee13},
rather than the normal adaptive one.
That is,
at the beginning of the game, 
the adversary is required to commit to its choice of test-session and its partner (if any).
Technically, 
at the beginning of the game the adversary must output two pairs $(U,i)$ and $(V,j)$,
with $i \in [1,\numsessions]$ and $j \in [0, \numsessions]$,
where $\numsessions$ is an upper bound on the number of sessions at each party,
and a choice of $j = 0$ means that $\oracle[U][i]$ is not intended to get a partner.
\cref{game_hop:3P-KD:guess_test-session} then proceeds as in Game~\prevgame{},
except that if either of the following events occur,
then the challenger penalizes the adversary by outputting a random bit at the end.
\begin{enumerate}[(i)]
	\item $\oracle[U][i]$ was not selected as the test-session by $\A$.
	
	\item $\oracle[U][i]$ gets a different partner than $\oracle[V][j]$
	(including the case that it \emph{gets} a partner if $j=0$).
	
%	\item $\oracle[U][i]$ was guessed to have a partner, 
%	but either gets none or someone different from $\oracle[V][j]$
\end{enumerate}

\begin{claim}\label{lemma:3P-KD:guess_test-session}
\begin{equation}
	\adv_{\protocol_3,f_3}^{\mathsf{G_{\prevgame{}}}}(\A) 
	\leq (\numsessions +1 )^2 \cdot  | \mathcal{I} \cup \mathcal{R} | ^2 \cdot \adv_{\protocol_3,f_3}^{\mathsf{G_\game}}(\A').
\end{equation}
\end{claim}

\begin{proof}
From an adversary $\A$ that wins against the adaptive game in Game~\prevgame{},
we create the following adversary $\A'$ that wins against the selective game in Game~\game{}. 
First, $\A'$ randomly selects a pair $(U,i) \getsr \left( \mathcal{I} \cup \mathcal{R} \right) \times [1,\numsessions]$ and a pair $(V,j)$, which,
depending on the role of~$U$,
is either selected as $(V,j) \getsr \mathcal{I} \times [0,\numsessions]$
or $(V,j) \getsr \mathcal{R} \times [0,\numsessions]$.
It outputs $(U,i)$ and $(V,j)$ as its choice to the selective security game it is playing.
$\A'$ then runs $\A$ and answers all of its queries by forwarding them to its own selective security game. 
When $\A$ stops with output $b'$,
then $\A'$ stops and outputs the same bit as well. 

Algorithm $\A'$ perfectly simulates Game~\prevgame{} for $\A$,
so $\A'$'s choice of selective security targets matches those of $\A$ with probability $1 / \left( (\numsessions + 1) \cdot  | \mathcal{I} \cup \mathcal{R} |  \right)^2$.
When $\A'$'s guess is correct it wins with the same probability as $\A$,
while when it is wrong,
$\A'$ gets penalized in its selective security game, 
hence wins with probability $1 / 2$ in Game~\game.
%\qed
\end{proof}

In the remaining games,
let $\oracle[U][i]$ and $\oracle[V][j]$ denote the targets that the adversary commits to in \cref{game_hop:3P-KD:guess_test-session};
$\oracle[U][i]$ being the test-session,
and $\oracle[V][j]$ being its (potentially empty) partner.
Define the \emph{co-partner} of $\oracle[U][i]$ to be the server session being involved in the protocol run between $\oracle[U][i]$ and $\oracle[V][j]$.
Specifically,
if $\oracle[U][i]$ has the initiator role,
then its co-partner is defined to be $ f_{1,T_1}(\oracle[U][i])$;
while if $\oracle[U][i]$ has the responder role,
then its co-partner is defined to be $f_{2,T_2}(\oracle[U][i])$.

Note that if $\oracle[U][i]$ has the initiator role,
then it does not necessarily have a  co-partner when it accepts.
On the other hand,
if $\oracle[U][i]$ has the responder role,
then its co-partner is guaranteed to exist by Game~\ref{game_hop:3P-KD:ACCE-EA}.
 



\newgame
\paragraph{Game~\game:}\label{game_hop:3P-KD:3P-AKE-swap-random}
This game proceeds as the previous one,
except that it replaces the intermediate key $\key_{AS}$ derived in sub-protocol $\protocol_1$
with a random key in the protocol run involving the test-session.
That is,
for the session out of $\oracle[U][i]$ and $\oracle[V][j]$ that has the initiator role in protocol $\protocol_3$,
the challenger replaces its intermediate key $\key_{AS}$ in sub-protocol $\protocol_1$ with a random key. 
Moreover,
the intermediate key derived by the co-partner of $\oracle[U][i]$ (if any)
is also replaced with the same random~key. 

\begin{claim}\label{lemma:3P-KD:2P-AKE-swap-random}
\begin{equation}
	\adv_{\protocol_3,f_3}^{\mathsf{G_{\prevgame{}}}}(\A) 
	\leq \adv_{\protocol_3,f_3}^{\mathsf{G_\game}}(\A) 
	+ 2 \cdot \adv_{\protocol_1,f_1}^{\operatorname{\mathsf{2P-\akefs}}}(\A[B]_2)  .
\end{equation}
\end{claim}

\begin{proof}
We show that if it is possible to distinguish Game~\prevgame{} and Game~\game{},
then we can create an algorithm $\A[B]_2$ that breaks the 2P-\akefstext security of sub-protocol $\protocol_1$.
Algorithm $\A[B]_2$ begins by drawing a random bit $\bsim$
and creates all the long-term PSKs for sub-protocol $\protocol_2$.
$\A[B]_2$ then runs $\A$ and forwards all its queries that pertain to sub-protocol $\protocol_1$ to its own 2P-\akefstext security game.
All queries that pertain to sub-protocol $\protocol_2$,
$\A[B]_2$ answers itself using the PSKs it created.
It also implements all the abort conditions of the previous games.
To answer $\A$'s $\Test(\oracle[U][i])$ query,
$\A[B]_2$ proceeds as follows.

If $b_{sim} = 1$, then $\A[B]_2$ responds as normal by drawing a random key and returning this to $\A$.
If $b_{sim} = 0$ and $\oracle[U][i]$ is an initiator session,
then $\A[B]_2$ makes a corresponding $\Test(\oracle[U][i])$ query to its own 2P-\akefstext security game to obtain a key $\key_{AS}^*$
(which is either $\oracle[U][i]$'s real session key in sub-protocol~$\protocol_1$
or a random key).
From $\key_{AS}^*$,
$\A[B]_2$  derives  $\key_{AB} \gets \mathsf{PRF}(\key_{AS}^*, A, B)$ which it returns back to $\A$
($A$, $B$ being the party identities accepted by the test-session).

If $b_{sim} = 0$ and $\oracle[U][i]$ is a responder session,
then $\oracle[U][i]$ must have a co-partner $\oracle[S][k]$ by Game~\ref{game_hop:3P-KD:ACCE-EA}.
To obtain the intermediate key $\key_{AS}^*$ needed to derive the session key $\key_{AB}$, 
$\A[B]_2$ does the same thing as above,
but this time by issuing the $\Test$ query to $\oracle[S][k]$.



When $\A$ outputs its guess $b'$,
then $\A[B]_2$ stops and outputs $0$ to its 2P-\akefstext security game if $b' = \bsim$,
and $1$ otherwise.
\medskip

Note that if the key $\key_{AS}^*$ returned from the $\Test$ query in $\A[B]_2$'s 2P-\akefstext security game is real, 
then $\A[B]_2$ perfectly simulates Game~\prevgame{}.
On the other hand,
if $\key_{AS}^*$ is a random key,
then $\A[B]_2$ perfectly simulates Game~\game{}.
Thus, the \lcnamecref{lemma:3P-KD:2P-AKE-swap-random} follows if we can show that the test-session chosen by $\A[B]_2$ in its own 2P-\akefstext security game is fresh according to predicate $\fresh_{\akefs}$ whenever the test-session $\oracle[U][i]$ chosen by $\A$ is fresh according to predicate $\fresh_{\akewfs}$.

If $\oracle[U][i]$ is an initiator session,
then $\A[B]_2$  uses the same session $\oracle[U][i]$ as the test-session target in its own 2P-\akefstext security game.
Since the freshness predicate $\fresh_{\akewfs}$ is strictly more restrictive than predicate $\fresh_{\akefs}$,
it follows that $\oracle[U][i]$ is fresh in $\A[B]_2$'s 2P-\akefstext game whenever it is fresh in $\A[B]_2$'s simulation for~$\A$.

If $\oracle[U][i]$ is a responder session,
then the test-session chosen by $\A[B]_2$ is $\oracle[U][i]$'s co-partner $\oracle[S][k]$.
We need to argue that $\oracle[S][k]$ is fresh in $\A[B]_2$'s 2P-\akefstext security game whenever $\oracle[U][i]$ is \akewfstext fresh in $\A[B]_2$'s simulation.
There are two cases to consider: either $\oracle[U][i]$ has an $f_3$-partner or it does not.
If $\oracle[U][i]$ has a partner
(which by Game~\prevgame{} must be $\oracle[V][j]$), 
then $\A$ cannot have made a $\Reveal(\oracle[V][j])$ query since this would violate the \akewfstext freshness of $\oracle[U][i]$.
Moreover,
since $f_3$ is constructed from $f_1$ and $f_2$,
$\oracle[V][j]$ must be $\oracle[S][k]$'s $f_1$-partner.
Consequently,
$\A[B]_2$ is also allowed to forward any $\Corrupt$ query to either $A$ or $S$ without violating the freshness of $\oracle[S][k]$ according to predicate $\fresh_{\akefs}$.

If $\oracle[U][i]$ does \emph{not} have an $f_3$-partner,
then $\A$ cannot have made any $\Corrupt$ query to $A$ or $S$ (since this would violate \akewfstext freshness).
Thus,
neither has $\A[B]_2$.
Furthermore,
if $\oracle[U][i]$ does not have an $f_3$-partner,
then this implies that its co-partner $\oracle[S][k]$ cannot have an $f_1$-partner either. 
Thus, 
$\A[B]_2$ can safely forward all of $\A$'s $\Reveal$ queries without violating the \akefstext freshness of~$\oracle[S][k]$.
%\qed
\end{proof}


\newgame
\paragraph{Game~\game:}\label{game_hop:3P-KD:PRF->RF}
This game proceeds as the previous one,
except that when deriving the session key $\key_{AB}$ in the protocol run involving the test-session $\oracle[U][i]$,
the challenger uses a random function $\$(\cdot, \cdot)$ rather than the function $\mathsf{PRF}(\key_{AS}, \cdot, \cdot)$.

More specifically,
if $\oracle[U][i]$ has the initiator role then its session key $\key_{AB}$ is derived using the random function $\$(\cdot, \cdot)$ instead of the function $\mathsf{PRF}(\key_{AS}, \cdot, \cdot)$.
Additionally,
if $\oracle[U][i]$ has a co-partner $\oracle[S][k]$, 
then $\oracle[S][k]$ uses the same random function to derive the key $\key_{AB}$ that it will forward in its $\Ckey$ message.

If $\oracle[U][i]$ has the responder role,
then it must have a co-partner $\oracle[S][k]$ by Game~\ref{game_hop:3P-KD:ACCE-EA}.
When  deriving the key $\key_{AB}$ that $\oracle[S][k]$ will use for its $\Ckey$ message,
the challenger uses the random function $\$(\cdot, \cdot)$ instead of the function $\mathsf{PRF}(\key_{AS}, \cdot, \cdot)$.

\begin{claim}\label{claim:game_hop:3P-KD:PRF->RF}
\begin{equation}
	\adv_{\protocol_3,f_3}^{\mathsf{G_{\prevgame{}}}}(\A) 
	\leq \adv_{\protocol_3,f_3}^{\mathsf{G_\game}}(\A) 
	+  2 \cdot \adv_{\mathsf{PRF}}^{\operatorname{\mathsf{prf}}}(\A[D])  .
\end{equation}
\end{claim}

\begin{proof}
We show that if it is possible to distinguish Game~\prevgame{} and Game~\game{},
then we can create a distinguisher algorithm $\A[D]$ against the PRF security of the function $\mathsf{PRF}$.
Distinguisher $\A[D]$ has access to an oracle $\mathcal{O}$
which either implements the function $\mathsf{PRF}(\widetilde{\key}, \cdot, \cdot)$ using an independent and uniformly distributed key $\widetilde{\key}$,
or it implements a random function $\$(\cdot, \cdot)$.
$\A[D]$ begins by drawing a random  bit $\bsim$ and creates all the long-term keys for sub-protocols $\protocol_1$ and $\protocol_2$.
%It also guesses a test-session $\oracle[U][i]$ and its partner $\oracle[V][j]$.
Next,
it runs $\A$ and answers all its queries according to Game~\prevgame{} by using the keys it created,
except that it answers $\A$'s $\Test(\oracle[U][i])$ query as follows.

If $\bsim = 1$, then $\A[D]$ returns a random key as normal. 
If  $\bsim  = 0$ and $\oracle[U][i]$ is an initiator session,
then $\A[D]$ answers with $\mathcal{O}(A,B)$,
where $A$, $B$ are the party identities accepted by $\oracle[U][i]$.
If $\bsim = 0$ and $\oracle[U][i]$ is a responder session,
then $\A[D]$ does the same,
but this time $A$, $B$ are the identities that the co-partner of $\oracle[U][i]$ received over the $A$--$S$ link in \cref{fig:generic_protocol}
(recall that if $\oracle[U][i]$ is a responder session then it is guaranteed to have a co-partner by Game~\ref{game_hop:3P-KD:ACCE-EA}). 

When $\A$ outputs its guess $b'$,
then $\A[D]$ stops and outputs $0$ to its PRF-game if $b' = \bsim$,
and $1$ otherwise.

\medskip
When $\A[D]$'s oracle $\mathcal{O}$ implements $\mathsf{PRF}(\widetilde{\key}, \cdot, \cdot)$,
then $\A[D]$ perfectly simulates Game~\prevgame{};
while if $\mathcal{O}$ implements a random function $\$(\cdot,\cdot)$, 
then $\A[D]$ perfectly simulates Game~\game{}.
For $x \in \lbrace \prevgame, \game \rbrace$, let $G_{x}^{\A} \Rightarrow 1$ denote the event that $\A$ wins in Game~$x$.
Then
\begin{align}
\adv_{F}^{\prf}(\A) 
	&= \Pr[\A^{\mathsf{PRF}(\widetilde{\key}, \cdot, \cdot)} \Rightarrow 1] - \Pr[\A^{\$(\cdot, \cdot)} \Rightarrow 1] \\
	&= \Pr[G_{\prevgame}^{\A} \Rightarrow 1] - \Pr[G_{\game}^{\A} \Rightarrow 1],
\end{align}
and the \lcnamecref{claim:game_hop:3P-KD:PRF->RF} follows.
\end{proof}

%Note that by the change in Game~\game{},
%the session key of $\oracle[U][i]$ and $\oracle[V][j]$ is derived using a random function rather then the pseudorandom function $\mathsf{PRF}$.
At this point one might expect that the adversary should be unable to distinguish the test-key from random
since the session key of $\oracle[U][i]$ is now derived using a random function rather than the pseudorandom function $\mathsf{PRF}$.
Unfortunately,
we cannot (currently) rule out that $\A$ might be able to learn something about the session key through the $\Ckey$ message delivered from the server to the responder.
Furthermore,
$\A$ could potentially also modify the $\Ckey$ message in such a way that it still decrypts to the same session key.
In this case $\oracle[U][i]$ and $\oracle[V][j]$ would end up with the same key while at the same time not being partners according to the definition of the partner function $f_3$.
Hence,
$\A$ could  reveal $\oracle[V][j]$ and trivially win in Game~\game.

In the following two games we show that neither of these scenarios are possible due to the ACCE security of sub-protocol $\protocol_2$.
In the first game we show that $\A$ is unable to successfully forge the $\Ckey$ message in the protocol run involving~$\oracle[U][i]$.
In the second game we show that $\A$ is unable learn anything about the session key from observing the $\Ckey$ message.


%In the following,
%let $\widetilde{\key}_{AB}$ denote the session key derived in this manner by the co-partner of $\oracle[U][i]$ (if it exists).


\newgame
\paragraph{Game~\game:}\label{game_hop:3P-KD:ACCE-int}
Suppose $\oracle[U][i]$ has a co-partner $\oracle[S][k]$ and that the ciphertext $C$ was the $\Ckey$ message produced by $\oracle[S][k]$ (if it created one at all).
Let $\oracle^*$ be the $f_2$-partner of $\oracle[S][k]$ in sub-protocol $\protocol_2$ required to exist by Game~\ref{game_hop:3P-KD:ACCE-EA}.
Game~\game{} proceeds as Game~\prevgame{},
but if $\oracle^*$ receives a $\Ckey$ message $C' \neq C$,
then $C'$ is automatically rejected,
i.e.,
it is  assumed to have decrypted to~$\bot$.
In this case $\oracle^*$'s session key is not set.



\begin{remark}
Note that if $\oracle[U][i]$ has the responder role,
then $\oracle^*$ is $\oracle[U][i]$ itself,
while if $\oracle[U][i]$ has the initiator role then $\oracle^*$ (if it exists) is some responder session.
We write ``if it exists'' because if $\oracle[U][i]$ has the initiator role
then it might not actually have a co-partner at all since sub-protocol $\protocol_1$ does not give any guarantees of explicit entity authentication.
However,
in that case there is no difference between Game~\prevgame{} and Game~\game{} since no relevant $\Ckey$ message is being created.
\end{remark}
 
\begin{claim}\label{lemma:3P-KD:ACCE-int}
\begin{equation}
	\adv_{\protocol_3,f_3}^{\mathsf{G_{\prevgame{}}}}(\A) 
	\leq \adv_{\protocol_3,f_3}^{\mathsf{G_\game}}(\A) 
	+ 2 \cdot \adv_{\protocol_2,f_2}^{\acce}(\A[B]'_3)  .
\end{equation}
\end{claim}

\begin{proof}[]
Assume $\oracle[U][i]$ has a co-partner $\oracle[S][k]$,
and that $\oracle[S][k]$ produced the ciphertext $C$ as its $\Ckey$ message.
Let session $\oracle^*$ be a session with the same definition as given in the game description above,
and let $F$ denote the event that $\A$ successfully forges the $\Ckey$ message being delivered to $\oracle^*$.
\begin{itemize}
\item \emph{Event $F$}: $\A$ sends to $\oracle^*$ a $\Ckey$ message $C' \neq C$,
and $C'$ decrypts to something other than $\bot$.
\end{itemize}

As long as event $F$ does not occur then Game~\prevgame{} and Game~\game{} are identical,
so by the Difference Lemma we have
\begin{equation}
	\adv_{\protocol_3,f_3}^{\mathsf{G_{\prevgame{}}}}(\A) 
		\leq \adv_{\protocol_3,f_3}^{\mathsf{G_\game}}(\A) +  \Pr[F] .
\end{equation}


To bound $\Pr[F]$ we create an adversary $\A[B]'_3$ that capitalizes on the event $F$ in order to break the ACCE security of sub-protocol $\protocol_2$.
Algorithm $\A[B]'_3$ begins by drawing a random bit $\bsim$
and creates all the long-term keys for sub-protocol $\protocol_1$.
%It also guesses the sessions $\oracle[U][i]$ and $\oracle[V][j]$ as in Game~\ref{game_hop:3P-KD:guess_test-session},
%and implements all of the abort conditions introduced so far.
All of $\A$'s $\Send$ queries that pertain to sub-protocol $\protocol_1$,
$\A[B]'_3$ answers itself using the long-term keys it created.
Particularly,
$\A[B]'_3$ can answer all $\Corrupt$ queries targeting the asymmetric long-term keys of initiators and servers itself. 
Moreover,
it can also answer all of $\A$'s $\Reveal$ queries that target initiator sessions,
and answer the $\Test$ query if $\oracle[U][i]$ if $U \in \mathcal{I}$
(using the bit $\bsim$).

$\Send$ queries that pertain to sub-protocol $\protocol_2$,
as well as $\Corrupt$ queries that target the PSKs shared between servers and responders,
are forwarded to $\A[B]'_3$'s 2P-ACCE security game.
Whenever a server session $\oracle$ accepts in sub-protocol $\protocol_2$, 
then $\A[B]'_3$ creates its $\Ckey$ message by querying $\Encrypt(\oracle, A \concat \key_{AB}, A \concat \key_{AB})$\footnote{Here
we are abusing notation and using ``$\oracle$'' to denote both the server session that $\A[B]'_3$ simulates for $\A$ in protocol $\protocol_3$,
as well as the corresponding ``proxy'' server session that $\A[B]'_3$ creates in its own ACCE security game in order to answer the queries to the former. 
}
 to its ACCE experiment,
where ``$A$'' is the party identity that $\oracle$ received on the $A$--$S$ link in \cref{fig:generic_protocol},  and $\key_{AB}$ is the session key that $\A[B]'_3$ derived from $\oracle$'s intermediate key $\key_{AS}$ in sub-protocol~$\protocol_1$.


Whenever $\A$ forwards a $\Ckey$ message to a responder session different from~$\oracle^*$,
then $\A[B]'_3$ first makes a $\Reveal$ query to that session in its ACCE security game in order to obtain its channel-key for sub-protocol $\protocol_2$.
Using this channel-key,
$\A[B]'_3$ decrypts the received $\Ckey$ message and simulates the responder session accordingly.
%(i.e., rejecting if the $\Ckey$ message didn't decrypt properly, 
%while accepting and setting the session-key and peers variables based on the decrypted $\Ckey$ message  if not). 
Consequently,
$\A[B]'_3$ can also answer all of $\A$'s $\Reveal$ queries targeting responder sessions different from $\oracle^*$. 

If $\A$ at any point stops during $\A[B]'_3$'s simulation,
not having sent a $\Ckey$ message to $\oracle^*$,
then $\A[B]'_3$ stops too and outputs $(\oracle, b')$ to its ACCE security game where $\oracle$ is an arbitrary session and $b'$ is a random bit,
i.e., $b' \getsr \bits$.

If $\A$ \emph{does} forward a $\Ckey$ message $C'$ to $\oracle^*$,
then $\A[B]'_3$ stops its simulation and outputs $(\oracle^*, b')$ to its ACCE game,
where the bit $b'$ is determined as follows.
If $C' = C$, 
where $C$ is the $\Ckey$ message produced by $\oracle[U][i]$'s co-partner $\oracle[S][k]$,
then this cannot be a forgery,
so $\A[B]'_3$ lets $b'$ be a random bit.
On the other hand, 
if $C' \neq C$,
meaning that $C'$ is a potential $\Ckey$ message forgery,
then $\A[B]'_3$ first makes the query $ \Decrypt(\oracle^*, C')$ to its ACCE game.
%If $m' \neq \bot$ then $b' = 1$,
%otherwise $b' \getsr \bits$.
%That is, 
If the $\Decrypt$ query returns something other than $\bot$,
$\A[B]'_3$ outputs $1$,
and in all other cases outputs a random bit.



\medskip
We now analyze $\A[B]'_3$.
If $\A$ does not send a $\Ckey$ message to $\oracle^*$ during $\A[B]'_3$'s simulation
then $\A[B]'_3$ outputs $(\oracle, b')$ to its 2P-ACCE security,
where $\oracle$ is an arbitrary session and $b'$ a random bit.
In this case $\A[B]'_3$ wins with probability~$1/2$.

If $\A$ \emph{does} send a $\Ckey$ message to $\oracle^*$,
then $\A[B]'_3$ picks $\oracle^*$ as its ACCE target.
We begin by arguing that $\oracle^*$ is fresh according to predicate $\fresh_{\acce}$,
provided $\A$'s test-target $\oracle[U][i]$ is fresh according to predicate $\fresh_{\akewfs}$.
Since $\A[B]'_3$'s simulation stops immediately once $\oracle^*$ accepts,
it never makes a $\Reveal$ query to $\oracle^*$,
and so we only have to consider the effects of $\Corrupt$ queries against the PSK shared between $\oracle^*$ and its server peer. 


If $\oracle[U][i]$ has the responder role then $\oracle[U][i] = \oracle^*$.
By Game~\ref{game_hop:3P-KD:ACCE-EA},
$\oracle^*$ must have an $f_2$-partner in sub-protocol $\protocol_2$ and its ACCE freshness follows immediately
since then any long-term key can legally be exposed;
in particular,
this includes the PSK shared between $\oracle^*$ and its server peer.

Now suppose $\oracle[U][i]$ has the initiator role.
If $\oracle[U][i]$ does not have a co-partner or this co-partner never reached the accept state
(hence not producing a $\Ckey$ message),
then there is nothing to prove since then there is also no $\oracle^*$ session.
On the other hand,
if $\oracle[U][i]$ has co-partner $\oracle[S][k]$ which created a $\Ckey$ message $C$,
then by Game~\ref{game_hop:3P-KD:ACCE-EA} there must be some session $\oracle^*$ being the $f_2$-partner of $\oracle[S][k]$.
If $\A$ forwards $C$ unmodified to $\oracle^*$,
then $\oracle[U][i]$ and $\oracle^*$ would be $f_3$-partners and so the ACCE freshness of $\oracle^*$ would again follow immediately.
%\footnote{Although 
%in this case it doesn't even matter that $\oracle^*$ is ACCE fresh since $\A[B]'_3$ would nevertheless output a random bit $b'$ if $\oracle^*$ and $\oracle[U][i]$ were partners
%(since event $F$ wouldn't have happened in this case).
%}
Conversely,
if $\A$ sends $C' \neq C$ to $\oracle^*$,
then $\oracle[U][i]$ and $\oracle^*$ would not be $f_3$-partners.
Hence,
if $\oracle[U][i]$ is to be fresh according to predicate $\fresh_{\akewfs}$,
then the long-term keys of its peers cannot have been exposed.
In particular,
this means that the PSK of $\oracle^*$ cannot have been exposed.
It follows that $\oracle^*$ is fresh according to predicate $\fresh_{\acce}$. 

\medskip
It remains to calculate $\A[B]'_3$'s winning probability when $\A$ forwards a $\Ckey$ message to $\oracle^*$.
That is,
if $\A[B]'_3$ picked $\oracle^*$ as its ACCE target.
If the $\Ckey$ message that $\oracle^*$ received  was forwarded unmodified from its $f_2$-partner $\oracle[S][k]$,
then $\A[B]'_3$ outputs a random bit and thus wins with probability $1 / 2$.
On the other hand,
if the $\Ckey$ message that $\oracle^*$ received was different from the one that $\oracle[S][k]$ sent out,
then there is a potential for event $F$ to occur. 
Note that $\A[B]'_3$ perfectly simulates Game~\prevgame{} until $\A$ sends a $\Ckey$ message to~$\oracle^*$,
so the probability that $F$ occurs in $\A[B]'_3$'s simulation is the same as the probability that $F$ occurs in Game~\prevgame{}.

Let $C'$ be the $\Ckey$ message that $\oracle^*$ received.
Recall that $\A[B]'_3$ outputs $1$ only if the $\Decrypt(\oracle^*, C')$ query returned something other than $\bot$,
and a random bit otherwise.
Since a $\Decrypt$ query in the ACCE experiment returns something other than $\bot$ only if $\oracle^*.b = 1$,
we have 
\begin{equation}
	\Pr[\Exp_{\protocol_2,\queryset}^{\acce}(\A[B]'_3) \Rightarrow 1 \mid F \land \oracle^*.b = 1] = 1,
\end{equation}
and 
\begin{equation}
	\Pr[\Exp_{\protocol_2,\queryset}^{\acce}(\A[B]'_3) \Rightarrow 1 \mid F \land \oracle^*.b = 0] = \frac{1}{2}.
\end{equation}

Finally,
notice that the value of $\oracle^*$'s secret bit $b$ in $\A[B]'_3$'s ACCE security game is independent of event $F$.
This is because there is nothing in $\A[B]'_3$'s simulation that depends on $\oracle^*.b$ before $\oracle^*$ receives the $\Ckey$ message,
and $\A[B]'_3$'s simulation stops immediately once this happens.
Thus 
\begin{equation}
	\Pr[\oracle^*.b = b \mid F] = \frac{1}{2} .
\end{equation}

Conditioning on event $F$ occurring, the winning probability of  $\A[B]'_3$ is
\small
\begin{align}
	\begin{split}
		\Pr[\Exp_{\protocol_2,\queryset}^{\acce}(\A[B]'_3) \Rightarrow 1 \mid F] &=
			\Pr[\Exp_{\protocol_2,\queryset}^{\acce}(\A[B]'_3) \Rightarrow 1 \mid F \land \oracle^*.b = 0] \cdot \frac{1}{2}	\\
			&\qquad + \Pr[\Exp_{\protocol_2,\queryset}^{\acce}(\A[B]'_3) \Rightarrow 1 \mid F \land \oracle^*.b = 1] \cdot 	\frac{1}{2} 
	\end{split} \\
		&= \frac{1}{2} \cdot \frac{1}{2} + 1 \cdot \frac{1}{2} = \frac{3}{4} .
\end{align}
\normalsize

Combing the above probability with the case when $F$ does not occur yields \cref{lemma:3P-KD:ACCE-int}.
%\qed
\end{proof}

The previous game established that $\A$ cannot modify the $\Ckey$ message in the protocol run involving the test-session $\oracle[U][i]$.
The next and final game shows that $\A$ also cannot learn anything about $\oracle[U][i]$'s session key by merely observing the $\Ckey$ message.


\newgame
\paragraph{Game~\game:}\label{game_hop:3P-KD:C_key_encrypt_0}
This game proceeds as the previous one,
but when creating the $\Ckey$ message of the co-partner of $\oracle[U][i]$,
the challenger encrypts the all-zero string $0^\keylen$
instead of the session key $\key_{AB}$.
If this $\Ckey$ message is eventually delivered to the intended responder session
(being either $\oracle[U][i]$ or $\oracle[V][j]$),
then its session key is still set to $\key_{AB}$.



\begin{claim}\label{lemma:3P-KD:Ckey_encrypt_0}
\begin{equation}
	\adv_{\protocol_3,f_3}^{\mathsf{G_{\prevgame{}}}}(\A) 
	\leq \adv_{\protocol_3,f_3}^{\mathsf{G_\game}}(\A) 
	+ 2 \cdot \adv_{\protocol_2,f_2}^{\acce}(\A[B]''_3)  .
\end{equation}
\end{claim}

\begin{proof}
We show that if it is possible to distinguish Game~\prevgame{} and Game~\game{},
then we can create an algorithm $\A[B]''_3$ that breaks the ACCE security of sub-protocol $\protocol_2$.
Algorithm $\A[B]''_3$ is almost identical to algorithm $\A[B]'_3$ in the previous proof of \cref{lemma:3P-KD:ACCE-int},
except for the following differences.
\begin{itemize}

	\item When creating the $\Ckey$ message of $\oracle[U][i]$'s co-partner (if it exists),
	say $\oracle[S][k]$,
	algorithm $\A[B]''_3$ makes the query $\Encrypt(\oracle[S][k], A \concat \key_{AB}, A \concat 0^{\keylen})$
	instead of the query $\Encrypt(\oracle[S][k], \allowbreak A \concat \key_{AB}, \allowbreak A \concat \key_{AB})$.
	
	\item If $\A$ sends a $\Ckey$ message $C'$ to $\oracle^*$
	(with the same definition of $\oracle^*$ as in the description of Game~\ref{game_hop:3P-KD:ACCE-int}),
	then $\A[B]''_3$ does \emph{not} stop its simulation.
	Instead $\A[B]''_3$ continues the simulation as follows.

	If $C'$ is equal to the $\Ckey$ message that was previously output by the co-partner of $\oracle[U][i]$
	using the $\Encrypt(\oracle[S][k], A \concat \key_{AB}, A \concat 0^{\keylen})$ query described above,
	then $\oracle^*$'s peer and session key variables are set based on the left message input to the $\Encrypt$ query.

	If $C'$ is not equal to the $\Ckey$ message,
	then $C'$ is automatically rejected in accordance with \cref{game_hop:3P-KD:ACCE-int}.
	
	

	\item Finally,
	when $\A$ outputs its guess $b'$,
	then $\A[B]''_3$ outputs the following to its 2P-ACCE game.
	If the test-session $\oracle[U][i]$ has a co-partner $\oracle[S][k]$,
	then $\A[B]''_3$ outputs $(\oracle[S][k],0)$ if $b' = \bsim$ and $(\oracle[S][k], 1)$ otherwise.
	If the test-session does not have a co-partner,
	then $\A[B]''_3$ outputs an arbitrary session together with a random~bit.
\end{itemize}

Note that if the test-session does not have a co-partner then there is no difference between Game~\prevgame{} and Game~\game{},
and $\A[B]''_3$ perfectly simulates it.

If the test-session \emph{has} a co-partner $\oracle[S][k]$,
and $\oracle[S][k].b = 0$ in $\A[B]''_3$'s 2P-ACCE security game,
then $\A[B]''_3$ perfectly simulates Game~\prevgame{}
since the $\Ckey$ message of $\oracle[S][k]$ is an encryption of the actual session key $\key_{AB}$.
On the other hand, if $\oracle[S][k].b = 1$ then $\A[B]''_3$ perfectly simulates Game~\game{}
since the $\Ckey$ message of $\oracle[S][k]$ is an encryption of~$0^{\keylen}$.

It remains to argue that whenever $\A[B]''_3$ uses $\oracle[S][k]$ as its ACCE target-session,
then it is fresh according to predicate $\fresh_{\acce}$ whenever $\oracle[U][i]$ is fresh according to predicate $\fresh_{\akewfs}$.
But this follows by the same arguments that was used to show that $\oracle^*$ was ACCE fresh in the proof of \cref{game_hop:3P-KD:ACCE-int},
hence we omit it.
\end{proof}

\paragraph{Concluding the proof of Theorem~\ref{thm:protocol_3:3P-AKE-}.}
We argue that $\adv_{\protocol_3,f_3}^{\mathsf{G_\game}}(\A)=0$.
By the change in Game~\ref{game_hop:3P-KD:PRF->RF}, 
the session key of the test-session $\oracle[U][i]$ is derived using a random function $\$(A, B)$,
where ``$A$'' and ``$B$'' are the identities of the initiator and responder that $\oracle[U][i]$ believes took part in this protocol run.
We claim that the only other session that holds a session key derived from $\$(\cdot, \cdot)$ using the same identities ``$A$'' and ``$B$'', 
is $\oracle[U][i]$'s partner $\oracle[V][j]$
(if it exists).

First,
note that the random function is evaluated at no more than two sessions:
one initiator session and one server session.
Second,
the session key derived by the server session is delivered to at most one responder session. 
Finally,
the identities used to evaluate $\$(\cdot, \cdot)$  at the initiator and server could potentially be different since $\A$ can modify the identities communicated at the $A$--$S$ link in~\cref{fig:generic_protocol}.

However,
if $\A$ modifies these identities,
then the initiator and server derive independent keys,
which means that the initiator and responder will ultimately  have independent keys too. 
Moreover,
since the communicated identities at the $S$--$B$ link in \cref{fig:generic_protocol} will be different,
the initiator and responder sessions  will not be partners 
(recall that $f_3$-partnering includes the sessions' recorded peers, 
and by Game~\ref{game_hop:3P-KD:ACCE-int} the adversary is unable to change the $\Ckey$ message).
On the other hand,
if the identities were the same,
then the initiator and responder session would necessarily be $f_3$-partners.
This follows because the initiator has the server session as its $f_1$-partner in sub-protocol $\protocol_1$
and the server session's $\Ckey$ message,
if delivered at all,
must be delivered honestly to its $f_2$-partner in sub-protocol $\protocol_2$.
Combined with their agreement on their peers,
this means the initiator and responder session would be partners by the definition of $f_3$.

Altogether,
since the session key of the test-session is derived using a random function,
and since the corresponding $\Ckey$ message leaks nothing about the session key by Game~\ref{game_hop:3P-KD:C_key_encrypt_0},
%$\A$ learns nothing about  session key by revealing any non-partnered session.
the adversary has zero advantage in Game~\game{} as claimed.


\medskip
Combining the bounds from \cref{lemma:3P-KD:ACCE-EA} to \cref{lemma:3P-KD:Ckey_encrypt_0}
we get
\begin{align*}
	\begin{split}
	\adv_{\protocol_3,f_3}^{\operatorname{\mathsf{3P-\akewfs}}} (\A) 
	&\leq \adv_{\protocol_2,f_2}^{\operatorname{\mathsf{\acce-\EA}}}(\A[B]_1) 
 		 + 2n^2 \cdot \adv_{\protocol_1,f_1}^{\operatorname{\mathsf{2P-\akefs}}}(\A[B]_2) 
		  +	2n^2 \cdot \adv_{\mathsf{PRF}}^{\prf}(\A[D]) \\
	&\qquad + 2n^2 \cdot \adv_{\protocol_2,f_2}^{\acce}(\A[B]'_3)
				+ 2n^2 \cdot \adv_{\protocol_2,f_2}^{\acce}(\A[B]''_3) ,			
	\end{split}
\end{align*} 
where $ n = (\numsessions +1 ) \cdot  | \mathcal{I} \cup \mathcal{R} | $.

By letting $\A[B]_3$ be the ACCE adversary that with probability $1/2$ either implements algorithm $\A[B]'_3$ or algorithm $\A[B]''_3$,
the concrete bound  in the statement of \cref{thm:protocol_3:3P-AKE-} follows.
%\qed
\end{proof} %% End of proof of Composition Theorem 1.

\begin{remark}\label{remark:UKS_attack_on_P3_without_channel_binding}
Note that the conclusion above only holds if protocol $\protocol_3$ employs channel binding.
If the identities of $A$ and $B$ where not included in the evaluation of the pseudorandom function $\mathsf{PRF}$,
then $\protocol_3$ would be vulnerable to the simple UKS attack described in \cref{sec:generic_composition_results:modeling_EAP}.
That is,
simply change the responder identity being sent over the (unauthenticated) $A$--$S$ link from ``$B$'' to ``$C$''.
Without channel binding,
$A$ and $C$ would obtain the same session key but disagree on their intended peers,
and hence not be partners.
\end{remark}










\section[Second composition theorem]{Second composition theorem:\\
{\large\texorpdfstring{3P-\akewfstext + 2P-\akenfstext $\implies$ 3P-\akefstext}{3P-AKEw + 2P-AKEnfs -> 3P-AKEfs}}}\label{sec:generic_composition_results:second_composition_result}

In this \lcnamecref{sec:generic_composition_results:second_composition_result} we state and prove our second composition theorem.
This result corresponds to the cryptographic core of full EAP.


\paragraph{Construction.}\label{sec:construction:PI_5}


From a 3P-AKE protocol $\protocol_3$ and a PSK-based 2P-AKE protocol $\protocol_4$,
we construct the 3P-AKE protocol $\protocol_5$ shown in \cref{fig:generic_protocol}.
Specifically,
protocol $\protocol_5$ works as follows.
First sub-protocol $\protocol_3$ is run between $A$, $B$ and $S$ in order to establish an intermediate key $\KDkey$.
Then sub-protocol $\protocol_4$ is run between $A$ and $B$ using $\KDkey$ as the their PSK.
The session key derived in sub-protocol $\protocol_4$ becomes $A$ and $B$'s final session key in protocol $\protocol_5$.






\paragraph{Result.}
Our second composition result shows that protocol $\protocol_5$ is 3P-AKE secure 
if sub-protocol $\protocol_3$ is 3P-\akewfstext  secure and sub-protocol $\protocol_4$ is 2P-\akenfstext secure with  explicit entity authentication. 
We remark that the last requirement is necessary in order for our proof to go through.
In fact,
$\protocol_5$ inherits the property of explicit entity authentication from sub-protocol $\protocol_4$.
Moreover,
while $\protocol_4$ does not necessarily achieve any forward secrecy on its own,
protocol $\protocol_5$ does.
The reason is that within $\protocol_5$, 
sub-protocol $\protocol_4$ is merely used to upgrade the security of $\protocol_3$ from weak forward secrecy to full forward secrecy.
%Specifically, does.
%we have the following results.

For technical reasons,
we additionally need to assume some minimal structure on sub-protocol $\protocol_4$ beyond it being a 2P-AKE protocol.
In particular,
we require that the probability that two sessions at the same party end up with the same local transcript $\localtranscript$ in sub-protocol $\protocol_4$ should be ``small''.
Formally,
we demand that the probability of such a transcript collision should be statistically bounded by some function $\transcriptcollisionfunc$ of the number of parties and sessions. 
This technical requirement is needed in order be able to apply a local partner function to the transcript of sub-protocol $\protocol_4$
(see the proof of \cref{lemma:3P-AKE-EA:EA-p4}).
%at a party and the number of messages that each such session has sent/received.

%\small
\begin{theorem}\label{thm:protocol_5:3P-AKE}
Let $\protocol_5$ be the 3P-AKE protocol constructed from the 3P-AKE protocol $\protocol_3$ and 2P-AKE protocol $\protocol_4$ as described in the construction above.
Let $f_3$ and $f_4$ be partner functions,
where $f_4$ is required to be local.
Then,
for any adversary $\A$ in security experiment \akefstext against protocol $\protocol_5$,
we can create a partner function~$f_5$ and algorithms $\A[B]_1$ and $\A[B]_2$,
such that
\begin{equation}\label{eq:thm:P5:AKE}
%	\begin{split}
	\adv_{\protocol_5,f_5}^{\operatorname{\mathsf{3P-\akefs}}} (\A) 
		\leq 3n^2 \cdot \adv_{\protocol_3,f_3}^{\operatorname{\mathsf{3P-\akewfs}}}(\A[B]_1) 
		+ 2n^2 \cdot \adv_{\protocol_4,f_4}^{\operatorname{\mathsf{2P-\akenfs}}}(\A[B]_2) + 2 \transcriptcollisionfunc, 
%	\end{split}
\end{equation}
where $n = (\numsessions+1) \cdot \left| \mathcal{I} \cup \mathcal{R} \right|$,
$\numsessions$ is the maximum number of sessions that $\A$ creates at each party,
and $\transcriptcollisionfunc = \transcriptcollisionfunc(\left| \mathcal{I} \cup \mathcal{R} \right|, \numsessions)$ is a function that bounds the probability that two sessions at the same party get the same local transcript in protocol $\protocol_4$.
\end{theorem}
\normalsize


The idea of the proof  is as follows.
Using that sub-protocol $\protocol_3$ is 3P-\akewfstext secure,
we can replace the intermediate key $\KDkey$ coming out of $\protocol_3$ with a random key for the test-session.
This then allows us to reduce the 3P-\akefstext security of protocol $\protocol_5$ to the 2P-\akenfstext of sub-protocol $\protocol_4$,
since the now random intermediate key of the test-session is identically distributed with the PSKs used in~$\protocol_4$. 
The partner function $f_5$ will be composed out of $f_3$ and $f_4$,
so that two sessions are $f_5$-partners if and only if they are $f_3$-partners and $f_4$-partners.

The main difficulty of the proof lies in the first step,
i.e., replacing the intermediate key of the test-session with a random key.
The issue is that $\A$ plays in a security game that has \emph{full} forward secrecy,
whereas the reduction $\A[B]_1$ to sub-protocol $\protocol_3$ plays in a security game with only \emph{weak} forward secrecy.
As such,
$\A$ is allowed strictly more $\Corrupt$ queries than what $\A[B]_1$ can do itself.
The question is how $\A[B]_1$ can simulate the 3P-\akefstext security game for $\A$
while still keeping the test-session fresh in its own 3P-\akewfstext security game. 

This is where we use that sub-protocol $\protocol_4$ provides explicit entity authentication.
Essentially,
it guarantees that the test-session must have a partner in protocol $\protocol_5$.
By definition of $f_5$,
this implies that it must also have an $f_3$-partner in sub-protocol $\protocol_3$.
But recall from \cref{table:security_models_summary} that when the test-session has a partner,
then there is no difference between the \akefstext and \akewfstext models!
Thus,
as long as we can show that the test-session has a partner in protocol $\protocol_5$,
we are fine.
Consequently,
we first prove as an initial lemma that protocol $\protocol_5$ provides explicit entity authentication.



\begin{proof}[Proof of \cref{thm:protocol_5:3P-AKE}]
We begin by defining the  partner function $f_5$  for protocol $\protocol_5$.
We construct $f_5$ from the partner functions $f_3$ and $f_4$ given for sub-protocols $\protocol_3$ and $\protocol_4$ 
as follows:
\begin{equation}
	f_{5,T_5}(\oracle) = \oracle' \Longleftrightarrow (f_{3,T_3}(\oracle) = \oracle') \land (f_{4,T_4}(\oracle) = \oracle')	,
\end{equation}
where $T_3$ and $T_4$ are the transcripts one gets from $T_5$ by restricting to the messages pertaining to sub-protocols $\protocol_3$ and $\protocol_4$,
respectively.
The soundness of $f_5$ follows directly from the soundness of $f_3$ and $f_4$. 
Moreover,
like in the proof of the first composition theorem (\cref{thm:protocol_3:3P-AKE-}),
we assume for simplicity that $f_3$ and $f_4$ have perfect soundness.
It follows that $f_5$ has perfect soundness too. 


\subsection{Explicit entity authentication}\label{sec:proof_of_EA_protocol_5}



\begin{lemma}\label{lemma:protocol_5:3P-AKE-EA}
With $f_5$ as defined above,
and everything else as otherwise stated in \cref{thm:protocol_5:3P-AKE},
we have that
\begin{equation*}
	\begin{split}
	\adv_{\protocol_5,f_5}^{\operatorname{\mathsf{3P-\akefs-\EA}}} (\A) 
		&\leq 2n^2 \cdot \adv_{\protocol_3, f_3}^{\operatorname{\mathsf{3P-\akewfs}}}(\A[B]_1) 
			+ n^2 \cdot \adv_{{\protocol_4},{f_4}}^{\operatorname{\mathsf{2P-\akenfs-\EA}}}(\A[B]_2) 
			+ \transcriptcollisionfunc .
	\end{split}
\end{equation*}
\end{lemma}

\begin{proof}\ 

%Recall that the partner function used to prove explicit entity authentication needs to be the same as when proving key-indistinguishability.
%Thus, the partner function $f_5$ we use below is the same as the one used in the proof of Theorem~\ref{thm:protocol_5:3P-AKE}.





%In the following games, 
%let $M_i$ denote the event that a session accepts maliciously in sub-protocol $\protocol_2$ in Game~$i$. 

\setcounter{gamehop}{0}
%\item
\paragraph{Game~\game:}\label{game_hop:3P-AKE:P5:EA:G0}
This is the original 3P-\akefstext-EA security experiment,
hence
\begin{equation}
	\adv_{\protocol_5,f_5}^{\operatorname{\mathsf{G_{\game}-\EA}}}(\A) = \adv_{\protocol_5,f_5}^{\operatorname{\mathsf{3P-\akefs-\EA}}} (\A) .
\end{equation}


\newgame
\paragraph{Game~\game:}\label{game_hop:3P-AKE:P5:EA:transcript_collision_Pi4}
In this game the challenger aborts if two sessions at the same party end up with the same local transcript $\localtranscript$ in sub-protocol $\protocol_4$.
By definition of the function $\transcriptcollisionfunc$ this gives
\begin{equation}
	\adv_{\protocol_5,f_5}^{\operatorname{\mathsf{G_{\prevgame}-\EA}}}(\A)
		\leq \adv_{\protocol_5,f_5}^{\operatorname{\mathsf{G_{\game}-\EA}}}(\A) +  \transcriptcollisionfunc .
\end{equation}


\newgame
\paragraph{Game~\game:}\label{game_hop:3P-AKE:P5:EA:guess_MA_target}
This game implements a selective security game where the adversary is required to commit to the session that will accept maliciously first.
Specifically,
at the beginning of the game the adversary must first choose a pair $(U,i)$,
with $i \in [1, \numsessions]$.
The game then proceeds as in Game~\prevgame{},
except that if some session accepts maliciously before $\oracle[U][i]$,
the challenger aborts the game and outputs $0$ 
(i.e., $\A$ loses).
In particular,
this includes the possibility that $\A$ makes a query that renders $\oracle[U][i]$ unfresh
(which would preclude $\oracle[U][i]$ from accepting maliciously).


\begin{claim}\label{lemma:3P-AKE-EA:selective-security:EA-target}
\begin{equation}
	\adv_{\protocol_5,f_5}^{\operatorname{\mathsf{G_{\prevgame}-\EA}}}(\A)
		\leq \numsessions \cdot | \mathcal{I} \cup \mathcal{R} | 
			\cdot 	\adv_{\protocol_5,f_5}^{\operatorname{\mathsf{G_{\game}-\EA}}}(\A') .
\end{equation}
\end{claim}

\begin{proof}
The proof is essentially the same as for Game~\ref{game_hop:3P-KD:guess_test-session} in \cref{thm:protocol_3:3P-AKE-} (\cref{lemma:3P-KD:guess_test-session}),
only that this time the selective security adversary guesses one session rather than two. 
\end{proof}

In the remaining games,
let $\oracle[U][i]$ denote the session that the adversary commits to in Game~\game{}. 
Note that $\oracle[U][i]$ is not necessarily the same as the test-session chosen by the adversary.

\newgame
\paragraph{Game~\game:}\label{game_hop:3P-AKE:P5:EA:guess_P3_partner}
This game extends the selective security requirement of Game~\prevgame{} by demanding that the adversary also commits to the partner of $\oracle[U][i]$ in sub-protocol $\protocol_3$ (if any).
Specifically,
at the beginning of the game the adversary must pick a pair $(U,i)$ as in Game~\prevgame{},
but it must also pick a pair $(V,j)$,
with $j \in [0, \numsessions]$.
\cref{game_hop:3P-AKE:P5:EA:guess_P3_partner} then proceeds as in Game~\prevgame{},
but it additionally aborts if $\oracle[U][i]$ gets a different $f_3$-partner than $\oracle[V][j]$ in sub-protocol $\protocol_3$.
This includes the case that $\oracle[U][i]$ \emph{gets} an $f_3$-partner if $j = 0$.


\begin{remark}
Note that there is no contradiction between $\oracle[U][i]$ accepting maliciously in protocol $\protocol_5$ according to partner function $f_5$,
while simultaneously having an $f_3$-partner in sub-protocol $\protocol_3$. 
\end{remark}

\begin{claim}\label{lemma:3P-AKE-EA:selective-security:EA-target-p3-partner}
\begin{equation}
	\adv_{\protocol_5,f_5}^{\operatorname{\mathsf{G_{\prevgame}-\EA}}}(\A)
	\leq (\numsessions + 1) \cdot \max \lbrace | \mathcal{I} |, | \mathcal{R} | \rbrace 
		\cdot \adv_{\protocol_5,f_5}^{\operatorname{\mathsf{G_{\game}-\EA}}}(\A').
\end{equation}
\end{claim}

\begin{proof}
Again,
from an adversary $\A$ that plays against the \emph{single} selective security requirement of Game~\prevgame{},
we can create an adversary $\A'$ against the \emph{two} selective security requirements of Game~\game{}.
Basically,
after $\A$ outputs its commitment to a pair $(U,i)$,
then $\A'$ guesses another pair $(V,j)$
(conditioned on the role of $U$),
and outputs $(U,i)$ and $(V,j)$ as its own commitments to Game~\game{}.
%\qed
\end{proof}

In the remaining games,
let $\oracle[V][j]$ denote the (possibly empty) $f_3$-partner of $\oracle[U][i]$ that the adversary commits to in Game~\game{} in addition to $\oracle[U][i]$.

\newgame
\paragraph{Game~\game:}\label{game_hop:3P-AKE:P5:EA:swap_P3_random}
This game proceeds as the previous one,
but it replaces the intermediate key $\KDkey$ of $\oracle[U][i]$ and $\oracle[V][j]$ in sub-protocol $\protocol_3$ with a random key $\KDkeyrand$.


\begin{claim}\label{claim:fantastic-claim}
\begin{equation}
	\adv_{\protocol_5,f_5}^{\operatorname{\mathsf{G_{\prevgame}-\EA}}}(\A)
	\leq \adv_{\protocol_5,f_5}^{\operatorname{\mathsf{G_{\game}-\EA}}}(\A)
	+ \adv_{\protocol_3,f_3}^{\operatorname{\mathsf{3P-\akewfs}}}(\A[B]_1)  .
\end{equation}
\end{claim}
\begin{proof}

We show that if it is possible to distinguish Game~\prevgame{} and Game~\game{},
then we can create an algorithm $\A[B]_1$ that breaks the 3P-\akewfstext security of sub-protocol $\protocol_3$.
Reduction $\A[B]_1$ begins by choosing a random bit $\bsim$.
%and guessing the two sessions $\oracle[U][i]$ and $\oracle[V][j]$ as in Game~\ref{game_hop:3P-AKE:P5:EA:guess_MA_target} and Game~\ref{game_hop:3P-AKE:P5:EA:guess_P3_partner}.
It then runs $\A$ and implements all the abort conditions introduced so far.
All of $\A$'s $\Send$ queries that pertain to the 3P-AKE sub-protocol $\protocol_3$,
$\A[B]_1$ forwards to its 3P-\akewfstext security game. 
For all  sessions different from $\oracle[U][i]$ and $\oracle[V][j]$,
$\A[B]_1$  obtains their intermediate keys $\KDkey$ in sub-protocol $\protocol_3$ by making a corresponding $\Reveal$ query to its 3P-\akewfstext game. 

On the other hand,
when the first session out of $\oracle[U][i]$ and $\oracle[V][j]$ accepts in sub-protocol $\protocol_3$,
then $\A[B]_1$ instead makes a $\Test$ query to obtain its intermediate key $\KDkey$ in protocol $\protocol_5$.
Let $\key^*$ denote this key.
When the second session out of $\oracle[U][i]$ and $\oracle[V][j]$ accepts in sub-protocol $\protocol_3$, 
it is assigned the same key $\key^*$ as its intermediate key in sub-protocol $\protocol_3$. 


$\A[B]_1$ simulates sub-protocol $\protocol_4$ itself using the intermediate keys it obtained for sub-protocol $\protocol_3$ as the PSKs for $\protocol_4$.
To answer $\A$'s $\Test$ query,
$\A[B]_1$ uses the bit $b_{sim}$ it drew in the beginning of the simulation.
Finally, 
when $\oracle[U][i]$ accepts in protocol $\protocol_5$,
then $\A[B]_1$ stops it simulation and outputs a 0 to its 3P-\akewfstext game if $\oracle[U][i]$ accepted maliciously,
and a 1 otherwise.
\smallskip

Before analyzing $\A[B]_1$'s advantage,
we argue that if $\oracle[U][i]$ accepts maliciously in $\A[B]_1$'s simulation,
then both $\oracle[U][i]$ and $\oracle[V][j]$ are valid  test-targets in its 3P-\akewfstext game,
i.e., fresh according to predicate $\fresh_{\akewfs}$.
Recall that $\A[B]_1$ selects its test-session based on which of $\oracle[U][i]$ and $\oracle[V][j]$  accepted first in sub-protocol $\protocol_3$.
We consider three cases: 
\begin{itemize}

	\item \emph{$j = 0$}:
	In this case $\oracle[U][i]$ is chosen as the test-session,
	and $\A[B]_1$ makes no $\Reveal$ query towards it in its 3P-\akewfstext game
	because it uses the $\Test$ query to obtain its intermediate key.
	Since $\oracle[U][i]$ does not have an $f_3$-partner ($j = 0$),
	there are of course no other $\Reveal$ queries that could have made $\oracle[U][i]$ unfresh.
	
	We claim that $\A[B]_1$ also never issued a $\Corrupt$ query to $\oracle[U][i]$'s peers.
	To see this,
	note that if $\oracle[U][i]$ is to accept maliciously in protocol $\protocol_5$,
	then it must be fresh according to predicate $\fresh_{\akefs}$.
	In particular,
	this means that $\A$ cannot issue any $\Corrupt$ queries to $\oracle[U][i]$'s peers \emph{before} $\oracle[U][i]$ accepted.\footnote{Recall 
	that predicate $\fresh_{\akefs}$ forbids any $\Corrupt$ query to a session's peers if (1) it does not have a partner, and (2) it has not accepted yet. 
	This corresponds exactly to the setting we are in when a session accepts maliciously.
	}  
	But
	$\A[B]_1$ stops its simulation immediately once $\oracle[U][i]$ accepts,
	so no $\Corrupt$ query will actually be forwarded to $\oracle[U][i]$'s peers  in $\A[B]_1$'s 3P-\akewfstext experiment in this case.

	


	\item \emph{$j \neq 0$ and $\oracle[U][i]$ chosen as test-session}:  
	Again, $\A[B]_1$ makes no $\Reveal$ query towards $\oracle[U][i]$ or $\oracle[V][j]$ in its 3P-\akewfstext game,
	since they are both handled by the $\Test$ query.
%	This is because it obtains their intermediate keys from sub-protocol $\protocol_3$ through the $\Test$ query (to $\oracle[U][i]$).
	Moreover,
	since $\oracle[U][i]$ has an $f_3$-partner ($j \neq 0$),
	it remains \akewfstext  fresh even if its peers are corrupted.
	

	
	
	\item \emph{$j \neq 0$ and $\oracle[V][j]$ chosen as test-session}:
	By symmetry of the $f_3$ partner function,
	$\oracle[V][j]$ has $\oracle[U][i]$ as its $f_3$-partner,
	and thus the argument is the same as for the above case.
%	Thus, $\A[B]_1$ makes no $\Reveal$ query towards $\oracle[V][j]$ or its $f_3$-partner $\oracle[U][i]$.
%%	\footnote{The
%%		potential issue was that $\oracle[V][j]$ could have had a different $f_3$-partner than $\oracle[U][i]$.
%%		If $\A$ then asked to reveal that session,
%%		$\A[B]_1$ would have answered with a $\Reveal$ query,
%%		hence breaking the \akewfstext  freshness of $\oracle[V][j]$ in its own 3P-AKE experiment.
%%	}
%	Moreover,
%	since $\oracle[V][j]$ has an $f_3$-partner,
%	it remains \akewfstext  fresh even if its peers are corrupted.
%
%


\end{itemize}

Taken together,
the above cases show that no-matter which one of $\oracle[U][i]$ and $\oracle[V][i]$ was selected as the test-session by $\A[B]_1$,
it will be fresh according to predicate $\fresh_{\akewfs}$ in $\A[B]_1$'s 3P-\akewfstext game if $\oracle[U][i]$ accepted maliciously.

\medskip

Finally,
we analyze $\A[B]_1$'s advantage.
Let $b$ denote the challenge-bit used to answer $\A[B]_1$'s $\Test$ query in its 3P-\akewfstext game.
%and let $\mathsf{MA}_i$ denote the event that $\oracle[U][i]$ accepted maliciously in Game~$i$.
%and let ``$\A[B]_1 \Rightarrow b_{out}$'' denote the event that $\A[B]_1$ outputs $b_{out}$ to its 3P-AKE experiment.
If $b=0$,
then $\A[B]_1$'s $\Test$ query is answered with a real key,
and $\A[B]_1$ simulates Game~\prevgame{} perfectly for $\A$ up until the point when $\oracle[U][i]$ accepts (in protocol $\protocol_5$).
Thus: 
\begin{align}\label{eq:proof:p5:EA:swap_keys_p3:MA_b=0}
	\Pr[\Exp_{\protocol_3,\queryset}^{\akewfs}(\A[B]_1) \Rightarrow 1 \mid b = 0] = \adv_{\protocol_5,f_5}^{\operatorname{\mathsf{G_{\prevgame}-\EA}}}(\A) .
\end{align}


On the other hand,
if $b=1$,
meaning that $\A[B]_1$'s $\Test$ query is answered with a random key,
then $\A[B]_1$ perfectly simulates Game~\game{}.
Since $\A[B]_1$ only outputs a $1$ to its 3P-\akewfstext security game if $\A$ loses in $\A[B]_1$'s simulation,
i.e.,
if $\oracle[U][i]$ does \emph{not} accept maliciously,
we have
\begin{align}\label{eq:proof:p5:EA:swap_keys_p3:MA_b=1}
	\Pr[\Exp_{\protocol_3,\queryset}^{\akewfs}(\A[B]_1) \Rightarrow 1 \mid b = 1] = 1 - \adv_{\protocol_5,f_5}^{\operatorname{\mathsf{G_{\game}-\EA}}}(\A)  . 
\end{align}


Thus, $\A[B]_1$'s advantage is
%\small
\begin{align}
	\begin{split}
	\adv_{\protocol_3,f_3}^{\operatorname{\mathsf{3P-\akewfs}}}(\A[B]_1)
		&= 2 \cdot \Pr[\Exp_{\protocol,\queryset}^{\akewfs}(\A[B]_1) \Rightarrow 1 ] - 1
	\end{split} \\
	\begin{split}
		&= \Pr[\Exp_{\protocol,\queryset}^{\akewfs}(\A[B]_1) \Rightarrow 1 \mid b = 0] \\
			& \qquad	+ \Pr[\Exp_{\protocol,\queryset}^{\akewfs}(\A[B]_1) \Rightarrow 1 \mid b = 1] - 1
	\end{split} \\
%	\begin{split}
%			&=   1 \cdot \Pr[\mathsf{MA}^{\A[B]_1} \mid b = 0] + \frac{1}{2} \cdot \left( 1 - \Pr[ \mathsf{MA}^{\A[B]_1} \mid b = 0] \right)   \\
%			&\quad - 1 \cdot   \Pr[\mathsf{MA}^{\A[B]_1} \mid b = 1]  - \frac{1}{2} \cdot \left( 1 - \Pr[\mathsf{MA}^{\A[B]_1} \mid b = 1] \right) 	
%		\end{split} \\
%	&\hspace{-12.25pt}\overset{\eqref{eq:proof:p5:EA:swap_keys_p3:MA_b=0}+\eqref{eq:proof:p5:EA:swap_keys_p3:MA_b=1}}{=}  
	&= \adv_{\protocol_5,f_5}^{\operatorname{\mathsf{G_{\prevgame}-\EA}}}(\A) 
			- \adv_{\protocol_5,f_5}^{\operatorname{\mathsf{G_{\game}-\EA}}}(\A) ,
\end{align}
as stated in the \lcnamecref{claim:fantastic-claim}.
\end{proof}


Since the intermediate key $\KDkey$ of $\oracle[U][i]$ and $\oracle[V][j]$ is replaced with an independent uniformly random key in Game~\game,
we can finally show that if $\oracle[U][i]$ accepts maliciously in Game~$\game$,
then it must have accepted maliciously in sub-protocol~$\protocol_4$.


\begin{claim}\label{lemma:3P-AKE-EA:EA-p4}
\begin{equation}
	\adv_{\protocol_5,f_5}^{\operatorname{\mathsf{G_{\game}-\EA}}}(\A)
	\leq  \adv_{\protocol_4,f_4}^{\operatorname{\mathsf{2P-\akenfs-\EA}}}(\A[B]_2)  .
\end{equation}
\end{claim}



\begin{proof}

If $\A$ wins in Game~\game{},
we can create the following algorithm $\A[B]_2$ which breaks the explicit entity authentication of sub-protocol $\protocol_4$.
Algorithm $\A[B]_2$ begins by creating all the long-term keys for sub-protocol $\protocol_3$
and drawing a random bit $b_{sim}$.
%and guesses the two sessions $\oracle[U][i]$ and $\oracle[V][j]$ as in Game~\ref{game_hop:3P-AKE:P5:EA:guess_MA_target} and Game~\ref{game_hop:3P-AKE:P5:EA:guess_P3_partner}.
It then runs $\A$ and simulates sub-protocol $\protocol_3$ itself
using the intermediate keys coming out of sub-protocol $\protocol_3$ as the PSKs for sub-protocol $\protocol_4$.
Additionally,
for all sessions different from $\oracle[U][i]$ and $\oracle[V][j]$,
$\A[B]_2$ also simulates sub-protocol $\protocol_4$ itself.
On the other hand,
to simulate sub-protocol $\protocol_4$ for $\oracle[U][i]$ and $\oracle[V][j]$,
$\A[B]_2$ instead forwards the corresponding queries to its own 2P-\akenfstext security game. 
Once $\A$ stops and outputs a guess $b'$,
then $\A[B]_2$ stops too.



For sessions different from $\oracle[U][i]$ and $\oracle[V][j]$,
$\A[B]_2$  simulates Game~\game{} perfectly since it created all the keys.  
However,
$\A[B]_2$ also perfectly simulates $\oracle[U][i]$ and $\oracle[V][j]$,
since by the change in Game~\game{},
their intermediate key $\KDkey$ from sub-protocol $\protocol_3$ is replaced with an independent uniformly random key $\KDkeyrand$.
This is identically distributed to the long-term PSK used in $\A[B]_2$'s 2P-\akenfstext security game,
so by forwarding the $\Send$ queries directed to $\oracle[U][i]$ and $\oracle[V][j]$ to its 2P-\akenfstext game,
$\A[B]_2$ perfectly simulates these sessions too.

It remains to argue that if $\oracle[U][i]$ accepts maliciously in Game~\game{},
then it must also have accepted maliciously in  $\A[B]_2$'s 2P-\akenfstext security game.
First we claim that session $\oracle[V][j]$ cannot be $\oracle[U][i]$'s $f_4$-partner in sub-protocol~$\protocol_4$.

\begin{itemize}
	\item If $j = 0$,
	then $\A[B]_2$ never creates a corresponding proxy session in its 2P-\akenfstext security game,
	hence $\oracle[U][i]$ cannot possibly have a partner there.

	\item If $j \neq 0$,
	then $\oracle[V][j]$ by definition
	(\cref{game_hop:3P-AKE:P5:EA:guess_P3_partner})
	must be $\oracle[U][i]$'s $f_3$-partner in sub-protocol $\protocol_3$.
	But if $\oracle[V][j]$ was \emph{also} the $f_4$-partner of $\oracle[U][i]$ in sub-protocol $\protocol_4$,
	then by the construction of $f_5$ from $f_3$ and $f_4$, $\oracle[V][j]$ would be $\oracle[U][i]$'s $f_5$-partner---contradicting the fact that $\oracle[U][i]$ was supposed to accept maliciously.



\end{itemize}

There is one subtlety with the arguments above:
technically we need to show that $\oracle[U][i]$ and $\oracle[V][i]$ are $f_4$-partners in Game~\game{} if and only if they are $f_4$-partners in $\A[B]_2$'s 3P-\akenfstext-EA security game.
However, the $T_4$ transcript from Game~\game{} contains \emph{many} sessions,
while the transcript $T_{\A[B]_2}$ from $\A[B]_2$'s  2P-\akenfstext security game contains at most two: $\oracle[U][i]$ and $\oracle[V][j]$.
In particular,
transcript $T_{\A[B]_2}$ is the restriction $\restrict{T_4}{\oracle[U][i], \oracle[V][i]}$ of $T_4$.
But evaluating the same partner function on these two transcripts does not necessarily have to yield the same answer.
This is where we use the assumption that $f_4$ is a local partner function
(see \cref{def:local_partnering}).
Namely,
since $f_4$ is local,
we have that
$\oracle[U][i]$ and $\oracle[V][i]$ are $f_4$-partners based on $T_4$ if and only if they are $f_4$-partners based on the restriction $\restrict{T_4}{\oracle[U][i], \oracle[V][i]}$---provided that $T_4$ is a unique transcript,
i.e., no two sessions at the same party have the same local transcript $\localtranscript$.
But this is exactly what the abort condition in \cref{game_hop:3P-AKE:P5:EA:transcript_collision_Pi4} ensures. 


Having shown that $\oracle[U][i]$ does not accept with an $f_4$-partner in $\A[B]_2$'s 2P-\akenfstext-EA security game,
we only have to show that $\oracle[U][i]$ is fresh according to predicate $\fresh_{\akenfs}$.
But this is true since $\A[B]_2$ makes no $\Corrupt$ query at all in its 2P-\akenfstext security game,
and also makes no $\Reveal$ query to $\oracle[U][i]$.
%\footnote{Whether or not $\A[B]_2$ makes a $\Reveal$ query to $\oracle[V][i]$ is irrelevant since $\oracle[U][i]$ and $\oracle[V][j]$ are not partners.
%}.
Thus $\oracle[U][i]$ accepts maliciously in $\A[B]_2$'s 2P-\akenfstext security game whenever it accepts maliciously in Game~\game{},
proving \cref{lemma:3P-AKE-EA:EA-p4}.
\end{proof}

Combining the bounds from \cref{game_hop:3P-AKE:P5:EA:transcript_collision_Pi4} to \cref{game_hop:3P-AKE:P5:EA:swap_P3_random} with \cref{lemma:3P-AKE-EA:EA-p4} yields \cref{lemma:protocol_5:3P-AKE-EA}.
%\qed
\end{proof} %% End of 2nd composition theorem (EA)






\subsection{\texorpdfstring{\akefstext}{AKEfs} security}

Given \cref{lemma:protocol_5:3P-AKE-EA},
which shows that $\protocol_5$ provides explicit entity authentication,
we can now proceed with the proof of \cref{thm:protocol_5:3P-AKE}.

\setcounter{gamehop}{0}
\item
\paragraph{Game~\game:}\label{game_hop:3P-AKE:P5:EA:AKE:G0}
This is the real 3P-\akefstext security game, hence
\begin{equation*}
	\adv_{\protocol_5,f_5}^{\mathsf{G_\game}}(\A) = \adv_{\protocol_5{},f_5}^{\operatorname{\mathsf{3P-}\akefs}}(\A) .
\end{equation*}

\newgame
\paragraph{Game~\game:}\label{game_hop:3P-AKE:P5:abort_on_MA}
In this game, the challenger aborts if a session accepts maliciously in protocol $\protocol_5$,
whence
\begin{equation}
	\adv_{\protocol_5,f_5}^{\mathsf{G_{\prevgame{}}}}(\A) 
	\leq \adv_{\protocol_5,f_5}^{\mathsf{G_\game}}(\A) 
	+ \adv_{\protocol_5,f_5}^{\operatorname{\mathsf{3P-\akefs-\EA}}} (\A).
\end{equation}
%and $\adv_{{\protocol_5},\A,{f_5}}^{\operatorname{\mathsf{3P-AKE-EA}}}(\keylen)$ will be bounded in \cref{lemma:protocol_5:3P-AKE-EA}.

The remaining game hops are essentially the same as those of \cref{lemma:protocol_5:3P-AKE-EA}.
Hence,
we merely state their descriptions and corresponding bounds, 
but omit the proofs.


\newgame
\paragraph{Game~\game:}\label{game_hop:3P-AKE:P5:EA:AKE:transcript_collision_Pi4}
In this game the challenger aborts if two sessions at the same party end up with the same local transcript $\localtranscript$ in sub-protocol $\protocol_4$.
By definition of the function $\transcriptcollisionfunc$ this gives
\begin{equation}
	\adv_{\protocol_5,f_5}^{\operatorname{\mathsf{G_{\prevgame}}}}(\A)
		\leq \adv_{\protocol_5,f_5}^{\operatorname{\mathsf{G_{\game}}}}(\A) + \transcriptcollisionfunc .
\end{equation}


\newgame
\paragraph{Game~\game:}\label{game_hop:3P-AKE:P5:guess-test-session-and-partner}
This game implements a selective security game where the adversary has to commit to a test-session and its partner
(required to exist by \cref{game_hop:3P-AKE:P5:abort_on_MA}).
Specifically,
at the beginning of the game the adversary must output two pairs $(U,i)$ and $(V,j)$.
The game then proceeds as in Game~\prevgame{},
except that if either of the following events occur,
then the challenger penalizes the adversary by outputting a random bit at the end.
\item
\begin{enumerate}[(i)]
	\item Neither $\oracle[U][i]$ nor $\oracle[V][j]$ were selected as the test-session by $\A$.
	
	\item $\oracle[U][i]$ and $\oracle[V][j]$ did not get partnered to each other.
	
%	\item $\A$ makes a $\Reveal$ or $\Corrupt$ query that would make either $(U,i)$ or $(V,j)$ unfresh.

\end{enumerate}


\begin{claim}
\begin{equation}
	\adv_{\protocol_5,f_5}^{\mathsf{G_{\prevgame{}}}}(\A) 
		\leq (\numsessions^2 \cdot \left| \mathcal{I} \right| \cdot \left| \mathcal{R} \right| ) / 2 \cdot \adv_{\protocol_5,f_5}^{\mathsf{G_\game}}(\A') .
\end{equation}
\end{claim}




In the remaining games,
let $\oracle[U][i]$ and $\oracle[V][j]$ denote the two sessions the adversary commits to in the selective security game.


\newgame
\paragraph{Game~\game:}\label{game_hop:3P-AKE:3P-KD-swap-random}
This game proceeds as the previous one,
but it replaces the intermediate key $\KDkey$ of $\oracle[U][i]$ and $\oracle[V][j]$ in sub-protocol $\protocol_3$ with a random key $\KDkeyrand$.

\begin{claim}\label{lemma:3P-AKE:3P-KD-swap-random}

\begin{equation}
	\adv_{\protocol_5,f_5}^{\mathsf{G_{\prevgame{}}}}(\A) 
	\leq \adv_{\protocol_5,f_5}^{\mathsf{G_\game}}(\A) 
	+ 2 \cdot \adv_{\protocol_3,f_3}^{\operatorname{\mathsf{3P-\akewfs}}}(\A[B]_1)  .
\end{equation}
\end{claim}


Finally, any successful attack on protocol $\protocol_5$ in Game~\game{} can be transformed into a successful attack on sub-protocol $\protocol_4$. 


\begin{claim}\label{lemma:3P-AKE:2P-AKE-swap-random}
\begin{equation}
	\adv_{\protocol_5,f_5}^{\mathsf{G_{\game{}}}}(\A) 
	\leq  \adv_{\protocol_4,f_4}^{\operatorname{\mathsf{2P-\akenfs}}}(\A[B]_2)  .
\end{equation}
\end{claim}

\paragraph{Concluding the proof of Theorem~\ref{thm:protocol_5:3P-AKE}.}

Combining the bounds from  \cref{game_hop:3P-AKE:P5:EA:AKE:G0} to \cref{game_hop:3P-AKE:3P-KD-swap-random} with \cref{lemma:3P-AKE:2P-AKE-swap-random}, we get
\begin{align}\label{eq:second_composition:collecting_probabilities}
	\begin{split}
	\adv_{\protocol_5,f_5}^{\operatorname{\mathsf{3P-\akefs}}} (\A) 
		&\leq \adv_{\protocol_5,f_5}^{\operatorname{\mathsf{3P-\akefs-\EA}}} (\A)
			 + 2n^2 \cdot \adv_{\protocol_3,f_3}^{\operatorname{\mathsf{3P-\akewfs}}}(\A[B]_1)   \\
		& \qquad + n^2 \cdot \adv_{\protocol_4,f_4}^{\operatorname{\mathsf{2P-\akenfs}}}(\A[B]_2) 
			+ \transcriptcollisionfunc .
	\end{split} 
\end{align}

The concrete bound of \cref{thm:protocol_5:3P-AKE} now follows by applying \cref{lemma:protocol_5:3P-AKE-EA} to \cref{eq:second_composition:collecting_probabilities}.
\end{proof} %% End of 2nd composition theorem












\section{Application to EAP}\label{sec:EAP}

Our two composition theorems (\cref{thm:protocol_3:3P-AKE-} and \cref{thm:protocol_5:3P-AKE})
apply to the basic and full variants of EAP respectively,
as defined in \cref{sec:generic_composition_results:modeling_EAP}.
Specifically,
in \cref{thm:protocol_3:3P-AKE-} we identify sub-protocol $\protocol_1$ with the EAP method run between the client and the server,
and sub-protocol $\protocol_2$ with the key-transport protocol between the server and the authenticator. 
By a suitable instantiation of these building blocks,
and assuming that the EAP method provides channel binding,
we immediately get from \cref{thm:protocol_3:3P-AKE-} that basic EAP ($\protocol_3$) is a secure 3P-AKE protocol in the weak forward secrecy model 3P-\akewfstext.
This result can then be combined with \cref{thm:protocol_5:3P-AKE},
where we identify sub-protocol $\protocol_3$ with basic EAP and sub-protocol $\protocol_4$ with the subsequent Security Association Protocol between the client and the authenticator.
It follows immediately that full EAP ($\protocol_5$) is a secure 3P-AKE protocol in the full forward secrecy model \akefstext.

In \cref{sec:EAP-TLS-security} we will show that the EAP method EAP-TLS~\cite{IETF:RFC5216:EAP-TLS} satisfies the requirements on sub-protocol $\protocol_1$ in \cref{thm:protocol_3:3P-AKE-}.
Likewise,
in \cref{sec:802.11} we show that IEEE~802.11~\cite{IEEE:2012:802.11} satisfies the requirements on sub-protocol~$\protocol_4$ in \cref{thm:protocol_5:3P-AKE}.
Thus,
it remains to demonstrate that the key-transport protocol between the server and the authenticator satisfies the requirements on sub-protocol $\protocol_2$ in \cref{thm:protocol_3:3P-AKE-}.

As we argued in \cref{sec:generic_composition_results:modeling_EAP:AAA},
the security properties provided by the MPPE encryption method employed by RADIUS are largely unknown.
It is therefore difficult to assess whether RADIUS alone can safely instantiate our first composition theorem.
On the other hand,
RADIUS is commonly run on top of a secure channel protocol like TLS or IPsec.
%This is also the case for RADIUS' successor protocol Diameter~\cite{IETF:RFC6733:DIAMETER},
%which is another popular key-transport protocol.
In this case the security reduces to that of the underlying secure channel protocol.
Both TLS and IPsec are well-studied,
and have received large amounts of formal analysis.
In particular,
a number of works have shown TLS to be a secure ACCE protocol
\cite{C:JKSS12,C:KraPatWee13,EPRINT:KohSchSch13,BrzuskaFSWW:2012:less_is_more,PKC:LSYKS14},
so in \cref{thm:protocol_3:3P-AKE-} we can for instance set sub-protocol $\protocol_2$ to be RADIUS-over-TLS~\cite{IETF:RFC6614:RADIUS_over_TLS}.


 



\subsection{EAP without channel binding}\label{sec:EAP:without_channel_binding}

\cref{thm:protocol_3:3P-AKE-} requires that the EAP method provides channel binding.
Without it,
EAP  becomes vulnerable to the UKS attack described in \cref{sec:generic_composition_results:modeling_EAP:EAP_method}.
Unfortunately,
many concrete EAP methods do not provide channel binding.
Because the communication between the client and the server is normally routed through the authenticator,
a malicious authenticator can trivially modify the information presented to the two sides. 
As a consequence,
without channel binding it suffices to compromise a \emph{single} authenticator in order to compromise an entire network. 
Since authenticators are typically low-protected devices,
such as wireless access points,
this represents  a substantial attack vector on larger enterprise networks.


Interestingly,
a very similar situation can be found in the  UMTS and LTE mobile networks.
UMTS and LTE employ a key exchange protocol called AKA which is structured almost identically to the EAP protocol:
a mobile client that wants to connect to a base station first has to authenticate to its home operator.
The home operator then transmits a list of so-called \emph{authentication vectors}
(which in particular includes a session key) to the base station in much the same way the server forwards the session key to the authenticator in EAP.  
Moreover,
similar to many EAP methods,
the AKA protocol lacks channel binding for its authentication vectors. 
In their analysis of the AKA protocol,
Alt et al.~\cite[§5]{ACNS:AFMOR16} noted this lack of channel binding,
and suggested a fix which is almost identical to the key-derivation approach analyzed in this \lcnamecref{sec:generic_composition_results}. 


\subsection{Channel binding scope}\label{sec:EAP:channel-binding_scope}
\cref{thm:protocol_3:3P-AKE-} assumes that the channel binding includes the identity of the client and the authenticator in order to bind them cryptographically to the session key.  
However,
this fine-grained scope of the channel binding might not be relevant all situations.
For example,
in a WLAN supported by many access points,
the client does not actually care about \emph{which} specific access point it connects to,
as long as it connects to a legitimate access point of that WLAN.
Thus, in this case the granularity of the channel binding does not have to be at the individual access point level,
but rather at the WLAN level,
defined by all the access points broadcasting the same network identifier (SSID).  
Of course,
by doing so the protection  provided by the channel binding will be weaker. 
In particular,
when channel binding occurs at the individual level,
then the corruption of a single access point will not influence clients connecting to access points having a \emph{different} identity.
On the other hand, 
when channel binding occurs at the network level,
then a single corrupted access point will affect \emph{all} connections within that WLAN.
In this case,
channel binding only protects connections occurring in networks having a different SSID.


More generally,
the information included in the channel binding defines the scope of the protection it provides,
and can include more than just identities.
For instance,
physical media types,
data rates, cost-information,
channel frequencies,
can all be used as input to the channel binding
(see \cite{ClancyH:2009:making_the_case_EAP_cb} for a discussion of these possibilities). 
The specifications for channel binding within EAP~\cite{IETF:draft:EAP-channel-binding,IETF:RFC6677:EAP-channel-binding}
leave open exactly what type of information should go into the binding,
because the amount of information that will be available to both the client and the server may vary.







