\section{Introduction}\label{sec:introduction}

%\subsection*{Douglas' introduction}

Secure channel protocols are widely used in practice to allow two parties to authenticate each other and securely transmit data.  
A common design paradigm is to use an \emph{authenticated key exchange (AKE)} protocol to authenticate parties based on public key certificates and to establish a session key, and then use a \emph{stateful authenticated encryption} scheme to encrypt and authenticate the transmission of application data.  
Real-world secure channel protocols such as TLS, SSH, IPsec, Google's QUIC, the EMV chip-and-pin system, and IEEE 802.11i all follow this paradigm.  

For theoreticians, this paradigm is desirable because it allows for modular proofs via composability.  
A classic result by Canetti and Krawczyk \cite{EC:CanKra01} shows how to provably construct a secure channel by running a key exchange protocol that satisfies standard key indistinguishability notions, and then using the key output by the AKE protocol as the symmetric key in authenticated encryption.  

For practitioners, this paradigm is desirable because it is efficient and allows to use and combine simple software and hardware components in a variety of ways to form the overall system.

Despite the merits of modularity, most real-world designs are not as clean.  
In TLS versions up to 1.2, a key exchange protocol, the so-called handshake protocol, 
is used to establish a premaster secret, 
which is then used to derive a master secret, 
which is then used to derive session keys.  
The final messages of the handshake protocol are encrypted using the session keys, and then application data can be sent, encrypted using the same session keys.  
SSH has a similar design.  
In this design, the session keys do \emph{not} satisfy the standard key indistinguishability notion for key exchange security: 
an adversary can decide whether they have been given the real session key or a random one simply by trial decrypting the encrypted handshake messages.  

Early work on proving the security of TLS avoided this problem by showing that a modified version 
of the TLS handshake yields indistinguishable session keys \cite{JC:MorSmaWar10},
 but this is unsatisfactory since it does not consider the TLS protocol as used in practice.  
In 2012, Jager, Kohlar, Sch{\" a}ge, and Schwenk (JKSS) \cite{C:JKSS12} introduced the \emph{authenticated and confidential channel establishment} (ACCE) security notion, 
which treats the key exchange and authenticated encryption as a single monolithic object, 
allowing them to prove security of the signed Diffie--Hellman ciphersuites in the unmodified TLS 1.2 protocol.  
ACCE has been applied or adapted to prove security of most other TLS ciphersuites \cite{EPRINT:KohSchSch13,C:KraPatWee13,PKC:LSYKS14}, 
as well as SSH~\cite{CCS:BDKSS14}, 
QUIC~\cite{SP:LJBN15,CCS:FisGue14},
and the EMV chip-and-pin system~\cite{CCS:BSWW13}.

The ACCE notion is not necessarily ideal to cryptographers; 
its monolithic nature can make modular analysis more difficult, 
and in particular individual components of ACCE-secure protocols cannot necessarily be used independently.  
For example, 
although TLS 1.2's signed Diffie--Hellman ciphersuite is ACCE-secure, 
one has no security assurance that the session key satisfies any independent security notion: 
we only have the assurance that the session key is safe to use with the corresponding authenticated encryption scheme in the manner described by the protocol.

Moreover, practitioners seem to like to use the TLS handshake in order to establish keying material for their own purposes.  
A prominent example is the EAP-TLS protocol~\cite{IETF:2008:RFC5216-EAP-TLS},
which uses the TLS handshake to derive a session key between two peers in the Extensible Authentication Protocol (EAP)~\cite{IETF:RFC3748:EAP}.
%\footnote{EAP can be used to provide authentication in a variety of network protocols, 
%such as the Point-to-Point Protocol 
%(PPP, Point-to-Point Tunnelling Protocol (PPTP), 
%Layer 2 Tunnelling Protocol (L2TP), 
%IEEE 802 wired networks,
%and IEEE 802.11 wireless networks.}  
More generally, the practice of \emph{exporting} additional keys from the master secret in the TLS handshake has been formalized in the proposed IETF standard RFC~5705 on TLS key material exporters~\cite{IETF:RFC:5705:TLS-key-exporters}.

However, 
is it actually safe to use keys exported from the master secret in the TLS handshake? 
Solely assuming ACCE security of TLS does not at first sight seem to say anything about the \emph{internal} variables of TLS,
such as the master secret. 
However, interestingly, 
inspired by Morrissey, Smart, and Warinschi (MSW)~\cite{JC:MorSmaWar10} we can show that the
ACCE security of TLS implies that the master secret is \emph{unpredictable}. 
If the master secret were predictable, then we would be able to break the security of the ACCE channel. 
This intuition lies at the heart of our proof which uses the ACCE property of TLS in a (semi-)black-box way.

\paragraph{Our contributions.}
In this paper we analyze the security of key exporters from ACCE protocols in the provable security setting.  Concretely,
for TLS we show that if one derives an additional exported key from the TLS master secret---independently 
of the other handshake messages---then TLS (outputting this additional exported key as the session key) constitutes a secure AKE protocol in the sense of Bellare and Rogaway \cite{C:BelRog93}.  
However, while our starting point is the TLS protocol, 
our result is in fact more general, 
pertaining to a wider class of protocols which we call \emph{TLS-like ACCE protocols}.  
Roughly speaking, 
these are protocols which satisfy the ACCE security notion and, 
like TLS, establish a master secret during the handshake, 
and from the master secret derive both the channel encryption key and the additional exported key.  
Apart from this requirement, our result has no other dependencies on the specifics of the protocol.
In other words, our main result is a general theorem showing that the transformation specified by EAP-TLS as a key exporter turns any ACCE protocol which has a concept of a master secret into an AKE protocol.

An immediate application of our result is a proof of security in the Bellare-Rogaway model for TLS Key Material Exporters~\cite{IETF:RFC:5705:TLS-key-exporters} and EAP-TLS~\cite{IETF:2008:RFC5216-EAP-TLS}.
The former has never been subject to a formal security analysis,
while the latter has only been analyzed  in the symbolic model by He et al.~\cite{CCS:HSDDM05}
who gave a proof in the context of IEEE~802.11.




%\paragraph{Overview of our result.}
\paragraph{Motivation for our approach.}
MSW~\cite{JC:MorSmaWar10} proved that a modified version of the TLS handshake yields indistinguishable session keys.
Specifically,
they considered a variant of TLS were the final messages are sent unencrypted.
As an intermediate step in their analysis, they showed that the TLS master secret is unpredictable,
i.e., that no adversary is able to output the full master secret of a fresh target session.
They modeled the key derivation function (KDF) in TLS as a random oracle, 
and as the inputs to the random oracle are unpredictable, 
the session keys derived from the master secret are indistinguishable from random.
%Note that indistinguishability of the session keys requires that the session keys not be used inside the
%handshake itself. 
%Hence,
%this is precisely the modification of TLS that MSW considered, namely
%sending the last messages of the handshake unencrypted.

Similar to MSW, we want to use the fact that the master secret is unpredictable to show that \emph{export} keys are indistinguishable
from random. 
This should be possible even for the \emph{unmodified} TLS protocol, 
because exported keys are not used to encrypt messages during the handshake phase.
One obvious approach would be to reuse one of the existing security proofs which shows TLS to be ACCE secure.
Specifically, in these proofs the master secret of a particular session is typically swapped out with a completely random value,
allowing the rest of the proof to continue on the assumption that the master secret is completely hidden from the adversary.
Due to the unpredictability of the master secret,
the adversary will not be able to detect the switch.
Using this truly random master secret,
we could extend the proof with one additional step where we derive the export key through a random oracle query.
It would then follow that the derived export key is indistinguishable from random. 

However,
such a result could not be re-used across different TLS ciphersuites,
nor hold for future versions of TLS. 
Instead, for every variant of TLS,
one would have to copy-paste the corresponding security proof and augment it accordingly to account for the extra export key.
This approach is of course inherently non-modular since it is tied to the innards of each particular proof.
Still,
it seems likely that most of these proofs would be fairly similar in terms of technique,
and also reasonably independent of the specific details of the TLS protocol itself.  

The question is whether we can isolate exactly those properties of the TLS protocol that these proofs rely on.
If so, we could extract a generic proof of TLS key exporters that works across different versions unmodified.
Moreover,
it would be even better if we could have a result that is not tied to TLS at all,
but rather one that targets an appropriate abstract security notion.

Essentially,
this is what we do in this paper.
We identify some features of the TLS protocol which,
when added to a generic ACCE protocol,
are sufficient to establish the indistinguishability of the export keys derived from the protocol.
Note that,
apart from the features that we identify,
the result is completely independent of the internals of TLS. 
Below we describe these features.
%
% security proof each time a new version of the TLS standard is published. 
%Furthermore, 
%the additional step does not rely on many of the specifics and internal description and details of TLS---but which specifics does such an
%additional step in the security proof rely on? 
%Or, put differently, what are good design criteria for key exporters in general? 
%Perhaps we can draw lessons from studying key exporters in the context of TLS.

%
%\paragraph{Our proof at a glance.}
%\paragraph{Technical overview of our proof.}
\paragraph{Technical overview of our result.}
Surprisingly,
the number of additional features we require in addition to a generic ACCE protocol is rather minimal
and consists of the following three requirements 
(which we make more precise in Sect.~\ref{sec:TLS-like:def}).
We call an ACCE protocol that satisfies these requirements \emph{TLS-like}.
%
%
%First,
%if we want 
%Since generic ACCE protocols do not have a concept of a master secret we have to 
%
%
%
%is fairly minimal.
%%We now discuss the properties on which the ``additional step in the proof for key exporters'' relies. 
%A first attempt would be to consider a generic ACCE protocol rather than TLS, 
%and from there show that the master secret is unpredictable in the vein of MSW~\cite{MorrisseySW:2008:ModularTLS}. 
%However, 
%Consequently, our result demands 
%that the protocol be ``TLS-like'' in the sense that it satisfies the following three
%a bit of additional structure from 
%satisfy three
% \emph{syntactical} requirements:
\begin{enumerate}[(i)]
	\item The handshake includes a random \emph{nonce} from each party.
	\item Each party maintains a value called the \emph{master secret} during the handshake.
	\item The session key is derived from the master secret, the nonces, and possibly some other public information.
\end{enumerate}


Our result can now be more precisely formulated as follows:
starting from an ACCE secure TLS-like protocol $\protocol$,
we create an AKE secure protocol $\protocolderived$,
where $\protocolderived$ consists of first running protocol $\protocol$ until a session accepts
(according to $\protocol$),
then deriving one additional key from the master secret and nonces of $\protocol$.
This key---which is distinct from the session key in the underlying protocol $\protocol$---becomes the session key of $\protocolderived$.
In our security proof the key derivation step will be modeled using a random oracle.
The construction of $\protocolderived$ from $\protocol$ precisely captures the definition used in TLS key exporters~\cite{IETF:RFC:5705:TLS-key-exporters}
and EAP-TLS~\cite{IETF:2008:RFC5216-EAP-TLS}.


Note that while we put no security requirements on the master secret of a TLS-like protocol,
it is pivotal in our proof to relate the indistinguishability of the session keys in $\protocolderived$ to the ACCE security of $\protocol$.
As mentioned previously,
we build on the idea used by MSW~\cite{JC:MorSmaWar10} to 
show that unless the adversary queries the random oracle on the exact master secret of a party,
it has no advantage in distinguishing the corresponding exported session key in $\protocolderived$.
MSW proved that an application key agreement protocol 
(having indistinguishable session keys) could be built out of a master key agreement protocol
(having unpredictable master secrets).  
In their security reduction
the simulator could simulate the application key agreement protocol since it had access to a \emph{perfect} key-checking oracle,
allowing it to test the validity of master secrets supplied to the random oracle. 
Our proof is complicated by the fact that we do a reduction to a (TLS-like) ACCE protocol for which there is no key-checking oracle available.
The main technical novelty of our proof is to show that we can still create an approximation of the key-checking oracle as long as we allow a (small) one-sided error probability.
This emulated key-checking oracle suffices to simulate the AKE experiment of protocol $\protocolderived$ in our reduction to the ACCE security of $\protocol$.

To give some intuition for our key-checking oracle within the ACCE setting,
suppose we want to test whether the value  $ms'$ is the master secret of some session $\oracle[][]$.
First, 
we use $ms'$,
the nonces $\oracle[][]$ accepted with,
and the KDF of $\protocol$
(all available due to the TLS-like requirement on $\protocol$)
to derive a \emph{guess} on $\oracle[][]$'s session key \emph{in $\protocol$}.
Next, we obtain a ciphertext $C$ of a random message under $\oracle[][]$'s \emph{actual} session key in $\protocol$,
using our access to a left-or-right encryption oracle in the ACCE game. 
Finally,
we \emph{locally} decrypt $C$ using the guessed session key of $\protocol$, 
i.e., we do not use the decrypt oracle of the ACCE game.
If this decryption gives back the random message we started with, 
we guess that $ms'$ was the correct master secret of $\oracle[][]$;
otherwise, we guess that it was incorrect.

In the above we tacitly assumed that different master secrets derive different session keys 
(using the same nonces).
Normally,
this would follow directly from the pseudorandomness of the KDF used in $\protocol$.
However,
since we do not require the master secrets to be independent and uniformly distributed,
we cannot invoke this property of the KDF.
Instead, 
we have to explicitly assume that different master secrets do not ``collide'' to the same session key.
We expect this property to hold for most real-world KDFs.
Concretely,
we show in Theorem~\ref{thm:tls.PRF:collision_resistance} (Appx.~\ref{sec:tls.PRF_collision_resistance})
that the HMAC-based KDF used in TLS has this property.
 



%Since the same master secret and nonces are used to derive the extra session keys in $\protocolderived$ (via the random oracle) 
%as in $\protocol$ (via the KDF),




\paragraph{Alternatives to using the ACCE security notion?}
The main reason for using the ACCE security notion in our analysis is that is has proved to be a very useful model for studying the security of two-stage channel establishment protocols. 
As already mentioned,
 it has been used repeatedly to analyze real-world protocols such as TLS, SSH, and QUIC. 
Since our result applies to \emph{any} ACCE protocol that is TLS-like,
it can be applied to all these protocols in a near black-box manner.
In particular, 
we can plug in any existing ACCE result without having to re-do any of the steps carried out in the (ACCE) proof of the protocol itself. 
For example, our result applies unmodified to every ciphersuite version of TLS for which there exist an ACCE proof.
Moreover, we can even apply our theorem to future versions of TLS, 
as long as these continue to be TLS-like and derive their channel keys using a collision resistant KDF.

Still, in the specific case of TLS, one might ask whether another approach could give a simpler,
yet equally modular proof of the same result,
namely that EAP-TLS (and more generally, TLS key exporters) constitutes a secure AKE.

Krawczyk, Paterson, and Wee (KPW)~\cite{C:KraPatWee13} showed that all the major handshake variants of TLS satisfy a security notion on its key encapsulation mechanism (KEM) called IND-CCCA~\cite{C:HofKil07}. 
If we could reduce the AKE security of EAP-TLS to the IND-CCCA security of the TLS-KEM,
then the results of KPW would give us all the major TLS ciphersuites ``for free''.

Unfortunately,
it is not obvious how such a result can be obtained in a black-box manner from the KEM in~\cite{C:KraPatWee13}.
Technically,
in order to reduce the AKE security of EAP-TLS to the IND-CCCA security of the TLS-KEM,
we need to be able to simulate the key derivation step in the AKE game of EAP-TLS.
This requires knowledge about the sessions' master secrets.
However,
the KEM defined by KPW does not contain the TLS master secret. 
%Instead, 
%the KEM key is defined to be the concatenation of the TLS \emph{channel key} 
%and the client's and servers finished messages
%(i.e., \emph{unencrypted} MAC's of the respective party's \emph{verify data}).
This means that an adversary against the TLS-KEM in the IND-CCCA game
cannot simulate the $\Q{Test}$-challenge for some adversary playing in the AKE game against EAP-TLS. 
Moreover,
as remarked by KPW~\cite[Remark~4]{C:KraPatWee13},
if the KEM key actually \emph{was} defined to be the TLS master secret,
then the resulting scheme would be insecure for TLS-RSA provided that RSA PKCS\#1v1.5 is re-randomizable\footnote{On the other hand, 
Bhargavan et al.~\cite{C:BFKPSZ14}
conjecture that re-randomizing RSA PKCS\#1v1.5 is infeasible, 
allowing the master secret to be used as the KEM key in TLS-RSA too.
We forgo the issue by not reducing to the KEM-security of TLS. 
}.

\paragraph{Other modular approaches to analyzing TLS.}
Canetti and Krawczyk~\cite{EC:CanKra01} presented a model that allows to analyze protocols in a modular way.
Unfortunately,
since TLS does not meet the stringent requirement of key indistinguishability, it cannot be analyzed within their framework.
K{\"u}sters and Thuengerthal~\cite{CCS:KusTue11} analyzed the core of TLS in their simulation-based universal composability model
called IITM.
Unlike some other UC models,
the IITM model has the appealing feature that it does not rely on pre-established session identifiers. 
Brzuska et al.~\cite{BrzuskaFSWW:2012:less_is_more} introduced a framework that uses so-called key-independent reductions and allows to analyze protocols such as TLS. 
Their analysis is in a game-based setting and,
up to some small technical differences between models,
implies ACCE security of TLS.
Kohlweiss et al.~\cite{EPRINT:KMOTV14} recently used the abstract cryptography framework by Maurer and Renner~\cite{ICS:MauRen11} for a modular analysis of TLS.

