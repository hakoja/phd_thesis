
\begin{definition}\label{def:TLS-like}
An ACCE protocol $\Pi$ is said to be \emph{TLS-like} if
\begin{enumerate}[label=(\roman*),ref=\ref*{def:TLS-like}.\roman*]
	\item\label{def:TLS-like:nonces_generated_and_sent} each session uniformly at random generates and transmits a distinguished \emph{nonce value} $n \getsr \bits^\noncelen$ during its run of the protocol,
	
	\item\label{def:TLS-like:ms_variable_present} each session holds a variable $\oracle.ms \in \bits^\keylen \cup \lbrace \bot \rbrace$, called the \emph{master secret},
	
	\item\label{def:TLS-like:kdf_use_ms_and_nonces} if $n_1, n_2$ are the two nonces on a session's transcript $T$,
	then the \emph{session key} is derived as 
	\begin{equation}\label{eq:TLS-like:key-derivation}
		\key \gets \mathsf{Kdf}(ms, n_1 \concat n_2, F_{\protocol}(T)), 
	\end{equation}
	where $\mathsf{Kdf} \colon \bits^\keylen \times \bits^{2\noncelen} \times \bits^* \to \bits^\keylen$ and $F_{\protocol} \colon \bits^* \to \bits^*$  are deterministic functions.
\end{enumerate}

\end{definition}


\begin{remark}
The function $F_{\Pi}$ is protocol specific and meant to capture any additional input that might be used to derive the session keys.
For example in TLS, $F_{\Pi}(T)$ is the empty string, 
while in IPsec (IKEv2), $F_{\Pi}(T)$ is the Security Parameter Index (SPI) of the initiator and the responder.
\end{remark}

\begin{remark}
Clearly TLS is TLS-like,
but most other real-world protocols,
like SSH, IPsec and QUIC,
belong to this class as well.
\end{remark}

%\begin{definition}[Generalized definition]\label{def:TLS-like:generalized}
%An ACCE protocol $\Pi$ is said to be \emph{TLS-like} if:
%\begin{enumerate}[label=\roman*),ref=\ref*{def:TLS-like}.\roman*]
%	
%	
%	\item\label{def:TLS-like:ms_variable_present} each session holds a variable $ms \in \bits^\secparam \cup \lbrace \bot \rbrace$, called the \emph{master secret};
%	
%	
%	\item\label{def:TLS-like:kdf_use_ms_and_nonces} the \emph{session key} is derived as 
%	\begin{equation}
%		\key \gets \mathsf{Kdf}(ms, F_{\protocol}(T)), 
%	\end{equation}
%	where $\mathsf{Kdf} \colon \bits^\secparam \times \bits^* \to \bits^\secparam$ and \todo{can the domain of $F_\Pi$ be infinite?}$F_{\protocol} \colon \bits^* \to \bits^*$  are deterministic functions,
%	and $T$ is the session's transcript;
%%	and $aux \in \bits^*$ is any other auxiliary data used by the key derivation process in protocol $\protocol$; 
%	
%%	\item\label{def:TLS-like:nonces_generated_and_sent};
%
%	\item\label{def:TLS-like:high_entropy_sid} 
%	$T_{sent}$ and $F_{\Pi}(T)$ have min-entropy $\secparam$ (see Def.~\ref{def:min_entropy});
%	
%	\item\label{def:TLS-like:nonces_are_in_sid} $F_{\protocol}(T)$ is part of the session identifier $\sid$;
% 	
%	\item\label{def:TLS-like:public_session_matching} the session identifier allows for \emph{public session matching} 
%	(see Def.~\ref{def:public_session_matching}).
%\end{enumerate}
%
%\end{definition}
%
%\begin{definition}\label{def:min_entropy}
%A random variable $X$ has \emph{min-entropy} $k$,
%if $\max\limits_{x} \Pr[X = x] \leq 2^{-k}$.
%\end{definition}



%The following simple observation on TLS-like protocols will be useful in proving our main result.
%\begin{proposition}\label{proposition:TLS-like:unique_nonces}
%If all nonces are unique in a TLS-like protocol $\protocol$,
%and two sessions $\oracle, \oracle[j][t]$ accepted with the same nonces without being partners,
%then either both sessions must be non-fresh or at least one must have accepted maliciously.
%\end{proposition}
%\begin{proof}
%%Assume $\oracle, \oracle[j][t]$ accepted with nonces $n_1,n_2$,
%%and $\oracle.\sid \neq \oracle[j][t].\sid$.
%Suppose $\oracle$ was fresh.
%Then we claim that it would have accepted without a partner,
%because $\oracle[j][t]$ was the only such candidate.
%This follows directly from the requirement that the nonces be part of the session identifier for a TLS-like protocol (Definition~\ref{def:TLS-like:nonces_are_in_sid})
%and the assumption that all nonces are unique. %\qed
%\end{proof}

