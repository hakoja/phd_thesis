\newcommand{\abort}{\operatorname{\mathsf{malicious-accept}}}
\newcommand{\CK}{\mathbf{CheckKey}}
\newcommand{\KOerror}{\mathsf{KOerror}}
\newcommand{\CKerror}{\mathsf{CKerror}}
\newcommand{\KDFcoll}{\operatorname{\mathit{ck}-\mathsf{coll}}}
\newcommand{\correct}{\mathsf{correctDec}}
\newcommand{\ncorrect}{\overline{\correct}}
\newcommand{\wrong}{\mathsf{wrongDec}}
\newcommand{\nwrong}{\overline{\wrong}}
\newcommand{\wrongMS}{\mathsf{wrongMS}}
\newcommand{\eventFresh}{\mathsf{fresh}}
\newcommand{\nfresh}{\operatorname{\mathsf{non-fresh}}}
\newcommand{\bwins}{\oracle.b = b'}
\newcommand{\KO}{\mathcal{KO}}


\chapter{Security of EAP-TLS}\label{sec:EAP-TLS-security}


\begingroup
\hypersetup{linkcolor=black}
\minitoc
\endgroup

\section{Motivation}\label{sec:EAP-TLS-security:intro}
In \cref{sec:generic_composition_results} we showed that EAP is a secure 3P-AKE protocol assuming, 
among other things,
that the EAP method between the client and the server is a secure 2P-AKE protocol.
Thus,
in order to complete the picture on EAP,
we need to establish that at least \emph{some} EAP method satisfies the 2P-AKE security notion. 
Fortunately,
such an EAP method already exists,
namely EAP-IKEv2~\cite{IETF:RFC5106:EAP-IKEv2}.
In particular,
since an EAP method is just a wrapper around some concrete AKE protocol,
the security of EAP-IKEv2 reduces to that of IKEv2, 
which has been proven secure by Canetti and Krawczyk~\cite{C:CanKra02}.

On the other hand,
probably the most widely supported EAP method of all---EAP-TLS---has no such proof.
In fact,
as we explained in \cref{sec:definitions:ACCE},
TLS in all versions up to TLS~1.2 is not a secure AKE protocol at all!
The reason is that TLS encrypts some of the handshake messages using the session key itself,
giving the adversary a trivial way of distinguishing the session key from random.
Thus,
it might appear that our results on EAP cannot be applied to the case when EAP-TLS is being used as the EAP method.

Fortunately,
it turns out that within the context of EAP-TLS,
TLS \emph{can} be proven to be a secure AKE protocol.
Recall from \cref{sec:EAP:description} that an EAP method is supposed to \emph{export} a master session key $\MSK$.
Crucially,
in EAP-TLS the $\MSK$ is \emph{not} the ordinary session key of TLS which is used to protect the channel.
Instead,
the $\MSK$ is derived as a separate key from the master secret established during the TLS handshake.
This fact makes it possible to prove that EAP-TLS is a secure AKE protocol,
by considering the $\MSK$ as the session key.


Concretely,
in this \lcnamecref{sec:EAP-TLS-security} we show that if one derives an additional \emph{export key} from the TLS master secret---independent 
of the other handshake messages---then TLS constitutes a secure AKE protocol by taking this export key to be the session key.  
Furthermore, while our starting point is the TLS protocol, 
our result will in fact be much more general.
Instead of focusing solely on TLS,
we generalize to a wider class of protocols which we call \emph{TLS-like ACCE protocols}.  
Roughly, 
TLS-like ACCE protocols are protocols that satisfy the ACCE security notion and, 
like TLS, establish a master secret during the handshake.
%and from the master secret derive both the channel encryption key and the additional exported key.  
Apart from this requirement, our result has no other dependencies on the specifics of the TLS protocol.
In other words, our main result is a general theorem showing that any ACCE protocol which has a concept of a master secret can be turned into an AKE protocol.

An immediate corollary of this result is of course that EAP-TLS is a secure 2P-AKE protocol.
However,
it also applies more broadly to the general practice of exporting additional keys from the master secret in TLS,
as has been formalized in RFC~5705: ``Keying Material Exporters for Transport Layer Security (TLS)''~\cite{IETF:RFC5705:TLS-key-exporters}
(which we call \emph{TLS Key Exporters} from now on).
%\footnote{In an upcoming version of TLS, TLS~1.3.,
%the concept of exporting keys is baked right into the standard.
%}




\paragraph{Motivation for our approach.}
For the moment,
suppose we only wanted to show that EAP-TLS was a secure AKE protocol,
leaving aside the possibility of further generalizations for now.
One obvious approach would be to reuse one of the many existing security proofs which shows that TLS is a secure ACCE protocol 
(e.g.,~\cite{C:JKSS12,EPRINT:KohSchSch13,PKC:LSYKS14}).
Specifically, in these proofs the master secret of a particular session is typically swapped out with a completely random value,
allowing the rest of the proof to continue on the assumption that the master secret is completely hidden from the adversary.
Due to the unpredictability of the master secret,
the adversary will not be able to detect the switch.
Using this truly random master secret,
we could then extend the proof with one additional step where we derive the export key using a random oracle.
It would then follow that the derived export key is indistinguishable from random. 

However,
a big downside of such a result is that it could not be re-used across different TLS ciphersuites,
nor would it hold for future versions of TLS. 
Indeed, for every variant of TLS one would have to redo the corresponding security proof and augment it accordingly to account for the extra export key.
Besides being tedious,
this approach is of course also inherently non-modular since it is tied to the innards of each particular proof.
Still,
it seems likely that most of these proofs would be fairly similar in terms of technique,
and also reasonably independent of the specific details of the TLS protocol itself.  

The question is whether we can isolate exactly those properties of the TLS protocol that these proofs rely on.
If so, we could extract a generic proof of TLS key exporters that works across different versions unmodified.
Moreover,
it would be even better if we could have a result that is not tied to TLS at all,
but rather one that targets an appropriate abstract security notion.

Essentially,
this is what we do in this \lcnamecref{sec:EAP-TLS-security}:
we identify certain features of the TLS protocol which,
when added to a generic ACCE protocol,
are sufficient to establish the indistinguishability of the export keys derived by the protocol.
Note that,
apart from the features that we identify,
the result is completely independent of the internals of TLS. 
Below we describe these features.
%
% security proof each time a new version of the TLS standard is published. 
%Furthermore, 
%the additional step does not rely on many of the specifics and internal description and details of TLS---but which specifics does such an
%additional step in the security proof rely on? 
%Or, put differently, what are good design criteria for key exporters in general? 
%Perhaps we can draw lessons from studying key exporters in the context of TLS.

%
%\paragraph{Our proof at a glance.}
%\paragraph{Technical overview of our proof.}
\paragraph{Technical overview of our result.}
Surprisingly,
the number of additional features that needs to be added beyond a generic ACCE protocol is rather minimal.
They consist of the following three requirements.
%(which we make more precise in \cref{sec:TLS-like:def}).
%
%
%First,
%if we want 
%Since generic ACCE protocols do not have a concept of a master secret we have to 
%
%
%
%is fairly minimal.
%%We now discuss the properties on which the ``additional step in the proof for key exporters'' relies. 
%A first attempt would be to consider a generic ACCE protocol rather than TLS, 
%and from there show that the master secret is unpredictable in the vein of MSW~\cite{MorrisseySW:2008:ModularTLS}. 
%However, 
%Consequently, our result demands 
%that the protocol be ``TLS-like'' in the sense that it satisfies the following three
%a bit of additional structure from 
%satisfy three
% \emph{syntactical} requirements:
\begin{enumerate}[(i)]
	\item The handshake includes a random \emph{nonce} from each session.
	\item Each session maintains a value called the \emph{master secret} during the handshake.
	\item The session key is derived from the master secret, the nonces, and possibly some other public information using a key derivation function (KDF).
\end{enumerate}

We call an ACCE protocol that satisfies these requirements \emph{TLS-like}.
Our result can now be more precisely formulated as follows:
starting from an ACCE secure TLS-like protocol $\protocol$,
we create an AKE secure protocol $\protocolderived$,
where $\protocolderived$ consists of running protocol $\protocol$ until a session accepts
(according to $\protocol$),
and then derives one additional key from the master secret and nonces of $\protocol$.
This key---which is distinct from the session key in the underlying protocol $\protocol$---becomes the session key of $\protocolderived$.
In our security proof the key derivation step will be modeled using a random oracle.
The construction of $\protocolderived$ from $\protocol$ precisely captures the definition used in TLS key exporters~\cite{IETF:RFC5705:TLS-key-exporters}
and EAP-TLS~\cite{IETF:RFC5216:EAP-TLS}.


Note that while we put no security requirements on the master secret of a TLS-like protocol,
it is pivotal in our proof to relate the indistinguishability of the session keys in $\protocolderived$ to the ACCE security of $\protocol$.
However,
at first sight it does not seem like merely assuming the ACCE security of TLS will allow us to say anything about the \emph{internal} variables of TLS,
and in particular its master secret. 
Nevertheless, 
inspired by Morrissey, Smart, and Warinschi~\cite{JC:MorSmaWar10},
we can show that the ACCE security of TLS implies that the master secret is \emph{unpredictable},
meaning that no adversary is able to output the full master secret of a fresh target session.
If the master secret was predictable, then we would be able to break the security of the ACCE channel. 
This intuition lies at the heart of our proof,
which uses the ACCE property of TLS in a (semi-)black-box way.


Specifically,
Morrissey et al.~\cite{JC:MorSmaWar10} proved that a secure AKE protocol can be built out of a secure \emph{master key agreement protocol},
which has the much weaker security requirement of having unpredictable master secrets.  
In their security reduction
they assumed to have access to a \emph{key-checking oracle} $\mathcal{O}$ that answers whether a supplied value equals the master secret of a given session.
Using the key-checking oracle $\mathcal{O}$,
they could simulate the session key derivation of the AKE protocol as well the random oracle.
Crucially,
however,
it required that  $\mathcal{O}$ was \emph{perfect},
meaning that it always answered correctly.



By contrast,
our proof is complicated by the fact that there is no perfect key-checking oracle available.
That is,
given only a (TLS-like) ACCE protocol,
there is no apparent mechanism for testing the master secret of a session with certainty.
The main technical novelty of our proof is to show that we can still create an approximation of the key-checking oracle as long as we allow a (small) one-sided error probability.
This emulated key-checking oracle suffices to simulate the AKE experiment of protocol $\protocolderived$ in our reduction to the ACCE security of~$\protocol$.

To give some intuition for our key-checking oracle in the ACCE setting,
suppose we want to test whether the value  $ms$ is the master secret of some session $\oracle[][]$.
First, 
we use $ms$,
the nonces $\oracle[][]$ accepted with,
and the KDF of $\protocol$
(all available since $\protocol$ is TLS-like)
to derive a \emph{guess} on $\oracle[][]$'s session key \emph{in $\protocol$}.
Next, we obtain a ciphertext $C$ of a random message under $\oracle[][]$'s \emph{actual} session key in $\protocol$
using our access to the ``left-or-right'' $\Encrypt$ query in the ACCE game. 
Finally,
we \emph{locally} decrypt $C$ using the guessed session key of $\protocol$, 
i.e., we do not use the $\Decrypt$ query of the ACCE game.
If the local decryption gives back the random message we started with, 
we guess that $ms$ was the correct master secret of $\oracle[][]$;
otherwise, we guess that it was incorrect.

In the above, we tacitly assumed that different master secrets derive different session keys 
(using the same nonces).
Normally,
this would follow directly from the pseudorandomness of the KDF used in $\protocol$.
However,
since we do not require the master secrets to be independent and uniformly distributed,
we cannot invoke this property of the KDF.
Instead, 
we have to explicitly assume that different master secrets do not collide to the same session key.
We expect this property to hold for most real-world KDFs.
Particularly,
we show in \cref{thm:tls.PRF:collision_resistance} (\cref{sec:tls.PRF_collision_resistance})
that the HMAC-based KDF used in TLS~1.2 has this property,
provided the underlying hash function in HMAC is collision-resistant.
 



%Since the same master secret and nonces are used to derive the extra session keys in $\protocolderived$ (via the random oracle) 
%as in $\protocol$ (via the KDF),





\subsection{Related work on EAP-TLS}

The classic result of Canetti and Krawczyk~\cite{EC:CanKra01} shows how to build secure channels from AKEs.
Our result can be seen as a kind of dual: building AKEs from secure channels.
Specifically,
we create a \emph{compiler} that on input a secure TLS-like ACCE protocol outputs a secure AKE protocol.
There is a long tradition for generic compiler results like this in the literature~\cite{STOC:BelCanKra98,C:KatYun03,AC:JKSS10,IMA:BoyGon11,ESORICS:CreFel12,CCS:Krawczyk16}.

On the specific topic of EAP-TLS we are not aware of any existing results.
There are results on other EAP methods, however.
For example, the already mentioned EAP-IKEv2 method is a secure AKE protocol
(following directly from the corresponding result on IKEv2~\cite{C:CanKra02}).
Likewise,
Küsters and Tuengerthal~\cite{CCS:KusTue11} have shown that the EAP-PSK method (see \cref{table:EAP_methods}) is 
%a 
secure 
%key exchange functionality 
in their IITM universal composability framework.


\section{TLS-like ACCE \texorpdfstring{$\implies$}{->} AKE}\label{sec:EAP-TLS-security:technical_result}

In this \lcnamecref{sec:EAP-TLS-security:technical_result} we state and prove our generic result which will be used to establish the AKE security of EAP-TLS in \cref{sec:EAP-TLS-security:application_EAP-TLS}.
The protocols analyzed in this \lcnamecref{sec:EAP-TLS-security:technical_result} are generic in the sense that they are not assumed to have any specific structure except for being TLS-like.

\subsection{TLS-like protocols}\label{sec:TLS-like:def}

%Here we formally define what it means for an ACCE protocol to be TLS-like.
Since the definition of TLS-like is motived by the structure of the TLS~1.2 protocol,
we first give a brief description of this protocol here. 
\cref{fig:TLS} shows a simplified version of the TLS~1.2 handshake parameterized on a key encapsulation mechanism (KEM) $\mathsf{KEM}$.
This presentation is inspired by Krawczyk et al.~\cite{C:KraPatWee13}
and allows us to treat all the main TLS handshake variants,
TLS-RSA, TLS-DH, TLS-DHE,
in a uniform manner by suitably instantiating the KEM.
Note that in order to do so,
the ``$\mathsf{Cert}^+$'' notation captures more than just one side's certificate.
For example,
for TLS-DHE we have that $\mathsf{Cert}_C^+$ includes the client's ephemeral Diffie-Hellman share,
a signature on the share,
as well as the client certificate itself.

\begin{figure}
	\centering
	\includestandalone[width=0.95\textwidth,mode=build]{tikz_TLS}
	\caption{Simplified TLS~1.2 handshake.}
	\label{fig:TLS}
\end{figure}

The TLS handshake begins with the client sending a random nonce $\nonce_{C}$.
The server responds with its own random nonce $\nonce_{S}$ and its contribution to the KEM,
denoted $\mathsf{Cert}_S^+$.
On receiving this message,
the client generates its own KEM contribution $\mathsf{Cert}_C^+$,
and from the two KEM values derives a \emph{master secret} key $ms$. 
From the master secret and the nonces,
the client also derives the TLS session key $\key$
using the TLS key derivation function $\mathsf{tls.PRF}$.
The client sends its KEM contribution together with a key confirmation message $\mathsf{Finished}_C$ to the server.
On receiving the client's KEM contribution,
the server derives the same $ms$ and $\key$ as the client and checks the validity of $\mathsf{Finished}_C$.
It ends the handshake by sending a key confirmation message $\mathsf{Finished}_S$ of its own.
 
Essentially,
a TLS-like protocol abstracts from the TLS~1.2 handshake the idea of having random nonces,
a master secret,
and a session key derived from the master secret and the nonces using a KDF.
In the definition below,
recall that a session's local transcript $\localtranscript$ consists of all the messages it has sent and received during the protocol run.


\begin{definition}[TLS-like protocols]\label{def:TLS-like}
An ACCE protocol $\Pi$ is \emph{TLS-like} if:
\begin{enumerate}[label=(\roman*),ref=\ref*{def:TLS-like}.\roman*]
	\item\label{def:TLS-like:nonces_generated_and_sent} each session transmits a random \emph{nonce value} $\nonce \getsr \bits^\noncelen$ during its run of the protocol,
	
	\item\label{def:TLS-like:ms_variable_present} each session holds a variable $ms \in \bits^\keylen \cup \lbrace \bot \rbrace$, called the \emph{master secret},
	
	\item\label{def:TLS-like:kdf_use_ms_and_nonces} if $\nonce_1, \nonce_2$ are the two nonces on a session's local transcript $\localtranscript$,
	then the \emph{session key} is derived as 
	\begin{equation}\label{eq:TLS-like:key-derivation}
		\key \gets \mathsf{Kdf}(ms, \nonce_1 \concat \nonce_2, F_{\protocol}(\localtranscript)), 
	\end{equation}
	where $\mathsf{Kdf} \colon \bits^\keylen \times \bits^{2\noncelen} \times \bits^* \to \bits^\keylen$ and $F_{\protocol} \colon \bits^* \to \bits^*$  are deterministic functions.
\end{enumerate}

\end{definition}


It should be clear from \cref{fig:TLS} that TLS~1.2 is indeed TLS-like.
But many other real-world protocols also belong to this class,
like SSH, IKEv2, and QUIC.
The function $F_{\Pi}$ is protocol-specific and meant to capture any auxiliary input that might be used to derive the session keys in addition to the nonces.
For example in TLS~1.2, $F_{\protocol}(\localtranscript)$ is the empty string, 
while in IKEv2, $F_{\protocol}(\localtranscript)$ is the Security Parameter Index (SPI) of the initiator and the responder.





\subsection{Construction}\label{sec:construction:PI+}
Let $\protocol$ be a TLS-like ACCE protocol with key derivation function $\mathsf{Kdf}$
and let $G \colon \bits^{\keylen} \times \bits^{2\noncelen} \times \bits^{*} \to \bits^\keylen$ be a random oracle.
From $\protocol$ and $G$ we create an AKE protocol $\protocolderived$ as follows.
Protocol $\protocolderived$ consists of first running protocol $\protocol$ as usual until a session accepts,
then it derives an additional key $ek \gets G(ms, \nonce, aux)$,
where $ms$ is the master secret of $\protocol$,
$\nonce_{C}$ and $\nonce_{S}$ are the nonces,
and $aux \in \bits^*$ is an (optional) string containing selected values from the session's local transcript $\localtranscript$.
The key $ek$ becomes the \emph{session key} in protocol~$\protocolderived$.


By construction,
a session in $\protocolderived$ derives (at least) two keys:
its ``true'' session key in the sense of the AKE model,
i.e., the key $ek$ derived from the random oracle $G$;
and the ``session key'' derived in the underlying protocol $\protocol$
using the KDF $\protocol.\mathsf{Kdf}$.
To avoid confusion,
we will call the former key the \emph{export key};
and the latter key the \emph{channel key} and denote it $ck$.
%Note that 
%
% also derives what would normally be considered the ``session key'' in protocol $\protocol$
%(since $\protocol$ is an ACCE protocol),
%however, 
%this key is only considered an internal value in protocol $\protocolderived$.
%To avoid confusion between the keys derived in $\protocolderived$,
%we call the session key derived from $G$ the \emph{export key} and denote it $ek$;
%while we call the ``session key'' derived in the underlying protocol $\protocol$,
%the \emph{channel key} and denote it $ck$.
In particular,
in the AKE game
the session key variable $\oracle.\key$ will store the export key $ek$,
while the channel key $ck$ will merely be one of $\oracle$'s other internal state variables,
written $\oracle.ck$.
The reason why the export key $ek$ needs to be derived using a random oracle will be explained below.


\subsection{Main result}

Informally, our main result shows that the construction described above transforms a secure TLS-like ACCE protocol $\protocol$ into a secure AKE protocol $\protocolderived$.
However, 
in our proof we need to make one additional assumptions besides that of ACCE security.
We need to assume that the key derivation function $\protocol.\mathsf{Kdf}$ does not have \emph{key collisions},
i.e.,
that two different master secrets produce the same output when given the same nonces and auxiliary data as input.



\begin{definition}[KDF collision resistance]\label{def:KDF_collision_resistance}
Let $\mathsf{KDF}$ be a function with the same domain and range as the function in \cref{eq:TLS-like:key-derivation}.
The \emph{KDF key collision advantage} of an adversary $\A$ is defined as
\begin{equation}\label{eq:def:KDF_collision}
	\adv_{\mathsf{KDF}}^{\kdfcoll}(\A) 
		\defeq \Pr \left[ ((ms,ms'), r,s) \gets \A :
				\parbox{13.5em}{\centering
					$\mathsf{KDF}(ms, r,s) = \mathsf{KDF}(ms', r,s)$ \\
%					$\land$ \\
					$ms \neq ms'$
				}
			\right] .
\end{equation}
A tuple $((ms,ms'),r,s)$ satisfying the criterion in \eqref{eq:def:KDF_collision} is called a \emph{key collision} for $\mathsf{KDF}$.
\end{definition}

\begin{remark}\label{remark:EAP-TLS:def:KDF_key_collision}
\cref{def:KDF_collision_resistance} is a variant of the more common notion of collision-resistant \emph{hash functions}.
The difference is that KDF collision resistance is about collisions in the \emph{keys},
not the messages. 
\end{remark}





\begin{theorem}\label{thm:AKE_from_TLS-like_ACCE_protocol_main_result}
Let $\protocolderived$ be the AKE protocol derived from a TLS-like ACCE protocol $\protocol$ and a random oracle $G$
using the construction described in \cref{sec:construction:PI+}.
%with random oracle $G$.
Let $f$ be a partner function with perfect soundness.
Then for any adversary $\A$ in the AKE security experiment against $\protocolderived$,
we can create adversaries $\A[B]$ and $\A[C]$ such that:
\begin{equation}
	\adv_{\protocolderived,f}^{\akefs}(\A)
		\leq 3 \cdot \adv_{\protocol,f}^{\operatorname{\mathsf{ACCE}}}(\A[B])
		+ 3 \cdot \adv_{\protocol.\mathsf{Kdf}}^{\mathsf{KDFcoll}}(\A[C])
		+ \frac{6 \unionbound}{2^{c}}
		+ \frac{(\numparties \numsessions)^2 }{2^{\noncelen+1}},
\end{equation}
where $\noncelen$ is the nonce length of protocol $\protocol$,
$\numparties$ is the number of parties,
$\numsessions$ is the max number of sessions that $\A$ creates at each party,
$q$ is $\A$'s number of random oracle queries,
and $c \in \N$ is an arbitrary constant.
\end{theorem}

The main idea behind the proof of \cref{thm:AKE_from_TLS-like_ACCE_protocol_main_result} is to relate the security of the derived export keys to the security of the channel keys in the underlying ACCE protocol $\protocol$.
Basically, 
by using the property that TLS-like protocols derive their channel keys from the master secret and nonces,
we establish that two sessions derive the same export key if and only if they derive the same channel key
(barring certain bad events which we bound).
%This fact will make it possible to derive the sessions' export keys in $\protocolderived$ 
%independently of their master secrets,
%and still fully simulate the random oracle $G$. 
The reason why the export key needs to be derived using a random oracle is because the master secret is not guaranteed to be uniform and independently distributed.
Because of this we cannot invoke the pseudorandomness of the KDF when deriving the export key from the master secret. 


%\section{Proof of \texorpdfstring{\cref{thm:AKE_from_TLS-like_ACCE_protocol_main_result}}{Theorem~\ref{thm:AKE_from_TLS-like_ACCE_protocol_main_result}} }\label{sec:proof_of_main_result_auth}

Like in the proofs of our compositions theorems in \cref{sec:generic_composition_results},
we assume for simplicity that the partner function $f$ has perfect soundness so that we can always take for granted that partners derive the same session key.



\begin{proof}

Let $\A$ be the adversary  in the \akefstext security game against protocol~$\protocolderived$.
Our proof proceeds through a sequence of games,
where each consecutive game aims to decrease the challenger's dependency on the sessions' master secrets and the random oracle $G$,
in order to derive the export keys in protocol $\protocolderived$.
Eventually, 
in the final game the random oracle $G$ will have been completely replaced by a local list $L_G$,
and the $\protocolderived$ export keys are derived independently of the sessions' master secrets.
At this point we can construct an algorithm $\A[B]$ against the ACCE security of the underlying protocol $\protocol$,
since $\A[B]$ will now be able to simulate the game.


\setcounter{gamehop}{0}

\item

\paragraph{Game~\game:} \label{game:EAP-TLS:G0}
This is the original \akefstext security game for protocol $\protocolderived$:
\begin{equation}
	\adv_{\protocolderived,f}^{\akefs}(\A)
		= \adv_{\protocolderived}^{\mathsf{G_{\game}}}(\A) . 
\end{equation}


\newgame
\paragraph{Game~\game:}  \label{game:nonce_collision}
Game~\game{} proceeds like in Game~\prevgame{},
but aborts if two sessions generate the same nonce value.
Since there are $\numparties \cdot \numsessions$ generated nonces,
the probability of there being at least one collision is bounded by $(\numparties  \numsessions)^2 \cdot 2^{-(\noncelen+1)}$.
By the Difference Lemma  we have
\begin{equation}
	\adv_{\protocolderived,f}^{\mathsf{G_{\prevgame{}}}}(\A) 
		\leq \adv_{\protocolderived,f}^{\mathsf{G_\game}}(\A) 
		 + \frac{(\numparties \numsessions)^2}{2^{\noncelen+1}}. 
\end{equation} 


The remaining games are aimed at removing the challenger's dependency on the random oracle
and enabling it to derive the $\protocolderived$ export keys without knowing the sessions' master secrets.
To this end,
the challenger will begin to maintain a list $L_G$
which it will use to simulate the  random oracle $G$
and derive the sessions' export keys.
The entries of $L_G$ are tuples of the form $(ms, \nonce, aux, ek, [*])$,
where $ms \in \bits^\keylen \cup \lbrace \bot \rbrace$,
$\nonce \in \bits^{2\noncelen}$,
$aux \in \bits^*$,
$ek \in \bits^\keylen$,
and $[*]$ denotes a list that contains zero or more session oracles.
Specifically, 
we use the notation ``$[\,]$'' to denote an empty list,
``$[\oracle]$'' for a list containing exactly $\oracle$,
``$[\oracle,*]$'' for a list containing $\oracle$ plus zero or more (unspecified) sessions,
and ``$[*]$'' for a list containing zero or more (unspecified) sessions.
$L_G$ is initially empty and is filled out either in response to $\A$'s random oracle queries or when a session reaches the $\accepted$ state.

All the remaining games either change the way export keys are derived for newly accepted sessions
%
%all the changes introduced in the following games 
%consists in either changing the way the challenger derives the export keys of a newly accepted session
(which we call the ``$\Send$-code''), 
%%since it is a $\Send$ query that triggers a session to accept),
or how they answer random oracle calls
%or in how it answers random oracle calls
(which we call the ``$G$-code'').
The evolution of the $\Send$-code from Game~\ref{game:introduce_L_G} through Game~\ref{game:remove_code_send} is shown in \cref{fig:code_for_game_hops_Send} on \cpageref{fig:code_for_game_hops_Send},
while the corresponding $G$-code is shown in \cref{fig:code_for_game_hops_G} on \cpageref{fig:code_for_game_hops_G}.
Here is how to read the $\Send$-code.
When a session $\oracle$ accepts with master secret $ms$,
nonces $\nonce = \nonce_{C} \concat \nonce_{S}$, and auxiliary data $aux$,
then we look for the existence of  a tuple $t \in L_G$ that matches these variables.
We use \textcolor{red}{red} color to indicate the variables that a given if/else-if clause uses to ``pattern-match'' against the corresponding variables of $\oracle$.
The $G$-code is read in a similar way.

We annotate the changes made in one game relative to the previous one using boxes.
Note that some games make changes to both the $\Send$-code and
the $G$-code at the same time
For the remainder of this proof we always use $\nonce$ to refer to the concatenation of the two nonces $\nonce_{C}$, $\nonce_{S}$ that a session has received.





\newgame
\paragraph{Game~\game:} \label{game:introduce_L_G} 
This game introduces the list $L_G$.
When a session $\oracle$ accepts with master secret $ms$,
nonces $\nonce = \nonce_{C} \concat \nonce_{AP}$, and auxiliary data $aux$,
the challenger uses the $\Send$-code shown in the panel labeled ``Game~\ref{game:introduce_L_G}'' in \cref{fig:code_for_game_hops_Send} to derive its export key.
It uses the $G$-code shown in the panel labeled ``Game~\ref{game:introduce_L_G}'' in \cref{fig:code_for_game_hops_G} to answer the adversary's random oracle queries.

\vspace*{-\baselineskip}
\item 
\begin{claim}
\begin{equation}
	\adv_{\protocolderived,f}^{\mathsf{G_{\prevgame{}}}}(\A) 
		= \adv_{\protocolderived,f}^{\mathsf{G_\game}}(\A) .
\end{equation} 
\end{claim}

\begin{proof}
Since the challenger considers all of the input values to the random oracle when answering from $L_G$ in this game---in 
particular, 
it explicitly looks at the master secrets of the sessions---and because a random oracle always returns the same value when given the same input twice, 
the answers in Game~\game{} are distributed exactly like in Game~\prevgame{}.
\end{proof}









In the remaining games,
we define $\KDFcoll_\mathsf{i}$ to be the event that during the run of Game~$i$,
the challenger calls the key derivation function $\protocol.\mathsf{Kdf}$ on two different master secrets $ms \neq ms'$,
but with the same nonces $\nonce$ and additional input $aux$,
such that
$\protocol.\mathsf{Kdf}(ms, \nonce, aux) 
= \protocol.\mathsf{Kdf}(ms', \nonce, aux)$.
We call event $\KDFcoll_{\mathsf{i}}$ a \emph{channel key collision}.


\newgame
\paragraph{Game~\game:} \label{game:match_fresh_sessions} 
In this game the $\Send$-code  is modified so that when a session accepts,
the challenger first checks whether the session's partner is present in a tuple on $L_G$ 
before deriving its export key
(see the panel labeled ``Game~\game{}'' in \cref{fig:code_for_game_hops_Send}).
The $G$-code remains unchanged.

\begin{claim}\label{claim:game:match_fresh_sessions}
\begin{equation}
	\adv_{\protocolderived,f}^{\mathsf{G_{\prevgame{}}}}(\A) 
		\leq \adv_{\protocolderived,f}^{\mathsf{G_\game}}(\A) 
		+ \Pr[\KDFcoll_{\mathsf{\game}}] .
\end{equation}
\end{claim}


\begin{proof}
We claim that unless a channel key collision occurs,
then Game~\prevgame{} and Game~\game{} are identical.
To see this,
suppose the if-check at \cref{alg:line:send_game_hops_match_fresh} of Game~\game{} matched two sessions $\oracle$ and $\oracle'$.
This means that $f_T(\oracle)  = \oracle'$,
which implies that they have the same channel key by our assumption of perfect soundness of $f$.
Then our assumption that no key collision occurs further
implies that they must also have the same master secret.
Hence, 
the else-if check at \cref{alg:line:send_game_hops_catch_all_check} would also have matched $\oracle$ and $\oracle'$ in Game~\prevgame{}.
This shows that Game~\prevgame{} and Game~\game{} matches exactly the same sessions when no channel key collision occurs.


To bound $\Pr[\KDFcoll_{\mathsf{\game}}]$
we create an algorithm $\A[C]_1$ that finds key collisions in $\protocol.\mathsf{Kdf}$.
Algorithm $\A[C]_1$ emulates adversary $\A$ and the challenger in an execution of Game~\game{} by
instantiating all the parties' long-term keys and running all the sessions according to the specification of the game. 
If event $\KDFcoll_{\mathsf{\game}}$ happened during this run,
say due to the calls $\protocol.\mathsf{Kdf}(ms, \nonce, F_{\Pi}(\localtranscript))$ and $\protocol.\mathsf{Kdf}(ms',\allowbreak \nonce, F_{\Pi}(\localtranscript))$,
then algorithm $\A[C]_1$ outputs $((ms,ms'), \nonce, F_{\Pi}(\localtranscript))$ as its collision for $\protocol.\mathsf{Kdf}$.

Since $\A[C]_1$ holds all the keys,
it can simulate Game~\game{} perfectly.
In particular,
it can correctly simulate the random oracle $G$ in those places where it is called inside of Game~\game{} 
(i.e., \cref{alg:line:send_game_hops_RO_call} of the $\Send$-code,
and \cref{alg:line:game_hops_code_G_derive_key} of the $G$-code).
Thus,
the probability that $\A[C]_1$ finds a collision in $\protocol.\mathsf{Kdf}$ is exactly the probability that event $\KDFcoll_\mathsf{\game}$ occurs during its simulation of Game~\game{} for $\A$. 
\end{proof}

\begin{remark}\label{remark:EAP-TLS:proof:game_hop:KDF_collision}
The reason we have to condition on there being no channel key collision in Game~\game{}
is because we do not assume that being partners necessarily implies having equal master secrets.
It is conceivable that two partner sessions might end up with the same channel key (and export key) even if their master secrets differ.
This would lead to a discrepancy in how $G$ queries are answered in Game~\prevgame{} and Game~\game{}.
\end{remark}




\newgame
\paragraph{Game~\game:} \label{game:match_non_fresh}
In this game the $\Send$-code is augmented by matching non-fresh sessions based on their channel keys
(see \cref{fig:code_for_game_hops_Send}). 
That is,
if two non-fresh sessions are found to have the same channel key
(and the same nonces and auxiliary data),
then they are given the same export key too.

\begin{claim}
\begin{equation}
	\adv_{\protocolderived,f}^{\mathsf{G_{\prevgame{}}}}(\A) 
		\leq \adv_{\protocolderived,f}^{\mathsf{G_\game}}(\A) 
		+ \Pr[\KDFcoll_{\mathsf{\game}}] .
\end{equation}
%\begin{equation}
%	\adv_{\protocolderived,f}^{\mathsf{G_{\prevgame{}}}}(\A) 
%		- \adv_{\protocolderived,f}^{\mathsf{G_\game}}(\A) 
%		\leq \Pr[\KDFcoll_{\mathsf{\game}}]
%		\leq \adv_{\protocol.\mathsf{Kdf}}^{\mathsf{KDFcoll}}(\A[C]_2) .
%\end{equation}
\end{claim}

\begin{proof}

Again,
as long as a channel key collision does not occur (event $\KDFcoll_{\mathsf{\game}}$),
then Game~\prevgame{} and Game~\game{} are identical.
To bound $\Pr[\KDFcoll_{\mathsf{\game}}]$ 
we build an algorithm $\A[C]_2$ against the collision resistance of $\protocol.\mathsf{Kdf}$
just like we created $\A[C]_1$ in the proof of \cref{claim:game:match_fresh_sessions}.
\end{proof}



\newcommand*{\codewidth}{0.7\linewidth}
\newcommand*{\codescale}{0.7}
\colorlet{hidecode}{lightgray!2}

\begin{figure}


\begin{leftfullpage}

%\scriptsize

\small

\setlength{\tabcolsep}{0.2pt}
\renewcommand{\arraystretch}{2}


%\begin{adjustbox}{tabular={cc},scale=\codescale,center}
\begin{adjustbox}{tabular={cc},Trim=0.9ex 0 0 0, width=\textwidth}
	\tikzmark{left}
	\begin{adjustbox}{minipage=[t]{\codewidth},fbox}
		\begin{algorithmic}[1]	
%			\Statex 
		

			
			\color{hidecode}
			\State \textcolor{hidecode}{\emph{// match partner sessions}}
			\color{black}
			\Comment Game~\ref{game:introduce_L_G}
			\color{hidecode}
			\If{$\exists (*, \nonce, aux, ek, [\oracle']) \in L_G \land f_T(\oracle) = \oracle'$}\label{alg:line:send_game_hops_match_fresh}
				\State $\oracle.\key \gets ek$
				\State update $(*, \nonce, aux, ek, [\oracle'])$ to $(*, \nonce, aux, ek, [\oracle', \oracle])$
			\tikzmark{begin}
			\tikzmark{end}
			\Codebox{0}{0}{0}
		
		%	\color{black}
		
			\State \textcolor{hidecode}{\emph{// match non-fresh sessions on their channel keys}}
			\ElsIf{$\left(\exists (*, \nonce, aux, ek, [\oracle']) \in L_G \right)$ 
					\Statex \hspace*{\algorithmicindent}\hspace*{\algorithmicindent}\hspace*{\algorithmicindent} $\land \left( \oracle, \oracle' \text{ non-fresh} \right) \land \left( \oracle.ck = \oracle'.ck \right)$}
				\State $\oracle.\key \gets ek$
				\State update $(*, \nonce, aux, ek, [\oracle'])$ to $(*, \nonce, aux, ek, [\oracle', \oracle])$	
			
			\EndIf 
%			\vspace{-2.9pt}
			
			\color{black}
			
		
			\State \codecomment{look at the master secret}
			\If{$\exists (\patternmatch{ms}, \patternmatch{\nonce}, \patternmatch{aux}, ek, [*]) \in L_G$}\label{alg:line:send_game_hops_catch_all_check}
				\State $\oracle.\key \gets ek$
				\State update $(ms, \nonce, aux, ek, [*])$ to $(ms, \nonce, aux, ek, [*,\oracle])$
				\label{alg:line:send_game_hops_catch_all_final_line}
			
			\color{black}
			
			\Else  
				\State \codecomment{no match found -- derive new key}
				\State $ek \gets G(ms, \nonce, aux)$\label{alg:line:send_game_hops_RO_call}
				\State $\oracle.\key \gets ek$
				\State $L_G \gets L_G \cup (ms, \nonce, aux, ek, [\oracle]) $
			\EndIf
			
		\end{algorithmic}
	\end{adjustbox}	
	
	&	
	
	\tikzmark{left}
	\begin{adjustbox}{minipage=[t][]{\codewidth},fbox}
	
		\begin{algorithmic}[1]
			
%			\Statex 
			
			\color{black}
			\State \codecomment{match partner sessions}
			\Comment Game~\ref{game:match_fresh_sessions}
			\tikzmark{begin}
			\If{$\left( \exists (*, \patternmatch{\nonce}, \patternmatch{aux}, ek, [\oracle']) \in L_G \right) \land \left( f_T(\oracle) = \oracle' \right)$}	
				\State $\oracle.\key \gets ek$
				\State update $(*, \nonce, aux, ek, [\oracle'])$ to $(*, \nonce, aux, ek, [\oracle', \oracle])$\tikzmark{end}
		
			\color{hidecode}
		
			\State \hspace*{-\algorithmicindent} \textcolor{hidecode}{\emph{// match non-fresh sessions on their channel keys}}		
			\ElsIf{$\left(\exists (*, \nonce, aux, ek, [\oracle']) \in L_G \right)$ 
				\Statex	\hspace*{\algorithmicindent}\hspace*{\algorithmicindent}\hspace*{\algorithmicindent} $\land \left( \oracle, \oracle' \text{ non-fresh} \right) \land \left( \oracle.ck = \oracle'.ck \right)$}
				\State $\oracle.\key \gets ek$
				\State update $(*, \nonce, aux, ek, [\oracle'])$ to $(*, \nonce, aux, ek, [\oracle', \oracle])$
			
			\color{black}
			\vspace{2.26pt}
		
			\State \hspace*{-\algorithmicindent} \codecomment{look at the master secret}
			\ElsIf{$\exists (\patternmatch{ms},\patternmatch{\nonce}, \patternmatch{aux}, ek, [*]) \in L_G$}
				\State $\oracle.\key \gets ek$
				\State update $(ms, \nonce, aux, ek, [*])$ to $(ms, \nonce, aux, ek, [*,\oracle])$
			
			\color{black}
			
			\Else 
				\State \codecomment{no match found -- derive new key}
				\State $ek \gets G(ms, \nonce, aux)$
				\State $\oracle.\key \gets ek$
				\State $L_G \gets L_G \cup  (ms, \nonce, aux, ek, [\oracle]) $
			\EndIf
			

			
		\end{algorithmic}
	\end{adjustbox}\Codebox{11.5}{2}{242}
	

	\\

	\tikzmark{left}
	\begin{adjustbox}{minipage=[t][]{\codewidth},fbox}
		\begin{algorithmic}[1]
		
			\State \codecomment{match partner sessions}
			\Comment Game~\ref{game:match_non_fresh}
			
			\If{$\left( \exists (*, \patternmatch{\nonce}, \patternmatch{aux}, ek, [\oracle']) \in L_G \right) \land \left( f_T(\oracle) = \oracle' \right)$}
				\State $\oracle.\key \gets ek$
				\State update $(*, \nonce, aux, ek, [\oracle'])$ to $(*, \nonce, aux, ek, [\oracle', \oracle])$\tikzmark{begin}
		
			\color{black}
		
			\State \hspace*{-\algorithmicindent} \codecomment{match non-fresh sessions on their channel keys}
			\ElsIf{$\left(\exists (*, \patternmatch{\nonce}, \patternmatch{aux}, ek, [\oracle']) \in L_G \right)$ 
					\Statex \hspace*{\algorithmicindent}\hspace*{\algorithmicindent}\hspace*{\algorithmicindent} $\land \left( \oracle, \oracle' \text{ non-fresh} \right) \land \left( \oracle.ck = \oracle'.ck \right)$}
				\State $\oracle.\key \gets ek$
				\State update $(*, \nonce, aux, ek, [\oracle'])$ to $(*, \nonce, aux, ek, [\oracle', \oracle])$\tikzmark{end}
			
			\color{black}
		
			\State \hspace*{-\algorithmicindent} \codecomment{look at the master secret}
			\ElsIf{$\exists (\patternmatch{ms}, \patternmatch{\nonce}, \patternmatch{aux}, ek, [*]) \in L_G$}
				\State $\oracle.\key \gets ek$
				\State update $(\patternmatch{ms},\patternmatch{\nonce}, \patternmatch{aux}, ek, [*])$ to $(ms, \nonce, aux, ek, [*,\oracle])$
			
			\color{black}
			
			\Else 
				\State \codecomment{no match found -- derive new key}
				\State $ek \gets G(ms, \nonce, aux)$
				\State $\oracle.\key \gets ek$
				\State $L_G \gets L_G \cup (ms, \nonce, aux, ek, [\oracle]) $
			\EndIf
			
		\end{algorithmic}
	\end{adjustbox}\Codebox{-0.2}{0.45}{238}

	&
	
	
	\tikzmark{left}
	\begin{adjustbox}{minipage=[t][]{\codewidth},fbox}
		\begin{algorithmic}[1]	
		
			\State \codecomment{match partner sessions}
			\Comment Game~\ref{game:drop_RO}
			
			\If{$\left( \exists (*, \patternmatch{\nonce}, \patternmatch{aux}, ek, [\oracle']) \in L_G \right) \land \left( f_T(\oracle) = \oracle' \right)$}
				\State $\oracle.\key \gets ek$
				\State update $(*, \nonce, aux, ek, [\oracle'])$ to $(*, \nonce, aux, ek, [\oracle', \oracle])$
		
			\color{black}
		
			
			\State \hspace*{-\algorithmicindent} \codecomment{match non-fresh sessions on their channel keys}
			\ElsIf{$\left(\exists (*, \patternmatch{\nonce}, \patternmatch{aux}, ek, [\oracle']) \in L_G \right)$ 
					\Statex \hspace*{\algorithmicindent}\hspace*{\algorithmicindent}\hspace*{\algorithmicindent} $\land \left( \oracle, \oracle' \text{ non-fresh} \right) \land \left( \oracle.ck = \oracle'.ck \right)$}
				\State $\oracle.\key \gets ek$
				\State update $(*, \nonce, aux, ek, [\oracle'])$ to $(*, \nonce, aux, ek, [\oracle', \oracle])$
			
			\color{black}
		
			\State \hspace*{-\algorithmicindent} \codecomment{look at the master secret}
			\ElsIf{$\exists (\patternmatch{ms},\patternmatch{\nonce}, \patternmatch{aux}, ek, [*]) \in L_G$}
				\State $\oracle.\key \gets ek$
				\State update $(ms, \nonce, aux, ek, [*])$ to $(ms, \nonce, aux, ek, [*,\oracle])$ \label{alg:line:send_game_hops:drop_RO:look_at_MS}
			
			\color{black}
			
			\Else
				\State \codecomment{no match found -- derive new key}
				\tikzmark{begin} 
				\color{black}
				\State $ek \getsr \bits^\keylen$\tikzmark{end}
				\color{black}
				\State $\oracle.\key \gets ek$
				\State $L_G \gets L_G \cup  (ms, \nonce, aux, ek, [\oracle]) $
			\EndIf
			
		\end{algorithmic}
	\end{adjustbox}\Codebox{0.15}{0.9}{83.5}
	
	\\
	
	\tikzmark{left}
	\begin{adjustbox}{minipage=[t][]{\codewidth},fbox}
		\begin{algorithmic}[1]			
		
			\State \codecomment{match partner sessions}
			\Comment Game~\ref{game:remove_code_send}
			
			\If{$\left( \exists (*, \patternmatch{\nonce}, \patternmatch{aux}, ek, [\oracle']) \in L_G \right) \land \left( f_T(\oracle) = \oracle' \right)$}
				\State $\oracle.\key \gets ek$
				\State update $(*, \nonce, aux, ek, [\oracle'])$ to $(*, \nonce, aux, ek, [\oracle', \oracle])$
		
			\color{black}
		
			\State \hspace*{-\algorithmicindent} \codecomment{match non-fresh sessions on their channel keys}
			\ElsIf{$\left(\exists (*, \patternmatch{\nonce}, \patternmatch{aux}, ek, [\oracle']) \in L_G \right)$ 
					\Statex \hspace*{\algorithmicindent}\hspace*{\algorithmicindent}\hspace*{\algorithmicindent} $\land \left( \oracle, \oracle' \text{ non-fresh} \right) \land \left( \oracle.ck = \oracle'.ck \right)$}
				\State $\oracle.\key \gets ek$
				\State update $(*, \nonce, aux, ek, [\oracle'])$ to $(*, \nonce, aux, ek, [\oracle', \oracle])$ \tikzmark{begin}
			
			\color{black}
			
			\State \hspace*{-\algorithmicindent} \codecomment{has $G$ already been queried on an $ms'$ valid for $\oracle$?}
			\ElsIf{$\left( \exists (ms', \patternmatch{\nonce}, \patternmatch{aux}, ek, [*]) \in L_G \right)$
				\Statex \hspace*{\algorithmicindent} \hspace*{\algorithmicindent}	$\land \left( \CK(\oracle, ms') = \TRUE \right)$} \label{alg:line:send_game_hops_CheckKey}\tikzmark{end}	
				\State $\oracle.\key \gets ek$
				\State update\hspace*{-0.9pt} $(ms', \nonce, aux, ek, [*])$ \hspace*{-0.5pt}to\hspace*{-0.5pt} $(ms', \nonce, aux, ek, [*,\oracle])$\hspace{-2pt}
			\color{black}		
			\Else 
				\State \codecomment{no match found -- derive new key}
				\State $ek \getsr \bits^\keylen$
				\color{black}
				\Codebox{-0.5}{1}{240}{}	
				\State $\oracle.\key \gets ek$\tikzmark{begin}
				\color{black}
				\State $L_G \gets L_G \cup  (\bot, \nonce, aux, ek, [\oracle])$\label{alg:line:send_game_drop_ms}\tikzmark{end}
			\EndIf
			
		\end{algorithmic}
	\end{adjustbox}\Codebox{0.5}{1.8}{180}
	
	&
	
\end{adjustbox}


\caption{How to the derive the export key $ek$ of a session $\oracle$ that accepted with master secret $\patternmatch{ms}$, 
nonces $\patternmatch{\nonce = \nonce_{C} \concat  \nonce_{S}}$, and auxiliary data $\patternmatch{aux}$,
in Game~\ref{game:introduce_L_G} to Game~\ref{game:remove_code_send}.
%Variables with underscores denote those that are `pattern matched'' against $\oracle$'s variables.
%For example, ``$(a,\patternmatch{b},\patternmatch{c}, \patternmatch{d},ek,[*]) \in L_G$'' matches $\oracle$ only if $\nonce_{C} = b$, $\nonce_{S} = c$, and $aux = d$.
}
\label{fig:code_for_game_hops_Send}

	\end{leftfullpage}

\end{figure}



\renewcommand*{\codewidth}{0.7\linewidth}
%\newcommand*{\codescale}{0.65}

\begin{figure}

\begin{fullpage}

\small

\setlength{\tabcolsep}{0.5pt}
\renewcommand{\arraystretch}{2}

%\begin{adjustbox}{tabular={cc},scale=\codescale,center}
\begin{adjustbox}{tabular={cc},Trim=0.5ex 0 0 0, width=\textwidth}
%	\tikzmark{left}
	\begin{adjustbox}{minipage=[t][]{\codewidth},fbox}
		\begin{algorithmic}[1]	
			\Statex 
			\Comment \cref{game:introduce_L_G}--\cref{game:match_non_fresh}
%			\Comment \cref{game:introduce_L_G}
		
			\State \codecomment{$G$ queried on the same value before?}
			
			\If{$\exists (\patternmatch{ms}, \patternmatch{\nonce}, \patternmatch{aux}, ek, [*]) \in L_G$}
		    	\State \Return $ek$
%		    \tikzmark{begin}
%   			\tikzmark{end}
%   			\Codebox{0}{0}{0}
		    	
		    \color{hidecode}
		    
		    \State \textcolor{hidecode}{\emph{// test if $ms$ matches any already accepted sessions}}	
			\ElsIf{$\left( \exists (\bot, \nonce, aux, ek, [\oracle,*]) \in L_G \right)$\\
   				\hspace*{\algorithmicindent}\hspace*{\algorithmicindent}\hspace*{\algorithmicindent} $\land \left( \CK(\oracle, ms) = \TRUE \right)$}		
				\State update $(*, \nonce, aux, ek, [\oracle,*])$ to $(ms, \nonce, aux, ek, [\oracle,*])$
				\State \Return $ek$
		
			\color{black}
			
			\Else\label{alg:line:G-code:else:clause}
				\State \codecomment{no match found -- derive new key}
				\State $ek \gets G(ms, \nonce, aux)$
				\State $L_G \gets L_G \cup  (ms, \nonce, aux, ek, [\,]) $
				\State \Return $ek$
			\EndIf
			
		\end{algorithmic}
	\end{adjustbox}
	
	\hspace*{-6.15pt}
	
	&
	
	\tikzmark{left}
	\begin{adjustbox}{minipage=[t][]{\codewidth},fbox}
		\begin{algorithmic}[1]
			\Statex
			\Comment \cref{game:drop_RO}
		
			\State \codecomment{$G$ queried on the same value before?}
			
			\If{$\exists (\patternmatch{ms}, \patternmatch{\nonce}, \patternmatch{aux}, ek, [*]) \in L_G$}\label{alg:line:game_hops_code_G_full_if}
		    	\State \Return $ek$
			
			\color{hidecode}
		
			\State \textcolor{hidecode}{\emph{// test if $ms$ matches any already accepted sessions}} 	
			\ElsIf{$\left( \exists (\bot, \nonce, aux, ek, [\oracle,*]) \in L_G \right)$\\
    			\hspace*{\algorithmicindent}\hspace*{\algorithmicindent}\hspace*{\algorithmicindent} $\land \left( \CK(\oracle, ms) = \TRUE \right)$} \label{alg:line:game_hops_code_G_CheckKey}	
				\State update $(\bot, \nonce, aux, ek, [\oracle,*])$ to $(ms, \nonce, aux, ek, [\oracle,*])$
				\State \Return $ek$
					
			\color{black}
			
			\Else
				\State \codecomment{no match found -- derive new key}
				\tikzmark{begin}
				\color{black}
				\State $ek \getsr \bits^\keylen$\label{alg:line:game_hops_code_G_derive_key}\tikzmark{end}
				\color{black}
				\State $L_G \gets L_G \cup  (ms, \nonce, aux, ek, [\,]) $
				\State \Return $ek$
			\EndIf
		\end{algorithmic}
	\end{adjustbox}\Codebox{0.1}{0.5}{83.5}
	
	\\ 	
	
	\tikzmark{left}
	\begin{adjustbox}{minipage=[t][]{\codewidth},fbox}
		\begin{algorithmic}[1]
			\Statex
			\Comment \cref{game:remove_code_send}
		
			\State \codecomment{$G$ queried on the same value before?}
			
			\If{$\exists (\patternmatch{ms}, \patternmatch{\nonce}, \patternmatch{aux}, ek, [*]) \in L_G$}
		    	\State \Return $ek$\tikzmark{begin} 
		   
			\color{black}
			
			\State \hspace*{-\algorithmicindent} \codecomment{test if $ms$ matches any already accepted sessions}
			\ElsIf{$\left( \exists (\bot, \patternmatch{\nonce}, \patternmatch{aux}, ek, [\oracle,*]) \in L_G \right)$\\
			    \hspace*{\algorithmicindent}\hspace*{\algorithmicindent}\hspace*{\algorithmicindent} $\land \left( \CK(\oracle, ms) = \TRUE \right)$}	\label{alg:proof:EAP-TLS:G-code:CheckKey}
				\State update $(\bot, \nonce, aux, ek, [\oracle,*])$ to $(ms, \nonce, aux, ek, [\oracle,*])$
				\State \Return $ek$\tikzmark{end} 	
			
			\color{black}
		
			\Else
				\State \codecomment{no match found -- derive new key}
				\State $ek \getsr \bits^\keylen$
				\State $L_G \gets L_G \cup (ms, \nonce, aux, ek, [\,]) $
				\State \Return $ek$
			\EndIf	
		\end{algorithmic}
	\end{adjustbox}\Codebox{0.5}{0}{242} %
	
	&
	
\end{adjustbox}


\caption{How $G$ queries (RO calls) of the form $G(\patternmatch{ms},\patternmatch{\nonce},\patternmatch{aux})$ are answered in Game~\ref{game:introduce_L_G} to Game~\ref{game:remove_code_send}.}
\label{fig:code_for_game_hops_G}

\end{fullpage}

\end{figure}


\newgame
\paragraph{Game~\game:} \label{game:drop_RO} 
In this game the challenger replaces the calls to the random oracle 
(both in the $\Send$-code and in the $G$-code)
with strings drawn uniformly at random.

\begin{claim}
\begin{equation}
	\adv_{\protocolderived,f}^{\mathsf{G_{\prevgame{}}}}(\A) 
		= \adv_{\protocolderived,f}^{\mathsf{G_\game}}(\A) .
\end{equation}
\end{claim}

\begin{proof}
We show that the challenger in Game~\prevgame{} never repeats a call to the random oracle on the same input.
Thus, replacing these calls with uniformly drawn strings in Game~\game{} yields exactly the same distribution on the export keys.

Suppose at some point during Game~\prevgame{} the challenger made the random oracle call $G(ms, \nonce, aux)$ for the first time
(either due to a session accepting,
or because $\A$ made this exact $G$ query).
Suppose the random oracle responded with $ek$,
and let $t = (ms, \nonce, aux, ek, [*])$ be the tuple that was added to $L_G$ in response to the call.

If the adversary later  makes a $G$ query on the same values,
i.e. a query of the form $G(ms, \nonce,aux)$,
then \cref{alg:line:game_hops_code_G_full_if} of the $G$-code will be used to answer the query.
Thus, the random oracle call on \cref{alg:line:game_hops_code_G_derive_key} of the $G$-code would never be made on the same values twice in Game~\prevgame{}.

Likewise, 
if a session $\oracle$ accepts with the same values,
i.e., master secret $ms$, nonces $\nonce = \nonce_{C} \concat \nonce_{S}$, 
and auxiliary data $aux$,
\emph{after} the initial $G$ query was made,
then the else-if check on \cref{alg:line:send_game_hops_catch_all_check} of the $\Send$-code would match $\oracle$ to $t$. 
Thus, the random oracle call on \cref{alg:line:send_game_hops_RO_call} of the $\Send$-code would not be made on the same values twice in Game~\prevgame{} either.
\end{proof}



In the final game hop the challenger will derive the sessions' export keys independently of their master secrets.
To do this,
the challenger will use a probabilistic \emph{key-checking} procedure called $\CK$ to test whether the adversary queried the random oracle at the correct master secret of a session.
The $\CK$ procedure is defined in \cref{alg:CheckKey}.
The statements in blue boxes can be ignored for now.

\begin{algorithm}


\caption{$\CK(\oracle, ms)$}
\label{alg:CheckKey}

\medskip

\textbf{Note:} The procedure is parameterized by some integer $c \in \N$.
Calls on the same input always return the same value,
i.e.
$\mathbf{CheckKey}$ caches its results for every input combination.
To simplify the presentation this is not shown below.
Statements shown in blue boxes are only executed by reduction algorithm $\A[B]$.

\medskip

\textbf{Precondition:} 
To every session $\oracle$,
$\CK$ associates a random bit $d$,
written $\oracle.d$.
Let $L_{\oracle} = \lbrace (C_{1}, H_1), \dotsc, (C_r,H_r), (C_{r+1}, \emptystring), \dotsc, (C_s, \emptystring) \rbrace$ be the list of all the encrypted handshake messages (if any) output by $\oracle$ during the run of $\protocolderived$,
as well as all the ciphertexts produced by previous calls to  $\CK(\oracle,*)$.


\tcbset{colback={CornflowerBlue!15},boxrule=0pt, sharpish corners, on line, boxsep=0pt,top=3pt, bottom=2pt,left=4pt,right=4pt}

\begin{algorithmic}[1]
	\Statex
	
	\State $x,y \getsr \bits^{c}$
	\State $m_0 \gets 0 \concat x$
	\State $m_1 \gets 1 \concat y$
	\State

	\State \codecomment{ $\nonce = \nonce_{C} \concat \nonce_{S}$ are the nonces $\oracle$ accepted with, and $aux \gets F_\protocol(\oracle.\localtranscript)$}
	\State\label{alg:CheckKey_derive_key}$ck' \gets \protocol.\mathsf{Kdf}(ms, \nonce, aux)$
%		\Comment \textcolor{blue}{\emph{$\nonce_{C}, \nonce_{S}$ are the nonces, and $T$ the transcript, $\oracle$ accepted with}}
	\State
	

	\If{$\oracle$ is non-fresh}
		\State $ck \gets \oracle.ck$ 	\hspace{1em} 	\tcbox{$ck \gets \Reveal(\oracle)$} \label{alg:CheckKey:obtain_ck}	
		\State \Return $ck \overset{?}{=} ck'$\label{alg:CheckKey:compare_CKs}	
	\Else
		
		\State \codecomment{obtain an encryption of $m_{\oracle.d}$ under $\oracle.ck$}

		\State $(C, \stE) \gets \stae.\staeenc(\oracle.ck, m_{\oracle.d}, \emptystring, \stE)$
		\hspace{1em}  \tcbox{$C \gets \Encrypt(\oracle, m_0, m_1, \emptystring)$} \label{alg:CheckKey:encrypt_mb}
		
			
		

		
		
%		\Comment \textcolor{blue}{\emph{obtain an encryption of $m_{\oracle.b}$ under $\oracle.ck$}}

		\State

		\State \codecomment{recreate a decrypt state $\stD'$}
		\State $(*, st'_D) \gets \stae.\staeinit$ \label{alg:CheckKey:recreate_state:init}
		\ForAll{$(C',H') \in L_{\oracle}$}
			\State $(*, st'_D) \gets \stae.\staedec(ck',C',H', \stD')$\label{alg:CheckKey:recreate_state:decrypt}
		\EndFor
%		\ForAll{$(C_r,H_r)$}
%			\State $(*, st'_D) \gets \mathsf{stE.Dec}(ck',C_r,H_r, st'_D)$
%		\EndFor
%		\ForAll{$C$'s from previous calls to $\CK(\oracle,*)$}
%			\State $(*, st'_D) \gets \mathsf{stE.Dec}(ck',C,H, st'_D)$
%		\EndFor
 
		\State
		\State \codecomment{decrypt $C$ using $ck'$ and $st'_D$}
		\State\label{alg:CheckKey_line_decrypt} $(m',*) \gets \stae.\staedec(ck', C, \emptystring, st'_D)$
%		\Comment \textcolor{blue}{\emph{locally decrypt $C$ using $ck'$ and $st'_D$}}

		\State
		\State \Return $m' \overset{?}{\in} \lbrace m_0, m_1 \rbrace$
	
	\EndIf
\end{algorithmic}


\end{algorithm}


The idea of $\CK$ is to test the validity of a supplied master secret \emph{indirectly} by checking whether it derives the same channel key as the one held by the session.
Of course,
this whole exercise might seem totally pointless,
since the challenger has direct access to the session's master secrets.
However,
the purpose of the game hop is to prepare the ground for a subsequent reduction to the ACCE security of protocol $\protocol$.
This reduction will \emph{not} have direct access to the session's master secrets,
hence it needs to be able to simulate this key-checking procedure.

The $\CK$ procedure can be explained as follows.
After being given a master secret $ms$ and a session $\oracle$,
it first derives a guess on $\oracle$'s channel key,
denoted $ck'$ (\cref{alg:CheckKey_derive_key} in \cref{alg:CheckKey}).
If $\oracle$ is non-fresh,
then $\CK$ simply compares $ck'$ with $\oracle.ck$ directly (\cref{alg:CheckKey:compare_CKs}).
On the other hand,
when $\oracle$ is fresh,
then $\CK$ tests the validity of $ck'$ by trying to decrypt a ciphertext $C$ that was legitimately created with the actual channel key of $\oracle'$.

Unfortunately,
this stage is complicated by the fact that the stAE scheme is stateful.
Recall that a stAE scheme maintains two counters $\stE$ and  $\stD$ for encryption and decryption,
respectively.
Before attempting to decrypt $C$,
$\CK$ first needs to recreate a valid decryption state $\stD$.
This is shown at \crefrange{alg:CheckKey:recreate_state:init}{alg:CheckKey:recreate_state:decrypt}.
Basically,
starting from the initial state of the stAE scheme,
$\CK$ chronologically decrypts each encrypted message output by the session during the handshake (if any).
Then it decrypts all ciphertext messages created in prior calls to $\CK$  
(because these advance the session's encrypt state $\stE$).
If the correct channel key was used,
then this process is guaranteed to generate a decryption state $st_D'$ that ``matches''\footnote{We write ``matches'' since the recreated state $st_D'$ will not necessarily be \emph{equal} to the decryption state held by $\oracle$, only that it has the property of yielding a valid decryption.}
the encrypt state $st_E$ which was used to create the ciphertext $C$. 
Finally, $\CK$ attempts the decryption of $C$ (\cref{alg:CheckKey_line_decrypt}).   


If $\CK$ was called on the correct master secret of a session $\oracle$,
then the above shows that it will always return $\TRUE$ since the derived channel key $ck'$ will equal $\oracle.ck$ because  $\protocol.\mathsf{Kdf}$ is deterministic.
Conversely,
if $\CK$ was called on a wrong master secret, then it is possible that it incorrectly returns $\TRUE$.
Namely,
if the derivation of the channel key $ck'$ at \cref{alg:CheckKey_derive_key} in \cref{alg:CheckKey} yields the \emph{same} channel key as $\oracle$,
then $\CK$ will erroneously return $\TRUE$ both when $\oracle$ is fresh and when it is non-fresh.
Moreover,
even if the derived channel key was wrong,
there is still a possibility of error when $\oracle$ is fresh:
by pure chance the decryption at \cref{alg:CheckKey_line_decrypt} of \cref{alg:CheckKey} could return one of the messages $m_0$ or $m_1$ even with the wrong key.

Thus,
$\CK$ has a one-sided error probability.
Let $\CKerror$ denote the event that a call to $\CK$ erroneously returns $\TRUE$.


\newgame
\paragraph{Game~\game:} \label{game:remove_code_send} 
The challenger in \cref{game:remove_code_send} proceeds as in \cref{game:drop_RO},
except that it starts using the $\CK$ procedure as indicated on \cref{alg:line:send_game_hops_catch_all_check} of the $\Send$-code and \cref{alg:proof:EAP-TLS:G-code:CheckKey} of the $G$-code
(\cref{fig:code_for_game_hops_Send} and \cref{fig:code_for_game_hops_G} respectively).
Additionally,
if a session accepts without a match on $L_G$,
then Game~\game{} omits its master secret from the tuple that gets added to $L_G$ in the $\Send$-code (\cref{alg:line:send_game_drop_ms}).

\begin{claim}\label{claim:EAP-TLS:introduce_CheckKey}
\begin{align}\label{eq:game_hop:remove_code_and_drop_MS}
	\adv_{\protocolderived,f}^{\mathsf{G_{\prevgame{}}}}(\A) 
		\leq \adv_{\protocolderived,f}^{\mathsf{G_\game}}(\A) 
		+ \Pr[\CKerror] .
\end{align} 
\end{claim}

\begin{proof}
By inspecting the $\Send$-code and $G$-code of Game~\prevgame{} and Game~\game{},
one sees that they proceed identically unless event $\CKerror$ occurs.
In particular,
provided $\CK$ does not make a mistake,
then the else-if clause on \cref{alg:line:send_game_hops_CheckKey} in the $\Send$-code of Game~\game{} matches $\oracle$ with a tuple on $L_G$ if and only the tuple contains the correct master secret of $\oracle$ (plus of course all the other input to the KDF, which we ignore here).
But this is exactly the same as what the else-if clause on \cref{alg:line:send_game_hops:drop_RO:look_at_MS} in the $\Send$-code of Game~\prevgame{} does too.
Similarly,
in the $G$-code of Game~\game{},
the else-if clause on \cref{alg:line:game_hops_code_G_CheckKey} assigns a master secret $ms$ to a tuple of $L_G$ if and only it matches the master secret of those sessions contained in the tuple.
Combined with the preceding argument for the $\Send$-code of Game~\game{},
this means that no more sessions gets matched to tuples in $L_G$ in the $\Send$-code of Game~\game{} than in Game~\prevgame{}.

Hence, 
provided $\CK$ does not make a mistake,
Game~\prevgame{} and Game~\game{} proceed identically
and the \lcnamecref{claim:EAP-TLS:introduce_CheckKey} follows.
\end{proof}




It remains to bound the right-hand side of \cref{eq:game_hop:remove_code_and_drop_MS}.
Recall that $\CKerror$ represents the event that $\CK$ erroneously returns $\TRUE$ on a wrong master secret.
Note that this can happen both with a fresh session and with a non-fresh session.
Let $\eventFresh$ denote that $\CK$ was called on a fresh session $\oracle$ according to predicate $\fresh_{\acce}$,
and let $\nfresh$ denote that $\CK$ was called on a non-fresh session.
Then we have:
\begin{equation}
\Pr[\CKerror] \leq \Pr[\CKerror \land \eventFresh] + \Pr[\CKerror \land \nfresh] .
\end{equation}

In the case of a non-fresh session,
$\CK$ can by design only make a mistake if the there is a key collision,
so the next claim follows at once.

\begin{claim}\label{claim:EAP-TLS:CKerror&not-fresh<KDFcoll}
\begin{equation}
	\Pr[\CKerror \land \nfresh] \leq \adv_{\protocol.\mathsf{Kdf}}^{\kdfcoll}(\A[C]_3) .
\end{equation}
\end{claim}

Consequently,
we are left to bound $\Pr[\CKerror \land \eventFresh]$.
To this end,
we define the following event:
\begin{equation}\label{eq:def:event_Q}
	Q: \CK \text{ returns $\TRUE$ when called on a fresh session}.
\end{equation}  

We stress that if event $Q$ happened,
say due to a call $\CK(\oracle,ms')$,
then this does not necessarily imply that $\oracle .ms = ms'$.
Event $Q$ also includes those cases where $\CK$ erroneously returns $\TRUE$.
We will later show that $\A$ has zero advantage in guessing the $\Test$-challenge correctly unless $Q$ happens (\cref{lemma:A_wins_not_Q}).
The probability $\Pr[\CKerror \land \eventFresh]$ can now be bounded in terms of the occurrence of event $Q$.

\begin{claim}\label{claim:EAP-TLS:CKerror&fresh<Q}
\begin{equation}
	\Pr[\CKerror \land \eventFresh] \leq 2 \cdot \Pr[Q] .
\end{equation}
\end{claim}


\begin{proof}

Event $\CKerror \land \eventFresh$ only occurs if the decryption of $C$ at \cref{alg:CheckKey_line_decrypt} of \cref{alg:CheckKey} returned one of the two messages $m_0$ and $m_1$.
We write $\correct$ for the event that $C$ got decrypted to $m_d$,
and $\wrong$ for the event that it got decrypted to $m_{\overline{d}}$,
where $d$ is the bit associated to the session $\oracle$ in the $\CK$ procedure.\footnote{Note that event $\correct$ can happen both legitimately 
($\oracle.ms = ms'$) and due to an error ($\oracle.ms \neq ms'$).
Event $\wrong$ can only happen because of an error.}
The events $\correct$ and $\wrong$ are mutually exclusive,
so 
\begin{equation}
	\Pr[\CKerror \land \eventFresh] = \Pr[\correct] + \Pr[\wrong] .
\end{equation}

Finally,
within the context of $\CK$,
both $\correct$ and $\wrong$ are sub-events of $Q$,
hence, $\Pr[\correct] + \Pr[\wrong] \leq 2 \cdot \Pr[Q]$. 
\end{proof}

The next claims shows that unless $Q$ happens in \cref{game:remove_code_send},
then $\A$ has zero advantage in  answering the $\Test$-challenge correctly.

\begin{claim}\label{lemma:A_wins_not_Q}
Suppose that $\A$ output $b'$ as its answer to the $\Test$-challenge in  Game~\ref{game:remove_code_send}.
Then,
\begin{equation}\label{eq:pr_not_Q_half}
	\Pr[b' = b \mid \overline{Q}] = \frac{1}{2}  .
\end{equation}
\end{claim}
\begin{proof}
If event $Q$ did not happen,
then $\CK$ never returned $\TRUE$ for any fresh session during Game~\ref{game:remove_code_send}.
Since $\CK$ is always correct when rejecting a key,
i.e., when outputting $\FALSE$,
this implies that $\A$ never queried the random oracle on the correct master secret of any fresh session.
In particular, 
this means that the derived export key of the test-session in Game~\ref{game:remove_code_send} is distributed exactly like that of a random key.
Thus,
the bit $b$ is independent of the derived export key from $\A$'s point of view. 
\end{proof}






\cref{lemma:A_wins_not_Q} implies that it is sufficient to bound the probability of event $Q$  to bound $\A$'s advantage in \cref{game:remove_code_send}. 
Furthermore, \cref{claim:EAP-TLS:CKerror&fresh<Q} showed that the probability of event $\CKerror \land \eventFresh$ is also bounded in terms of $Q$.
Thus,
the only thing that remains in order to bound the right-hand side of \cref{eq:game_hop:remove_code_and_drop_MS} is to bound $\Pr[Q]$.
To this end,
we construct an ACCE adversary $\A[B]$ against protocol $\protocol$ such that
\begin{equation}
	\Pr[Q] \leq  \adv_{\protocol,f}^{\acce}(\A[B]) + \frac{2 \unionbound}{2^{c}} , \label{eq:bound:Q}
\end{equation}
where $q$ is the number of random oracle calls made by $\A$ and $c \in \N$ is the free parameter value of the $\CK$ procedure.


\paragraph{Description of algorithm \texorpdfstring{$\A[B]$}{B}.}
Algorithm $\A[B]$ plays in an ACCE security experiment against protocol $\protocol$.
It simulates Game~\ref{game:remove_code_send} for $\A$ by using the sessions in its own ACCE experiment to represent the sessions in Game~\ref{game:remove_code_send}.
Basically,
$\A[B]$ forwards all of $\A$'s queries to its own ACCE game
(to simulate the $\Test$ query,
$\A[B]$ draws a mock bit $\bsim$).
To simulate the sessions' export keys in \cref{game:remove_code_send},
$\A[B]$ maintains the list $L_G$ which it fills out,
and answers from, 
according to the $\Send$-code and $G$-code of \cref{game:remove_code_send}.
%(shown in \cref{fig:code_for_game_hops_Send} and \cref{fig:code_for_game_hops_G},
%respectively). 
However,
$\A[B]$ implements the $\CK$ procedure slightly different from what the challenger in \cref{game:remove_code_send} does. 

Specifically,
at \cref{alg:CheckKey:obtain_ck} and \cref{alg:CheckKey:encrypt_mb} in \cref{alg:CheckKey},
$\A[B]$ executes the statements shown in blue boxes instead of the respective statements at those lines.
The blue boxes represents queries to $\A[B]$'s ACCE game.
To compare the key $ck'$ with the real channel key of $\oracle$ when $\oracle$ is non-fresh,
$\A[B]$ uses the $\Reveal$ query.
To obtain a valid ciphertext under $\oracle$'s real channel key when $\oracle$ is fresh,
$\A[B]$ uses the ``left-or-right'' $\Encrypt$ query.


Finally,
$\A[B]$ stops and outputs a guess $(\oracle, b')$ in its ACCE game if one of the following events happen.


%\begin{enumerate}[label=\alph*)]
\begin{itemize}
	\item \emph{Two sessions generated the same nonce}: select $\oracle$ arbitrarily among the fresh sessions and draw $b'$ randomly.
	
	\item \emph{$\A$ outputs a guess for the $\Test$-challenge}:
	select $\oracle$ arbitrarily among the fresh sessions and draw $b'$ randomly.
	
	\item \emph{A call to $\CK(\oracle, ms)$ returned $\TRUE$ for a fresh session $\oracle$}:
	
	This means that the decryption at \cref{alg:CheckKey_line_decrypt} of \cref{alg:CheckKey} either returned $m_0$ or $m_1$.
	If the result was $m_0$ then $\A[B]$ outputs $(\oracle, 0)$ to its ACCE game.
	If the result was $m_1$ then $\A[B]$ outputs $(\oracle, 1)$ to its ACCE game.
	
%	\item \emph{A session accepted maliciously}: 
%	select $\oracle$ arbitrarily among the fresh sessions and draw $b'$ randomly.
	

	
\end{itemize}





\paragraph{Analysis of \texorpdfstring{$\A[B]$}{B}.}
Note that the only thing that differs between $\A[B]$'s simulation and Game~\game{} 
is $\A[B]$'s implementation  of the $\CK$ procedure.
However,
$\A[B]$'s usage of $\Reveal$ and $\Encrypt$ queries perfectly simulates the respective lines in \cref{alg:CheckKey}.
Particularly,
the secret bit of a session $\oracle$ in $\A[B]$'s ACCE game,
i.e., $\oracle.b$,
simulates the bit associated to $\oracle$ in the $\CK$ procedure,
i.e., $\oracle.d$.
Thus,
if event $Q$ happens,
then $\A[B]$'s output in its ACCE experiment will be directly related to the value of $\oracle.b$.
On the other hand,
if $Q$ does \emph{not} happen,
then $\A[B]$ by design  wins in its ACCE game with probability $1 / 2$. 

Formally,
suppose $\A[B]$ output $(\oracle, b')$ for some fresh session $\oracle$.
The probability that $\A[B]$ wins in its ACCE security game is then:
\begin{align}
		\Pr[\bwins] 
			&=\Pr[\bwins \mid Q] \cdot \Pr[Q] 
				+ \Pr[\bwins \mid \overline{Q}] \cdot \Pr[\overline{Q}] \label{eq:B_wins:first_equality} \\
		&\overset{(a)}{=}	\Pr[\bwins \mid Q] \cdot \Pr[Q] 
				+ \frac{1}{2}(1 - \Pr[Q])  \\
		&\overset{(b)}{=} \Big( 
				\overbrace{\Pr[\bwins \mid Q \land \correct]}^{=1} \cdot \Pr[\correct \mid Q] \nonumber \\
				&\qquad  + \overbrace{\Pr[\bwins \mid Q \land \wrong]}^{=0} \cdot \Pr[\wrong \mid Q] 			
			\Big) \cdot \Pr[Q] \\
			&\qquad\qquad + \frac{1}{2}(1 - \Pr[Q]) \nonumber \\
		&= \Pr[\correct \mid Q] \cdot \Pr[Q]  + \frac{1}{2} (1 - \Pr[Q]) \\
		&= \Pr[\correct \land Q]  - \frac{1}{2} \cdot \Pr[Q] + \frac{1}{2} \\
		&\overset{(c)}{=} \big( \Pr[Q] - \Pr[\wrong \land Q]  \big) - \frac{1}{2} \Pr[Q] + \frac{1}{2} \\
		&= \frac{1}{2} \Pr[Q] - \Pr[\wrong \land Q]  + \frac{1}{2} \\		
		&= \frac{1}{2} \Pr[Q] - \Pr[\wrong]  + \frac{1}{2} \\		
		&\geq \frac{1}{2} \Pr[Q] 
			- \frac{q \numparties \numsessions}{2^{c}}  
			+ \frac{1}{2}\label{eq:B_wins:final_inequality}  . 		
\end{align}

In $(a)$ we used the fact that $\A[B]$ outputs a random bit when $Q$ does not happen,
while $(b)$ and $(c)$ used that event $Q$ is the union of the mutually exclusive events $\correct$ and $\wrong$.
The final inequality is proved as follows.


%Next consider event $\CKerror \land \eventFresh$. 
%Since this is a sub-event in $\correct \cup \wrong$ 
%(by \eqref{eq:event:CKerror_AND_fresh}),
%it follows that $\Pr[\CKerror \land \eventFresh] \leq \Pr[\correct] + \Pr[\wrong]$.
%To bound $\Pr[\wrong]$,
%suppose $\CK(\oracle, ms)$ was a call where $\oracle$ was fresh
%and $ms$ was \emph{not} the correct master secret of $\oracle$.
Let $\overline{b} = 1 - \oracle.b$ and let $(m_0,m_1)$ be the two messages associated to the pair $(\oracle,ms)$ in $\CK$.
Since $m_{\overline{b}}$ is independent of the ciphertext $C$ produced at \cref{alg:CheckKey:encrypt_mb} of \cref{alg:CheckKey},
the probability that $C$ decrypts to $m_{\overline{b}}$ at \cref{alg:CheckKey_line_decrypt} is statistically bounded by $2^{-c}$ for any key $\key$.
By taking the union bound over all parties, 
the number of sessions per party,
and the number of random oracle calls,
we get that $\Pr[\wrong] \leq q \numparties \numsessions / 2^{c}$.

Solving \eqref{eq:B_wins:final_inequality} and \eqref{eq:B_wins:first_equality} for $\Pr[Q]$ yields
\begin{equation}\label{eq:EAP-TLS:Q<ACCE_B}
	\Pr[Q] \leq \adv_{\protocol,f}^{\acce}(\A[B]) + \frac{2 \unionbound}{2^{c}} , 
\end{equation}
which is what we wanted to prove.

 

\paragraph{Concluding the proof of \texorpdfstring{\cref{thm:AKE_from_TLS-like_ACCE_protocol_main_result}}{Theorem~\ref{thm:AKE_from_TLS-like_ACCE_protocol_main_result}}.}


%Applying \crefrange{claim:EAP-TLS:CKerror&not-fresh<KDFcoll}{lemma:A_wins_not_Q},  
Applying \cref{claim:EAP-TLS:CKerror&not-fresh<KDFcoll},  
\cref{claim:EAP-TLS:CKerror&fresh<Q},
\cref{lemma:A_wins_not_Q}
and 
\cref{eq:EAP-TLS:Q<ACCE_B}
to \cref{eq:game_hop:remove_code_and_drop_MS},
we get
\begin{gather}\label{eq:game_hop:remove_code_and_drop_MS_expanded}
	\adv_{\protocolderived,f}^{\mathsf{G_{\prevgame{}}}}(\A)
		\leq 3 \cdot \adv_{\protocol,f}^{\acce}(\A[B])
		+ \frac{6 \unionbound}{2^{c}}
		+\adv_{\protocol.\mathsf{Kdf}}^{\kdfcoll}(\A[C]_3) .
\end{gather}

Collecting all the probabilities from \cref{game:EAP-TLS:G0} to Game~\ref{game:drop_RO}
we get 
\begin{equation}
	\adv_{\protocolderived,f}^{\akefs}(\A)
		\leq 3 \cdot \adv_{\protocol,f}^{\acce}(\A[B])
		+ \frac{6 \unionbound}{2^{c}}
		+ \frac{(\numparties \numsessions)^2 }{2^{\noncelen+1}}
		+ \sum_{i=1}^{3} \adv_{\protocol.\mathsf{Kdf}}^{\kdfcoll}(\A[C]_i)
\end{equation}

Let $\A[C]$ be the algorithm that with probability $1 / 3$ implements one of the $\A[C]_i$'s,
then
\begin{equation}
	\sum_{i=1}^{3} \adv_{\protocol.\mathsf{Kdf}}^{\kdfcoll}(\A[C]_i) \leq 3 \cdot \adv_{\protocol.\mathsf{Kdf}}^{\kdfcoll}(\A[C]),
\end{equation}
and \cref{thm:AKE_from_TLS-like_ACCE_protocol_main_result} follows.
\end{proof}



\section{Application to EAP-TLS}\label{sec:EAP-TLS-security:application_EAP-TLS}


In EAP-TLS~\cite{IETF:RFC5216:EAP-TLS} the export key $ek$  is derived as follows:
\begin{equation}\label{eq:TLS-key-exporters:key_derivation}
	ek \defeq \mathsf{tls.PRF}(ms, ``\mathtt{client\ EAP\ encryption}", \nonce_{C} \concat \nonce_{S})  .
\end{equation}
%and $\mathtt{label} = ``\mathtt{client\ EAP\ encryption}"$.

More generally,
RFC~5705: ``Keying Material Exporters for Transport Layer Security (TLS)''~\cite{IETF:RFC5705:TLS-key-exporters}
defines how export keys should be derived from the TLS handshake for any type of application.
In particular,
the export key in RFC~5707 is derived as follows:
\begin{equation}
	ek \defeq \mathsf{tls.PRF}(ms, ``\mathtt{label}", \nonce_{C} \concat \nonce_{S}, aux) ,
\end{equation}
where $\mathtt{label}$ is some application dependent label,
and $aux$ is an optional auxiliary input that can be added into the key derivation together with the nonces.
%For both EAP-TLS and RFC~5705 the security requirement on $ek$ is that it be indistinguishable from random.

Thus,
we can apply \cref{thm:AKE_from_TLS-like_ACCE_protocol_main_result} to EAP-TLS and TLS Key Exporters by in the theorem setting $\protocol = \mathsf{TLS}$ and  $\protocolderived = \operatorname{\mathsf{EAP-TLS}}$ or $\protocolderived = \operatorname{\mathsf{TLS-EXPORT}}$,
with $F_\protocol$ being $\emptystring$ or $aux$,
respectively. 
However,
we still have to argue that TLS is in fact a secure TLS-like ACCE protocol and that the TLS KDF is key collision resistant.
In the following two sections we address these questions.








\subsection{TLS security}\label{sec:EAP-TLS-security:application:TLS-security}


There is a large body of existing analysis on TLS.
Here we only focus on a small sample of these results,
based on how relevant they are to our current analysis on  EAP-TLS.
The first thing that should be acknowledged is that each study of TLS comes with its own unique security model.
Our own work is no exception in this regard.
As a consequence,
most of the existing results on TLS cannot be applied verbatim to our setting,
but will need some reinterpretation within our formal models.
Fortunately,
most of the differences are quite minor,
concerning superficial things like choice of notation and so on.
But there are also some differences that are more substantial
and which we feel are worthy to point out.
Mainly,
these have to do with the choice of corruption model
and 
%the choice of 
partnering mechanism.
Below we survey a few of the existing results on TLS and discuss how they pertain to our result on EAP-TLS.


\paragraph{Jager, Kohlar, Schäge, and Schwenk (JKKS)~\texorpdfstring{\cite{C:JKSS12}}{}.} 
JKKS were the first to conduct an analysis of the unmodified TLS~1.2 protocol,
looking specifically at the TLS-DHE ciphersuite.
They showed that TLS-DHE constitutes a secure ACCE protocol. 
The security model used by JKKS largely mirrors our own,
but with some slight differences.
First,
their corruption model is a little weaker than the 
%full forward secrecy 
\akefstext model we have used in \cref{thm:AKE_from_TLS-like_ACCE_protocol_main_result}.
Specifically,
in our  \akefstext model the adversary is allowed to corrupt the long-term keys of the test-session's peers \emph{before} it accepted provided it has a partner.
On the other hand,
in the model of JKKS these corruptions always need to happen \emph{after} the test-session accepts.
\cref{thm:AKE_from_TLS-like_ACCE_protocol_main_result} allows us to conclude that $\protocolderived = \operatorname{EAP-TLS-DHE}$ achieves the stronger security guarantees of the \akefstext model only by making a similarly strong assumption on $\protocol = \operatorname{TLS-DHE}$.
However,
\cref{thm:AKE_from_TLS-like_ACCE_protocol_main_result} can be modified to work with the weaker model of JKKS as well,
but then with a correspondingly weaker conclusion for $\protocolderived$.

Second,
JKKS used matching conversations in their analysis of TLS,
while we use partner functions.
Since matching conversations can be recast in terms of partner functions in a straightforward manner this is not a big issue.\footnote{See
\cref{sec:802.11:4WHS:AKE} for how to recast SIDs as partner functions.
The procedure for matching conversations is completely analogous.
} 
However,
there is a subtle technical difference between the ACCE model defined in this thesis and the ACCE model defined by JKSS,
stemming from the difference in choice of partnering mechanism.	
Specifically,
in JKSS's definition
of ACCE~\cite[Def.~11]{C:JKSS12} one must forbid the adversary from issuing a $\Reveal$ query towards the server after it sent out its last message,
but \emph{before}  the client to which it has a matching conversation received it.
%	Note that the client does \emph{not} have a matching conversation with the server.
This is to avoid a trivial attack whereby the adversary can re-encrypt the final message towards the client,
getting it to accept maliciously.
%\footnote{This
%restriction on $\A$'s $\Reveal$ queries was not included in the original published version \cite{C:JKSS12},
%but was later added to the online and journal versions~\cite{EPRINT:JKSS11,JagerKKS:2017:TLS-DHE:JC}.}

By contrast,
the definition of ACCE used in this thesis allows all $\Reveal$ queries.
It should be noted that the trivial attack in JKSS's model does not imply any obvious weakness in TLS,
but rather highlights a peculiarity of using matching conversations as the partnering mechanism when defining ACCE. 

\paragraph{Krawczyk, Paterson, and Wee (KPW)~\texorpdfstring{\cite{C:KraPatWee13}}{}.}

As mentioned in \cref{sec:EAP-TLS-security:intro},
KPW showed that all the main handshake variants of TLS,
i.e.,
TLS-DHE, TLS-DH, and TLS-RSA, satisfy a security notion on its key encapsulation mechanism (KEM) called IND-CCCA~\cite{C:HofKil07}. 
Additionally, 
starting from the assumption that the TLS KEM is IND-CCCA secure,
they generically proved that the full TLS protocol is a secure ACCE protocol,
Hence,
by combining these two sets of results,
KPW could show that TLS is a secure ACCE protocol for all the handshake variants TLS-DHE, TLS-DH and TLS-RSA.


Remarks similar to those we made for JKKS~\cite{C:JKSS12} also apply to KPW 
regarding their result's applicability to our analysis of EAP-TLS.
Specifically,
the ACCE model of KPW is mostly the same as JKKS's,
except for one major difference:  KPW do not treat forward secrecy at all.
Thus,
since KPW only allow us to assume (ACCE) security of TLS in our weakest corruption model,
we can correspondingly only conclude with (AKE) security of EAP-TLS in our weakest corruption model \akenfstext as well. 

%when using their result in an analog variant of \cref{thm:AKE_from_TLS-like_ACCE_protocol_main_result}.
%we only get correspondingly weaker conclusion for EAP-TLS, namely security in the \akenfstext model.





\paragraph{Kohlar et al.~\texorpdfstring{\cite{EPRINT:KohSchSch13}}{} and Li et al.~\texorpdfstring{\cite{PKC:LSYKS14}}{}.}
In parallel work to KPW \cite{C:KraPatWee13},
Kohlar et al.~\cite{EPRINT:KohSchSch13} also proved that the TLS variants TLS-RSA and TLS-DH satisfy the ACCE notion.
However,
their result is less modular,
essentially following the original approach of JKKS~\cite{C:JKSS12}.
Later,
Li et al.~\cite{PKC:LSYKS14} complemented the work of JKKS and Kohlar et al.~\cite{EPRINT:KohSchSch13},
by conducting an analysis of the pre-shared key variants of TLS,
i.e., they showed that the TLS-DHE-PSK, TLS-DH-PSK, TLS-RSA-PSK and TLS-PSK ciphersuites all satisfy the ACCE security notion (in appropriate corruption models).
As before,
the results of Kohlar et al.~\cite{EPRINT:KohSchSch13} and Li et al.~\cite{PKC:LSYKS14}
can be used to instantiate our \cref{thm:AKE_from_TLS-like_ACCE_protocol_main_result},
again given the relevant caveats on the assumed security of the underlying TLS variant and the corresponding conclusion we can draw for EAP-TLS.
Note that the result of Li et al.~\cite{PKC:LSYKS14} is also particularly interesting from the viewpoint of our first composition theorem (\cref{thm:protocol_3:3P-AKE-}),
seeing as RADIUS is based on symmetric shared secrets.
In particular,
if one wants migrate from plain RADIUS to RADIUS-over-TLS~\cite{IETF:RFC6614:RADIUS_over_TLS},
shared secrets can still be supported by using one of the TLS-PSK ciphersuites.



\paragraph{Brzuska et al.~\texorpdfstring{\cite{BrzuskaFSWW:2012:less_is_more}}{}.}

Given that the above results can be applied more or less directly to  our \cref{thm:AKE_from_TLS-like_ACCE_protocol_main_result} in order to obtain a result on EAP-TLS,
it is interesting to discuss another result where this \emph{cannot} be done.
Specifically,
Brzuska et al.~\cite{BrzuskaFSWW:2012:less_is_more} developed a generic composition framework which allowed them to show that the TLS variants TLS-DHE, TLS-DH, and TLS-RSA,
all satisfy the ACCE security notion.
However,
in their analysis---which used SIDs as the partnering mechanism---Brzuska et al.~\cite{BrzuskaFSWW:2012:less_is_more} defined the SID to consist of the parties' nonces, identities,
\emph{and}  the TLS pre-master secret.
Basing the SID upon secret values does not in general allow for public partnering.
For instance,
if the KEM used in the TLS handshake was a \emph{re-randomizable} encryption scheme~\cite{C:CanKraNie03,C:PraRos07},
then the choice of Brzuska et al.~\cite{BrzuskaFSWW:2012:less_is_more} would not allow for public partnering
(see also \cite{CCS:BFWW11} for further details).
Unfortunately,
this also means that we cannot use Brzuska et al.'s result within our \cref{thm:AKE_from_TLS-like_ACCE_protocol_main_result} since partner functions are defined on public data.






\paragraph{On TLS~1.3.}
The IETF is currently in the process of standardizing a new version of TLS, denoted TLS~1.3~\cite{IETF:DRAFT:TLS_1.3}.
Unlike the prior versions,
TLS~1.3. does not use the derived session key within the handshake itself,
and so it avoids the issue that prevented the other versions from being a secure AKE protocol.
In fact,
several preliminary analyses have already proven that different draft versions of TLS~1.3 satisfy the AKE security notion~\cite{CCS:DFGS15,EPRINT:KraWee15,EPRINT:DFGS16}.
Moreover,
in the context of exporting keys from the TLS handshake,
TLS~1.3 even defines a dedicated exporter key much like how EAP-TLS and TLS Key Exporters~\cite{IETF:RFC5705:TLS-key-exporters} do it.
Thus,
our generic result for turning an ACCE protocol into an AKE is unnecessary for TLS~1.3.
Nevertheless,
TLS~1.3 \emph{is} still a TLS-like protocol,
and so our result could in principle be applied to TLS~1.3 as well,
albeit redundantly.
Interestingly,
however,
the existing analysis of TLS~1.3 would  not work for this purpose,
because of an issue similar to the one we had with the analysis of Brzuska et al.~\cite{BrzuskaFSWW:2012:less_is_more}.
That is,
a new feature of TLS~1.3 is that many of the handshake messages are encrypted with a temporary handshake key.
But the analyses in \cite{CCS:DFGS15} and \cite{EPRINT:DFGS16},
define the SID over the \emph{unencrypted} messages.
Thus,
in trying to use these results in our \cref{thm:AKE_from_TLS-like_ACCE_protocol_main_result},
we would run into the same problems with public partnering as we had with the analysis of Brzuska et al.~\cite{BrzuskaFSWW:2012:less_is_more}.
The reason why Dowling et al.~\cite{CCS:DFGS15, EPRINT:DFGS16} are still able to carry out their analysis,
is because they leverage the fact that TLS~1.3 provides so-called \emph{multi-stage} security~\cite{CCS:FisGue14},
where different stage keys are computationally independent.
By modifying \cref{thm:AKE_from_TLS-like_ACCE_protocol_main_result} to assume a ``TLS~1.3-like'' structure on protocol $\protocol$,
and by incorporating the multi-stage assumption,
we could potentially be able to obtain a similar black-box result for EAP-TLSv1.3 as we have for EAP-TLS(v1.2).




\paragraph{Other results on TLS.}
Two other works that also analyze the TLS~(1.2) protocol are \cite{INDOCRYPT:KMOTV15} and \cite{C:BFKPSZ14}.
However,
the models used in these analyses are significantly different from ours,
making their use in \cref{thm:AKE_from_TLS-like_ACCE_protocol_main_result} difficult.
On a different note,
Bhargavan et al.~\cite{C:BFKPSZ14} showed that the \emph{full} TLS protocol,
including resumption and renegotiation,
is vulnerable to an unknown key-share attack \cite{PKC:BlaMen99}.
The attack allows an adversary to synchronize the master secret and nonces of two non-partnered sessions,
leading them to derive  the same channel key. 
While the attack carries over to EAP-TLS,
it does not invalidate our results,
since our model does not consider resumption and renegotiation. 
However,
it should be noted that this has been done for the sake of simplicity,
not because of an essential limitation in our analysis.
Our result can be extended to incorporate features like renegotiation, resumption or ciphersuite and version negotiation,
either by using the \emph{multi-phase} ACCE model of Giesen et al.~\cite{CCS:GieKohSte13}
or the \emph{multi-ciphersuite} ACCE model of Bergsma et al.~\cite{CCS:BDKSS14}.
The former has been used to prove results on TLS with renegotiation~\cite{CCS:BDKSS14},
while the latter has been used to prove results on SSH and TLS with ciphersuite and version negotiation~\cite{CCS:BDKSS14,ACISP:DowSte15}.
Since our proof uses the underlying ACCE protocol in an almost black-box way,
by adopting one of the above models we could  inherit their corresponding results for EAP-TLS as well.
%Specifically,
%the fundamental problem of resumption and renegotiation in TLS is the insufficient binding between the different phases of the protocol. 
%Both Bhargavan et al.~\cite{Bhargavan:2014:TLS-triple-handshake} and Giesen et al.~\cite{GiesenKS:2013:TLS-renogiation} 
%endorsed the \emph{Renegotiation Indication Extension} (RIE) of TLS~\cite{RFC5746:TLS:RIE},
%but additionally proposed to cryptographically bind the different phases together using a running \emph{session-hash} of some,
%or all,
%the previously sent handshake and record-layer messages.
%
%
%
%Since our result does not put any requirements on the master secret,
%our analysis can be left completely unchanged and still ``inherit'' the upgrade in security to TLS.
%Additionally,
%our use of auxiliary data in the derivation of export keys
%already covers the possibility of including more data into the EAP-TLS key derivation.
%
%
%
%While the attack restricted exclusively to the TLS protocol only enables an attacker to inject data into the channel towards the server and get it accredited to the client,
%the attack on EAP-TLS is more severe.
%In particular,
%the attacker is able to fully impersonate the client towards the server using session resumption.
%First the attacker acts as a man-in-the-middle during a full (mutually authenticated) TLS handshake between a client and a server\footnote{The attack assumes that the adversary is in possession of certificates trusted by the client and the server.}.
%Next the attacker initiates an EAP-TLS connection towards the server using the session identifier of the previous TLS handshake.
%This will trigger an abbreviated EAP-TLS session resumption handshake from the server,
%which crucially does not rely on any certificates.
%Since the attacker at this point knows both the master secret and nonces of the server,
%it can correctly derive the EAP-TLS export key,
%and use this to impersonate the client. 
%first the client connects to the attacker (acting as server),
%then the attacker connects towards the server (acting as a client),
%relaying everything sent by the client except its certificate which it substitutes with its own.




\paragraph{Alternatives to the ACCE security notion.}
%The ACCE notion is not necessarily ideal to cryptographers; 
%its monolithic nature can make modular analysis more difficult, 
%and in particular individual components of ACCE secure protocols cannot necessarily be used independently.  
%
The main reason for using the ACCE security notion in our analysis is that is has proved to be a very useful model for studying real-world protocols that intermix the key exchange stage with the channel stage. 
Since our result applies to \emph{any} ACCE protocol that is TLS-like,
it can be applied to all of these protocols in a nearly black-box manner.
In particular, 
we can plug in any existing ACCE result without having to re-do any of the steps carried out in the (ACCE) proof itself. 
For example, our result applies unmodified to every ciphersuite version of TLS for which there exist an ACCE proof.
Moreover, we can even apply our theorem to future versions of TLS, 
as long as these continue to be TLS-like and derive their channel keys using a key collision resistant KDF.
Even so, in the specific case of TLS, one might ask whether another approach could have given a simpler,
yet equally modular proof of the same result,
namely that EAP-TLS constitutes a secure AKE protocol.

Krawczyk, Paterson, and Wee (KPW)~\cite{C:KraPatWee13} showed that all the major handshake variants of TLS satisfy a security notion on its key encapsulation mechanism (KEM) called IND-CCCA~\cite{C:HofKil07}. 
If we could reduce the AKE security of EAP-TLS to the IND-CCCA security of the TLS KEM,
then the results of \cite{C:KraPatWee13} would automatically give us a corresponding result on EAP-TLS for all the major TLS ciphersuites.

Unfortunately,
it is not obvious how such a result could be obtained in a black-box manner from the KEM defined by KPW.
Technically,
in order to reduce the AKE security of EAP-TLS to the IND-CCCA security of the TLS KEM,
we need to be able to simulate the key derivation step in the AKE game of EAP-TLS.
This requires knowledge about the sessions' master secrets.
However,
the KEM defined by KPW does not contain the TLS master secret. 
%Instead, 
%the KEM key is defined to be the concatenation of the TLS \emph{channel key} 
%and the client's and servers finished messages
%(i.e., \emph{unencrypted} MAC's of the respective party's \emph{verify data}).
This means that an adversary against the TLS-KEM in the IND-CCCA game
cannot simulate the $\Test$-challenge for some adversary playing in the AKE game against EAP-TLS. 
Moreover,
as remarked by KPW~\cite[Remark~4]{C:KraPatWee13},
if the KEM key \emph{was} actually defined to be the TLS master secret,
then the resulting scheme would be insecure for TLS-RSA,
provided that RSA PKCS\#1v1.5 is re-randomizable.
On the other hand, 
Bhargavan et al.~\cite{C:BFKPSZ14}
conjecture that re-randomizing RSA PKCS\#1v1.5 is infeasible, 
allowing the master secret to be used as the KEM key in TLS-RSA too.
We forgo the whole issue by not reducing to the KEM-security of TLS at all. 


We stress that the KEM used to explain the TLS handshake in \cref{fig:TLS} is only meant for illustratory purposes,
and is \emph{not} the same as the KEM used by KPW~\cite{C:KraPatWee13}.


\subsection{On the key collision resistance of the TLS KDF}\label{sec:tls.PRF_collision_resistance}


\newcommand{\HMAC}{\overline{H}}
\newcommand{\padI}{\mathtt{ipad}}
\newcommand{\padO}{\mathtt{opad}}


The TLS key derivation function $\mathsf{tls.PRF}$ is an iterated construction based on the  HMAC~\cite{IETF:RFC2104:HMAC} function,
which itself is based on some underlying  hash function $H$. 
%Let $H$ be a hash function,
Let $\HMAC$ denote the HMAC function using $H$ as its underlying hash function,
that is,
\begin{align}
	\HMAC(\Key, M) &\overset{\text{def}}{=} H \left( (\Key \oplus \padO) \concat H\left( (\Key \oplus \padI) \concat M \right) \right) \label{eq:HMAC:def} , % \\
%	\intertext{}
%	\padI &= \mathtt{0x5c5c...5c}\\
%	\padO &= \mathtt{0x3636...36}\\
\end{align}
where $\padI$ and $\padO$ are distinct constants.

The TLS~1.2 KDF is defined as follows,
where the variable $t$ depends on how much keying material is needed:
\begin{align}
	\mathsf{tls.PRF}(\Key, M) &\overset{\text{def}}{=} \bigparallel_{i=1}^t \HMAC(\Key, A(i) \concat ``\mathtt{key\ expansion}" \concat M)\label{eq:tls.PRF:def}, \\
	 \intertext{with}
	 A(1) &= \HMAC(\Key, ``\mathtt{key\ expansion}" \concat M) \label{eq:A(1):def} \nonumber \\
	 A(i) &= \HMAC(\Key, A(i-1)).	\nonumber
\end{align}

In TLS,
$M = \nonce_{C} \concat \nonce_{S}$ is the concatenation of the client and server nonce.
Note that $\mathsf{tls.PRF}$ does not take any auxiliary input. 

\begin{theorem}\label{thm:tls.PRF:collision_resistance}
A key collision in $\mathsf{tls.PRF}$  implies a collision in $H$.
\end{theorem}
\begin{proof}

Suppose $\mathsf{tls.PRF}(\Key, M) = \mathsf{tls.PRF}(\Key', M)$, with $\Key \neq \Key'$,
and let $S = ``\mathtt{key\ expansion}" \concat M$.
By \eqref{eq:tls.PRF:def} we have in particular that 
\begin{equation}\label{eq:tls.PRF:collision_in_first_block}
	\HMAC(\Key, A(1) \concat S) = \HMAC(\Key', A'(1) \concat S),
\end{equation}
where $A'(1) = \HMAC(\Key',S)$.
%Let $X = A(1) \concat S$ and $Y = A'(1) \concat S$.
Expanding \eqref{eq:tls.PRF:collision_in_first_block} using \eqref{eq:HMAC:def} we get: 
\begin{equation}
	\begin{gathered}\label{eq:Case1}
		H \left(\Key \oplus \padO \concat H \left(\Key \oplus \padI \concat A(1) \concat S \right) \right) \\
		= \\
		H \left(\Key' \oplus \padO \concat H \left(\Key' \oplus \padI \concat A'(1) \concat S \right) \right) .
	\end{gathered}
\end{equation}
%\begin{equation}\label{eq:Case1}
%	H \left((\Key \oplus \padO) \concat H((\Key \oplus \padI) \concat n) \right)
%	=
%	H \left((\Key' \oplus \padO) \concat H((\Key' \oplus \padI) \concat n) \right).
%\end{equation}

Letting $X = H \left(\Key \oplus \padI \concat A(1) \concat S \right)$ and $Y = H \left(\Key' \oplus \padI \concat A'(1) \concat S \right)$ denote the ``inner'' hash function values,
\eqref{eq:Case1} becomes:
\begin{equation}\label{eq:}
	H (\Key \oplus \padO \concat X )
	=
	H (\Key' \oplus \padO \concat Y )  .
\end{equation}

Since $\Key \oplus \padO  \neq \Key' \oplus \padO$,
it follows that  $(\Key \oplus \padO \concat X, \Key' \oplus \padO \concat Y)$ constitute a collision in $H$. 
\end{proof}

\begin{remark}\label{remark:EAP-TLS:tls.PRF:key_collision}
The construction of $\mathsf{tls.PRF}$ in TLS~1.0/1.1 is different from the one in TLS~1.2
(shown in \cref{eq:tls.PRF:def}).
In versions prior to TLS~1.2,
$\mathsf{tls.PRF}$ is defined as $P_{\mathrm{MD5}} \oplus P_{\mathrm{SHA1}}$,
where $P_{\mathrm{MD5}}$ and $P_{\mathrm{SHA1}}$ are equal to the right-hand side of \cref{eq:tls.PRF:def} with $\HMAC$ using $\operatorname{\mathrm{MD5}}$ and $\operatorname{\mathrm{SHA1}}$,
respectively. 
Theorem~\ref{thm:tls.PRF:collision_resistance} only applies to the construction used in TLS~1.2.
\end{remark}

\begin{remark}
It is interesting to note that HMAC in general is \emph{not} key collision resistant. 
As observed by Dodis et al.~\cite{C:DRST12},
HMAC has two large classes of so-called \emph{weak keys} with exactly the property that $\operatorname{HMAC}(\Key, M) = \operatorname{HMAC}(\Key', M)$.
These weak keys arise due to an ambiguity in how HMAC handle different-length keys.
For example,
if $d$ is the block size of the underlying hash function used in HMAC and $| \Key | < d$,
then $\Key$ and $\Key' = \Key \concat 0$ lead to a key collision.
Similarly,
if $| \Key | > d$ and $\Key' = H(\Key)$ we also get a key collision.
On the other hand,
the way HMAC is used within TLS does not lead to key collisions since TLS only uses fixed-length keys.
\end{remark}

