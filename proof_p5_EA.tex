\section{Proof of Lemma~\ref{thm:protocol_5:3P-AKE-EA}}\label{sec:proof_of_EA_protocol_5}

%\small
\EAlemma*
\normalsize

\begin{proof}
%Recall that the partner function used to prove explicit entity authentication needs to be the same as when proving key-indistinguishability.
%Thus, the partner function $f_5$ we use below is the same as the one used in the proof of Theorem~\ref{thm:protocol_5:3P-AKE}.

In the following games, 
let $M_i$ denote the event that a session accepts maliciously in sub-protocol $\protocol_2$ in Game~$i$. 

\setcounter{gamehop}{0}
%\item
\paragraph{Game~\game:}
This is original 3P-AKE security experiment,
hence
\begin{equation}
	\Pr[M_0] = \Pr[M].
\end{equation}


\newgame
\paragraph{Game~\game:}\label{game_hop:3P-AKE:P5:EA:guess_MA_target}
This game implements a selective security game  similar to  Game~\ref{game_hop:3P-KD:guess_test-session} in the proof of Theorem~\ref{thm:protocol_3:3P-AKE-}.
However,
this time the adversary is required to commit to the session that will accept maliciously first.

Specifically,
at the beginning of the game the adversary must first choose a pair $(U,i)$,
with $i \in [1, \numsessions]$.
The game then proceeds as in Game~\prevgame{},
except that if some session accepts maliciously before $\oracle[U][i]$,
the challenger aborts the game and outputs $0$ 
(i.e., $\A$ loses).
In particular,
this includes the possibility that $\A$ makes a query that renders $\oracle[U][i]$ unfresh
(which would preclude $\oracle[U][i]$ from accepting maliciously).
\begin{lemma}\label{lemma:3P-AKE-EA:selective-security:EA-target}
$
	\Pr[M_{\prevgame}] 
	\leq \numsessions \cdot | \mathcal{I} \cup \mathcal{R} | \cdot \Pr[M_{\game}].
$
\end{lemma}

\begin{proof}
The proof is essentially the same as for Game~\ref{game_hop:3P-KD:guess_test-session} in Theorem~\ref{thm:protocol_3:3P-AKE-} (Lemma~\ref{lemma:3P-KD:guess_test-session}),
only that this time the selective security adversary guesses one session rather than two. 
\end{proof}

In the remaining games,
let $\oracle[U][i]$ denote the session that the adversary commits to in Game~\game{}. 
Note that $\oracle[U][i]$ is not necessarily the same as the test-session eventually chosen by the adversary.

\newgame
\paragraph{Game~\game:}\label{game_hop:3P-AKE:P5:EA:guess_P3_partner}
This game extends the selective security requirements of Game~\prevgame{} by demanding that the adversary also commits to the partner of $\oracle[U][i]$ in sub-protocol $\protocol_3$ (if any).

Specifically,
at the beginning of the game the adversary must pick a pair $(U,i)$ as in Game~\prevgame{},
and also pick a pair $(V,j)$,
with $j \in [0, \numsessions]$.
The game then proceeds as in Game~\prevgame{},
but it additionally aborts if $\oracle[U][i]$ gets a different $f_3$-partner than $\oracle[V][j]$ in sub-protocol $\protocol_3$ 
(including the case that $\oracle[U][i]$ \emph{gets} an $f_3$-partner when~$j = 0$).


\begin{remark}
Note that there is no contradiction between $\oracle[U][i]$ accepting maliciously in protocol $\protocol_5$ according to partner function $f_5$,
while simultaneously having an $f_3$-partner in sub-protocol $\protocol_3$. 
\end{remark}

\begin{lemma}\label{lemma:3P-AKE-EA:selective-security:EA-target-p3-partner}
$
	\Pr[M_{\prevgame}]
	\leq (\numsessions + 1) \cdot \max \lbrace | \mathcal{I} |, | \mathcal{R} | \rbrace \cdot \Pr[M_{\game}],
$
\end{lemma}

\begin{proof}
Again,
from an adversary $\A$ that plays against the \emph{single} selective security requirement of Game~\prevgame{},
we can create an adversary $\A'$ against the \emph{two} selective security requirements of Game~\game{}:
after $\A$ outputs its commitment to a pair $(U,i)$,
then $\A'$ guesses another pair $(V,j)$
(conditioned on the role of $U$),
and outputs $(U,i)$ and $(V,j)$ as its own commitments to Game~\game{}.
%\qed
\end{proof}

In the remaining games,
let $\oracle[V][j]$ denote the (possibly empty) $f_3$-partner of $\oracle[U][i]$ that the adversary commits to in Game~\game{}
(in addition to $\oracle[U][i]$).

\newgame
\paragraph{Game~\game:}
This game proceeds as the previous one,
but when the first session out of $\oracle[U][i]$ and $\oracle[V][j]$ accepts in sub-protocol $\protocol_3$, 
then the challenger replaces  its intermediate key $\KDkey$
%output by protocol $\protocol_3$
with a random key $\KDkeyrand$. 
When the other session accepts in~$\protocol_3$, 
the challenger replaces its intermediate key with the same random key~$\KDkeyrand$.
%\item
%\vspace{-\baselineskip}

\begin{lemma}\label{claim:fantastic-claim}
$
	\Pr[M_{\prevgame}]
	\leq \Pr[M_{\game}]
	+ 2 \cdot \adv_{\protocol_3,\A[B]'_1,f_3}^{\operatorname{\mathsf{3P-AKE}}^w}(\secparam)  .
$
\end{lemma}
\begin{proof}[Proof (sketch)]
Reduction $\A[B]'_1$ begins by choosing a random bit $b_{sim}$.
%and guessing the two sessions $\oracle[U][i]$ and $\oracle[V][j]$ as in Game~\ref{game_hop:3P-AKE:P5:EA:guess_MA_target} and Game~\ref{game_hop:3P-AKE:P5:EA:guess_P3_partner}.
It then runs $\A$ and implements all the abort conditions introduced so far.
All of $\A$'s $\Q{Send}$ queries that pertain to the 3P-AKE sub-protocol $\protocol_3$,
$\A[B]'_1$ forwards to its own 3P-AKE experiment. 




When the first session out of $\oracle[U][i]$ and $\oracle[V][j]$ accepts in sub-protocol $\protocol_3$, 
$\A[B]'_1$ makes a $\Q{Test}$ query to its own 3P-AKE experiment in order to obtain its session key $\key^*$
(which is either its actual key or a random key).
$\A[B]'_1$ uses $\key^*$ as that session's intermediate key $\KDkey$ in protocol $\protocol_5$.
When the second session out of $\oracle[U][i]$ and $\oracle[V][j]$ accepts in sub-protocol $\protocol_3$, 
then $\A[B]'_1$ again uses the key $\key^*$ as its intermediate key $\KDkey$. 
For all  sessions different from $\oracle[U][i]$ and $\oracle[V][j]$ $\A[B]'_1$  obtains their intermediate keys by making a corresponding $\Reveal$ query to its own 3P-AKE experiment. 



All of $\A$'s $\Q{Send}$ queries that pertain to sub-protocol $\protocol_4$,
$\A[B]'_1$ answers itself using the intermediate keys it obtained from sub-protocol $\protocol_3$ as the ``long-term keys'' for sub-protocol $\protocol_4$.
In particular,
$\A[B]'_1$ derives the session keys of every session in sub-protocol $\protocol_4$ 
(and hence in protocol $\protocol_5$) itself.
Using the session keys so-obtained,
together with the bit $b_{sim}$ it drew in the beginning of the simulation,
$\A[B]'_1$ can answer $\A$'s $\Q{Test}$-query.




Finally, 
when $\oracle[U][i]$ accepts in protocol $\protocol_5$,
then $\A[B]_1$ stops it simulation and outputs a bit $b_{out}$ to its own 3P-AKE experiment defined as follows.
If $\oracle[U][i]$ accepted maliciously in protocol $\protocol_5$,
then $b_{out} = 0$,
otherwise, $b_{out}$ is defined to be a random bit.
\medskip

Recall that $\A[B]'_1$ picks its test-session based on which of $\oracle[U][i]$ and $\oracle[V][j]$ accepted first in sub-protocol $\protocol_3$.
We argue that if $\oracle[U][i]$ accepts maliciously in $\A[B]'_1$'s simulation,
then both $\oracle[U][i]$ and $\oracle[V][j]$ are valid  test-targets in $\A[B]'_1$'s own 3P-AKE experiment
(i.e., fresh according to predicate $\fresh_{\mathsf{AKE}^w}$).
We consider three cases: 

\begin{itemize}
	\item \emph{$\oracle[U][i]$ chosen as test-session and $j \neq 0$:}  
	In this case $\A[B]'_1$ makes no $\Reveal$ query towards $\oracle[U][i]$
	or its $f_3$-partner $\oracle[V][j]$
	in its own 3P-AKE experiment
	because it obtains their intermediate keys from sub-protocol $\protocol_3$ through the $\Q{Test}$ query (to $\oracle[U][i]$).
	Moreover,
	since $\oracle[U][i]$ has an $f_3$-partner,
	it remains AKE$^w$-fresh even if its peers are corrupted.
	
	\item \emph{$\oracle[U][i]$ chosen as test-session and $j = 0$:}  
	Again, $\A[B]'_1$ makes no $\Reveal$ query towards $\oracle[U][i]$ in its own 3P-AKE experiment,
	and since $\oracle[U][i]$ doesn't have an $f_3$-partner,
	$\A[B]'_1$ of course makes no $\Reveal$ query to that either.
	We claim that $\A[B]'_1$ also never issued a $\Corrupt$ query to $\oracle[U][i]$'s peers.
	To see this,
	note that if $\oracle[U][i]$ is to accept maliciously in protocol $\protocol_5$,
	then it must be fresh according to predicate $\fresh_{\mathsf{AKE}}$.
	In particular,
	this means that $\A$ cannot issue any $\Corrupt$ queries to $\oracle[U][i]$'s peers \emph{before} $\oracle[U][i]$ accepted.\footnote{Recall 
	that predicate $\fresh_{\mathsf{AKE}}$ forbids any $\Corrupt$ query to a session's peers if (1) it does not have a partner, and (2) it has not accepted yet. 
	This corresponds exactly to the setting we are in when a session accepts maliciously.}  
	But
	$\A[B]'_1$ stops its simulation immediately once $\oracle[U][i]$ accepts,
	so no $\Corrupt$ query will actually be forwarded towards $\oracle[U][i]$'s peers  in $\A[B]'_1$'s 3P-AKE experiment in this case.

	
	
	
	\item \emph{$\oracle[V][j]$ chosen as test-session:}
	Note that by symmetry of the $f_3$ partner function,
	$\oracle[V][j]$ has $\oracle[U][i]$ as its $f_3$-partner.
	Thus, $\A[B]'_1$ makes no $\Reveal$ query towards $\oracle[V][j]$ or its $f_3$-partner ($\oracle[U][i]$),
	since this is answered with the $\Q{Test}$ query instead'
%	\footnote{The
%		potential issue was that $\oracle[V][j]$ could have had a different $f_3$-partner than $\oracle[U][i]$.
%		If $\A$ then asked to reveal that session,
%		$\A[B]'_1$ would have answered with a $\Reveal$ query,
%		hence breaking the AKE$^w$-freshness of $\oracle[V][j]$ in its own 3P-AKE experiment.
%	}
	Moreover,
	since $\oracle[V][j]$ has an $f_3$-partner,
	it remains AKE$^w$-fresh even if its peers are corrupted.




\end{itemize}

Taken together,
the above cases show that no-matter which one of $\oracle[U][i]$ and $\oracle[V][i]$ was selected as the test-session by $\A[B]'_1$,
it will be fresh according to predicate $\fresh_{\mathsf{AKE}^w}$ in $\A[B]'_1$'s 3P-AKE experiment if $\oracle[U][i]$ accepted maliciously.

\medskip

Finally,
we analyze $\A[B]'_1$'s advantage.
Let $b$ denote the challenge-bit used in $\A[B]'_1$'s own 3P-AKE experiment
and let $\mathsf{MA}^{\A[B]'_1}$ denote the event that $\oracle[U][i]$ accepted maliciously in~$\A[B]'_1$'s simulation.
%and let ``$\A[B]'_1 \Rightarrow b_{out}$'' denote the event that $\A[B]'_1$ outputs $b_{out}$ to its 3P-AKE experiment.



If $b=0$,
then $\A[B]'_1$'s $\Q{Test}$ query is answered with a real key.
In this case $\A[B]'_1$ simulates Game~\prevgame{} perfectly for $\A$ up until the point when $\oracle[U][i]$ accepts (in protocol $\protocol_5$),
thus:
\begin{align}\label{eq:proof:p5:EA:swap_keys_p3:MA_b=0}
	\Pr[\mathsf{MA}^{\A[B]'_1} \mid b = 0] &= \Pr[M_{\prevgame}] .
\end{align}


On the other hand,
if $b=1$,
meaning that $\A[B]'_1$'s $\Q{Test}$ query is answered with a random key,
then $\A[B]'_1$ simulates Game~\game{} perfectly for $\A$,
thus:
\begin{align}\label{eq:proof:p5:EA:swap_keys_p3:MA_b=1}
	\Pr[\mathsf{MA}^{\A[B]'_1} \mid b = 1] &= \Pr[M_{\game}] . 
\end{align}

Furthermore,
the probability that $\A[B]'_1$ outputs $b_{out} = 0$ when the event $\mathsf{MA}^{\A[B]'_1}$ occurs is $1$, and
and when it don't occurs the probability is $1/2$
(independently of the value of $b$).
Thus, $\A[B]'_1$'s advantage is:
%\small
\begin{align}
	\begin{split}
	\adv_{\protocol_3,\A[B]'_1,f_3}^{\operatorname{\mathsf{3P-AKE}}^w}(\secparam)
		&= \Pr[\Exp_{\protocol,\queryset,\A}^{\keyind^w}(\secparam) \Rightarrow 1 \mid b = 0] \\
			& \qquad	- \Pr[\Exp_{\protocol,\queryset,\A}^{\keyind^w}(\secparam) \Rightarrow 0 \mid b = 1]
	\end{split} \\
%		\begin{split}
%			&=   \Pr[\A[B]'_1 \Rightarrow 0 \mid b = 0 \land \mathsf{MA}^{\A[B]'_1}] \cdot \Pr[\mathsf{MA}^{\A[B]'_1} \mid b = 0] \\
%			&\qquad + \Pr[\A[B]'_1 \Rightarrow 0 \mid b = 0 \land \neg \mathsf{MA}^{\A[B]'_1}] \cdot \Pr[\neg \mathsf{MA}^{\A[B]'_1} \mid b = 0]   \\
%			&\phantom{=} - \Pr[\A[B]'_1 \Rightarrow 0 \mid b = 1 \land \mathsf{MA}^{\A[B]'_1}] \cdot  \Pr[\mathsf{MA}^{\A[B]'_1} \mid b = 1]  \\
%			&\qquad  - \Pr[\A[B]'_1 \Rightarrow 0 \mid b = 1 \land \neg \mathsf{MA}^{\A[B]'_1}] \cdot \Pr[\neg \mathsf{MA}^{\A[B]'_1} \mid b = 1] 	 	
%		\end{split} \\
	\begin{split}
			&=   1 \cdot \Pr[\mathsf{MA}^{\A[B]'_1} \mid b = 0] + \frac{1}{2} \cdot \left( 1 - \Pr[ \mathsf{MA}^{\A[B]'_1} \mid b = 0] \right)   \\
			&\quad - 1 \cdot   \Pr[\mathsf{MA}^{\A[B]'_1} \mid b = 1]  - \frac{1}{2} \cdot \left( 1 - \Pr[\mathsf{MA}^{\A[B]'_1} \mid b = 1] \right) 	
		\end{split} \\
	&\hspace{-12.25pt}\overset{\eqref{eq:proof:p5:EA:swap_keys_p3:MA_b=0}+\eqref{eq:proof:p5:EA:swap_keys_p3:MA_b=1}}{=} \frac{1}{2} \cdot  
		\Pr[M_{\prevgame}] - \frac{1}{2} \cdot \Pr[M_{\game}] ,
\end{align}
\normalsize
which proves the lemma.
%\qed
\end{proof}


Finally, we show that if $\oracle[U][i]$ accepts maliciously in Game~$\game$,
then it must also have accepted maliciously in sub-protocol $\protocol_4$.
Here we use that the AKE$^\mathsf{static}$ experiment allows key-registration of long-term PSKs.
\begin{lemma}\label{lemma:3P-AKE-EA:EA-p4}
$
	\Pr[M_{\game}] 
	\leq  \adv_{\protocol_4,\A[B]'_2,f_4}^{\operatorname{\mathsf{2P-AKE^{static}-EA}}}(\secparam)  .
$
\end{lemma}

\begin{proof}[Proof (sketch)]
Reduction $\A[B]'_2$ begins by creating all the long-term keys for sub-protocol $\protocol_3$, 
%drawing a random bit $b_{sim}$.
%and guesses the two sessions $\oracle[U][i]$ and $\oracle[V][j]$ as in Game~\ref{game_hop:3P-AKE:P5:EA:guess_MA_target} and Game~\ref{game_hop:3P-AKE:P5:EA:guess_P3_partner}.
then runs $\A$.
For all of $\A$'s $\Q{Send}$ queries that pertain to sub-protocol $\protocol_3$,
$\A[B]'_2$ answers itself using the keys it created.
Once an initiator or responder session $\oracle$ accepts in sub-protocol $\protocol_3$,
then $\A[B]'_2$ forwards all of $\A$'s remaining $\Q{Send}$ queries towards $\oracle$ to its own 2P-AKE$^\mathsf{static}$ experiment.


Specifically,
once a session $\oracle$ accepts in in sub-protocol $\protocol_3$
then $\A[B]'_2$ creates a corresponding ``proxy'' session in its own 2P-AKE$^\mathsf{static}$ experiment with a $\Q{NewSession}$ query.
If $\oracle$ is one of the two selective security targets $\oracle[U][i]$ and $\oracle[V][j]$,
then $\A[B]'_2$ uses a $\Q{NewSession}$ query \emph{without} key registration.
On the other hand,
if $\oracle$ is different $\oracle[U][i]$ and $\oracle[V][j]$,
then $\A[B]'_2$ uses a $\Q{NewSession}$ query \emph{with} key registration,
where the key that $\A[B]'_2$ registers is the session key it derived for $\oracle$ in sub-protocol~$\protocol_3$. 
 


To answer $\A$'s $\Corrupt$ queries,
$\A[B]'_2$ uses the long-term keys it created for sub-protocol $\protocol_3$.
$\A$'s  $\Reveal$ queries,
as well as $\Q{Test}$ query,
$\A[B]'_2$ forwards to its own 2P-AKE$^\mathsf{static}$ experiment.



Finally, 
when $\oracle[U][i]$ accepts in protocol $\protocol_5$,
then $\A[B]'_2$ stops it simulation and outputs a random bit to its 2P-AKE$^\mathsf{static}$ experiment.\footnote{We 
do not care about the value of $\A[B]'_2$'s output since we are only interested in its ability to get a session to accept maliciously in sub-protocol $\protocol_4$.
} 
\medskip


Note that $\A[B]'_2$ provides a perfect simulation of Game~\game{}.
Specifically,
it provides a correct simulation of sub-protocol $\protocol_3$ since it created all the keys itself.
Moreover,
the simulation of sub-protocol $\protocol_4$ is also perfect,
since the long-term keys used by the proxy sessions in $\A[B]'_2$'s own 2P-AKE$^\mathsf{static}$ experiment are identically distributed to the ``intermediate'' keys in  protocol $\protocol_5$.
In particular,
for sessions different from $\oracle[U][i]$ and $\oracle[V][j]$ the use of key registration ensures that the PSK used by the corresponding proxy sessions is exactly the same as intermediate keys derived in protocol $\protocol_5$. 
On the other hand,
when $\A[B]'_2$ creates the proxy sessions corresponding to $\oracle[U][i]$ and $\oracle[V][j]$,
it does so \emph{without} key registration.
By definition,
this means that in $\A[B]'_2$'s 2P-AKE$^\mathsf{static}$ experiment these proxy sessions  will use an 
an independent, uniformly distributed PSK.
However,
by the change in Game~\game{},
$\oracle[U][i]$ and $\oracle[V][j]$'s intermediate keys in protocol $\protocol_5$ are replaced with an independent, randomly distributed key $\widetilde{\Key}$,
hence the distribution is the same.


\medskip
It remains to argue that if $\oracle[U][i]$ accepts maliciously in Game~\game{},
then it must also have accepted maliciously in sub-protocol $\protocol_4$---which by extension implies that the corresponding proxy session in $\A[B]'_2$'s 2P-AKE$^\mathsf{static}$ experiment accepted maliciously too.

First we claim that session $\oracle[V][j]$ cannot be $\oracle[U][i]$'s partner in sub-protocol~$\protocol_4$.
To see this,
note that if $j = 0$,
then $\A[B]'_2$ never creates a corresponding proxy session in its 2P-AKE$^\mathsf{static}$ experiment,
hence $\oracle[U][i]$ cannot possibly have a partner in sub-protocol $\protocol_4$.
On the other hand,
if $j \neq 0$,
then by the definition of $\oracle[V][j]$  (ref. Game~\ref{game_hop:3P-AKE:P5:EA:guess_P3_partner}) 
it must be $\oracle[U][i]$'s $f_3$-partner in sub-protocol $\protocol_3$.
But if $\oracle[V][j]$ was \emph{also} the $f_4$-partner of $\oracle[U][i]$ in sub-protocol $\protocol_4$,
then it would necessarily be $\oracle[U][i]$'s $f_5$-partner as well---contradicting the fact that $\oracle[U][i]$ was supposed to accept maliciously.

Second,
we claim that $\oracle[U][i]$ is fresh according to predicate $\fresh_{\mathsf{AKE^{static}}}$.
This is true because $\A[B]'_2$ makes no $\Corrupt$ query at all in its 2P-AKE$^\mathsf{static}$ experiment,
and also makes  no $\Reveal$ query to the proxy session of $\oracle[U][i]$.\footnote{As argued by the previous claim,
$\oracle[V][j]$ is not $\oracle[U][i]$'s $f_4$-partner in sub-protocol $\protocol_4$,
so a $\Reveal$ query to it does not affect the AKE$^{\mathsf{static}}$-freshness of $\oracle[U][i]$.
}

Together,
these two claims show that the proxy session corresponding to $\oracle[U][i]$ accepted maliciously in $\A[B]'_2$'s 2P-AKE$^\mathsf{static}$ experiment. 
\end{proof}

Combining Lemma~\ref{lemma:3P-AKE-EA:selective-security:EA-target} to Lemma~\ref{lemma:3P-AKE-EA:EA-p4} yields Lemma~\ref{thm:protocol_5:3P-AKE-EA}.
%\qed
\end{proof} %% End of 2nd composition theorem (EA)